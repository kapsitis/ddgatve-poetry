\documentclass[11pt]{article}
\usepackage{ucs}
\usepackage[utf8x]{inputenc}
\usepackage{changepage}
\usepackage{graphicx}
\usepackage{amsmath}
\usepackage{gensymb}
\usepackage{amssymb}
\usepackage{enumerate}
\usepackage{tabularx}
\usepackage{lipsum}

\setlength{\parskip}{\baselineskip}%
\setlength{\parindent}{0pt}%

\oddsidemargin 0.0in
\evensidemargin 0.0in
\textwidth 6.27in
\headheight 1.0in
\topmargin 0.0in
\headheight 0.0in
\headsep 0.0in
%\textheight 9.69in
\textheight 9.00in

\setlength\parindent{0pt}

\newenvironment{myenv}{\begin{adjustwidth}{0.4in}{0.4in}}{\end{adjustwidth}}
\renewcommand{\abstractname}{Anotācija}
\renewcommand\refname{Atsauces}

\newenvironment{uzdevums}[1][\unskip]{%
\vspace{3mm}
\noindent
\textbf{#1:}
\noindent}
{}

\newcommand{\subf}[2]{%
  {\small\begin{tabular}[t]{@{}c@{}}
  #1\\#2
  \end{tabular}}%
}



\newcounter{alphnum}
\newenvironment{alphlist}{\begin{list}{(\Alph{alphnum})}{\usecounter{alphnum}\setlength{\leftmargin}{2.5em}} \rm}{\end{list}}


\makeatletter
\let\saved@bibitem\@bibitem
\makeatother

\usepackage{bibentry}
%\usepackage{hyperref}


\begin{document}

\begin{center}
{\LARGE \bf Tālu ceļu vējš}
\end{center}



\section{Tā puksti sirds}

\begin{quote}
{\em
Vienreiz gadā atnāk brīži tādi,\\
Bet lai cauri dzīvei dzirdam to ---\\
Ledus lūst ar dobju kanonādi,\\
Palu straumes spēku atzīstot!
}
\end{quote}

{\bf Dzimtenei}

Kaut arī jūlijā\\
Tāpat reiba no birztalu tvana ---\\
Tās bija citādas vasaras,\\
Ausīs vēl tagad skan\\
Dziesma, ko dziedāja sanošās vārpas,\\
Kuras nebrieda man.\\
Taču jau toreiz tu biji\\
Dzimtene mana...\\
Un tomēr nebiji mana!

Pa dzimto zemi ejot,\\
Visur man kliedza: --- Nav tava! ---\\
Un pat robežas smilgainā grāvī\\
Nebija zemes man.\\
Nebija dzimtās zemes.\\
Vectēvs to atrada gan ---\\
Jo, kad to nesa\\
Zem kapsētas kļavām,\\
Neviens neteica: --- Tā nav tava. ---

Un bez dzimtenes\\
Izdzisa dzīve viena ---\\
Saujiņai bija,\\
Bet tūkstošiem nebija Latvijas.\\
... Un tad\\
Sarkans, ugunīgs karogs\\
Pār Daugavu atvijās,\\
Pirmo reizi kā dzimtajā\\
Varēja dzimtajā birzī ienākt\\
Tavā pirmajā dzimšanas dienā.

... Atkal ir jūlijs\\
Un var noreibt no birztalu tvana,\\
No dzimto birztalu tvana,\\
No dzimtenes vārpām, kas san,\\
Un ikviens no mums zin ---\\
Tagad tās nobriedīs man.

Nepatīk bagātiem kungiem\\
Karoga krāsa tava,\\
Vairākkārt viņi pēc tevis\\
Izstiepa viņi pēc tevis\\
Izstiepa pirkstus badīgos.

Blakus ar piecpadsmit māsām\\
Cauri gadiem ej\\
Piecpadsmit-, divdesmit-, piecdesmitgadīga,\\
Un, tāpat kā zem piecpadsmit karogiem,\\
Arī zem karoga tava\\
Ies slava!

\newpage

{\bf Maskava, māt!}

Ar tevi būt mēs ļoti sen jau alkām,\\
Nu ejam kopā darba slavu krāt,\\
Nu nākam mēs ar Dzintarjūras šalkām\\
Pie tevis, Maskava, pie tevis, māt!

Tev ielas dzied kā kalnu upes ašas\\
Par miera zemes miera dzīvi šo,\\
Un dziesmas pat uz visas zemes plašās\\
Nav lielākas un skaistākas par to.

Mums daudz bij gadu simtos cirstu rētu,\\
Bet atkal stipri nākam blakus stāt ---\\
Nav tāda spēka, kas mūs uzvarētu\\
Ar tevi, Maskava, ar tevi, māt!

Ik diena mūs pret jaunām tālēm aiznes,\\
Lai kādi bargi vēji pretī elš, ---\\
Mums spēka daudz. Mūs vada tavas zvaitznes,\\
Un, kur tās ved, tur nepazudīs ceļš.

Šķir plašums mūsu upju viļņu vālus,\\
Bet mēs ar sirdi pieaugām tev klāt ---\\
Nav tādas vietas, kur meš būtu tālu\\
No tevis, Maskava, no tevis, māt!

\newpage

{\bf Vecais ostinieks sagaida kuģi}

Nāk pie krasta viļņi sudrabaini,\\
Stājas tvaikonis, --- No kurienes tu tāds? ...\\
Atkal mūsu pusē, "Jāni Raini"! ---\\
Celtņa roku sniedzu: --- Sveicināts!

TIklīdz ļaujos atmiņām un domām ---\\
Augsta piere, sirmu matu sniegs...\\
Liekas, nevis kuģi šūpo joma, \\
Bet mēs sastopamies, dziesminiek.

Spožām acīm tevī raugās nami,\\
Viļņos spuldžu atspīdumi dziest.\\
Reiz kā draugi bijām pazīstami\\
Piektā gada brāzmās, atminies.

Pagrabos bij saslodzīti vēji,\\
Stāvošs ūdens kļuva aļģēm segts,\\
Tad tu pacēlies un vētru sēji,\\
Cēlies sirdīs zvaigžņu liesmas degt.

Tos, kam krūtīs šaubu aukstums bija,\\
Dzeju lapas pusēs slēptais rīts\\
Dzīvinošiem stariem sasildīja,\\
Palu straumē rāva citiem līdz. ---

... Parka kokos putni daino klusi,\\
Mēness maurā ēnu rakstu auž,\\
Nakts jau lēnām iet uz rīta pusi,\\
Pirmā vēsma kuģa mastu glauž.

Visas zīmes rāda labu laiku,\\
Tātad rītu ceļš tev tālāk ies.\\
Labu ceļa vēju! Pilnu tvaiku!\\
Komunisma ostā tiksimies!

\newpage

{\bf Varoņu kapos}

Šeit nav nāves. Nav un nevar būt,\\
Jo, kad dzīvē kādreiz pagurt sākam,\\
Šurp ar viņiem parunāties nākam,\\
Nevis raudāt --- nākam stipri kļūt.

Citā vietā sacīsim varbūt:\\
Viņi krita. Kauja bij par sīvu ---\\
Šeit mēs tiekamies kā dzīvs ar dzīvu.\\
Šeit nav nāves. Nav un nevar būt.

\newpage

{\bf Pēdējais raunds}

Pēdējais raunds. Tas būs asāks kā citi,\\
Rokās milzīgu spēku gribas dzirkstele kurs.\\
...Melnais pasaules čempions Bertlings Siti\\
Tuvu uzvarai. Baltais pretinieks gurst.

Pēdējais raunds. Stāv pie bulvāra stūra\\
Sprogains puisītis. Namam grib tuvāk kļūt,\\
Gaida tēti, bet priekšā --- publikas jūra.\\
Siti uzvar, un... vai tas drīkst būt?

--- Linčot! --- kāds izbrēc. Saucienu atkārto citi,\\
Naidā sašķiebtām sejām ringā slepkavas skrien.\\
Melnais pasaules čempions Bertlings Siti\\
Pēdējā raundā stājas pret simtiem --- viens...

Bija cilvēks... Nē, misteriem tas nebūs lēti ---\\
Sprogainais puisītis izaugs, pieņemsies spēkā un tad\\
Lielās taisnības dienā slepkavām prasīs par tēti\\
Un par visu, ko nevar piedot nekad!

\newpage

{\bf No pasaran\footnote{Viņi neizlauzīsies!}}

Ložu celtos putekļos vērpies\\
Lielceļš, kur tagad nebraukā rati, ---\\
Pa to viņš atnāca, vienkārši tērpies,\\
Vienkāršs kā Spānija pati.

Kurš gan domāja toreiz par slavu,\\
Gājām kopā, vai kalns vai dzelme.\\
Rītā frankisti aplenca gravu ---\\
Tīrā uguns un svelme!

Gaisā frankistu granātas kauciens,\\
Lūpās sāji un lipīgi sviedri,\\
Atkal, atkal pār tranšeju sauciens:\\
--- Tomēr {\em no pasaran}, biedri! ---

Zeme drebēja. Šķembu virmā\\
Drīz vien mēs nokvēpām melni.\\
Tad no putekļu mākoņa sirmā\\
Iznira priekšā šie velni.

Viņš, kam bij jaunības skaņa vārdos,\\
Bet sarmā mirdzēja mati,\\
Pāri tranšejai kaujas dārdos\\
Cēlās kā Spānija pati.

Viņu pazina. Un nav brīnums ---\\
Varoņus pazīst karā.\\
Viņu sauca, un tas ar smīnu\\
Iegāja frankistu barā.

--- Gribat ar mani kopā staigāt?\\
Iesim! --- nošalca spontāns.\\
...Tur, kur bij viņi, izplauka baigā\\
Granātu sprādziena fontāns.

Tā nav pasaka, teika gan ---\\
Kopš tās dienas, kāds stāstīja man,\\
Visos Spānijas vējos skan:\\
--- {\em No pasaran!}

\newpage

{\bf Sirds meklē dziesmu}

Zvaigžņu miglājā tālums tinas,\\
Tādā miglājā staru daudz,\\
Bet vai zvaigžņu pulku jūs zināt,\\
Ko par Kahovkas zvaigznāju sauc?

Grūti zināt, vēl grūtāk sacīt,\\
Zvaigžņu atlantu velti šķirt,\\
Bet es redzēju savām acīm,\\
Ka tāds zvaigznājs patiešām ir!

Vējš tik nemierīgs noglauž āri,\\
Lietus sadomā zemi pērt,\\
Bet tur tālē --- Kahovkai pāri ---\\
Zvaigžņu piebārstīts plašums vērts.

Mašīna tuvāk blāzmai aiznes,\\
Prasu šoferim: --- Kā jums šķiet, \\
Nav taču zināmas tādas zvaigznes,\\
Kas caur negaisa mākoņiem zied? ---

Vēlāk kauns, kad pie krasta kraujas\\
Visu varēja saskatīt ---\\
Zilas raķetes kā pirms kaujas\\
Mierīgā Dņepras dzelmē krīt.

Un sirds meklē man tādas dziesmas,\\
Kurās apdziedāt ļaudis šos,\\
Autogēnu zilganās liesmas\\
Augsto sastatņu debešos.

Īstāko vārdu atrast grūti,\\
Ilgu laiku tas meklēts jau,\\
Stāsti sev, kad gurumu jūti:\\
Klausies, viņiem tur vieglāk nav.

...Atkal Kahovku dunam dzirdi,\\
Redzi, kā sastatnēs zvaigznes mirdz.\\
Viņi darbā ielika sirdi,\\
Tā savā dziesmā man jāieliek sirds!

\newpage

{\bf Kaklautu noraisot}

Kā lai noņem to, kas pieaudzis pie sirds?\\
Nevar noņemt. Kļūst uz mirkli drūmi,\\
Jo aiz loga tāle saucot mirdz\\
Un kūp pionieru ugunskuru dūmi.\\
Kā lai noņem to, kas pieaudzis pie sirds.

Visi tālie ceļi, kas šais gados bij,\\
Vis, ko ugunīgais kaklauts devis,\\
Lai sirds kļūtu rada ugunij, ---\\
Šodien viss sauc atpakaļ pie sevis,\\
Visi tālie ceļi, kas šais gados bij.

Mezgls atraisīts. Deg ugunīgais zīds,\\
Pašreiz jūti visu to, kas noiets.\\
Bet vēl nejūti, cik liels šis rīts\\
Un ka savu bērnību tu noliec ---\\
Mezgls atraisīts. Deg ugunīgais zīds.

Nevar noņemt to, kas pieaudzis pie sirds,\\
Un lai paliek ugunskura liesma ---\\
Tajā lai tev acu skatiens mirdz\\
Jauns un nemierīgs kā pionieru dziesma.\\
Nevar noņemt to, kas pieaudzis pie sirds.

Tālāk --- jaunība un ceļš pret kalnu būs,\\
Un ne vienmēr tur būs gluda iela.\\
Šodien tu kā strauts, kas šalcot plūst,\\
Ieej straumē varenā un lielā ---\\
Tālāk --- jaunība un ceļš pret kalnu būs!

\newpage

{\bf Atvadu dziesma skolai}

{\em Draugiem no Cesu 1. vidusskolas}

Šodien šķiršanos jūt,\\
Tāpēc dziesma varbūt\\
Arī neiznāk līksma, kā vajag.\\
Smieties negribas mums,\\
Varbūt labāk, ja skumst\\
Mūsu šķiršanās minūtē šajā.

Kā to vārdā lai sauc,\\
Laikam mīļuma daudz\\
Mums šais gados pret tevi ir krājies,\\
Un tik aicinošs skan\\
Tavu gaiteņu zvans,\\
Taču mums vairs uz klasi nav jāiet...

Tavas atmiņas --- būs!\\
Tās nav sniegi, kas kūst\\
Un uz jūru ko aizskalo pali, ---\\
Viss, ko devi mums līdz,\\
Šodien krūtīs mums trīc,\\
Un uz mūžu tas krūtīs paliks.

Celsies ziemelis ass,\\
Taču mums vairs nemaz\\
Nav no vētrām un puteņiem bailes:\\
Vēji brāzmaini trauc,\\
Un caur tālumu sauc\\
Kalnu augstās un sniegotās smailes.

Mūsu ceļā lai rīt\\
Dārdot lavīnas krīt,\\
Trako vēji un kupenas sasnieg,\\
Bet uz pasaules jau\\
Tādu virsotņu nav,\\
Kuras drosmīgie nespētu sasniegt.

...Šeit mēs atnācām sen,\\
Vēji lapas kad dzen,\\
Šeit mēs sapņus par nākotni vijām,\\
Taču sirdis varbūt\\
Tikai šķiroties jūt\\
To, cik dārgi viens otram mēs bijām.


\newpage

{\bf Pūš tālu ceļu vējš}

Ir mūsu mašīnai pat logu stikli silti,\\
Šķiet, ka tos kausē staru kūlis spējš.\\
Pār mežiem dūmaka. Kūp sakarsušās smiltis,\\
Un sejā mums pūš tālu ceļu vējš.

Ceļš kalnā uzvijas. Visapkārt bālē\\
Zils mežu loks, un kaut kur ezers mirdz.\\
Tā, skatam klīstot dūmakainā tālē,\\
Sāk krūtīs nemierīgu dziesmu sirds.

Zils tālumi, kur saule peld līdz rietam,\\
Bet atnāks laiks un pavērs skatam tos ---\\
Tie mazie cilvēki, kas vēl pie spēļu lietām,\\
Reiz raķetē uz zvaigznēm aizlidos!

... Pār mežiem dūmaka. Tik karsts, ka elpot grūti,\\
Mēs braucam, klausoties, kā mežā dzenis kaļ.\\
Iet blakus puisēns kaklautu uz krūtīm\\
Un katru brīdi skatās atpakaļ.

Mēs gribējām to aizvizināt mājās,\\
Viņš apmulsa un piesarka kā zieds:\\
Viņš --- skolu pabeidzis. Viņš grib iet kājām\\
Pa ceļu to, kur daudzus gadus iets.

Šis ceļš ir kļuvis ļoti mīļš un ierasts,\\
Bet... skola pabeigta, un mulsums domas jauc ---\\
Bet tā jau ir, kad beidzas kaut kas pierasts\\
Un kaut kas nezināms uz priekšu sauc.

... Un mums tāds svinīgs satraukums bij radies,\\
LKā pašus vajadzētu tālam ceļam post.\\
Bet zēns... var būt, ka tieši viņš pēc gadiem\\
Reiz raķetē uz zvaigznēm aizlidos!

Man droši vien jau matos stīdzēs salna,\\
Bet tieši tad es atcerēšos spējš,\\
Kā, plandot kaklautam, zēns stāv uz kalna\\
Un sejā tam pūš tālu ceļu vējš.


\newpage

{\bf Gaujienas vidusskolai}

Kaut grūti būs iespraukties šaurajā solā\\
Un apsēsties pagrūti būs,\\
Bet saņem vēl reizi mūs, mīļotā skola,\\
Kā toreiz tu saņēmi mūs!

Mēs klāt jau. Birst lapas no ceļmalas kļavām,\\
Tās paceļas vējos kā spiets,\\
Un meitene maza ar grāmatām savām\\
Caur dzeltenu puteni iet.

Tāds pats bija rudens ar skarbumu balsī,\\
Kad saviļņots ienācu es...\\
Lai šodien mūs visus tā vakara valsis\\
Pār Donavas plašumiem nes.

Tad atnāks tās domas, tie sapņi, tās ilgas,\\
Kas visus šos gaiteņus vij,\\
{at atnāks uz brīdi tās debesis zilgās,\\
Kas toreiz pār gaujienu bij.

Drīz jābrauc. Riets izkāris sarkanu jostu,\\
Uz sliedēm stāv vilciens un elš,\\
Bet, tāpat kā kuģis mīl iegriezties ostā,\\
No kuras tam iesācies ceļš, ---

Mēs atnāksim, atkal būs dzeltenas kļavas,\\
Būs rudens, mēs atnāksim tad ---\\
No tevis, tāpat kā no bērnības savas,\\
Mēs nevaram aiziet nekad!

\newpage

{\bf Dēls izaudzis}

Ir salda smarža liepu ziedu tējai\\
Tāpat kā agrāk. Ceriņziedu skaras,\\
Tāpat kā agrāk, atdod smaržu vējam,\\
Bet tu pie galda jau par visiem garāks.

Jūs runājat par to, ka tēvi sirmo,\\
Par mērķiem, ko sev krūtīs jaunie glabā,\\
Un savā dzīvē saki reizi pirmo\\
Tos vārdus: --- Tēvs, ne tā, bet tā būs labāk. ---

Tāpat kā agrāk, smaržo lauku tēja,\\
Tāpat kā agrāk, jūti saldo tvaiku,\\
Tēvs ilgi skatās tavā jaunā sejā:\\
--- Pa kuru laiku, dēls, pa kuru laiku? ... ---

Jo tu ar galvu gandrīz griestos tiecies,\\
Par visiem garāks daudz, ja blakus stājas,\\
Bet tēvam vēl tu tāds pats maziņš liecies,\\
Kāds reiz zem loga smiltīs rotaļājies.

Ka tev ir darbs, ar to viņš sāk jau aprast,\\
Pie vilciena teic: --- Dēls, lai labi sokas! ---\\
Bet pie tam ilgi domā, nevar saprast,\\
Kā dzīvē iztiec tu bez viņa rokas.

Vai tiešām vecums? Nē --- kūp zeme tvaikā,\\
Un bango rudzu zaļums vējos brīvos.\\
Nē, vecums nedrīkst būt, jo tādā laikā\\
Uz tādas zemes gribas ilgi dzīvot.

Nes tevi vilciens prom, met saule karstu kvēli,\\
Tu domā par šo zemi, ziediem sēto,\\
Uz kuras tēviem izaug stipri dēli\\
Un dzīvē paveic tēvu iecerēto.

\newpage

{\bf Lasot romānu "Pret kalnu"}

Skan kuranti, un diena maina dienu,\\
Dun sliedēs tramvajs, varbūt pēdējais,\\
Un gaišu logu neredz vairs nevienu.\\
... Man apkārt atkal Lejasciema gaiss.

Daudz esmu gājis lauku ceļos tajos,\\
Un tāpēc viss, kas lapas pusēs mirdz,\\
Ir kādreiz pārdzīvots, ir kādreiz sajusts\\
Un kļuvis daļiņa no paša sirds.

Bet reizēm liekas --- te par mani sacīts,\\
Tad skatiens ātrāk burtu rindās skrej.\\
Tai ugunī, kas kvēlo Mirdzai acīs,\\
Kvēl skatiens arī manai meitenei.

Var būt, ka viņa kalnupceļu grūto\\
No Mirdzas mācījas, jo tas spēj vest\\
Pie uzvarām, vien katru brīdi jūtot\\
Sev priekšā nesasniegtās virsotnes.

Ir jau tik vēls. Nakts melna skatās logā,\\
Bet cauri naktij tagad skatos es\\
Uz kalnu, kurā rīt ar stiprām rokām\\
Ir mūsu laimes karogs jāuznes.

\newpage

{\bf Briesmas garām}

Kad baltā palātā jūt ziedu dvesmu,\\
Nāk apskate, kāds puisis klusi min:\\
--- Es, biedri ārst, tik vesels šodien esmu... ---\\
Un pasmaida, jo ārsts jau labāk zin.

Aiz loga --- maija padebeši zilgi,\\
Uz pieres karsto saules skūpstu jūt.\\
Trīs mēneši --- tas ir bezgala ilgi,\\
Un darba ierindā tā gribas būt!

Balts celiņš. Tulpes sarkanas kā liesmas.\\
Viņš iziet, acīs atkal saule mirdz,\\
Jo slimību un tajā slēptās briesmas\\
Ir uzveikusi viņa jaunā sirds.

Un kas gan izbēgs jaunā acu skatam:\\
Gan upīte, kur dzelmes sudrabs blāv,\\
Gan... Baltā halātā vīrs sirmiem matiem,\\
Ar acīm aizejošo pavadīdams stāv.

Skats aizslīd jaunā vingro soļu pēdās,\\
Un sajūt ārsts, ka viņš nav vecs un lieks.\\
Kāpj acīs asaras. No lielām bēdām?\\
Nē, tas ir milzīgs, neizsakāms prieks ---

Par to, ka slimais baltā ceļa oļos\\
Tik droši atkal savus soļus liek,\\
Par to, ka jaunību jūt viņa soļos,\\
Par to, ka viņam laime jāsatiek.

Par to, ka tulpes uzkvēlo kā liesmas,\\
Ka ķirsis liecas ziedu sniegā balsts\\
Tam cilvēkam, kam lielas, lielas briesmas\\
Aiz muguras.\\
Uz priekšu --- dzīve šalc.

\newpage
{\bf Mums jau divdesmit}

Cik ātri gadus veido mūsu dienas!\\
Jau, skaties, divdesmit --- uz vaiga asa bārda,\\
Un tie, kam tikai devītais gads pienāks,\\
Jau tevi sveicina, tic katram tavam vārdam. \\
Vēl sprāga kaut kur paslēpušās mīnas,\\
Kad mātes viņus sildīja pie krūtīm,\\
Vēl kaut kur augstu gāja lidmašīnas,\\
Bet nakts jau bija ar gaišām logu rūtīm.\\
Un, kad tie pirmos bērna soļus spēra\\
Un bērna acīm dzīvē ieskatījās, ---\\
Pār viņiem mierīgs debess zilums vērās ---\\
Nekāda kara viņiem nebij bijis...\\
Strauts šalkdams lauzās cauri cietam iezim,\\
No zemes spēku dzēra zaļie kvieši,\\
Un dabai taisnība! --- Kad cilvēks piedzimst,\\
Tai jābūt skaistai. Jābūt tādai tieši.\\
... Es piedzimu uz kartupeļu lauka,\\
Tu --- varbūt rudzu tīrumā vai rijā.\\
Tur nebij bērnības, vien barga auka,\\
Un tajā mēs jau pieauguši bijām,\\
Bij vismaz jābūt tādiem --- rīta rasā\\
Jau agri nācās nogurumu pazīt,\\
Just katrā šūnā sāpes asas, asas,\\
Mums nebij laika toreiz palikt maziem...\\
Mums, kas stāv dzīvē jau uz abām kājām,\\
Mums, kuriem katru kalnu apgāzt viegli, ---\\
Ir grūti reizēm paiet garām mājām,\\
Kur atskan gaiši pionieru smiekli.\\
Kad ugunskura dūmi stīdz pār pļavu\\
Vai pionieru taka kalnos vijas,\\
Mēs domājam par bērnību par savu,\\
To, kuru vecā dzīve nolaupīja.

\newpage
{\bf Četrās sienās}

Sētā pikodamies ņemas bērni,\\
Žilbst aiz loga zeme ziemas sniegā,\\
Bet pie tevis viss kā grīšļi pērnie ---\\
Nogurdinošs, apnicīgs un miegains.\\
--- Šodien jaunu, labu filmu rāda,\\
Aiziesim! ---\\
Tu tikai ņurdi klusi:\\
--- Redzi, cik viss nejēdzīgi gadās ---\\
Mašīnas nav, sieva aizbraukusi. ---

Mīkstā krēslā atlaidies, kā ronis\\
Šauras dzīves saulgozītē peries.\\
Draugiem zvani tu pa telefonu,\\
Adreses nevienam neatceries.\\
Istabā tev nejūt vēju brīvo,\\
Nejūt dzīves nemieru kā senāk ---\\
Gribas aizmirst ielu, kur tu dzīvo,\\
Gribas aiziet un nekad vairs nenākt.\\
Eju projām.\\
Niknums mani urda,\\
Un uz ielas, jūtot sala svelmi,\\
Es kā zivs, kas izbēgot no murda,\\
Pēkšņi atkal atgūst savu dzelmi,\\
Skatos Rīgas putekļainā dienā,\\
Sildos Rīgas februāra salā,\\
Un pēc tādas elles četrās sienās\\
Ir uz ielas labāk galu galā.

\newpage
{\bf * * *}

Eju ielā ziemas naktī kādā,\\
Mūsu pašu mīļās Rīgas ielā ---\\
Logu simti deg ---\\
Tur ļaudis strādā\\
Aizrautībā lielā, steigā lielā.\\
Nakts.\\
Bet tur jau gatavojas rītam,\\
Tikai dažos logos melno klusums,\\
It kā viņi rītu negaidītu,\\
It kā lācis tur uz ziemas dusu\\
Ierīkojies. Bet rīt dzīve skaļā\\
No šīm dažām alām nenobīsies,\\
Atraus durvis, atraus logus vaļā,\\
Miega pārņemtajās sienās ielauzīsies.

Tā ir mūsu zemes vēju brīve,\\
Un tās skarbumā nav nekā neparasta,\\
Jo no daudzu, daudzu ļaužu dzīvēm\\
Mūsu lielās zemes dzīve sastāv.\\
Un lai nemiers kvēlotu ikvienā\\
Tajā mūsu lielās zemes šūnā,\\
Lai mums neļauj puteņainā dienā\\
Apgulties uz miegainības dūnām,\\
Lai tev istabā jūt vēju brīvo,\\
Lai jūt dzīves nemieru kā senāk,\\
Draugs lai nevar aizmirst\\
Ielu, kur tu dzīvo,\\
Atnācis lai otrreiz\\
Nevar nenākt!

\newpage

{\bf Ledus lūst}

... Atspīd Gaujā padebeši zili,\\
Dzelmē palojošā redzu es:\\
Upe šturmē ledus "Ziemas pili",\\
Sagrauj to un drupas lejup nes.

Ledus lūst ar dobju kanonādi,\\
Palu straumes spēku atzīstot,\\
Vienreiz gadā atnāk brīži tādi,\\
Atnāk domām palu spēku dot.

Lai cik liels ir pavasaris dabā,\\
Tas zūd, rudens salnām atnākot, ---\\
Dzīves pavasari sirdī glabā,\\
Un nekādas salnas neskars to!

Un ikreiz, kad krasta koku zari\\
Liecas palu dzelmes mutuļos,\\
Domāju par citu pavasari,\\
Citā ledus dunā ieklausos.

Dzirdu straumes izlaužamies brīvē,\\
Petrogradas ielās palojot.\\
Ir! Ir pavasaris arī dzīvē ---\\
"Aurora" ar zalvi sāka to!

VIenreiz gadā atnāk brīži tādi,\\
Bet lai cauri dzīvei dzirdam to:\\
Ledus lūst ar dobju kanonādi,\\
Palu straumes spēku atzīstot!


\newpage

{\bf Pavasarī}

Šķīst sniegs uz palodzes, un ārā līst,\\
Skrien strumes duļķainas,\\
Un dobji ledus plīst.\\
Šķīst sniegs uz palodzes, un ārā līst.\\
Kā šodien visa daba sniegu nīst!\\
Kaut īsto pavasara sauli\\
Tā vēl nepazīst ---\\
Šķīst sniegs uz palodzes un ārā līst.


\newpage

{\bf Maijā}

No savas sirds tu šorīt dziesmu ņemi,\\
Un tā lai ceļas kvēlojošai blāzmai līdz ---\\
Mēs šorīt ieraudzīsim tādu zemi,\\
Kam tūkstoš ziedu acīs asaras\\
No rīta laimes trīc.

Mēs šorīt ieraudzīsim dzimto zemi,\\
Kas maija dzīvības un plauksmes ilgās tvīkst, ---\\
Tiem tādas zemes nav, kas viņu nemīl\\
No karoga līdz smilgai pēdējai,\\
Kas ceļa malā līkst.


\newpage
{\bf Rudens rītā}

Tikko kļūst zilgana tumsa aklā,\\
Tiklīdz var izšķirt jau dažu krāsu,\\
Sāk zem kājām kā brīnums lāsot\\
Rudens brīnišķo lapu paklājs.

ARvienu skaistāks paklājs kļūst\\
Tās dienas staros, kas būs.

Paglaudi to, un, lapām švīkstot,\\
Rudens zelts tev pie rokām skaras,\\
Nu, bet pasakiet --- vai mēs varam\\
Līdzi paņemt šo paklāju mīksto?

Lai tad, ja kādreiz smagi kļūs,\\
Šis skaistums dara stiprus mūs!

Nevar to paņemt līdzi ceļā\\
Un pat saglabāt nevar ilgāk.\\
SKaties --- kļūst pelēka tāle zilgā ---\\
Rudens milzīgie vēji ceļas.

Jau paklājs pirmā brāzmā trīc.\\
Nē, viņu nevar paņemt līdz!

Brāzmainais vējš pa mežu ārdās,\\
Dzeltenās lapas noplēš saujām.\\
Koki sten. Tā ir varena kauja ---\\
Un ne nāves, bet dzīvības vārdā.

SImt pumpuru slēpj katrs zars,\\
Un viņiem pieder pavasars.

Katrs nākošais vēju cirtiens\\
Kokus ņem ciešāk savā varā ---\\
Jo no spēka pārpilnā zara\\
Vecām lapām ir grūti šķirties.

Krāc vējš, lūst zari, trako viss,\\
Šķīst rudens lapu ugunis.


\newpage

{\bf Tā puksti sirds}

Uz zemes\\
Vēl mūsu paaudzes nebij tad,\\
Kad sērās atsedza galvu\\
Jaunie un vecie.\\
Mūsu paaudze Ļeņinu\\
Nesatika nekad,\\
Bet šodien tā pārņem\\
Dzīvi uz saviem pleciem.

Un kā straume\\
Šī dzīve\\
Pret nākotni skries,\\
Glabājot aprīļa vēju\\
Sava karoga krokās.\\
Ieročus,\\
Ar ko par nākotni cīnīties,\\
Dzimtene, liec\\
Mūsu jaunajās rokās!

Liec rokās traktora stūri,\\
Un uzplauks viskrāšņākais zieds\\
Zemē, kas pirmoreiz tiekas\\
Ar cilvēka skatu,\\
Liec rokās veseri,\\
Otu vai spalvu liec\\
Un, ja vajadzēs, ---\\
Liec automātu!

Tāpat kā maijam\\
Šodienas tērces mirdz,\\
Lai skaistai, briedīgai vasarai\\
Kūp mūsu pirmā vaga.\\
Tā puksti, sirds,\\
Kā pukstētu Ļeņina sirds,\\
Ja tavā vietā\\
Pa dzīvi viņš soļotu tagad!






\section{Dziesma}

\begin{quote}
{\em
Nāc, kad riets ar sārtu liesmu\\
\mbox{}\hspace{10pt} Tālē zied,\\
Klausīties, cik skaistu dziesmu\\
\mbox{}\hspace{10pt} Ezers dzied.

Dziesmu tad no sirdīm abām\\
\mbox{}\hspace{10pt} sāksim mēs,\\
Ezers gribēs dziedāt labāk ---\\
\mbox{}\hspace{10pt} Nevarēs!
}
\end{quote}


{\bf Kolhozā}

Kā mājās izkārtoju visas mūsu lietas,\\
Pat radio, ko veda līdzi draugs,\\
Bet jutu --- sirds nav tomēr savā vietā\\
Un arī paša drauga nav un nav.\\
Tad durvis pēkšņi atveras ar dārdu!\\
Viņš nosvīdis un satraukts iejoņo,\\
Tā aizelsies, ka tikai divus vārdus\\
Pār lūpām izdabūt var: --- Zini ko... ---

... Neviens tai naktī ciemā negulēja,\\
Bij palu briesmās milzīgs rudzu lauks,\\
Tai naktī atkusušās zemes vējus\\
Pa īstam abi ieelpojām, draugs,\\
Un sajutām, kā ir, kad nomet svārkus,\\
Bet karstums tomēr smagi dvašot liek.\\
No slapjām drēbēm pārvērtās par mārku\\
Mums istaba, No kājām gāza miegs.

... Mūs no šī pirmā lauku miega cietā\\
Jau vesels mēnesis, pilns steigas, šķir;\\
Mums istabā ir daudzas jaukas lietas\\
Un pavasara ziedi vāzē ir.\\
Ir kaut kā aizmirsušās Rīgas ielas\\
Un labāk patīk lauku lielceļš mīksts,\\
Bet... draugam dzīvē sācies kaut kas lielāks,\\
Un nezinu, vai par to runāt drīkst.\\
Ir tieši tāpēc ziedi mūsu mājās ---\\
Es tikai vakar uzzināju to,\\
Kad, vēlu pārnācis, viņš mani modināja\\
Un, acīm mirdzot, teica: --- Zini ko!...





\end{document}
