\documentclass[11pt]{article}
\usepackage{ucs}
\usepackage[utf8x]{inputenc}
\usepackage{changepage}
\usepackage{graphicx}
\usepackage{amsmath}
\usepackage{gensymb}
\usepackage{amssymb}
\usepackage{enumerate}
\usepackage{tabularx}
\usepackage{lipsum}

\usepackage{xcolor}

\usepackage{makecell}

\oddsidemargin 0.0in
\evensidemargin 0.0in
\textwidth 6.27in
\headheight 1.0in
\topmargin 0.0in
\headheight 0.0in
\headsep 0.0in
%\textheight 9.69in
\textheight 9.00in

\setlength\parindent{0pt}

\newenvironment{myenv}{\begin{adjustwidth}{0.4in}{0.4in}}{\end{adjustwidth}}
\renewcommand{\abstractname}{Anotācija}
\renewcommand\refname{Atsauces}

\newenvironment{uzdevums}[1][\unskip]{%
\vspace{3mm}
\noindent
\textbf{#1:}
\noindent}
{}

\newcommand{\subf}[2]{%
  {\small\begin{tabular}[t]{@{}c@{}}
  #1\\#2
  \end{tabular}}%
}



\newcounter{alphnum}
\newenvironment{alphlist}{\begin{list}{(\Alph{alphnum})}{\usecounter{alphnum}\setlength{\leftmargin}{2.5em}} \rm}{\end{list}}


\makeatletter
\let\saved@bibitem\@bibitem
\makeatother

\usepackage{bibentry}
%\usepackage{hyperref}


\begin{document}

\begin{center}
{\LARGE \bf Raiņa dzejas poētika}
\end{center}



\section{Tropi un Figūras}

Tropi ir mākslinieciskās izteiksmes līdzeklis, kuram pamatā vārdi
pārnestā nozīmē. Uz tiem īstenībā balstās dzejas valoda. Taču dažādos
literāros virzienos dominē kāda noteikta tropu grupa vai atsevišķi
tropu veidi. Tā klasicisms pēc iespējas izvairās no
metaforām,\footnote{Angļu daramturgs, dzejnieks un literatūras
teorētiķis Bens Džonsons par metaforām raksta: "Izmantot metaforas
vajag tikai nepieciešamības gadījumā vai ērtības nolūkos.. kad mums nav
precīza vārda, lai izteiktu kādu jēdzienu.. vai arī lai aizstātu
nepieklājīgu vārdu. Šajā gadījumā metafora uzlabo valodu, piešķirot
tai lielāku nozīmību. Pārāk mākslotas metaforas tiek saprastas ar grūtībām,
bet pārāk pušķotas nevar būt smalkas. Visi mēģinājumi izgudrot jaunas
metaforas ir bīstami. Tās griež ausīs, kamēr, tās ilgāk lietojot,
pie tām nepierod" -- Cit. pēc: Литературные манифесты западноевропейских
классицистов. -- М., 1980. -- С. 185-186.}
priekšroku dodot metonīmijām un alegorijām, turpretī
romantisms un plašāk -- romantiskā poētiskā tipa lirika kultivē tieši
metaforas. Spilgts liecinājums tam ir Edvarta Virzas neoklasicistiskā
posma dzeja (galvenokārt krājums "Dzejas un poēmas", 1933.g.) un
akmeistiskā posma dzeja (krājums ``Laikmets un lira'', 1923.g.).
Grāmatā ``Laikmets un lira'' dzejnieks izvairās no metonīmijām,
priekšroku dodot epitetiem, metaforām, piemēram:


\begin{quote}
Caur vecas ejas tumši zaļiem lokiem, \\
Kur pamirdz kādas nimfas rožains plecs,\\
Vēl kolonādi ieapaļu redz\\
No balta marmora zem augstiem kokiem.\\
("Kādreizējie")\footnote{{\em Virza E.} Laikmets un lira. --
R. 1923. -- 75.lpp.}
\end{quote}


\begin{quote}
Jūs' svārki bij' jums bura, mazā roka airs.\\
("Madame de Maintenon")\footnote{Turpat. -- 74.lpp.}
\end{quote}

Turpretī krājumā "Dzejas un poēmas" dominē tieši
metonīmijas: daļa veselā, vienskaitlis daudzskaitļa vietā
u.tml. Piemēram, satīras "Deputāts", "Liberāls",
"Redaktors" u.c.:

\begin{quote}
Ikkatram latvietim ir Markss vai Luters zināms,\\
Ar to, kas priekšā tam, tas nava apmierināts.\\
("Dzeja un dzejnieks")\footnote{Virza. E.
Dzejas un poēmas. -- R. 1933. -- 20.lpp.}
\end{quote}

{\color{red}
Pirms pāriet pie Raiņa dzejas tropu konkrētās analīzes,
nepieciešams īss ieskats šīs mākslinieciskās izteiksmes
īpatnības vēsturē un iespējamos cēloņos.
Dzejniekam (īpaši romantiskā mākslinieciskā tipa pārstāvjiem)
nepatīk rakstīt tieši, bet vairāk simboliskā valodā,
lietojot salīdzinājumus, epitetus, personifikācijas,
metaforas, u.tml. } Ar kādu nolūku tas tiek darīts,
un kāpēc lasītājs pieņem šos spēles noteikumus?
Kāpēc, piemēram, rakstot par vienu no mūžīgajām tēmām --
mīlestību --, dzejnieks parasti nesaka skaidri un tieši
"Es tevi mīlu, Anna", bet šīs jūtas atklāj "caur puķēm"?
Tā Rainim dzejoļu krājumā "Addio bella!" starp daudzajiem
mīlas dzejoļiem ir arī šāds:

\begin{quote}
Divi balti spilventiņi\\
Zeltītiem galiņiem --\\
Kas viņos pagulējis,\\
Augšā celties negribējis.\\
(III, 68)
\end{quote}

Šo dzejoli ir gandrīz neiespējami uztvert, ja lasītājs
nepārzina tautasdziesmu simboliku. Un, ja viņš to
nepārzina, tad šī dzejoļa satura skaidrošana var iegūt
šķietami negaidītus pavērsumus. Tā kāds lasītājs,
kuram zināšanas par latviešu folkloru patiešām bija
minimālas, lūgts Raiņa dzejoli komentēt, saistīja to ar
nomiršanu (pamatojoties uz pēdējo rindu
"Augšā celties negribēja"). "Divi balti spilventiņi" --
spilveni nomirējam zārkā. Otrs lasītājs, kuram latviešu
kultūras reāliju uzkrājums bija bagātīgāks, bet kurš
nebija speciāli interesējies par folkloru, pirmajā
brīdī par Raiņa dzejoli teica: "Nezinu, kaut kāds murgs."
Taču tad, kad šim lasītājam tika doti papildu
norādījumi ("Tas ir mīlestības dzejolis"),
viņš bez šaubīšanās divus baltos spilventiņus saistīja
ar sievietes krūtīm.
Trešais lasītājs (ar zināmu uzkrājumu foklorā)
vispirms sasaistīja Raiņa dzejoļa pantmēru
(četrpēdu trohaju) ar dainu pantmēru. Un Raiņa dzejolī
lietotās semantēmas nozīme "uznira" no zemapziņā
iekodētā dainu simboliskā režģa, iespējams, ar
nojausmu par kādu no dainām, kur figurē divi balti
kukulīši ("Man pašai azotē / Divi balti kukulīši"
LD 19900) vai spilventiņi:

\begin{quote}
Bērza klucis ievēlās\\
Mūs' māsiņas spilvenos.\\
Ej ārā, bērza kluci,\\
Lai gul viena mūs' māsiņa!\\
(LD 24910)
\end{quote}

Rainis dzejoli veidojis saiņu pantiņu struktūrā -
vienā lapaspusē ievietotas četras četrrindes un viena
(noslēdzošā) divrinde -- visas četrpēdu trohaja
pantmērā. Bez citētās četrrindes, kas simbolikā
vistuvākā tautasdziesmā, nebūtu iespējama vai būtu
apgrūtināta pārējo Raiņa četrrinžu un noslēdzošās
divrindes tēlu atšifrēšana. Katrs nākamais dzejolis pārņem
konstantu daļu ("divi balti") no pirmās četrrindes
simbolikas, izvēršot to jaunās sakarībās:

\begin{quote}
Divus baltus kamoliņus\\
Laime puisim atmetusi -- \\
Puisis skrēja meklēdams,\\
Azotē atrazdams.\\
* *\\
*\\
Divi balti mākonīši\\
Gar mēnesi aizpeldēja --- ---\\
(III, 68)
\end{quote}

SImbols, kas veido pēdējo divrindi, būtu nesaprotams un
neatšifrējams bez iepriekšējiem.

Iemesli dzejnieku darbos lietotajiem vārdiem ar pārnestu
nozīmi var būt vairāki: 1) ja rakstīsi par "mūžīgajām"
tēmām tā, kā rakstījuši tavi priekšteči, neviens nelasīs,
jo cilvēks pēc dabas allaž meklē ko jaunu un uzmanību
piesaistošu; 2) dzejnieks grib vēstīt par
prieku, laimi vai skumjām, bet parasti vairās saukt
savu jūtu objektu vārdā un minēt kādas reālijas, kas
varētu norādīt uz konkrēto adresātu;
3) politiski, sabiedriski tabu neļauj runāt "skaidru"
valodu; 4) "Dažiem jēdzieniem valodā trūkst atbilstošu
vārdu, bet tomēr var atrast analogu
metaforu"\footnote{Aristotelis. Poētika. -- R., 1959. -- 91.lpp.}
{\color{red}
5) metaforas, salīdzinājumi, personifikācijas,
paralēlismi u.c. ļauj koncentrētāk pateikt domu, izvairoties
no plaša un secīga izklāsta, u.c.
}

Bez nosauktajiem var minēt vēl virkni citu iemeslu,
kāpēc dzejā tiek lietoti tropi. Un tomēr šķiet, ka
galvenais iemesls vēsturiski ir bijis pavisam cits.
Tā, protams, daļēji taisnība, ka valodā vārdu skaits ir
ierobežots un, lai izteiktu kādu jaunu nojēgumu,
tam vislabāk noder metafora, arī metonīmija,
personfikācija u.tml. Bet kāpēc tādā gadījumā
zinātnes valoda tradicionāli izvairās no metaforiskas
izteiksmes -- zinātniekiem taču tik bieži jāanalizē lietas
un parādības, kurām nav nosaukumu?
Daļēji patiess ir arī apgalvojums, ka tropu lietojums
dzejā atsvaidzina izteiksmi, palīdz lasītājam negarlaikoties.
Un tomēr, ja izsekojam dzejas vēsturei, krietna daļa
epitetu, salīdzinājumu, metaforu, simbolu, alegoriju
u.c. laika gaitā saglabājas nemainīgi vai tikai daļēji
mainās --- un tomēr lasītājs lasa...

Galvenais iemesls, kāpēc dzeja tiek veidota pārsvarā
pārnestas nozīmes (simboliskā) valodā, šķiet, meklējams
folklorā un plašāk --- senā cilvēka domāšanas
īpatnībā. Tā visas mītiskās dziesmas ir --- mūsdienu
terminoloģijā --- metaforiskas un simboliskas.
Tām raksturīga enigmātiskā jeb mīklu veida
(t.i. slēptā) izteiksme.

Pirmārt, tāpēc ka tie ir sakrālie teksti, kuru īsto jēgu
drīkstēja zināt un saprast ne visi, bet tikai iesvaidītie.

Otrkārt, mītisko domāšanu un līdz ar to --- mītisko
izteiksmes veidu raksturo priekšstats par mikrokosmosa
un makrokosmosa identiskumu, tāpēc, piemēram,
cilvēku kāzas nereti tiek tēlotas kā debesu dievību kāzas
(līgavainis --- Mēness vai Dieva dēls, līgava --- Saules meita)
vai zvaigznes tiek sauktas par acīm, debesis --- par sudraba
kalnu ("Simtacīts, Dieva dēls / Ar sudraba kalniņā",
LD 28 833, 5) u.tml. Šī dažādo līmeņu objektu līdzvērtība
un savstarpējā atbilsme veido to, ko mēs saucam par
paralēlismiem un metaforām. Tā mītiskajā
domāšanā cilvēka asinis bieži tiek pielīdzinātas
upei (sakrālajiem ūdeņiem), un tāpēc, piemēram, buramvārdos
tās tiek sauktas par straujupīti:

\begin{quote}
Līku loku upe tek\\
No kalniņa lejiņā;\\
Aiztecēja mīļa Māra, \\
Saturēja straujupīti.\\
(LD 34 132)
\end{quote}

Treškārt, dainās atspoguļojas totēmiskie
priekšstati, tāpēc nereti meitas sauktas par
irbēm, vāverēm, liepām, puiši --- par vanagiem,
vilkiem, ozoliem u.tml. Tad, kad
totēmiskie priekšstati tiek uztverti vairs tikai
kā māņi, kad no tautas apziņas sāk izzust to
funkcionālā jēga, totēma vārds pārvēršas par salīdzinājumu,
paralēlismu vai tiek uztverts kā metafora:

\begin{quote}
Kupla liepa ceļu gāja,\\
Zari vien locījās;\\
Ozoliņš ceļu grieza\\
Ar visām bitītēm.\\
(LTD 172, IV)
\end{quote}

\begin{quote}
Es nelauzu liepas zaru,\\
Pate augu kā liepiņa;\\
Ozolam, tam nolauzu\\
Visas zaru pazarītes.\\
(LD 9779)
\end{quote}

\begin{quote}
Kupla liepa jūrmalāi,\\
Pilna baltu gaigalīšu;\\
Viena meita māmiņai,\\
Pilna sēta precinieku.\\
(LD 14 248)
\end{quote}

Ceturtkārt, semantēmas, ko šodien parunās, izteicienos
uztveram metaforiski ("Paņem acis rokā",
"Ņem acis pirkstos, ka neredzi", "Mest ar acīm"),
folklorā un folkloristiskajā domāšanā kādreiz
nozīmēja tiešamību.
Sal. latviešu pasaku "mešana ar acīm":
"Nu velns devis citu darbu, lai apkopjoties pa māju,
ko jau nu redzot, un tad lai steidzoties viņam pakaļ
viesībās, viņš iešot tūliņ. Bet, ja atnākot viesībās,
lai tad pametot viņam ar acīm vien, gan viņš tad
nopratīšot, vai ir
apkopts viss vai ne.
Labi, velns aiziet, un Juris nu ņemas ar darbiem..
izņēma velna bērniem acis un aizgāja viesoties.
Noiet tur --- velns skatās: Kur Juris?
Bet Juris paķēra velna bērnu acis un iemeta velnam
klēpī. Velns traks, tūliņ mājās prom
un grib Juri vai piestā sagrūst; bet Juris brīnās:
"Par ko tad, vai pats neteici, lai ar acīm
metu?""\footnote{Latviešu tautas pasakas par
velniem. --- Kopenhāgena, 1953. --- 15.-16.lpp.}

Piektkārt, enigmātiskos tekstus (visbiežāk --- mīklas
vai mīklu dziesmas) senie cilvēki minēja kritiskos brīžos ---
gadu mijā, veļos kādu aizvadot, kāzās u.c.
Ticēja, ka mīklas atminot, viņi sekmē kritiskā brīža
atrisināšanu labvēlīgi. Turpretī tiešā veidā izteikti
vārdi neprasa radošu piepūli, tos uztverot, tāpēc
tiešai runai nevar būt maģiskas iedarbes spēka.

Sestkārt, svarīgas lietas, būtnes, ko pielūdza vai/un
no kurām baidījās, nesauca īstajā vārdā, bet aizstāja
ar kādu tabu apzīmējumu. Ne velti tautā mēdz teikt:
"Kā vilku/velnu piesauc, tā vilks/velns klāt."
Vilku (kas arī ir tabu vārds\footnote{"..vārdam {\em vilks}
ir tā pati cilme, kas verbam {\em vilkt}. Vilka
apzīmējums ar šās saknes vārdu ir samērā jauna
indoeiropiešu parādība. Šis vārds stājies agrākā vilka un
suņa kopapzīmējuma vietā.. Sākotnēji.. vilks bijis
segvārds ('plēsējs, laupītājs'). Kad šis vārds bija
kļuvis par dzīvnieka īsto nosaukumu, tam blakus izveidojās
jauni segvārdi (latviešu valodā pelēcis,
mežainis, mežavīrs, tēvainis, vecbrālis u.c.)." ---
Karulis K. Latviešu etimoloģijas vārdnīca. -- R., 1992. --
2.sēj. --- 530. lpp.}),
tāpat kā vairākus citus sakrālus dzīvniekus, rāpuļus,
tautasdziesmās bieži sauc kādā no segvārdiem. Čūskai vien
latviešu valodā tādu ir vairāk nekā
astoņdesmit.\footnote{Sk.: Johansons A. Die lettischen
Benennungen der Schlange // Donum Balticum. ---
Stockholm, 1970. ---- S. 222-230.}
Ja šos tabu apzīmējumus (vismaz to veidu) nezinām,
daļa tautasdziesmu mums paliek nesaprasta, piemēram:

\begin{quote}
Dzeniet, gani, kur dzeniet,\\
Purvā, gani, nedzeniet:\\
Purvā guļ iebavieši,\\
Zaļas acis bolīdami.\\
{\em Tdz 50 522}
\end{quote}

Vēlākos laikos, kad tabu priekšstati zaudēja savu spēku
un tos sāka uzskatīt par vecu laiku māņticību, vilka
apzīmējums -- "pelēcis", lapsas -- "rudaste",
čūskas -- "iebavietis" u.c. pārtapa par ornamentālu ---
stilistiski rotājošu epitetu, metaforu, u.tml.

Septītkārt, dzejas simboliskās izteiksmes avots meklējams
arī t.s. metamorfozēs --- seno cilvēku ticībā,
ka cilvēki uz laiku vai pavisam spēj pārvērsties dzīvniekos,
putnos u.c. (piemēram, vilkačos; pārdabisku uzdevumu veikšanai ---
par ērgli, zivi vai vilku u.c.), nedzīvos priekšmetos,
lietās (piemēram, pasaka par brāļiem, kurus ragana pārvērš
par akmeņiem), bet nedzīvi priekšmeti, lietas spēj
atdzīvoties (no māla vai koka gabala tiek izveidots cilvēks).
Metamorfozes bija viens no biežākajiem sižetiem
sengrieķu mītoloģijā (dievību Zeva, Proteja u.c. pārvēršanās
dažādos dzīvniekos, putnos u.c.; Narciss, kas pārvēršas
puķē --- narcisē, miniādas, kuras Dionīss pārvērš par
sikspārņiem, Arahne, kuru Atēna pārvērš par zirnekli, utt.)
un vēlāk antīkajā literatūrā (sk. Ovidija "Metamorfozes",
Apuleja "Metamorfozes jeb zelta ēzelis"). Priekšstati
par iespējamo pārvēršanos sakņojās gan totēmismā, gan ticībā
dvēseļu ceļošanai un pārtapšanai. Jaunlaiku literatūrā
metamorfozes iegūst vainu neomītisku traktējumu
(F.Kafkas "Pārvēršanās"), vai kļūst par metaforām,
salīdzinājumiem ("Viņš stāvēja kā pārakmeņojies"),
vai balansē starp metamorfozi un metaforu, kā, piemēram,
V.Plūdoņa balādē "Tīreļa noslēpums":

\begin{quote}
..Un dakšām un izkaptīm bruņoti,\\
Ciema ļaudis drīz salasās tīrelī.\\
Tu brīnums! Kur ganam trīsdesmit aitu\\
Pie grāvmales ēda, tur tādu pat skaitu\\
Redz avju vietā akmeņu,\\
Citu pie cita sakrautu\\
Kā kapa kopiņu...\footnote{{\em Plūdonis V.} Raksti. ---
R., 1974. --- 1. sēj. --- 91. lpp.}
\end{quote}

Tā, protams, ir tikai neliela daļa no iespējamiem dzejas
tropu avotiem folklorā. Te svarīgi pasvītrot, ka folklorā
simboli (un plašāk --- slēptā valoda) bija daudznozīmīgi
(atkarībā no konteksta), bet arī noturīgi, jo tos uzskatīja
par dievišķā zīmēm, kuras nedrīkst mainīt, lai nezustu
tajās ieliktais enerģētiskais spēks.

Kristietiskā kultūra, kas nāca ar pagāniskās kultūras
simbolisko zīmju noliegumu, neiznīcināja tās
(vai iznīcināja tikai daļēji), bet vairākumā gadījumu
pārkodēja. Tā kosmiskajam --- Saules kokam tika
pārklāta Bībeles Labā un ļaunā atzīšanas, Nemirstības
koka kontūras, Māra kā viens no Lielās pirmmātes
iemiesojumiem latviešu folklorā papildus ieguva
svētās Marijas vaibstus u.tml.

Kristietiskā mītoloģija (īpaši Jaunā derība) arī krietni
paplašināja Eiropas tautu kultūras simbolisko kodu.
Līdzības par pazudušo dēlu, par putniem gaisā un puķēm
laukā, par sējēju u.c. ar laiku iegājušas universālo
simbolu kategorijā un tiešā veidā vai pārfrazētas turpina
savu eksistenci.

Divdesmitā gadsimta modernā un postmodernā kultūra radīja
krasu lūzumu tradicionālo zīmju --- simbolu, metaforu
u.c. sistēmā, pirmkārt, atmetot tās, otrkārt, padarot
bezjēdzīgas (ievietojot starp citām bezjēdzīgām zīmēm),
treškārt, apzināti pārvēršot par seklām, paņemot no šīm
zīmēm tikai to virspusi, piemēram:

\begin{quote}
Es paskatos pulkstenī: vienpadsmit jau.\\
Mēs divi vien esam. Šeit citu nav.

Es nošauju greizi, čirkst nazis pa šķīvi.\\
Mēs tiesājam olas un spriežam par dzīvi.

Tu juriste jauna, es vīrietis tavs,\\
Kas domā, ko pārnācis sacīs tavs tēvs.

Tev iedrebas pleci kā Temīdas svari:\\
--- Ne es ko drīkstu, ne tu ko vari. ---

Es nolieku nazi uz šķīvja malas\\
Un jūtos kā būda uz pamestas salas.

Kā notiesātais tad ceļos un eju.\\
Tu nomazgā traukus, es pāršķirstu dzeju.

Tad iekrītam gultā un jūtamies brīvi.\\
Tu piemini olas... Ne čirksta par dzīvi.\footnote{{\em Zirnis
E.} Rudu lapsu kāsis. --- R.,1985. --- 48.lpp.}
\end{quote}

Te desakralizēts gan skaitlis ("vienpadsmit jau"),
gan "es --- tu" ("Tu juriste jauna, es vīrietis tavs"),
gan tādi Eiropas kultūrā ietilpīgi simboli kā
"Temīdas svari", "pamesta sala" u.c.
Sakrālais modelis aizstāts ar profāno, sakrālās
zīmes ar profānajām ("Tu nomazgā traukus, es
pāršķirstu dzeju", "Tad iekrītam gultā un jūtamies brīvi").
Salīdzinājums (pleci kā Temīdas svari; kā būda uz
pamestas salas; kā notiesātais), kas tradicionāli
dzejā lietots, lai paplašinātu vai padziļinātu
priekšstatu par salīdzināmo, te --- pretēji --- to
paseklina. Var teikt, ka šeit un līdzīgos postmodernistiskā
tipa tekstos tradicionālo kultūrzīmju (simbolu,
salīdzinājumu, epitetu u.tml.) desemantizācija
novesta gandrīz līdz galējai robežai, kad ne teksta rakstīšanai,
ne lasīšanai nav īpašas jēgas. Simboliskā izteiksme
vienmēr prasījusi no lasītāja līdzradīšanu --- atšifrējot
zīmes. Kad līdzradīšanas nav, zūd interese par pašu tekstu.
Protams, visvieglāk būtu pārmest postmodernistiem
apzinātu vai neapzinātu dzejas pamata --- tēlainās,
simboliskās valodas iznīcināšanu. Taču, redzams,
20.gs. tradicionālais --- tēlainās izteiksmes un līdz
ar to arī poētiskās domāšanas veids bija lielā mērā izsmēlis
sevi. Bija jānāk sastindzināto mītisko shēmu --- simbolu
(Dievs, Laima, Māra, Lāčplēsis, Koknesis, Burtnieks,
Melnais bruņinieks u.tml.) "pazemināšanai"
(kā, piemēram, J.Turbada darbā "Ķēves dēls Kurbads"),
sastindzināto simetrisko ritmu un metru
(jambu, trohaju) asimetrizācijai verlibriskās
struktūrās, lai tad vai nu pārradītu tās,
vai nogremdētu aizmirstības dzīlēs uz visiem laikiem.
Postmodernistiskās poētiskās domāšanas pašreizējā
ielenkumā gandrīz vai šķiet, ka simboliem ---
tradicionālajām zīmēm (līdz ar visu tradicionālo
tropu virteni) vieglāk būs nogrimt pavisam
(dainas, pasakas tādā gadījumā veidos pamestu un
nevienam nesaprotamu, nevajadzīgu lūžņu salu
21.gs. cilvēkam) nekā atdzimt no jauna.
Un ttomēr, vērojot to pašu postmodernistu dzeju,
nevar nepamanīt, ka arī viņi (apzināti vai neapzināti)
nespēj iztikt ja arī ne bez tradicionālajiem simboliem,
tad bez tā, ko šveiciešu psihologs Karls Gustavs Jungs
nosaucis par arhetipiem jeb dominantēm --- universālām
konstantām psihiskajām shēmām, kas kopš
neatminamiem laikiem zemapzinīgi tiek pārradītas un
aktualizētas rituālos, mītos, ticējumkos, sapņos,
mākslā.\footnote{Par to sīkāk sk.: Юнг К.Г.
Архетипы коллективного бессознательного //
Юнг К.Г. Психология бессознательного. --- М.,
1994. --- С. 135-165.}
Arhetipi, kas rodas no cilvēka zemapziņā snaudošiem
tēliem, pēc K.G.Junga uzskata, ir universālie
simboli, kuriem piemīt vislielākā patstāvība.
Mākslas darbu iedarbība slēpjas tieši šī arhetipiskā
dziļslāņa aktualizācijā.

Šos pirmtēlus --- arhetipus, kas saistīti ar ūdens,
uguns, debesu, zemes, kosmiskā koka, dzīvības, nāves
u.c. priekšstatu simbolizāciju, --- nav izdevies
iznīdēt arī postmodernistiem, tieši otrādi,
tropus desemantizējot vai pat iespējami "izmetot"
no dzejas, viņi aktualizē pirmtēlu līmeni.
Un pirmtēli --- šīs vārdos grūti izsakāmās,
definējamās zīmes --- ļauj pārmest saprašanās tiltu
starp rakstītāju un lasītāju. Vēl vairāk ---
impresionistiskā, vēlāk sirreālistiskā, vēl vēlāk ---
postmodernistiskā tipa dzeja tieši akcentē šo
kolektīvās zemapziņas --- arhetipisko slāni:

\begin{quote}
Šai dziļajā rudens aizmirstībā,\\
kur zirneklis dievmātes dziju auž,\\
es redzu --- no zvaigžņainā Vēža spīles\\
izaug mūsu Tikšanās kok.s

Tā zari liecas pār Nāves jūru\\
(tik daudz ziedos tavs profils zied),\\
tā zari liecas pār tuksnešu smiltīm\\
(tu varbūt vēl šaipus anemonēm).

Es tagad jau esmu viņpus Ēnas\\
(ai zilgme, kāds caurspīdīgs satraukums!).\\
Kaut kur tālu, tālu šūpoļu matos\\
atkal iemmirdzas maldugunis...\footnote{{\em Helds J.
} Naktsputna testaments. --- R., 1977. --- 79. lpp.}
\end{quote}

Šajā dzejolī rudens, zirneklis, kas dievmātes dziju
auž, zvaigznes, Tikšanās koka zari, Nāves jūra,
tuksneša smiltis, ziedi, ēna, šūpoļu mati,
maldugunis ir kādu senu pirmtēlu varianti.
Apziņas līmenī traktēti, tie veido šķietami
neloģisku virknējumu. Zemapziņas līmenī
(tajā, kas katram no mums aktualizējas
naktī --- sapņos) katrs no šiem tēliem uzjundī
vai potenciāli spējīgs uzjundīt sezonu
(rudens), aušanas (zirneklis, Dievmāte Marija,
likteņdievības kā kosmosa audējas), kosmiskā
koka, dzīvības un nāves ūdeņu u.c.
senās kolektīvās atmiņas zīmes.
"Vēsturei nav izdevies mainīt senā simbolisma
struktūru pamatos. Vēsture pastāvīgi pievieno
jaunas nozīmes, neizjaucot simbola struktūru
kā tādu."\footnote{{\em Eliade M.} The Sacred
and the Profane: The Nature of Religion. ---
N.Y.; Evanston, 1959. --- P. 137.}

{\color{red}
Rezumējot šīs piezīmes par enigmātisko izteiksmes
veidu (tropiem) dzejā, var secināt:
\begin{enumerate}[1)]
\item senajam cilvēkam tas, ko mēs saucam par
tropiem, apzīmēja dažāda līmeņa lietu, būtņu,
dievību identumu (A vienāds ar B; puisis ir lācis);
\item jauno laiku cilvēkam, kuram pakāpeniski
mītisko domāšanu aizstāja vēsturiskā, tropi dzejā ir
nosacītas zīmes, kas norāda uz tuvāku vai tālāku
sakaru starp A un B (puisis kā lācis, resp., stiprs);
\item 20.gs., īpaši postmodernistiskā laikmeta cilvēkam
tropi kā tādi lielā mērā kļuvuši par nolietotām zīmēm.
To vietā dzejā tiek aktualizēti arhetipiski
modeļi --- pirmtēli, kuriem jānodrošina saikne
(saprašanās mēģinājuma zīme) starp lasītāju un
rakstītāju. Taču bieži, ja ne visbiežāk šiem
pirmtēliem tiek "atņemts" to sensenais sakrālais
pildījums, tie aktualizē tikai dzīvības --- nāves,
mīlestības, agresijas u.c. zīmju profāno --- virsējo
slāni. Lai gan nevar nepamanīt, ka
līdzās šai simboliskās dzejas valodas elementu
demītoloģizācijas tiecībai pastāv arī
to pārradīšana, cilvēces izstrādāto tradicionālo
simbolisko zīmju aktualizācija (I.Ziedoņa, L.Brieža,
K.Skujenieka u.c. dzeja).
\end{enumerate}
}

Līdzīgu evolūciju izgājušas arī t.s. figūras (īpaši
izteicieni, kas atšķiras no parastā vārdu kārtojuma; arī
--- konstrukcijas, ko izlieto valodas izteiksmības
pastiprināšanai) dzejā. Dažādie atkārtojumu veidi
(anafora, divkāršojums, epanortoze, epifora,
hiasms) tautas dzejā bija cieši saistīti
ar ticību maģiskai iedarbībai. Atkārtojot pastiprina
teiktā iespaidu. To labi var redzēt buramvārdos,
kur atkārtojumam ir ļoti svarīga loma, piemēram, dūrēja
vārdi: "Dūrējs dur; lai trīs pērkoņi to sasper. Dūrējs dur
durdams, lai deviņi pērkoņi to sasper. Dūrējs
dur padurdams; lai trejdeviņi pērkoņi to
sasper."\footnote{{\em Straubergs K.} Latviešu
buramie vārdi. --- R., 1939. --- 1.sēj. ---
385.lpp.} Otrs dažādo atkārtojuma veidu
(pakares, divkāršojuma u.c.) iemesls slēpjas
valodas senajā struktūrā, proti,
"aizstājējvārdu" (vietniekvārdu) samērā
vēlā ienākšanā valodā vai arī vārdu, kas apzīmētu
kādas darbības intensitātes pakāpi ("ļoti"), kā
arī palīgvārdu trūkumā.
Tā piemēram, mītiskajās dziesmās personu vai
norādāmie vietniekvārdi ("viņš", "tas") tikpat kā
neparādās, tā vietā tiek divreiz atkārtots viens
un tas pats substantīvs (veidojot pakari --- vārds
rindas beigās sāk nākamo rindu;
epanortozi --- viena vai vairāku vārdu vai veselas
rindas atkārtojums):

\begin{quote}
Ej, māsiņa, klausīties,\\
Kādu vēsti zīle nes!\\
--- Zīle nesa tādu vēsti,\\
Būs brāļami karā iet.\\
{\em LTdz 22 180}
\end{quote}

\begin{quote}
Ļaudis ēda, ļaudis dzēra,\\
Dievs aiz loga klausījās,\\
Dievs aiz loga klausījās,\\
Vai Dieviņu pieminēja.\\
{\em LTdz 13 646}
\end{quote}

Vārda "ļoti" (stipri, daudz, maz u.tml.)
vietā tautasdziesmās parasti tiek divreiz
atkārtots viens un tas pats vārds, piemēram:
"Aši aši zīle brēca / Vārtu staba galiņā"
(LTdz 22 185).

Pakāpeniski, zūdot uzskatam par dzejas
valodu kā sakrālu valodu, ar kuras palīdzību
iespējams ietekmēt dažādas dzīves
norises, figūras, tāpat kā tropi pārvēršas nosacītos
mākslinieciskās izteiksmes līdzekļos, kuru uzdevums ir
gan "savādot" dzejas valodu, uzsverot tās
atšķirību no sarunvalodas un prozeas valodas, gan
zemapziņā saglabāt atkārtojumus u.c., lai
panāktu lielāku iedarbību uz lasītāju. Romantiskā
mākslinieciskā tipa dzejā figūru, tāpat kā tropu nozīme
ir lielāka nekā klasistiskajā vai reālistiskajā dzejā.
Pirmais tips orientējas uz lasītāja uztveri
"caur jūtām", otrais un trešais ---
"caur prātu", tāpēc pēdējiem dažādu figūru
lietojums šķietas esam traucēklis dzejoļa jēgas precīzai atklāsmei.
Un otrādi --- šķietamais apkārtceļš, kas bagātināts
ar dažādiem tropiem, figūrām, dzejnieka --- romantiķa
uztverē palīdz lasītājam dziļāk izprast
tekstā slēpto jēgu.

Daiļrades gaitā mainās attieksme pret dažādiem
tropu un figūru veidiem arī Raiņa dzejā. Šīs
izmaiņas pamatā saistītas ar literārā virziena
maiņu, lai gan ne tik konsekventi kā, piemēram,
izmaiņas metrikā.

Raiņa jaunromantiskajā\footnote{Par virzieniem
Raiņa dzejā sīkāk sk. šīs grāmatas nodaļā
Mākslinieciskās sistēmas attīstība".} dzejā,
ko visspilgtāk pārstāv krājums "Tālas noskaņas
zilā vakarā", un daļēji arī agrīnā simbolistiskā
posma krājumos ("Vētras sēja", "Klusā grāmata")
dominē {\bf alegorija} --- tropa veids, kura pamatā
ir līdzība. No divām lietām (A un B) alegorija
min tikai vienu (B), bet minētā liek lasītājam
apjaust arī vārdos neminēto (A), jo
starp abām pastāv kāda pastāvīga
līdzība\footnote{{\em Dziļleja K.} Poētika. ---
[B.v.]: P. Mantnieka apgāds, 1949, 1949. ---
46.-47.lpp.} Alegorija ir radniecīga simbolam,
taču atšķirībā no tā alegorija norāda tikai uz vienu
no potenciālajām nozīmēm. {\bf Simbols}
savā būtībā un arī dzejā parasti ir daudznozīmīgs,
alegorija turpretī viegli iekļaujas binārajā opozīcijā
"labais --- ļaunais"\footnote{Литературный
энциклопедический словарь. --- М., 1987. ---
С.20.}  Alegorija kā tropu veids tradicionāli
saistīta ar tādiem žanriem kā fabula, satīra,
groteska, parabola un jo īpaši --- līdzība
(uz alegorijām balstās, piemēram, līdzības Jaunajā derībā).
Alegorija bija īpaši iecienīta viduslaiku, baroka,
kā arī klasicisma posma literatūrā.
Romantiskā poētiskā tipa dzeja parasti šo tropu
veidu tikpat kā nelieto, vērtējot to (īpaši ar atskatu
uz klasicisma pārracionalizētajām alegorijām) kā pārlieku
tiešu, vienpusīgu mākslinieciskās izteiksmes līdzekli.
Raiņa jaunromantiskā posma dzeja liekas apgāžam
šo vispārējo romantiskā un jaunraomantiskā virziena
tendenci. Taču tam var atrast izskaidrojumu.
"Tālas noskaņas zilā vakarā", kā arī simbolistiskā
posma krājumi "Vētras sēja" un "Klusā grāmata" veidoti
kā pretestības dzeja pret indivīda, tautas apspiestību
20.gs. sākuma patvaldnieciskajā Krievijā. Tieša valoda
bija neiespējama cenzūras spaidu dēļ, pārlieku
"daudznozīmīga" arī ne, jo Rainis savu
dzeju bija iecerējis pēc iespējas ātrākai un spēcīgākai
iedarbībai uz lasītāju. Šai iecerei vislabāk atbilda
alegorijas, kuru nozīme gan bija no cenzūras formāli slēpta, bet
lasītājam īstenībā viegli caurredzama. Ne velti pēc
grāmatas "Tālas nozskaņas zilā vakarā" iznākšanas
konservatīvajās aprindās sacēlās protesta vētra,\footnote{Par
to sīkāk sk.: {\em Rainis J.} Kopoti raksti. --- R., 1977. ---
1.sēj. --- 458.-459.lpp.} kuras rezultātā
Krievijas Galvenā preses pārvalde izdeva rīkojumu
aizliegt atkārtotus dzejoļu krājuma izdevumus.

Lielai daļai alegoriju pamatā ir Jaunās derības līdzības.
Šī formālā dzejoļu piesaiste Jaunās derības līdzībām
lielā mērā palīdzēja dzejoļus, neskatoties uz tajos
ielikto dumpiniecisko saturu, krājumā saglabāt. Tā grāmatas
cenzors M.Remiķis, rakstot taisnošanās vēstuli
Galvenajai preses pārvaldei, uzsver, ka tā ir tikai
dabas un reliģiskā dzeja.\footnote{Turpat. --- 459.lpp.}

Reliģisko virskārtu "Tālās noskaņās zilā vakarā",
"Vētras sējā", kā arī "Klusajā grāmatā" Rainim
palīdz saglabāt Jaunās derības līdzības par pazudušo dēlu
(dzejolis "Pazudušais dēls" --- sal. Lūkas evaņģēliju 15, 3-32),
par lauka lilijām, putniem un zāli ("Lauka lilijas"
--- sal. Lūkas evaņģēliju 12, 22.32), par sējēju
("Sirdsasinis", "Ko sējējs bēdā", "Dzīves sējējs" ---
sal. Mateja vaņģēliju 13, 1-8) par laicīgu mantu
krāšanas veltīgumu ("Vecas gudrības" --- sal.
Mateja evaņģēliju 6, 19--20), par namu, kas celts
uz klints ("Akmens nams" --- sal. Mateja
evaņģēliju 7, 24-26) u.c. Tiesa, viena daļa šo
Raiņa dzejoļu arī dziļslānī nav pretrunā ar Jaunās
derības līdzību semantisko ievirzi. Atšķirība var būt tikai
adresātā, tā, piemēram, Jaunajā derībā Jēzus Kristus ar
līdzību par namu, kas būvēts uz klints, vēršas pie saviem mācekļiem
un visiem, kas par tādiem vēlētos kļūt,
turpretī Rainis dzejolī "Akmens nams" vēršas pie savas
tautas un tās brīvības izkarošanas lietas:

\begin{quote}
Uz akmens celts mūsu nams,\\
Tas akmens ir krams,\\
No tērauda lieti ir stabi, ---\\
Mūsu lieta stāv labi.\\
{\em (I, 405)}
\end{quote}

Sal. Mateja evaņģēlijā: "(7,24) Tāpēc ikviens, kas šos
manus vārdus dzird un dara, līdzināms gudram vīram, kas
savu namu cēlis uz klints. (25) Kad stiprs lietus lija
un straumes plūda un vēji pūta un brāzās šim namam
virsū, nams tomēr nesabruka, jo tas bija celts uz
klints."\footnote{Šeit un turpmāk Jaunās derības teksts
citēts pēc: Jaunā derība. --- [B.v.]: Latvijas ev. lut.
baznīcas konsistorija, 1988.}

Mainot adresātu, piemērojot Bībeles alegoriju konkrētam
laikam un konkrētai vietai, Rainim šķietami reliģiskā ietvarā izdodas
ielikt savu saturu.

Tomēr lielākā daļa Bībeles alegoriju Raiņa pirmo krājumu
dzejā tiek pārfrazēta. Dažos dzejoļos tas notiek
nemanāmi, dažos pretējā jēga tiek īpaši akcentēta.
Tā dzejolis "Lauka lilijas" sākumā pilnīgi seko
Jaunās derības alegorikai:

\begin{quote}
Jūs nezūdaties savas dzīvības,\\
Ko ēdīsat, ko dzersat rītu,\\
Nedz arī savas miesas apsegas,\\
Iekš kā to mīlīgi un silti tītu.

Jūs esat it kā putni gaisos brīvi:\\
Nedz viņi sēj, nedz viņi pļauj,\\
Nedz klētīs krāj, nedz kaudzēs krauj, ---\\
Un vai tiem maizes trūkst, ko elpēt dzīvi?

Jūs baro jūsu tēvs pārpilnībā\\
Un zemie zemes pīšļi tīrumā,\\
Ko jūsu kājas min.

Jūs esat skaistās lauka lilijas. ---\\
Kā viņas zied un sārtojas!\\
Tas krāšņais, priecinošais skats! --- ---\\
Nedz viņas strādā, nedz ar vērpj.

Un es jums saku: lielais Zālamans pats,\\
Kaut sevi visgreznākās drēbēs tērpj,\\
Ar visām pasauls lepnībām\\
Nau līdzīgs vienai vienīgai no tām.\footnote{Sal.
Lūkas evaņģēlijā: "(12,22) Un viņš sacīja saviem
mācekļiem: "Tāpēc es jums saku: nezūdaities savas
dzīvības pēc, ko ēdīsit, ne arī savas miesas pēc,
ar ko ģērbsities, (23) jo dzīvība ir labāka nekā
barība un miesa labāka nekā apģērbs. (24) Ņemiet
vērā kraukļus, kas nedz sēj nedz pļauj, kam nav ne klēts, ne
šķūņa, bet Dievs viņus uztur.. (27) Ņemiet vērā
lilijas, kas ne vērpj, ne auž. Bet es jums saku: pat
Zālamans visā savā greznībā nav bijis tā apģērbts
kā viena no tām.""}
\end{quote}

Turpretī Raiņa dzejoļa nobeigums ir pretrunā ar
Jēzus Kristus izteiktās līdzības daļu par zāli:\footnote{Sal.
turpat: "(12, 28) Bet, ja jau zāli laukā, kas šodien
aug bet rīt tiek krāsnī mesta, Dievs tik skaisti
apģērbj, cik vairāk jūs, jūs mazticīgie! (29) Tāpēc
neraizējieties arī jūs par to, ko ēdīsit un ko dzersit, un
neuztraucieties, (30) jo visu to meklē pasaules tautas.
Jūsu Tēvs jau zina, ka jums viss tas vajadzīgs.
(31) Dzenieties vairāk pēc Dieva valstības, tad jums arī
šīs lietas tiks piemestas."}


\begin{quote}
Tik --- lepnā zāle, --- to tie pīšļi zin, ---\\
Šo dienu stāv un rīt kļūs aprīta\\
No zemes pīšļiem un no tīruma,\\
Ko tā, iz viņa spēku sūkdama,\\
Ar kājām min.\\
{\em (I, 54)}
\end{quote}


Šajā dzejoļa noslēguma daļā alegoriju nomaina
simbols (ne tradicionālais, bet dzejnieka paša
--- individuālais, tiesa arī viegli "caurredzamais"),
kurā ielikta Bībeles tekstā neparedzēta slēptā nozīme.
"Lepnā zāle" simbolizē varas nesējus,
augstākās, apspiedējšķiras, "pīšļi", "tīrums" ---
zemes pamatu --- tos, kas varas hierarhiskajās kāpēs ir pašā
apakšā, --- tautu. Lai arī zāle sūc zemes auglīgos
spēkus, "ar kājām min", drīz to pašu aprij tīrums
un pīšļi.

Tradicionālās, no Jaunās derības līdzības pārņemtās
alegorijas akcentu neuzsvērta, bet būtiska pārbīde
pārvērš Raiņa dzejoļa nobeiguma daļas tēlus no
viennozīmīgām alegorijām ja ne gluži daudznozīmīgos,
tad neierastas nozīmes, līdz ar to grūtāk atšifrējamos
simbolos (zāle, tīrums, pīšļi).

Turpretī, piemēram, Raiņa dzejolī "Pazudušais dēls" Jaunās
derības līdzības par pazudušo dēlu alegoriju nozīme kardināli
pārfrāzēta. Lūkas evaņģēlijā Jēzus Kristus farizejiem
un rakstu mācekļiem stāsta līdzību par jaunāko
dēlu, kurš no tēva saņēmis sev pienākošos mantas daļu,
aiziet pasaulē un to notriec. Atjēdzies viņš nāk mājās,
grēkus nožēlodams un tēvam piedošanu izlūgdamies:
"(15,21) Bet dēls tam sacīja: Tēvs, es esmu
grēkojis pret debesīm un pret tevi, es neesmu vairs cienīgs,
ka mani sauc par tavu dēlu." Raiņa dzejolī
"Pazudušais dēls" alegorijas "skrandās ģērbts",
"basām kājām", "kuslām miesām" norāda uz "pazudušā dēla"
sociālo piederību (zemais, apspiestais) un uz
atriebes, dominēt, valdīt gribu:

\begin{quote}
Nē, nenāk viņš, lai jūgā plecus liektu, ---\\
Viņš nāk kā tiesātājs, lai jūs iz tempļa triektu.\\
{\em (I, 30)}
\end{quote}

Otrs nozīmīgs Raiņa pirmo dzejoļu grāmatu alegoriju
avots ir latviešu folklora. Sociālo tematiku
dzejoļos "Senatne", "Karaļmeita", "Gaismas pils",
"Čūska" u.c. palīdz ietonēt latviešu pasaku, teiku,
tautasdziesmu tēli, arī sakāmvārdi. Tā dzejolī "Caura muca"
alegorijas par cauru mucu un retu sietu priekšmetisko
līmeni veido latviešu sakāmvārdi: "Nenes ūdeni sietā";
"Sietā ūdeni neiesmelsi"; "Caura maisa nepiebērsi un ar sietu
ūdeņa nepienesīsi", "Kā caurā maisā" u.c. Virknē
gadījumu pat grūti precīzi definēt, vai šie pārnestās nozīmes
vārdi ir alegorijas vai simboli (tādā gadījumā ---
semantiski viennozīmīgi, kas simboliem parasti nav
raksturīgs), piemēram:

\begin{quote}
Tanī niknā dienāa\\
Asins šļācot šķīda:\\
Grima zemē pils un karaļmeita...

Daiļā karaļmeita\\
Sēd tur dziļā pilī,\\
Sēd un vērpj jau seši simti gadu.\\
\mbox{}[..]

Melnais suns pa laikam\\
Norūc, paceļ galvu,\\
Tālu augšā juzdams asins smaku...

Viņā niknā dienā,\\
Asins šaltēm šļācot,\\
Celsies augšā pils un karaļmeita..\\
{\em (Karaļmeita", I, 52)}
\end{quote}

No vienas puses, "pils" "karaļmeita" --- brīvības,
"melnais suns" --- brīvības atņēmēju simbols,
no otras puses, kā šie tā visi pārējie pārnestās nozīmes vārdi
dzejolī ir alegorijas, kurām tieši raksturīga viennozīmība.

Stingrāks un noteiktāks alegorijas un simbola nodalījums
ienāk ar dzejoļu krājumu "Klusā grāmata".
Tā dzejolī "Veļu laiva" vairs nevar runāt par
alegorijām. Kā "velu upe/ veļu Daugava", tā
"veļu airi", "veļu laiva", "veļu rokas" ir
daudznozīmīgi simboli, kur 1905. g. revolūcijas sakāves
plāns ir tikai viens no iespējamiem. Katrs no šiem
simboliem ietver kā mītisko (veļu valsts, veļu
ūdeņi), tā sociālo un vispārceilvēcisko plānu.
Simboli šai laikā Raiņa dzejā sāk iegūt dziļumu,
alegorijas ar savu nozīmes vieniespējamību
noiet izteiksmes līdzekļu perifērijā un, sākot ar krājumu
"Gals un sākums", izzūd pavisam.

{\color{red}
Atgriežoties pie tropiem pirmajos --- jaunromantiskā
un simbolistiskā posma --- Raiņa dzejoļu krājumos,
jāmin, ka līdzās alegorijām nozīmīga vieta ierādīta
arī simboliem un personifikācijām. }
Simboli balstīti romantismam raksturīgajos binārajos
pretstatos (brīvība --- nebrīvība, tagadne ---
pagātne, nākotne u.tml.). Daļa simbolu
(īpaši krājumos "Vētras sēja" un "Klusā grāmata"),
lai arī dzejnieka orientēta uz konkrētu sabiedrisko
situāciju, ar laiku ieguvusi vispārinātu zīmju raksturu
(pateicoties tradicionālo simbolu individualizācijai).
Tā garajā dzejolī "Pavasaras vētra" simboliskā tēlu rinda,
kurā ietilpst "putenis", "debesjūra", "viļņi", "kaivas",
"vārnu bari", "ledus", "kalni", "vējš" u.tml.,
šodien vairs nav traktējama tikai kā aizejošās vecās
un atnākošās jaunās varas konkrēto laiku brīvības un
nebrīvības izteicēja. Šie simboli ietver kā kādreiz tajos ieliktās
konkrētā laika situācijas, tā arī jebkura laika pasaules
pārradīšanas, atjaunošanas zīmes:

\begin{quote}
Gar visu vakarpusi putens ceļas,\\
Pār kalniem atnāk ledus lauzējs vējš,\\
Kā vilnis debesjūrā zemjup veļas,\\
Brūk lejā krākdams, neturami spējš.\\
{\em (I, 182)}
\end{quote}

{\color{red}
{\bf Personifikācija} parasti tēlo priekšmetu vai abstraktu
jēdzienu (A) kā dzīvu personu (P), piedēvējot
tam dzīva organisma īpašības (A=P).\footnote{{\em Dziļleja K.}
Poētika. --- 47.lpp.} Šis pārnestās nozīmes
izteiksmes līdzeklis ir iemīļots visos dzejnieka krājumos.
Tomēr starp personifikāciju daiļrades sākuma un beigu
posmā ir jūtama atšķirība. "Tālās noskaņās zilā vakarā",
"Vētras sējā" un "Klusajā grāmatā" A=P nenozīmē, ka
Rainis, līdzīgi kā tas ir folkloras tekstos un mītiskajā
domāšanā, uzskatītu dzīvu un nedzīvu būtni par identām.
Rainim personifikācija šeit veic galvenokārt to pašu funkciju,
ko alegorija un simbols, --- palīdz aizsardzībai pret
cenzūru. Piedēvējot saulei, vējam, vētrai u.c.
cilvēku īpašības, darbības veidu kā nevainīgu poētisku
rotājumu (tas cenzoram ļauj šos tekstus pieskaitīt pie
dabas lirikas), Rainis īstenībā raksta sociālo noskaņu dzeju,
kur personifikācija ir tikai tehnisks līdzeklis, kā
apslēpt īsto sakāmo:

\begin{quote}
Un vējus es dzirdēju\\
Ņirdzīgi smejam..\\
{\em (I, 19)}
\end{quote}

\begin{quote}
Tad augšā kāps pilskalns..\\
{\em (I, 51)}
\end{quote}

\begin{quote}
Bet mēmā zeme\\
Tad muti vērs\\
Un aprīs jūs,\\
Un kūpēs sērs.\\
{\em (I, 152)}
\end{quote}

\begin{quote}
Tālē tumsa glūna ar melnām acīm.\\
{\em (I, 341)}
\end{quote}
}

{\bf Salīdzinājums} un {\em epitets}, lai arī
pirmajos krājumos sastopams, tomēr nepieder pie
svarīgākajiem tropu rindā. Jāpiezīmē arī, ka
jaunromantiskā un agrīni simboliskā posma Raiņa
dzejas salīdzinājumi ir monosemantiski. Saules stars
"kā dzenātas stirnas bēg" (I, 123), dvēsele mirdz
balta "kā kalnu sniegi" (I, 155), varoņa vaigs
ir "ciets kā tērauds" (I, 324), savu dvēseli viņš nes
"kā rīta sauli sārtos mākoņos" (1, 47), pazudušais
dēls nāk mājās "kā tiesātājs" (I, 30) u.c.
Līdzīgi --- viennozīmību pasvītrojoši --- ir arī
epiteti: "Krūtīs tev gaiša / Un degoša sirds" (I, 126);
"Nopietnas ūsas, / Prātīgas lūpas, / Deguns miermīlīgs /
Pamatīgs vēders!" ("Filistris", I, 42).

{\bf Metafora} --- Sava veida saīsināts salīdzinājums,
kad viena priekšmeta, lietas īpašības piedēvē kādam
citam priekšmetam, lietai vai būtnei. Taču atšķirībā
no salīdzinājuma, kas norāda uz līdzīgumu (īstu
vai tikai iedomātu) starp A un B, metafora parasti
norāda uz kādu konstantu pazīmi, kas vieno A ar B.
Pirmajos Raiņa dzejoļu krājumos metafora lietota
ļoti reti un tai nav īpašas nozīmes tropu sistēmā.

Metafora tiek aktualizēta Raiņa filozofiskā un jo īpaši
impresionistiskā posma dzejā ("Gals un sākums",
"Addio bella!", "Sudrabota gaisma", "Mēness meitiņa").
Tā norāda uz būtisku sakaru starp kādām divām
parādībām (A un B), ar elementa B palīdzību padziļinot
priekšstatu par elementu A: "Vējš smiltis-ļaudis putina" (II, 279);
"Klūst tauta --- avis, kurām aizbēg gani" (III, 125);
"Skrej putnis, paliek zeme --- kviešu statnis"
(III, 136). Nereti Rainis lieto A un B elementus
apgrieztā kārtībā (t.i., nevis "..acis tavas ---
zaļas zvaigznes", bet otrādi: "..zaļas zvaigznes acis tavas",
III, 392).

{\bf Paralēlisms}, kas ir pastiprināts salīdzinājums, minot
līdztekus divus priekšmetus un to īpašības
(A un B)\footnote{{\em Dziļleja K.} Poētika. --- 46.lpp.},
pirmajos Raiņa dzejoļu krājumos tikpat kā nav sastopams.
Lai arī pēdējā --- impresionistiskā posma ("Addio bella!",
"Mēness meitiņa") dzejā paralēlisms nepieder pie biežāk
lietotajiem tropu veidiem, tomēr tam ir pietiekoši svarīga vieta.
Līdzīgi tautasdziesmu paralēlismiem, tas Raiņa dzejas
tekstos sasaista, satuvina divu dažādu plānu lietas, parādības
vai arī, piemēram, augu un cilvēku valsti:

\begin{quote}
Deviņi balti narcisi,\\
Desmitā sarkana roze ---\\
Deviņi slaiki brālīši,\\
Desmitā dailiņa māsiņa,\\
Desmitā māsiņa roze!\\
{\em (III, 38)}
\end{quote}

\begin{quote}
Zaļas bija vasar' lapas,\\
Rudens rudas padarīja, ---\\
Balta bija dzīve mana,\\
Bēdas melnu padarīja..\\
{\em (III, 430)}
\end{quote}

Otrajā piemērā krāsu pāri "zaļš --- ruds" un
"balts --- melns" veido semantisko paralēlismu,
kura pamatā binārais pretstats "dzīvība --- nāve,
prieks --- skumjas".

{\color{red}
Ļoti nozīmigs tropu veids Raiņa vēlīnajā dzejā ir
personifikācija. Līdzīgi paralēlismam un metaforai,
tā kļūst par nozīmīgu cikliskās laika un telpas
izteiksmes veidu. Personifikācija te nav vairs vienas
vienīgas traktējuma iespējas (sociālajā plānā)
demonstrētāja, bet ietver daudznozīmību, tuvojoties
dažāda līmeņa elementu līdzvērtības izpratnei,
kā mītiskajā domāšanā. Koks, puķe un cilvēks,
tauriņš un cilvēks ir viens: "Pa manu ceļu ķirsis
ziedus kaisa" (III, 61); "Pa pļavu skraida baltas
puķītes / Ar ziliem tauriņiem" (III, 62).
}

Semantiski ietilpīgi un daudznozīmīgi  Raiņa vēlīnajā
dzejā ir arī epiteti un salīdzinājumi. Kā zināms
starpposms starp agrīno (pirmo triju dzejoļu krājumu)
un vēlīno dzeju ir filozofiskā posma darbi (galvenokārt
krājums "Gals un sākums"). Epiteti, salīdzinājumi
jau te iegūst padziļinājumu, funkcionējot kā
abstraktu jēdzienu iedzīvinātāji dzejā. ---
"Tu paliec viens, un apkārt bālē/ Tik tukša, sausa,
balta smilts" (II, 292); "Aiz divām nāvēm /
Vēl trešā mīt, --- / Kā sēkliņa akā / Dzīve krīt"
(II, 323).

Raiņa divdesmito gadu dzejā epitets un salīdzinājums
kļūst par mākslinieciskās izteiksmes līdzekli, kas
palīdz sasaistīt dažādus laikus (esošo ar bijušo un
nākamo, pirmlaiku ar savu laiku) un atrast dzīvās un
šķietami nedzīvās dabas elementu kopsakaru pasaulē.
Salīdzinājumos tiek aktualizēti kolektīvajā atmiņā
iekodētie pirmtēli: mēness, saule, zvaigznes, putas,
laiva, tauriņš, putns u.c. Piemēram: "Es skaistu sapņu
tā kā laivas līgots" (III, 303), "Kā mēness spīdu es
ij pusnaktī" (III, 366); "Uz tavām krūtīm atdusēt, /
Kā putas dus uz ūdens" (III, 67); "Man atliek tikai bēgt
kā saulei ziemā" (III, 161).

Savukārt Raiņa divdesmito gadu krājumos lielāko tiesu
epitetu veido krāsu kods. Atšķirībā no agrīnās dzejas
krāsu epitetiem ar vienas nozīmes izcelšanu te krāsa
ir tēlojamās pasaules dziļuma un daudznozīmības zīme:

\begin{quote}
Zem kājām ezers man mainās:\\
Sudrabs un zaļš, un zils..\\
{\em (III, 57)}
\end{quote}

\begin{quote}
Aiz melniem kalniem rožu saule laidās:\\
Zaļš, oranžs, violets\\
Un zils un sārts --- sirds sapņo ilgās, gaidās.\\
{\em (III, 375)}
\end{quote}

\begin{quote}
Rožaini smaidi,\\
Sudraboti smiekļi..\\
{\em (III, 450)}
\end{quote}

Visnozīmīgākais pārnestās izteiksmes veids Raiņa
divdesmito gadu dzejā ir simbols --- pirmtēls, kas
balstīts arhetipiskajā pieredzē. Tas vairs nav
jaunromantiskā un agrīnā simbolistiskā posma simbols ---
alegorija ar iekodētu vienu un nepārprotamu nozīmi.
Simbola --- pirmtēla pamatā ir tautas un vēl plašāk ---
cilvēces kolektīvās zemapziņas nozīmju uzkrājums, kas ietver
kā sava laika nozīmju uzslāņojumu, tā visu iepriekšējo laiku
semantiskos slāņus. Pirmajās dzejoļu grāmatās simbols
"saule" vedināja lasītāju tikai uz konkrētā laika
brīvības alkām (tautas, indivīda tiecība uz brīvību),
bet pēdējās (īpaši impresionistiskajā poētiskajā manierē
veidotajās) "saule" ietver gadu tūkstošos uzkrāto nozīmju
dažādos slāņus. No: saule --- augstākais kosmiskais spēks,
radošā sākotne; Saule --- Mēness kā debesu dievību pāris;
dzīvības devēja un uzturētāja; auglību, ražību vairojošais
spēks; gaismas nesēja; nemirstīgā vai cikliski mirstošā
un atdzimstošā; neizsmeļamā enerģijas krātuve; līdz:
saule --- brīvība (individuālā, tautas, cilvēces,
kosmosa), mīlestība; siltums; tuvums, mīļums;
sievišķais; cerība; atjaunotne; patiesība, taisnība;
mūžība:

\begin{quote}
"Vēl tevī pagātne ir nepārciesta, ---\\
Vēl nespīdi tu pats,\\
Tik tava dvēsle ir uz sauli griezta."\\
{\em (III, 141)}
\end{quote}

\begin{quote}
Kad citam ziema, --- saule ver man durvis.\\
{\em (III, 314)}
\end{quote}

\begin{quote}
Divas saules\\
No iekšas, no āras\\
Satikās ---\\
Vienā skūpstā.\\
{\em (III, 490)}
\end{quote}

\begin{quote}
Saul' un mēness uz sudraba palaga\\
Zelta gultā kopā gulēja..\\
{\em (III, 448)}
\end{quote}

\begin{quote}
--- Zini, ka saules mīla\\
Ved uz aizsaules malu.\\
{\em (III, 362)}
\end{quote}

\begin{quote}
Ne pārvērš svētku brīnums ikdienību,\\
Bet pārsauļo,\\
Un saule lēni plaucē nākamību.\\
{\em (III, 349)}
\end{quote}

\begin{quote}
--- Kaut drīzi es kā saule būtu,\\
Kas peļķēs kāpj un nejūt riebumu!\\
{\em (III, 528)}
\end{quote}

\begin{quote}
Mēs sāksim citu sauli, kad šī gaisīs.\\
{\em (III, 385)}
\end{quote}

\begin{quote}
"Kur ej?"\\
--- Uz sauli...\\
{\em (III, 719)}
\end{quote}

Kas ir tas, kas atļauj lasītājam uztvert
šīs dažādās simbola "saule" nozīmes un
nozīmju nianses? Pirmkārt, tā ir jau minētā
kolektīvā zemapziņas atmiņa, kas visasāk
aktivizējas sapnī un dzejā, plašāk --- mākslā
vispār. Otrkārt, tā ir katra cilvēka (lasītāja)
individuālā pieredze un tēlainās valodas
uzkrājums. Šis kolektīvā un individuālā
sakausējums, precīzāk --- arhetipisko semantisko
struktūru pārradījums atkarībā no katra personisko
pārdzīvojumu pieredzes tiek uztverts ļoti plašā
amplitūdā.

Biežāk lietotie arhetipiskie simboli --- pirmtēli
Raiņa divdesmito gadu dzejā ir jau minētā saule,
mēness, zvaigznes, nakts/tumsa --- diena/gaisma,
rudens ---  ziema --- pavasaris --- vasara, koks,
puķe, zāle, jūra/ūdens, uguns, kalns, dārzs, sēkla,
vārti, logs, ceļš, čūska, acis, atmiņa, sapnis.

Varētu gaidīt, ka arhetipisko tēlu aktualizācija
"Dagdas skiču burtnīcās" un plašāk --- vēlīnajā Raiņa
dzejā vispār gūs plašu atbalsi lasītājos.
Tomēr tā nenotika. Tieši šī dzeja joprojām lasītājiem
ir vissvešākā. "Tālu noskaņu zilā vakarā",
"Vētras sējas", "Klusās grāmatas", arī vēl
"Gala un sākuma" dzejoļus ar iekšēju patiku vai nepatiku,
bet gandrīz katrs vidusmēra lasītājs zina un atceras,
un šad tad pārlasa (vai skolā, augstskolā viņam
to liek darīt). Raiņa pēdējo gadu dzeja, neskatoties
uz labvēlīgajām (bet vairākumā gadījumu --- virspusēji
slavinošajām) kritikas atsauksmēm, palika no latviešu
dzejas it kā nostatu. Varbūt tāpēc, ka te Rainis uzsāk
latviešu dzejā pilnīgi citu un citādu izteiksmes veidu, nekā
tas tradicionāli pierasts. Proti, 19.--20.gs.
pirmās puses dzejnieki orientējās (noliedzot vai
apliecinot) uz nozīmēs iepriekšparedzamu tēlu virkni.
Lai šos tēlus uztvertu, lasītājam parasti bija jāizvēlas
starp vienu vai divām iespējām (iepriekšējās tradīcijās
jau izstrādātām). Tas bija jaunais jau labi pazīstamajā,
galvenokārt apziņas līmenī iekodētajā.
Rainis "Dagdas skiču burtnīcās" piedāvā klasicisma,
sentimentālisma, romantisma, reālisma "nekustīgo",
stabilo, stingri diferencēto tēlu nozīmju skalas
paplašinājumu un padziļinājumu.
Lai uztvertu, kas tā par sauli un mēnesi, kas guļ vienā
gultā, lasītājam ir ne tikvien jāatbrīvojas no racionalizētas
tēlu uztveres (tēls --- kādas vienas idejas zīme), bet arī
jārosina vēsturiskā atmiņa un galvenais --- mītiskais/cikliskais
uztveres un domāšanas veids. Atmiņa (kaut vai tikai
zemapziņā ieslēgtā) ir tā, kas palīdz sasaistīt
folkloras un Raiņa dzejas mēness un saules, kā arī
katra konkrētā "es" (lasītāj) kopējo pārdzīvojumu vilni,
atrast tiltus, kas pārmetami starp dievību un cilvēku,
svešu un savu mīlestību, dzīvi un nāvi:

\begin{quote}
Nu mēness tālē manas ceļa jūtis:\\
Es nerodu to mīlu jaunadari,\\
Tik atmiņā spīd saulesloga rūtis.\\
{\em (III, 341)}
\end{quote}

Bet 1920.--1940. g. vidusmēra dzejas lasītājs bija
orientēts uz ierobežotu atmiņu, kas lielākoties
pagātnē reducējās uz tēliem --- shēmām (Dievs un
Velns, Lāčplēsis un Laimdota u.c. --- katram pa
vienai un vienkrāsas funkcijai), nevis tēliem ---
polivalentu nozīmju tvertnēm. Lai "Dagdas skiču
burtnīcas" kļūtu ne tikai par literatūras, bet arī
lasītāju apziņas--zemapziņas faktu, bija vajadzīgs,
lai šī citāda tipa dzeja nogulsnējas, atrod turpinājumu
vai pārradījumu (to izdarīja galvenokārt I. Ziedonis)
vai atdzimst stilizācijās (to izdarīja 1960.--1970. g.
Raiņa epigoņi). Raiņa "Mēness meitiņas" mīlas
dzejoļi varēja tikt komponēti (R.Pauls), uztverti
(resp. kolektīvi izjusti) tikai pēc gandrīz
deviņdesmit gadu nogulēšanas, pasīvas eksistences
perioda.

{\bf Metonīmija} --- tropu veids, kur viens jēdziena
apzīmējums ņemts otra vietā tā iemesla dēļ, ka vienam
(A) ar otru (B) ir kāds pastāvīgs sakars, piemēram, kā
daļai ar veselo, materiālam ar lietu, radītājam ar
radītāju, traukam ar saturu\footnote{Par metonīmiju
sīkāk sk.: {\em Dziļleja K.} Poētika. --- 44.--45.lpp.;
{\em Valeinis V.} Ievads literatūras zinātnē. --- R.,
1978. --- 106.--107.lpp.} u.tml.
Metonīmija Raiņa dzejai nav raksturīga.
Protams, var atrast pa atsevišķam metonīmijas paraugam
("Galvā tev zelts, un tur tūkstots krāsas", III, 50;
u.c.), bet salīdzinājumā ar metaforu, simbolu vai
salīdzinājumu, epitetu šis tropu veids Raiņa dzejā
ir mazsvarīgs.

{\bf Figūras} ieguva savu klasifikāciju jau antīkajos
laikos, tās tālāk detalizēja un tām īpašu nozīmi piešķīra
Eiropas baroka un klasicisma dzejā. Romantisms,
kas nāca kā klasicisma stingrās žanru un metru sistēmas
nojaucējs, netteicās no figūrām kā tādām, bet sajauca tās,
saplūdinot robežas starp atsevišķiem to veidiem.
Divdesmitā gadsimta dzejā visai problemātiski
runāt par figūrām. Būtu pārspīlējums teikt, ka to
dzejā nav vai ka tām dzejā nav nekādas nozīmes.
Dzejnieki teiktā ekspresivitātes pastiprināšanai
lieto kā divkāršojumu, tā daudzkāršojumu,
kā anaforu, tā epiforu u.tml. Beet,
pirmkārt, pazūd stingrs nošķīrums starp atsevišķiem
figūru veidiem, otrkārt, figūras tiek lietotas
pēc iekšējas izjūtas un vajadzības, ne pēc
apzinātiem priekšrakstiem, kā tas bija, piemēram,
klasicismā. Tas izskaidrojams gan ar formas kā
saturelementa nozīmes vājināšanos vispār (folklorā forma ---
tā, kas dod teiktajam maģisku spēku; antīkajā
retorikā, barokā, klasicismā ir uzskats, ka
noteikts saturs, tematika saderas ar noteiktām formām),
gan ar romantisma aizsāktās dažādu stilu, žanru,
metru, strofu, tropu u.c. izteiksmes līdzekļu robežu
apzinātu saplūdināšanu.

Šī robežu pazaudēšana, saplūšana starp atsevišķajiem
izteiksmes līdzekļiem un analizējamo izteiksmes līdzekļu
skaita samazināsānās salīdzinājumā ar klasicisma laikiem
vērojama ne tikvien dzejā, bet arī dažādu autoru poētikās.
Tā Jānis Kalniņš savā "Latviešu rakstniecības
teorijā"\footnote{{\em Kalniņš J.} Latviešu rakstniecības
teorija. --- Jelgava, 1892.} nenodala līdzibu ---
alegoriju no paralēlisma (abus apvienojot figūru
kategorijā), savukārt Kārlis Dziļleja paralēlismu
ietilpina tropu jēdzienā,\footnote{{\em Dziļleja K.}
Poētika. --- 46.--47.lpp.} kurpretī Vitolds Valeinis
paralēlismu pieskaita (vēsturiski pilnīgi pamatoti)
pie figūrām.\footnote{{\em Valeinis V.} Ievads
literatūras zinātnē. --- 114.lpp.} J. Kalniņam
aplūkojamo tēlainās izteiksmes līdzekļu sastāvā
ietilpināti ne tikai epiteti, salīdzinājumi,
metaforas, sinekdohas, metonīmijas, bet arī
asindetoni, polisindetoni, epanortozes, piemēri
jeb paraugi (no latīņu {\em exemplum}), bet poētikās,
kas iznākušas pēc otrā pasaules kara, analizējamo
tropu un figūru saraksts ir krietni nabagāks. Resp., arī
poētikas pakāpeniski atspoguļo daudzu agrāk
funkcionālu tropu, figūru nozīmes zudumu dzejā.

Arī šajā pētījumā nav stingras robežas starp tropiem un
figūrām. Mēs vismaz 20.gs. dzejā vairs nevaram
novilkt norobežojošu svītru starp pārnestas
nozīmes vārdiem (tropiem) un figūrām ---
vārdiem, kas atšķiras no ikdienā parastā vārdu
kārtojuma.

Kurš var pateikt, vai epitetā, salīdzinājumā,
paralēlismā, asindetonā vārdi saglabā tiešo nozīmi
vai iegūst citu? Jebkurš izteiksmes līdzeklis, vai
tas būtu epitets vai metafora u.tml., reiz
dzejā lietots, iegūst kādu citu nozīmes niansi
un padziļinājumu.

Tāpēc, runājot par figūrām Raiņa dzejā, īstenībā tiek
turpināta saruna par tropiem vai plašāk ---
tēlainas izteiksmes līdzekļiem.

{\bf Anafora} (vārdu atkārtojums rindu sākumā)
un {\bf divkāršojums} (blakus stāvošu vārdu atkārtojums)
ir figūras, kas atrodamas visos Raiņa dzejoļu
krājumos, piemēram:

\begin{quote}
Kur debess ar zemi tiekas,\\
Kur tālums neplūst vairs,\\
Tur balta laime man liekas,\\
Tur tu, tur nogrimst dairs.\\
{\em (I, 76)}
\end{quote}

\begin{quote}
Uz priekšu, uz priekšu! tur darbs, tur spēks..\\
{\em (I, 118)}
\end{quote}

\begin{quote}
Kā pļava tevi sauc!\\
Kā smilgas zelta spilviņas tev kārsta!\\
Tev ziedi drēbēs spīdu pārslas bārsta!\\
* * *
Cik ziedi sārti!
Cik zaļa zāle, debess zils, kā saule plaukst!
Mums veras visa mīlas valsts un vārti!
\end{quote}

\begin{quote}
Mēs kalām, kalām,\\
Ka dzirkstis lēca.\\
{\em (II, 148)}
\end{quote}

Ne tikai Raiņa, bet lielā mērā visai 20. gs.
sākuma romantiskā poētiskā tipa dzejai raksturīgs
divkāršojuma veids ir viena vārda apvienojums
saliktenī (bieži pirmajam atmetot galotni un
apvienojot abus vienā ar defises palīdzību):
"Dziļ-dziļi dvēselē zaigums" (I, 109);
"Lēn-lēni dziļāk / Tumsa uz zemi līkst" (I, 77).

Anafora aptuveni vienādi bieži lietota visos
Raiņa dzejoļu krājumos, savukārt divkāršojumam
vērojama tendence ar laiku samazināties.
Tas visvairāk sastopams 1903.--1911. g. darbos,
daudz retāk 20. gadu dzejā.

Vēlīnajā dzejā Rainis iecienījis figūru, kas nav
ne īsti {\bf elipse} (kāda teikuma locekļa
izlaidums), ne {\bf noklusējums} (domas
nepabeigtība).\footnote{Par šīm figūrām sīkāk sk.
{\em Valeinis V.} Ievads literatūras zinātnē. --- 119.lpp.}
Elipsē parasti var restaurēt, kādi konkrēti
vārdi ir izlaisti, Raiņa dzejā tas nav iespējams.
Domuzīme, kas parasti noslēdz rindu vai pusrindu,
vienīgi norāda uz neizteikto, bet neatļauj
šo neizteikto vārdiski konkretizēt.
Noklusējums parasti raksturīgs romantiskā
poētiskā tipa dzejā, un visbiežāk grafiski to
iezīmē ar daudzpunkti. Turpretī Rainis nekad to
neiezīmē ar daudzpunkti, bet tikai ar domuzīmi.
Domuzīme veido kā tiltu, kas vedina lasītāju turpināt
pašam domu, pārmest pusaprauto domu uz nākamo pusrindu,
rindu, pantu vai pat dzejoli:

\begin{quote}
Efeja vija ---\\
Pār senām drupām zāļu segu auž,\\
Kā tu ar skaistumu sedz to, kas bija.\\
Olivia ---\\
{\em (III, 39)}
\end{quote}

\begin{quote}
Reiz motorlaivā braucu --- ko lai stāstu? ---\\
Pēc sapnī šūpots jutos, jūras aina\\
Un krasti svaidījās --- --- grib miera glāstu,\\
Šī sirds lūdz līdzsvara, --- jo tava vaina.\\
{\em (III, 66)}
\end{quote}

Raiņa dzejā (īpaši vēlīnajā) viens no iecienītiem
figūru veidiem ir {\bf asindetons} jeb bezsaikļu
konstrukcija. Jau antīkajās retorikās atrodam
norādes uz asindetonu kā svarīgu mākslinieciskās
izteiksmes līdzekli. Tā, piemēram, Halikarnas
Dionīsijs "Vēstulē Pompejam" raksta:
"Jāatceras, ka stila spēcīgumu no visām figūrām
visvairāk sekmē saistvārdu izlaidums."\footnote{Cit.
pēc: Античные риторики. --- М., 1978. --- С. 179.}

Konstrukcijas bez saikļiem palīdz dzejā panākt domas
spraigumu:

\begin{quote}
Kas dzīve, saule, es --- kas tas, kas dzisa?\\
{\em (III, 91)}
\end{quote}

\begin{quote}
Jo bērnu gadus kopā nodzīvoju\\
Ar upi, smilti, zāli, sauli, vēju..\\
{\em (III, 97)}
\end{quote}

\begin{quote}
Jūra, priedes, saule, vakars, rīts ---\\
Viss, kas dārgs man bij, ir sagandīts.\\
{\em (III, 405)}
\end{quote}

Un otrādi --- {\bf polisindetons}\footnote{Par
asindetonu un polisindetonu kā figūrām latviešu
dzejā rakstījis J.Kalniņš (Ltviešu rakstniecības
teorija. --- 39.--40.lpp.). Asindetonu viņš
dēvē par vārdu virkni, polisindetonu par vārdu
vīteni; A.Bračs pirmo sauc par virkni, otro ---
par viju. --- {\em Bračs A. Rakstniecības
teorija. --- Cēsis, 1920. --- 1.d. --- 124--125.lpp.}}
jeb daudzsaikļu konstrukcija (Raiņa dzejā tā arī
semantiski nozīmīga figūra) dzejas domas ritējumu padara
lēnāku, rimtāku:

\begin{quote}
Ik sīķa asins lāsa\\
Man rit un tek, un steidz uz sirdi lecot..\\
{\em (III, 72)}
\end{quote}

Tomēr polisindetons salīdzinājumā ar asindetonu
Raiņa dzejā lietots daudz retāk. Iespē'jams, tas ir
sakarā ar Raiņa tiecību visos iespējamos līmeņos
padarīt dzejas domu spraigāku, īsāku, koncentrētāku,
ko polisindetons pretēji bezsaikļu konstrukcijai
neveicina.

Pirmajiem Raiņa dzejoļiem raksturīgas arī tādas
figūras kā {\bf retoriskais jautājums} un
{\bf retoriskā uzruna}. Retoriskais jautājums ir
spriedums, kas izteikts kā jautājums, kurš neprasa
atbildi, jo jautājamā forma ir tikai apstiprinājuma
līdzeklis.\footnote{{\em Valeinis V.} Poētika. ---
R., 1961.g. --- 215.lpp.} Savukārt retoriskā uzruna
ir "..tāda uzruna, ar ko dzejnieks pievēršas
parādībām, kuras nevar atsaukties (nedzīvas vai
abstraktas parādības, klātneesošas personas,
bieži --- iedomāts lasītājs u.tml.)"\footnote{{\em
Valeinis V.} Poētika. --- 215.--216.lpp.} Raiņa
pirmo dzejoļu krājumu emocionāli kāpinātajā, augstajā
izteiksmes stilā balstītajā dzejas valodā retoriskais
jatājums un uzruna ir svarīgi formas elementi:

\begin{quote}
--- --- --- Ko jūs ļāvāt ikdienībai\\
Saules smieklus slāpēt sirdī?\\
{\em (I, 134)}
\end{quote}

\begin{quote}
Raudat sev, raudat, ---\\
Kas gļēvos ies žēlot?\\
{\em (I, 39)}
\end{quote}

\begin{quote}
Ko, senais sapnis,\\
Tu dvēsli moci,\\
Tās tālās dienas\\
Pa miglu loci?\\
{\em (I, 79)}
\end{quote}

\begin{quote}
"Ko esat, pantiņi, tik īsas elpas,\\
Kā pušu plēstas biklas nopūtas?"\\
{\em I, 91}
\end{quote}

Vēlākajās Raiņa grāmatās (īpaši tajās, kas
nāca klajā 20.gados) retoriskā uzruna un jautājums
pazūd no dzejoļiem. Tas arī saprotami, jo
Rainis te ārēji nemanāmi, bet konsekventi
bināro pretstatu (augstais --- zemais;
patiesība --- nepatiesība; dzīvība -- nāve u.tml.)
dzeju nomaina pret ternāro modeli (dzīvība ---
nāve --- atdzimšana) vai modeli bez krasi izteikta
tēlu semantikas norobežojuma. Akcents tiek
likts nevis uz toņiem, bet pustoņiem, kas palīdz
atklāt lietu un būtņu neviennozīmību. Tā "migla"
(arī "dūmi") pirmajos dzejoļu krājumos simbolizē
šķērsli (ceļā uz brīvību), savukārt pēdējos
tā drīzāk ir vienotāja vai -- gan vienotāja,
gan šķīrēja, sal.:

\begin{quote}
Aiz pelēkās miglas dobji dūc\\
Kā apspiesti vaidi tik drūmi,\\
Kā zvani dun, kā klaudz, kā rūc,\\
Kā raud... bet visu māc dūmi.\\
{\em (I, 221)}
\end{quote}

\begin{quote}
Ar manu sirdi gribat jūs mani vilt,\\
Lai maigās jūtās norimtu tumšots prāts,\\
Lai liegi izplūstošās miglās\\
Izzustu plaisa starp jums un mani.\\
{\em (III, 46)}
\end{quote}

Raksturīga figūra Raiņa dzejā ir arī {\bf retoriskais
izsauciens} --- domu un jūtu izpausme
izsaukuma teikuma veidā.\footnote{Sīkāk par to
sk.: {\em Valeinis V.} Poētika. --- 216.--217.lpp.}
Ja Raiņa agrīnajā dzejā patiešām var runāt par retorisko
izsaucienu kā dzejas domas izpausmi izsaukuma
jautājuma veidā, tad 20. gadu dzejā tie drīzāk ir
izsauksmes vārdi (visbiežāk "ak") ---
emocionāli sakāpinātas runas signalizatori, ne izsaukuma
teikumi tradicionālā izpratnē, sal.:

\begin{quote}
Augšā! uz priekšu!\\
Uzvaras drosmā..\\
{\em (I, 128)}
\end{quote}

\begin{quote}
"Brāz bangas, tu, naidīgā pretvara ---\\
Mēs tāles sniegsim, kur laimība!\\
Tu vari mūs šķelt, tu vari mūs lauzt ---\\
Mēs sniegsim tāles, kur saule aust!"\\
{\em (I, 137)}
\end{quote}

\begin{quote}
Ak, tumšo mirtu ---\\
Es allaž tevi uzskatu ar bailēm:\\
Man ir, it kā es pats vai mīla mirtu.\\
{\em (III, 74)}
\end{quote}

\begin{quote}
Ak, saule, --- pieņem mīlu, dod un žēlo,\\
No tās, pēc tās man dvēsle mūžam kvēlo.\\
{\em (III, 106)}
\end{quote}

"Dagdas skiču burtnīcās" retoriskais izsauciens
bieži apvienots ar retorisku uzrunu vai jautājumu
(kā, piemēram, iepriekš citētajā III, 74).

Dzeja tradicionāli ir monoloģisks vēstījums,
bieži "es" formā (semantiski ---
"es" skatījumā) ietverts. Tā tas ir arī Raiņa pirmajos
dzejoļu krājumos. Taču sākot ar grāmatu "Gals un sākums",
Rainis īpašu uzmanību sāk pievērst {\bf dialogam} ---
"es" sarunai ar īstu vai iedomātu oponentu vai
domubiedru, plašāk --- sarunu biedru.
Dialoga forma ir dzejnieka skatu leņķa paplašināšanās,
dažādu viedokļu vienādiespējamības formālais izteiksmes
līdzeklis. Kā krājumā "Gals un sākums", tā
"Dagdas skiču burtnīcās" krietns skaits dzejoļu
sacerēts jautājumu --- atbilžu virknes formā:

\begin{quote}
"Es gribu redzēt, kur tavs prieks?" --- Nau tuvu. ---\\
"Būs grūti jāmeklē?" --- Tev acis gurs. ---\\
"Gan saulē žibēs? ies pa zaļu druvu?"\\
--- Būs nakts, tik priekšā spīdēs ugunskurs. ---

"Būs ilgi jāiet?" --- Ilgas tukšas dienas. ---\\
"Kur iet tavs prieks, zied ceļš?" --- Deg svelošs tvans. ---
"Un kur tam ceļam gals?" --- Pie pasauls sienas. ---

"Kā prieku pazīt?" --- Sēd pie uguns gans\\
Un avi sedz. --- "Tas citu prieks!" --- Tas mans. ---
\end{quote}

Tipoloģiski šī dialoga forma radniecīga daudzu
tautu folklorā iecienītajiem jautājumu --- atbilžu
tekstiem, kas parasti bija vai nu
iniciācijas rituālu piederums
(neiesvaidītais jautā, iesvaidītais atbild; pirmais ir tas,
kas vēl sakrālo nepārzina, otrais ir tas, kas to pārzina un
slēptu zīmju --- simbolu valodā izsaka), vai tika
rečitēti pasaules pārradīšanas rituālos u.c. ciltij,
saimei svarīgos brīžos. Tā liela daļa latviešu
mītisko dziesmu ir rituāli dialogi, kur kāds (acīmredzot ---
neiesvaidītais) uzdod jautājumus par dievībām, kosmosa
kārtību un otrs (iesvaidītais, kas bieži runā dievības
vietā, atbild ar dievības muti) atbild:

{\color{red}
\begin{quote}
Kur tas rīta Ausekliņš,\\
Ka neredz uzlecot?\\
--- Auseklinš Vāczemē\\
Samta svārkus šūdināja.\\
{\em (LD 33 831)}
\end{quote}

Te simboliskā izteiksmē (šķiet, pastāvēja uzskats, ka
tieši izteikto kaitētāji gari vai cilvēki var
noskaust, izjaukt) jautāts par to, kur vakarā pazūd
Rīta zvaigzne. Atbildē Rīta zvaigznes cikliskā
nomiršana tēlota kā nokļūšana Vāczemē (t.i. rietumos,
kas senajiem latviešiem asociējās ar veļu valsti).

Raiņa dzejā dialogi vairumā gadījumu ir saruna pašam
ar sevi, ar paša pretrunīgajiem dažādajiem "es".
Varbūt vienīgi pēdējā dzejoļu grāmatā "Mēness meitiņa"
nojaušams reāls sarunu, pārdomu partneris ("es" un
iemīļotā). Neskatoties uz to, ka vairākiem dialogiem
par pamatu varēja būt reāla sarakste, domu apmaiņa
starp diviem iemīļotajiem, dalogi bieži projicēti
mītiskā pirmpantā. Mēness --- Saule (piemēram,
III, 467) vai "es" (folkloras kultūrvaroņa
veidols) un mākons, es un vējš, es un saule, es un
mēness.

\begin{quote}
Kur ir ceļš uz tālo laimes zemi?\\
"Es to nezinu, man laimes nava,\\
Man tik ilgas ir, --- bet zinās mākons,\\
Tas tik balts un laimīgs peld pa gaisu."

Mākoni, kur ceļš uz laimes zemi?\\
Tu tik balts un laimīgs peld' pa gaisu.

"Es to nezinu, man laimes nava,\\
MMan tik ilgas ir, kad tālēs tiecos\\
Balts un starojošs, tad es aiz ilgām\\
Gaisā izgaistu, --- bet zinās vēji,\\
Tiem jau visa dzīve ir kā viena deja."

Vēji, kur ir ceļš uz laimes zemi?\\
Jums jau visa dzīve ir kā deja.

"Mēs to nezinām, mums laimes nava,\\
Mums tik ilgas ir, mēs sērās gaužam,\\
Visu zemes virsu apskraidījām,\\
Laimes neradām, --- bet zinās saule,\\
Tā jau visu redz un mirdz no laimes."

Saule, kur ir ceļš uz laimes zemi?\\
Tu jau visu redz' un mirdz' no laimes.

"Es tā nezinu, man laimes nava,\\
Man tik ilgas ir --- kopš gariem mūžiem\\
Visiem gaismu dodu, pate akla\\
Dzenos pirmsaules, --- bet zināsmēness,\\
Tas jau visu redz, ij to, kas tumsā."

Mēness, kur ir ceļš uz laimes zemi?\\
Tu jau visu redz', ij to, kas nakti.

"Es to nezinu, man laimes nava,\\
Man tik ilgas ir pēc tā, kas nava,\\
Tumsā meklēju, kas bij reiz gaismā,\\
Ko nu es, --- bet zinās mana meita,\\
Tā, kas zin visdziļāko, ar dvēsli!"

Mēness, dod man savu mēnessmeitu!

"Ej pats bildināt, kad ir tev drosme!"\\
Ak, man bailes, bailes, reiba galva.\\
{\em (III, 469--470)}
\end{quote}

Dialogs citētajā Raiņa dzejolī "Ceļš uz laimes
zemi" veidots pēc latviešu un cittautu folklorā
atrodamā iniciācijas (šai gadījumā --- kāzu
iniciācijas) tekstu parauga. Proti, kultūrvaronis
(Raiņa dzejolī --- "es") uzdod jautājumu,
virkne uzjautāto (tie, kam kāds pārdabisks spēks
vai zināšanas) mēģina uz to atbildēt.\footnote{Sal.
līdzīgas struktūras pasakas par brāli, kas dodas
meklēt pazudušās māsas. Ceļā viņš satiek veceni
(viens no pārdabisko palīgu variantiem --- te:
labā burve). Viņš tai lūdz, "..vai nevarot
pateikt, kur viņa māsas atrodas. Vecene saka:
"Es tev, brālīt, nevaru pateikt, kur tavas
māsas atrodas, kad tu noiesi pie manas otrās māsas,
tad tā varbūt gan to pateiks."" --- Latviešu tautas
pasakas. --- R., 1956. --- 188.--189.lpp.; parasti
tikai otrā vai visbiežāk trešā māsa zina īsto
atbildi.}
Mēness, kas noslēdz šo pārdabisko atbildētāju virkni,
iesaka jautātājam iet pašam bildināt mēness meitu
(iniciācijā tas nozīmē --- iet pārvarēt visus
ar kāzu iesvētīšanu saistītos pārbaudījumus).
Raiņa dzejolī atšķirībā no folkloras iniciācijas
dialogiem "es" (kultūrvaroņa veidolā) nevis dodas
veikt uzliktos pārbaudījumus, bet modernajam,
pretrunu un šaubu pārņemtajam cilvēkam raksturīgi
atbild: "Ak, man bailes, bailes, reiba galva."

Pievēršanās dialoģiskajām struktūrām (līdzīgi
personifikācijām, paralēlismam, pirmtēliem ---
arhetipiem u.c.) vēlīnajā dzejā Rainim saistīta
ar mītiskās domāšanas aktualizāciju. Protams,
tā nav mītiskā struktūra, kāda tā kādreiz bija
tautasdziesmā vai pasakā (sal. kaut vai dzejoļa
"Ceļš uz laimes zemi" nobeigumu ar iniciācijas
tekstiem folklorā). "Divreiz vienā un tajā pašā
upē nevar iekāpt", bet var saglabāt vai atjaunot
atmiņu par iepriekšējām "iekāpšanas reizēm".
Tā arī Rainim --- divdesmito gadu dzejas neomītiskās
tendences nenozīmē cikliskās domāšanas restaurēšanu
pilnībā, bet atsevišķi tās elementi (formā un
saturā) bagātina dzeju, padara to semantiski
ietilpīgāku.
}

Šeit aplūkotie tropi un figūras nebūt pilnībā
neizsmeļ visu to pārnestās nozīmes vārdu un
izteicienu klāstu, kas ir Raiņa dzejā. Tika
aplūkoti nozīmīgākie no tropiem un figūrām,
mēģinot tos skatīt evolūcijā --- dažādos
dzejnieka daiļrades posmos. Tā acīm redzami,
ka pirmajos dzejoļu krājumos iecienīti
ir tie tropi un figūras, kas palīdz veidot pasaules
divdalījumu (labais --- ļaunais; apspiedēji ---
apspiestie u.tml.), turpretī vēlīnajā dzejā
galvenokārt tie, kas palīdz atklāt
tēlotā daudznozīmību, dziļumu. Sākumā kultivētais
klaji retoriskais stils līdz ar tam raksturīgajiem
tropiem un figūrām pakāpeniski tiek aizstāts ar
neuzkrītoši retorisko --- slēptās zīmēs ---
pirmtēlos, metaforās, elipsēs u.c. --- saturu
izteicošo.

\end{document}

\section{Poētikas terminu vārdnīca}

Te skaidroti
tikai tie termini, kas lietoti grāmatā vai bez kuriem
nebūtu iespējama kādu tekstā sastopamo jēdzienu
pilnīga izpratne. (Tā, piemēram,
grāmatā sastopams termins {\em polimetrija}, bet nav ---
{\em monometrija}. Vārdnīcā skaidroti abi.)
Netiek skaidroti vienkāršākie un praksē visbiežāk
sastopamie versifikācijas un poētikas jēdzieni (pēda,
vārsma, atskaņas, salīdzinājums, epitets, metafora
u.tml., kā arī literārie virzieni.) To skaidrojumu
var atrast jebkurā literatūrzinātnes terminu vai
enciklopēdiska rakstura vārdnīcā.

\vspace{10pt}

{\bf Aleksandrietis} (franču {\em alexandrin})
--- 1) franču dzejā --- divpadsmitzilbju vārsmas
ar blakusatskaņām un cezūru (vai divām cezūrām),
kas dala rindu divās (vai trijās) līdzīgās
daļās. Nosaukums radies no 12.gs. franču poēmas par
Maķedonijas Aleksandru, kas bija sacerēta šajā
pantmērā; 2) latviešu dzejā ar terminu "aleksandrietis"
apzīmē sešpēdu jambā rakstītu dzejoli ar cezūru pēc trešās
pēdas un blakusatskaņām, sal. A.Čakam:

\begin{quote}
Reiz peļķe, bruģa zieds, kā tverta Dieva elpā,\\
Tu viegli pacelsies un būsi visā telpā.\\
{\bf ("Peļķe, bruģa zieds")}
\end{quote}

\vspace{10pt}

{\bf Aliterācija} (latīņu {\em alliteratio} no
{\em ad} 'pie' un {\em littera} 'burts') --- vienādu
līdzskaņu  (parasti vārdu sākumā) atkārtošana
dzejas rindās, piemēram:

\begin{quote}
{\bf V}ēji {\bf v}ilka dūmu {\bf v}īju\\
Pāri meža {\bf v}irsotnītēm,\\
Līdzi mana dziesma gāja,\\
{\bf V}ēja {\bf v}alkā {\bf v}īvinot.\\
{\em (Rainis, II, 194)}
\end{quote}

\vspace{10pt}

{\bf Alkaja I strofa\footnote{Alkaja II strofa latviešu
dzejā sastopama ļoti reti (piemēram, J.Plauža
dzejolī "Elementi", krāj. "putnu ceļš"), tāpēc šeit to
neapskatām.}} --- nosaukta sengrieķu dzejnieka Alkaja
7.--6.gs. pr.Kr.) vārdā. Sastāv no četrām rindām --- pirmās
divas ir vienpadsmitzilbju, trešā --- deviņzilbju,
ceturtā --- desmitzilbju rinda. Metriskā shēma antīkajā
dzejā:

\begin{quote}
v\footnote{Ar "v" antīkajā dzejā tiek apzīmēta
īsa zilbe, ar "-"  --- gara, "/" --- metriski uzsvērta,
bet ar divām vertikālām svītrām "||" --- cezūra.
Tur, kur metriskajā shēmā vienlaikus tiek iezīmēta
kā gara, tā īsa zilbe (te katras rindas, izņemot pēdējo,
sākumā un katras rindas, izņemot trešo, beigās), tas
nozīmē, ka reāli tā varēja vienlīdz būt kā gara,
tā īsa. Šādu garuma ziņā mainīgu zilbi sauca par
ancepsu (lat. {\em anceps} --- te: divdabīgs).} - v - - || - v v - v -\\
v - v - - || - v v - v -\\
v - v - v - v - v\\
- v v - v v - v -v
\end{quote}

Latviešu dzejā Alkaja I strofu tradicionāli atveido šādi:

\begin{quote}
v\footnote{Jaunlaiku Eiropas, arī latviešu dzejā ar "v"
tiek apzīmēta metriski neuzsvērta, "/" --- metriski
uzsvērta zilbe jeb ikts (sk.).
Cezūru, tāpat kā antīkajā dzejā, apzīmē ar "||"} / v / v / v v / v /\\
v / v / v / v v / v /\\
v / v / v / v / v\\
/ v v / v v / v / v
\end{quote}

\begin{quote}
Par daudz tu mini: --- saule un dailes dārzs!\\
Kā turp tu laistos --- ziemels un mēness bālst!\\
Par daudz ir raibi tavi spārni ---\\
Vienkrāsas netaps ne mēness gaismā.\\
(Rainis, III, 43)
\end{quote}

\vspace{10pt}

{\bf Anafora} (grieķu {\em anaphora}) -- viena vai vairāku
vārdu atkārtojums dzejas rindu sākumā:

\begin{quote}
{\bf Jo} palīdzības lūgties dvēsle kaunas;\\
{\bf Jo} nau, kas pats nāk tevi rokās slēgt.\\
{\em (Rainis, III, 131)}
\end{quote}

\vspace{10pt}

{\bf Anakrūza} (grieķu {\em anakrusis}) --- metriski
neuzsvērtās zilbes dzejas rindu sākumā līdz pirmajam
iktam (sk.) resp., līdz pirmajai metriski uzsvērtajai
zilbei. Anakrūza var būt 0 zilbj (trohajā, daktilā),
vienzilbīga (jambā, amfibrahijā), divzilbīga (anapestā).
Tā var būt pastāvīga vai mainīga. Mainīgas anakrūzas
paraugs:

\begin{quote}
No cīņas ar nakti \hspace{10pt} 1\\
Tā sviedru rasa: \hspace{12.5pt} 1\\
To zelta traukā \hspace{18.5pt} 1\\
Saulīte lasa. \hspace{32pt} 0\\
{\em (Rainis, I, 338)}
\end{quote}

\vspace{10pt}

{\bf Asindetons} (grieķu {\em asyndetos}) ---
sintaktiska konstrukcija, kurā vienlīdzīgi teikuma
locekļi vai salikta teikuma daļas tiek saistītas
bez saikļu palīdzības. Asindetons ļoti bieži tiek
lietots tautasdziesmās, lai apzīmētu kopjēdzienus
("tēvs māmiņa" --- vecāki; "priede egle" --- skuju
koki; u.tml.). Asindetons mākslas dzejā palīdz
koncentrēt poētisko domu, ar šo konstrukciju panāk
lielāku izteiksmes spraigumu:

\begin{quote}
Pats cīnies, palīdz, domā, spried un sver,\\
Pats esi kungs, pats laimei durvis ver.\\
{\em (Rainis, I, 28)}
\end{quote}

\vspace{10pt}

{\bf Asklepiāda I strofa} --- nosaukta sengrieķu
dzejnieka Asklepiāda vārdā. Četrrinde, kas
sastāv no četrreiz atkārtotas Asklepiāda vārsmas:

\begin{quote}
- - - v v - || - v v - v -\\
- - - v v - || - v v - v -\\
- - - v v - || - v v - v -\\
- - - v v - || - v v - v -
\end{quote}

Latviešu dzejā tā parasti tiek atveidota šādi:

\begin{quote}
/ v / v v / || / v v / v /\\
/ v / v v / || / v v / v /\\
/ v / v v / || / v v / v /\\
/ v / v v / || / v v / v /
\end{quote}

Visbiežāk garās divpadsmitzilbju rindas grafiski
tiek sadalītas it kā divās (6+6 zilbes), kā,
piemēram, Jānim Plaudim:

\begin{quote}
Sniegos kristāli viz ---\\
\mbox{}\hspace{10pt} tuksnesī smiltis kaist,\\
Jūrās ūdeņi guldz ---\\
\mbox{}\hspace{10pt} debešos tvaiki kāpj.\\
Zvaigznes tālumos mirdz ---\\
\mbox{}\hspace{10pt} tuvumā sēro sirds;\\
Sēro tuvumā sirds;\\
\mbox{}\hspace{10pt} ilgojas tālēs būt.\\
{\em (J. Plaudis. "Cilvēka sirds")}
\end{quote}

\vspace{10pt}

{\bf Asklepiāda II strofa} --- četrrinde, kas sastāv
no trim Asklepiāda divpadsmitzilbju vārsmām (sk.
Asklepiāda I strofu) un vienas t.s.
glikoneja\footnote{Glikonejs --- antīkais pantvmērs,
kas nosaukts grieķu dzejnieka Glikona vārdā:
- - - v v - v -.} rindas:

\begin{quote}
- - - v v - || - v v - v -\\
- - - v v - || - v v - v -\\
- - - v v - || - v v - v -\\
- - - v v - v -
\end{quote}

Ltviešu dzejā šis ir reti lietots Asklepiāda strofas
variants. Tā metriskā shēma:

\begin{tabular}{ccc}
\makecell[l]{
/ v / v v / || / v v / v v\\
/ v / v v / || / v v / v v\\
/ v / v v / || / v v / v v\\
/ v / v v / v v
} & vai & \makecell[l]{
/ v / v v / || / v v / v /\\
/ v / v v / || / v v / v /\\
/ v / v v / || / v v / v /\\
/ v / v v / v /
}
\end{tabular}

Asklepiāda II strofu pārsvarā atrodam tikai
antīkās dzejas tulkojumos, piemēram:

\begin{quote}
Daudziem apraudams, viņš nāvē ir aizgājis,\\
Visu vairāk tak tev, Vergīlij; bijībā\\
Velti dievus tev lūgt, viņi lai atdod tev\\
To, kas aizdots uz laiku bij.\\
{\em (Horācijs, K. Strauberga tulk.)\footnote{Sk.
Horātija dzejas. III. Dziesmas. --- R., 1936. ---
1.--2.grām. --- 38.lpp.}}
\end{quote}

\vspace{10pt}

{\bf Asklepiāda III strofa} --- Sastāv no divām Asklepiāda
divpadsmitzilbju vārsmām, vienas ferekrāta\footnote{Sengrieķu
dzejnieka Ferekrāta vārdā nosaukts pantmērs:
- - - v v -.} un vienas glikoneja rindas:

\begin{quote}
- - - v v - || - v v - v -\\
- - - v v - || - v v - v -\\
- - - v v - -\\
- - - v v - v -
\end{quote}

Latviešu dzejā tā\footnote{Dažreiz Asklepiāda III
stofu mēdz dēvēt par IV, sk., piem., V. Eglīša
dzejoli "Versus Asclepiadeus quartus". --
{\em Eglītis V.} Baltie akmeņi. --- R. 1990. -- 52.lpp.}
ir no visām Asklepiāda strofām biežāk lietotā:

\begin{tabular}{cc}
\makecell[l]{
/ v / v v / || / v v / v /\\
/ v / v v / || / v v / v /\\
/ v / v v / v\\
/ v / v v / v /\\
\mbox{}
} & \makecell[l]{
Katra būtne tik tvīkst --- sākumam galu jaust,\\
Katru vielu tā spēj --- galībā vien tik jēgt,\\
Galu nesniegt ir šausmas, ---\\
Labāk pašai tad galā zust.\\
{\em (Rainis, II, 393)}
}
\end{tabular}

\vspace{10pt}

{\bf Asklepiāda IV strofa} --- pirmā un trešā
rinda --- glikonejs, bet otrā un ceturtā ---
Asklepiāda vārsmas. Latviešu tradīcijā tās metriskā
shēma:

\begin{tabular}{cc}
\makecell[l]{
/ v / v v / v /\\
/ v / v v / || / v v / v /\\
/ v / v v / v\\
/ v / v v / || / v v / v /\\
\mbox{}
} & \makecell[l]{
Mūza, jaunava mīlīgā,\\
Katram dzimstot kādreiz laipna tu smaidīji,\\
Tādam laimes nav dzīvojot,\\
Tikai dziesmas un sirds, kurai vairs miera nav.\\
{\em (V. Eglītis. "Versus Asclepiadeus secundus")\footnote{
V.Eglītis to dēvē par Asklepiāda II strofu.
}}
}
\end{tabular}

\vspace{10pt}

{\bf Asklepiāda V strofa} --- četrrinde, kuras
metriskā shēma visās rindās vienāda --- 16 zilbju ar
divām cezūrām katrā rindā (pēc sestās un desmitās
zilbes):

\begin{quote}
- - - v v - || - v v - || - v v - v -
\end{quote}

Latviski sastopama tulkojumos (/ v / v v / || v v / || / v v / v /):

\begin{quote}
Jautāt nevajag tev, zināt ir grēks, kādu reiz tev un man\\
Galu dievi grib dot; Leukonoj, liegts Bābeles skaitļus tev\\
Saprast. Labāki būs visu, kas lemts, panest un uzņemties,\\
Lai vēl ziemu jo daudz Jupiters mums dāvā vai pēdējo.\\
{\em (Horācijs, K. Strauberga tulk.)\footnote{Horātija
dzejas. --- 29.lpp.}}
\end{quote}

\vspace{10pt}

{\bf Asonanse} (no latīņu {\em assono}) --- vienādu
patskaņu, divskaņu atkārtojums vienas, divu, arī vairāk
rindu vai pat visa dzejoļa ietvaros. Asonanses ļoti
bieži sastopamas tautas dzejā. Mākslas
dzejā tās tiek aktualizētas romantiskā poētiskā tipa tekstos
(tautiskajiem romantiķiem --- Auseklis; simbolistiem,
jaunromantiķiem - V.Plūdonis, Rainis u.c.):

\begin{quote}
Bet t{\bf au}rētājs ar savu t{\bf au}ri\\
L{\bf au}žas viens pats c{\bf au}r sienu c{\bf au}ri.\\
{\em (Rainis, I, 135)}
\end{quote}

Šaurākā nozīmē par asonansi sauc arī daļējās atskaņas,
kurās sakrīt tikai uzsvērtais patskanis, bet nesakrīt
līdzskaņi (zirgs: pirkt; cirks: vilkt u.tml.).

\vspace{10pt}

{\bf Astrofija} --- dzeja bes dalījuma pantos, pretstatā
strofiskai (pantos dalītai) un brīvās strofikas
(dažāda garuma pantu) dzejai:

\begin{quote}
Pēc lietus kluss bija vakars. Savāds mākons,\\
Kā balta roka izstiepta pēc tāles,\\
Mazs gabals varavīksnes pirkstiem pāri\\
Kā gredzens, mirdzošs dārgiem akmeņiem,\\
Tā ilgu laiku roka stiepās tālēs,\\
Līdz bālot zuda gredzens līdzi rokai,\\
Un nakti jūra tumšā segā sedzās, ---\\
Vai tava roka sniedzās man no tāles?\\
{\em (Rainis, III, 549)}
\end{quote}


\vspace{10pt}

{\bf Balāde} (franču {\em ballade}, angļu {\em ballad})
--- sākotnēji deju dziesma. Balādei jaunlaiku Eiropas
dzejā ir divi galvenie avoti un līdz ar to paveidi:

\begin{enumerate}[(1)]
\item t.s. angļu balāde, kuras izcelsme meklējama 14.-16.gs.
tautas dzejā. Šī tipa balāde bija saistīta ar kara cīņu,
varoņdarbu un varoņu tematiku (piemēram, balādes par
Robinu Hudu), bet arī ar cīņu pret pārdabiskiem pretiniekiem,
ar pusmītiskiem --- pusvēsturiskiem varoņiem u.tml.
Sentimentālisma un romantisma dzejā no šī angļu tautas
balādes atzara izveidojās vēsturiskā balāde (V. Skots,
J.V. Gēte, Ā. Mickevičs, M. Ļermontovs; latviešiem ---
J.Alunānam, Auseklim, J. Esenberģim --- galvenokārt
tulkojumos un lokalizējumos; vēlāk ---
V.Plūdonim, Rainim, E. Virzam, E. Ādamsonam, K. Ābelem
u.c.). Šim balādes veidam, ko nosacīti var saukt par
vēsturisko balādi, raksturīgs traģisms,
noslēpumainības un misticisma nokrāsa, formā ---
sižetiskums, nereti --- dramatiski dialogi.
Ne pantforma, ne pantmērs nav stingri noteikts.
Kārlis Ābele, kas pētījis šo balādes paveidu, raksta:
"Balādes būtiska pazīme --- augšējā un apakšējā norise.
Augšējā norise ir nereāla un tinas liktenīgā noslēpumā,
beidzot saskaras ar apakšējo norisi, kas rit reālā, visiem
pārredzamā plaknē. Šī abu norišu saskaršanās ir
arī balādes --- parasti dramatiskais, nereti pat
traģiskais --- atrisinājums."\footnote{{\em Ābele K.}
Vēl par balādi // Ceļš. --- 1948. --- Nr. 4/6. --- 168.lpp.}
Tā kā šī tipa balādes parasti ir samērā apjomīgas, norādīsim
tikai uz dažiem to avotiem (Rainis I, 336; V.Plūdonis
"Jumis --- atriebējs", "Tīreļa noslēpums", u.c.);
\item t.s. franču balāde --- stingrā strofiskā forma,
kas sastāv no trim astoņrindu pantiem ar vienādām
atskaņām (ABAB BCBC)\footnote{Te ar lielajiem sākuma
burtiem apzīmēts tikai atskaņu izkārtojums, ne pats atskaņu
veids (sievišķās, vīrišķās vai daktiliskās). Turpmāk,
arī J.Medeņa dzejolī "Balāde par kamenēm", ar
mazajiem sākuma burtiem (piemēram, a vai b) apzīmētas
vīrišķās, ar lielajiem --- sieviškās (piemēram,
A vai B, vai C u.tml.) atskaņas.}
un noslēdzošas pusstrofas --- četrrindes (BCBC).
Visos pantos noslēguma rinda ir identa, veidojot
sava veida refrēnu. Pantmērs --- visbiežāk
četrpēdu jambs. Franču balādes forma
kļuvusi populāra, pateicoties F.Vijona dzejoļiem.
Latviešu dzejā šajā pantformā rakstījuši V.Strēlerte,
A.Kaugars, bet visvairāk --- J.Medenis:

\begin{tabular}{ll}
Ceļ žiglos spārnus bišu spiets, & a\\
Ko dzemdē zeme, miglā tītā, & B\\
Un smaršīgs vējš to tālē triec & a\\
Pa puķu klajiem Jāņu rītā: & B\\
Nu logā, zāļu kroņiem pītā, & B\\
Kur noliecamies --- tu un es --- & c\\
Pie pirmā rožu zara šķītā, & B\\
Dūc skumji samta kamenes. & c\\[6pt]
Kur zālē zvaigžņu mirdzums liets, & a\\
Tās pavada mūs lēnā svītā; & B\\
Un, kur vien vieglo soli liec --- & a\\
Vai druvā, saules apveltītā, & B\\
Vai birzī lapu ēnām vītā, & B\\
Vai dārzā glaudi jasmīnes, --- & c\\
Kā mākonī, šurp vētras dzītā, & B\\
Dūc skumji samta kamenes. & c\\[6pt]
Kad logā puķes raksta riets & a\\
Bes smaršas --- trauslā sarmas krītā, & B\\
Tu galvu karstās rokās liec --- & a\\
Un jautā sapnī aizvadītā: & B\\
Kur laime staigā, kur gan mīt tā, & B\\
Kas viņas skūpstus garām nes? & c\\
Bet rūtī, uguns apzeltītā, & B\\
Dūc skumji samta kamenes. & c\\[6pt]
Tās piedzimst tavu acu spītā. & B\\
Tu saldās mokās velti dves: & c\\
Vai mūžam būs vai tikai bij tā? & B\\
Dūc skumji samta kamenes. & c\\
{\em (J. Medenis. "Balāda par kamenēm")} & \mbox{}
\end{tabular}
\end{enumerate}



\end{document}
