\documentclass[14pt]{extarticle}
\usepackage{ucs}
\usepackage[utf8x]{inputenc}
\usepackage{changepage}
\usepackage{graphicx}
\usepackage{amsmath}
\usepackage{gensymb}
\usepackage{amssymb}
\usepackage{enumerate}
\usepackage{tabularx}
\usepackage{lipsum}

\setlength{\parskip}{\baselineskip}%
\setlength{\parindent}{0pt}%

\oddsidemargin 0.0in
\evensidemargin 0.0in
\textwidth 6.27in
\headheight 1.0in
\topmargin 0.0in
\headheight 0.0in
\headsep 0.0in
%\textheight 9.69in
\textheight 9.00in

\setlength\parindent{0pt}

\newenvironment{myenv}{\begin{adjustwidth}{0.4in}{0.4in}}{\end{adjustwidth}}
\renewcommand{\abstractname}{Anotācija}
\renewcommand\refname{Atsauces}

\renewcommand{\baselinestretch}{1.2}

\newenvironment{uzdevums}[1][\unskip]{%
\vspace{3mm}
\noindent
\textbf{#1:}
\noindent}
{}

\newcommand{\subf}[2]{%
  {\small\begin{tabular}[t]{@{}c@{}}
  #1\\#2
  \end{tabular}}%
}



\newcounter{alphnum}
\newenvironment{alphlist}{\begin{list}{(\Alph{alphnum})}{\usecounter{alphnum}\setlength{\leftmargin}{2.5em}} \rm}{\end{list}}


\makeatletter
\let\saved@bibitem\@bibitem
\makeatother

\usepackage{bibentry}
%\usepackage{hyperref}


\begin{document}

\begin{center}
{\LARGE \bf Tālu ceļu vējš}
\end{center}

{\normalsize

\section{Tā puksti sirds}

\begin{quote}
{\em
Vienreiz gadā atnāk brīži tādi,\\
Bet lai cauri dzīvei dzirdam to ---\\
Ledus lūst ar dobju kanonādi,\\
Palu straumes spēku atzīstot!
}
\end{quote}

{\large \sc Dzimtenei}

Kaut arī jūlijā\\
Tāpat reiba no birztalu tvana ---\\
Tās bija citādas vasaras,\\
Ausīs vēl tagad skan\\
Dziesma, ko dziedāja sanošās vārpas,\\
Kuras nebrieda man.\\
Taču jau toreiz tu biji\\
Dzimtene mana...\\
Un tomēr nebiji mana!

Pa dzimto zemi ejot,\\
Visur man kliedza: --- Nav tava! ---\\
Un pat robežas smilgainā grāvī\\
Nebija zemes man.\\
Nebija dzimtās zemes.\\
Vectēvs to atrada gan ---\\
Jo, kad to nesa\\
Zem kapsētas kļavām,\\
Neviens neteica: --- Tā nav tava. ---

Un bez dzimtenes\\
Izdzisa dzīve viena ---\\
Saujiņai bija,\\
Bet tūkstošiem nebija Latvijas.\\
... Un tad\\
Sarkans, ugunīgs karogs\\
Pār Daugavu atvijās,\\
Pirmo reizi kā dzimtajā\\
Varēja dzimtajā birzī ienākt\\
Tavā pirmajā dzimšanas dienā.

... Atkal ir jūlijs\\
Un var noreibt no birztalu tvana,\\
No dzimto birztalu tvana,\\
No dzimtenes vārpām, kas san,\\
Un ikviens no mums zin ---\\
Tagad tās nobriedīs man.

Nepatīk bagātiem kungiem\\
Karoga krāsa tava,\\
Vairākkārt viņi pēc tevis\\
Izstiepa viņi pēc tevis\\
Izstiepa pirkstus badīgos.

Blakus ar piecpadsmit māsām\\
Cauri gadiem ej\\
Piecpadsmit-, divdesmit-, piecdesmitgadīga,\\
Un, tāpat kā zem piecpadsmit karogiem,\\
Arī zem karoga tava\\
Ies slava!

\newpage

{\large \sc Maskava, māt!}

Ar tevi būt mēs ļoti sen jau alkām,\\
Nu ejam kopā darba slavu krāt,\\
Nu nākam mēs ar Dzintarjūras šalkām\\
Pie tevis, Maskava, pie tevis, māt!

Tev ielas dzied kā kalnu upes ašas\\
Par miera zemes miera dzīvi šo,\\
Un dziesmas pat uz visas zemes plašās\\
Nav lielākas un skaistākas par to.

Mums daudz bij gadu simtos cirstu rētu,\\
Bet atkal stipri nākam blakus stāt ---\\
Nav tāda spēka, kas mūs uzvarētu\\
Ar tevi, Maskava, ar tevi, māt!

Ik diena mūs pret jaunām tālēm aiznes,\\
Lai kādi bargi vēji pretī elš, ---\\
Mums spēka daudz. Mūs vada tavas zvaitznes,\\
Un, kur tās ved, tur nepazudīs ceļš.

Šķir plašums mūsu upju viļņu vālus,\\
Bet mēs ar sirdi pieaugām tev klāt ---\\
Nav tādas vietas, kur meš būtu tālu\\
No tevis, Maskava, no tevis, māt!

\newpage

{\large \sc Vecais ostinieks sagaida kuģi}

Nāk pie krasta viļņi sudrabaini,\\
Stājas tvaikonis, --- No kurienes tu tāds? ...\\
Atkal mūsu pusē, ``Jāni Raini''! ---\\
Celtņa roku sniedzu: --- Sveicināts!

TIklīdz ļaujos atmiņām un domām ---\\
Augsta piere, sirmu matu sniegs...\\
Liekas, nevis kuģi šūpo joma, \\
Bet mēs sastopamies, dziesminiek.

Spožām acīm tevī raugās nami,\\
Viļņos spuldžu atspīdumi dziest.\\
Reiz kā draugi bijām pazīstami\\
Piektā gada brāzmās, atminies.

Pagrabos bij saslodzīti vēji,\\
Stāvošs ūdens kļuva aļģēm segts,\\
Tad tu pacēlies un vētru sēji,\\
Cēlies sirdīs zvaigžņu liesmas degt.

Tos, kam krūtīs šaubu aukstums bija,\\
Dzeju lapas pusēs slēptais rīts\\
Dzīvinošiem stariem sasildīja,\\
Palu straumē rāva citiem līdz. ---

... Parka kokos putni daino klusi,\\
Mēness maurā ēnu rakstu auž,\\
Nakts jau lēnām iet uz rīta pusi,\\
Pirmā vēsma kuģa mastu glauž.

Visas zīmes rāda labu laiku,\\
Tātad rītu ceļš tev tālāk ies.\\
Labu ceļa vēju! Pilnu tvaiku!\\
Komunisma ostā tiksimies!

\newpage

{\large \sc Varoņu kapos}

Šeit nav nāves. Nav un nevar būt,\\
Jo, kad dzīvē kādreiz pagurt sākam,\\
Šurp ar viņiem parunāties nākam,\\
Nevis raudāt --- nākam stipri kļūt.

Citā vietā sacīsim varbūt:\\
Viņi krita. Kauja bij par sīvu ---\\
Šeit mēs tiekamies kā dzīvs ar dzīvu.\\
Šeit nav nāves. Nav un nevar būt.

\newpage

{\large \sc Pēdējais raunds}

Pēdējais raunds. Tas būs asāks kā citi,\\
Rokās milzīgu spēku gribas dzirkstele kurs.\\
...Melnais pasaules čempions Bertlings Siti\\
Tuvu uzvarai. Baltais pretinieks gurst.

Pēdējais raunds. Stāv pie bulvāra stūra\\
Sprogains puisītis. Namam grib tuvāk kļūt,\\
Gaida tēti, bet priekšā --- publikas jūra.\\
Siti uzvar, un... vai tas drīkst būt?

--- Linčot! --- kāds izbrēc. Saucienu atkārto citi,\\
Naidā sašķiebtām sejām ringā slepkavas skrien.\\
Melnais pasaules čempions Bertlings Siti\\
Pēdējā raundā stājas pret simtiem --- viens...

Bija cilvēks... Nē, misteriem tas nebūs lēti ---\\
Sprogainais puisītis izaugs, pieņemsies spēkā un tad\\
Lielās taisnības dienā slepkavām prasīs par tēti\\
Un par visu, ko nevar piedot nekad!

\newpage

{\large \sc No pasaran\footnote{Viņi neizlauzīsies!}}

Ložu celtos putekļos vērpies\\
Lielceļš, kur tagad nebraukā rati, ---\\
Pa to viņš atnāca, vienkārši tērpies,\\
Vienkāršs kā Spānija pati.

Kurš gan domāja toreiz par slavu,\\
Gājām kopā, vai kalns vai dzelme.\\
Rītā frankisti aplenca gravu ---\\
Tīrā uguns un svelme!

Gaisā frankistu granātas kauciens,\\
Lūpās sāji un lipīgi sviedri,\\
Atkal, atkal pār tranšeju sauciens:\\
--- Tomēr {\em no pasaran}, biedri! ---

Zeme drebēja. Šķembu virmā\\
Drīz vien mēs nokvēpām melni.\\
Tad no putekļu mākoņa sirmā\\
Iznira priekšā šie velni.

Viņš, kam bij jaunības skaņa vārdos,\\
Bet sarmā mirdzēja mati,\\
Pāri tranšejai kaujas dārdos\\
Cēlās kā Spānija pati.

Viņu pazina. Un nav brīnums ---\\
Varoņus pazīst karā.\\
Viņu sauca, un tas ar smīnu\\
Iegāja frankistu barā.

--- Gribat ar mani kopā staigāt?\\
Iesim! --- nošalca spontāns.\\
...Tur, kur bij viņi, izplauka baigā\\
Granātu sprādziena fontāns.

Tā nav pasaka, teika gan ---\\
Kopš tās dienas, kāds stāstīja man,\\
Visos Spānijas vējos skan:\\
--- {\em No pasaran!}

\newpage

{\large \sc Sirds meklē dziesmu}

Zvaigžņu miglājā tālums tinas,\\
Tādā miglājā staru daudz,\\
Bet vai zvaigžņu pulku jūs zināt,\\
Ko par Kahovkas zvaigznāju sauc?

Grūti zināt, vēl grūtāk sacīt,\\
Zvaigžņu atlantu velti šķirt,\\
Bet es redzēju savām acīm,\\
Ka tāds zvaigznājs patiešām ir!

Vējš tik nemierīgs noglauž āri,\\
Lietus sadomā zemi pērt,\\
Bet tur tālē --- Kahovkai pāri ---\\
Zvaigžņu piebārstīts plašums vērts.

Mašīna tuvāk blāzmai aiznes,\\
Prasu šoferim: --- Kā jums šķiet, \\
Nav taču zināmas tādas zvaigznes,\\
Kas caur negaisa mākoņiem zied? ---

Vēlāk kauns, kad pie krasta kraujas\\
Visu varēja saskatīt ---\\
Zilas raķetes kā pirms kaujas\\
Mierīgā Dņepras dzelmē krīt.

Un sirds meklē man tādas dziesmas,\\
Kurās apdziedāt ļaudis šos,\\
Autogēnu zilganās liesmas\\
Augsto sastatņu debešos.

Īstāko vārdu atrast grūti,\\
Ilgu laiku tas meklēts jau,\\
Stāsti sev, kad gurumu jūti:\\
Klausies, viņiem tur vieglāk nav.

...Atkal Kahovku dunam dzirdi,\\
Redzi, kā sastatnēs zvaigznes mirdz.\\
Viņi darbā ielika sirdi,\\
Tā savā dziesmā man jāieliek sirds!

\newpage

{\large \sc Kaklautu noraisot}

Kā lai noņem to, kas pieaudzis pie sirds?\\
Nevar noņemt. Kļūst uz mirkli drūmi,\\
Jo aiz loga tāle saucot mirdz\\
Un kūp pionieru ugunskuru dūmi.\\
Kā lai noņem to, kas pieaudzis pie sirds.

Visi tālie ceļi, kas šais gados bij,\\
Viss, ko ugunīgais kaklauts devis,\\
Lai sirds kļūtu rada ugunij, ---\\
Šodien viss sauc atpakaļ pie sevis,\\
Visi tālie ceļi, kas šais gados bij.

Mezgls atraisīts. Deg ugunīgais zīds,\\
Pašreiz jūti visu to, kas noiets.\\
Bet vēl nejūti, cik liels šis rīts\\
Un ka savu bērnību tu noliec ---\\
Mezgls atraisīts. Deg ugunīgais zīds.

Nevar noņemt to, kas pieaudzis pie sirds,\\
Un lai paliek ugunskura liesma ---\\
Tajā lai tev acu skatiens mirdz\\
Jauns un nemierīgs kā pionieru dziesma.\\
Nevar noņemt to, kas pieaudzis pie sirds.

Tālāk --- jaunība un ceļš pret kalnu būs,\\
Un ne vienmēr tur būs gluda iela.\\
Šodien tu kā strauts, kas šalcot plūst,\\
Ieej straumē varenā un lielā ---\\
Tālāk --- jaunība un ceļš pret kalnu būs!

\newpage

{\large \sc Atvadu dziesma skolai}

{\em Draugiem no Cēsu 1. vidusskolas}

Šodien šķiršanos jūt,\\
Tāpēc dziesma varbūt\\
Arī neiznāk līksma, kā vajag.\\
Smieties negribas mums,\\
Varbūt labāk, ja skumst\\
Mūsu šķiršanās minūtē šajā.

Kā to vārdā lai sauc,\\
Laikam mīļuma daudz\\
Mums šais gados pret tevi ir krājies,\\
Un tik aicinošs skan\\
Tavu gaiteņu zvans,\\
Taču mums vairs uz klasi nav jāiet...

Tavas atmiņas --- būs!\\
Tās nav sniegi, kas kūst\\
Un uz jūru ko aizskalo pali, ---\\
Viss, ko devi mums līdz,\\
Šodien krūtīs mums trīc,\\
Un uz mūžu tas krūtīs paliks.

Celsies ziemelis ass,\\
Taču mums vairs nemaz\\
Nav no vētrām un puteņiem bailes:\\
Vēji brāzmaini trauc,\\
Un caur tālumu sauc\\
Kalnu augstās un sniegotās smailes.

Mūsu ceļā lai rīt\\
Dārdot lavīnas krīt,\\
Trako vēji un kupenas sasnieg,\\
Bet uz pasaules jau\\
Tādu virsotņu nav,\\
Kuras drosmīgie nespētu sasniegt.

...Šeit mēs atnācām sen,\\
Vēji lapas kad dzen,\\
Šeit mēs sapņus par nākotni vijām,\\
Taču sirdis varbūt\\
Tikai šķiroties jūt\\
To, cik dārgi viens otram mēs bijām.


\newpage

{\large \sc Pūš tālu ceļu vējš}

Ir mūsu mašīnai pat logu stikli silti,\\
Šķiet, ka tos kausē staru kūlis spējš.\\
Pār mežiem dūmaka. Kūp sakarsušās smiltis,\\
Un sejā mums pūš tālu ceļu vējš.

Ceļš kalnā uzvijas. Visapkārt bālē\\
Zils mežu loks, un kaut kur ezers mirdz.\\
Tā, skatam klīstot dūmakainā tālē,\\
Sāk krūtīs nemierīgu dziesmu sirds.

Zils tālumi, kur saule peld līdz rietam,\\
Bet atnāks laiks un pavērs skatam tos ---\\
Tie mazie cilvēki, kas vēl pie spēļu lietām,\\
Reiz raķetē uz zvaigznēm aizlidos!

... Pār mežiem dūmaka. Tik karsts, ka elpot grūti,\\
Mēs braucam, klausoties, kā mežā dzenis kaļ.\\
Iet blakus puisēns kaklautu uz krūtīm\\
Un katru brīdi skatās atpakaļ.

Mēs gribējām to aizvizināt mājās,\\
Viņš apmulsa un piesarka kā zieds:\\
Viņš --- skolu pabeidzis. Viņš grib iet kājām\\
Pa ceļu to, kur daudzus gadus iets.

Šis ceļš ir kļuvis ļoti mīļš un ierasts,\\
Bet... skola pabeigta, un mulsums domas jauc ---\\
Bet tā jau ir, kad beidzas kaut kas pierasts\\
Un kaut kas nezināms uz priekšu sauc.

... Un mums tāds svinīgs satraukums bij radies,\\
LKā pašus vajadzētu tālam ceļam post.\\
Bet zēns... var būt, ka tieši viņš pēc gadiem\\
Reiz raķetē uz zvaigznēm aizlidos!

Man droši vien jau matos stīdzēs salna,\\
Bet tieši tad es atcerēšos spējš,\\
Kā, plandot kaklautam, zēns stāv uz kalna\\
Un sejā tam pūš tālu ceļu vējš.


\newpage

{\large \sc Gaujienas vidusskolai}

Kaut grūti būs iespraukties šaurajā solā\\
Un apsēsties pagrūti būs,\\
Bet saņem vēl reizi mūs, mīļotā skola,\\
Kā toreiz tu saņēmi mūs!

Mēs klāt jau. Birst lapas no ceļmalas kļavām,\\
Tās paceļas vējos kā spiets,\\
Un meitene maza ar grāmatām savām\\
Caur dzeltenu puteni iet.

Tāds pats bija rudens ar skarbumu balsī,\\
Kad saviļņots ienācu es...\\
Lai šodien mūs visus tā vakara valsis\\
Pār Donavas plašumiem nes.

Tad atnāks tās domas, tie sapņi, tās ilgas,\\
Kas visus šos gaiteņus vij,\\
{at atnāks uz brīdi tās debesis zilgās,\\
Kas toreiz pār gaujienu bij.

Drīz jābrauc. Riets izkāris sarkanu jostu,\\
Uz sliedēm stāv vilciens un elš,\\
Bet, tāpat kā kuģis mīl iegriezties ostā,\\
No kuras tam iesācies ceļš, ---

Mēs atnāksim, atkal būs dzeltenas kļavas,\\
Būs rudens, mēs atnāksim tad ---\\
No tevis, tāpat kā no bērnības savas,\\
Mēs nevaram aiziet nekad!

\newpage

{\large \sc Dēls izaudzis}

Ir salda smarža liepu ziedu tējai\\
Tāpat kā agrāk. Ceriņziedu skaras,\\
Tāpat kā agrāk, atdod smaržu vējam,\\
Bet tu pie galda jau par visiem garāks.

Jūs runājat par to, ka tēvi sirmo,\\
Par mērķiem, ko sev krūtīs jaunie glabā,\\
Un savā dzīvē saki reizi pirmo\\
Tos vārdus: --- Tēvs, ne tā, bet tā būs labāk. ---

Tāpat kā agrāk, smaržo lauku tēja,\\
Tāpat kā agrāk, jūti saldo tvaiku,\\
Tēvs ilgi skatās tavā jaunā sejā:\\
--- Pa kuru laiku, dēls, pa kuru laiku? ... ---

Jo tu ar galvu gandrīz griestos tiecies,\\
Par visiem garāks daudz, ja blakus stājas,\\
Bet tēvam vēl tu tāds pats maziņš liecies,\\
Kāds reiz zem loga smiltīs rotaļājies.

Ka tev ir darbs, ar to viņš sāk jau aprast,\\
Pie vilciena teic: --- Dēls, lai labi sokas! ---\\
Bet pie tam ilgi domā, nevar saprast,\\
Kā dzīvē iztiec tu bez viņa rokas.

Vai tiešām vecums? Nē --- kūp zeme tvaikā,\\
Un bango rudzu zaļums vējos brīvos.\\
Nē, vecums nedrīkst būt, jo tādā laikā\\
Uz tādas zemes gribas ilgi dzīvot.

Nes tevi vilciens prom, met saule karstu kvēli,\\
Tu domā par šo zemi, ziediem sēto,\\
Uz kuras tēviem izaug stipri dēli\\
Un dzīvē paveic tēvu iecerēto.

\newpage

{\large \sc Lasot romānu "Pret kalnu"}

Skan kuranti, un diena maina dienu,\\
Dun sliedēs tramvajs, varbūt pēdējais,\\
Un gaišu logu neredz vairs nevienu.\\
... Man apkārt atkal Lejasciema gaiss.

Daudz esmu gājis lauku ceļos tajos,\\
Un tāpēc viss, kas lapas pusēs mirdz,\\
Ir kādreiz pārdzīvots, ir kādreiz sajusts\\
Un kļuvis daļiņa no paša sirds.

Bet reizēm liekas --- te par mani sacīts,\\
Tad skatiens ātrāk burtu rindās skrej.\\
Tai ugunī, kas kvēlo Mirdzai acīs,\\
Kvēl skatiens arī manai meitenei.

Var būt, ka viņa kalnupceļu grūto\\
No Mirdzas mācījas, jo tas spēj vest\\
Pie uzvarām, vien katru brīdi jūtot\\
Sev priekšā nesasniegtās virsotnes.

Ir jau tik vēls. Nakts melna skatās logā,\\
Bet cauri naktij tagad skatos es\\
Uz kalnu, kurā rīt ar stiprām rokām\\
Ir mūsu laimes karogs jāuznes.

\newpage

{\large \sc Briesmas garām}

Kad baltā palātā jūt ziedu dvesmu,\\
Nāk apskate, kāds puisis klusi min:\\
--- Es, biedri ārst, tik vesels šodien esmu... ---\\
Un pasmaida, jo ārsts jau labāk zin.

Aiz loga --- maija padebeši zilgi,\\
Uz pieres karsto saules skūpstu jūt.\\
Trīs mēneši --- tas ir bezgala ilgi,\\
Un darba ierindā tā gribas būt!

Balts celiņš. Tulpes sarkanas kā liesmas.\\
Viņš iziet, acīs atkal saule mirdz,\\
Jo slimību un tajā slēptās briesmas\\
Ir uzveikusi viņa jaunā sirds.

Un kas gan izbēgs jaunā acu skatam:\\
Gan upīte, kur dzelmes sudrabs blāv,\\
Gan... Baltā halātā vīrs sirmiem matiem,\\
Ar acīm aizejošo pavadīdams stāv.

Skats aizslīd jaunā vingro soļu pēdās,\\
Un sajūt ārsts, ka viņš nav vecs un lieks.\\
Kāpj acīs asaras. No lielām bēdām?\\
Nē, tas ir milzīgs, neizsakāms prieks ---

Par to, ka slimais baltā ceļa oļos\\
Tik droši atkal savus soļus liek,\\
Par to, ka jaunību jūt viņa soļos,\\
Par to, ka viņam laime jāsatiek.

Par to, ka tulpes uzkvēlo kā liesmas,\\
Ka ķirsis liecas ziedu sniegā balsts\\
Tam cilvēkam, kam lielas, lielas briesmas\\
Aiz muguras.\\
Uz priekšu --- dzīve šalc.

\newpage
{\large \sc Mums jau divdesmit}

Cik ātri gadus veido mūsu dienas!\\
Jau, skaties, divdesmit --- uz vaiga asa bārda,\\
Un tie, kam tikai devītais gads pienāks,\\
Jau tevi sveicina, tic katram tavam vārdam. \\
Vēl sprāga kaut kur paslēpušās mīnas,\\
Kad mātes viņus sildīja pie krūtīm,\\
Vēl kaut kur augstu gāja lidmašīnas,\\
Bet nakts jau bija ar gaišām logu rūtīm.\\
Un, kad tie pirmos bērna soļus spēra\\
Un bērna acīm dzīvē ieskatījās, ---\\
Pār viņiem mierīgs debess zilums vērās ---\\
Nekāda kara viņiem nebij bijis...\\
Strauts šalkdams lauzās cauri cietam iezim,\\
No zemes spēku dzēra zaļie kvieši,\\
Un dabai taisnība! --- Kad cilvēks piedzimst,\\
Tai jābūt skaistai. Jābūt tādai tieši.\\
... Es piedzimu uz kartupeļu lauka,\\
Tu --- varbūt rudzu tīrumā vai rijā.\\
Tur nebij bērnības, vien barga auka,\\
Un tajā mēs jau pieauguši bijām,\\
Bij vismaz jābūt tādiem --- rīta rasā\\
Jau agri nācās nogurumu pazīt,\\
Just katrā šūnā sāpes asas, asas,\\
Mums nebij laika toreiz palikt maziem...\\
Mums, kas stāv dzīvē jau uz abām kājām,\\
Mums, kuriem katru kalnu apgāzt viegli, ---\\
Ir grūti reizēm paiet garām mājām,\\
Kur atskan gaiši pionieru smiekli.\\
Kad ugunskura dūmi stīdz pār pļavu\\
Vai pionieru taka kalnos vijas,\\
Mēs domājam par bērnību par savu,\\
To, kuru vecā dzīve nolaupīja.

\newpage
{\large \sc Četrās sienās}

Sētā pikodamies ņemas bērni,\\
Žilbst aiz loga zeme ziemas sniegā,\\
Bet pie tevis viss kā grīšļi pērnie ---\\
Nogurdinošs, apnicīgs un miegains.\\
--- Šodien jaunu, labu filmu rāda,\\
Aiziesim! ---\\
Tu tikai ņurdi klusi:\\
--- Redzi, cik viss nejēdzīgi gadās ---\\
Mašīnas nav, sieva aizbraukusi. ---

Mīkstā krēslā atlaidies, kā ronis\\
Šauras dzīves saulgozītē peries.\\
Draugiem zvani tu pa telefonu,\\
Adreses nevienam neatceries.\\
Istabā tev nejūt vēju brīvo,\\
Nejūt dzīves nemieru kā senāk ---\\
Gribas aizmirst ielu, kur tu dzīvo,\\
Gribas aiziet un nekad vairs nenākt.\\
Eju projām.\\
Niknums mani urda,\\
Un uz ielas, jūtot sala svelmi,\\
Es kā zivs, kas izbēgot no murda,\\
Pēkšņi atkal atgūst savu dzelmi,\\
Skatos Rīgas putekļainā dienā,\\
Sildos Rīgas februāra salā,\\
Un pēc tādas elles četrās sienās\\
Ir uz ielas labāk galu galā.

\newpage
{\large \sc * * *}

Eju ielā ziemas naktī kādā,\\
Mūsu pašu mīļās Rīgas ielā ---\\
Logu simti deg ---\\
Tur ļaudis strādā\\
Aizrautībā lielā, steigā lielā.\\
Nakts.\\
Bet tur jau gatavojas rītam,\\
Tikai dažos logos melno klusums,\\
It kā viņi rītu negaidītu,\\
It kā lācis tur uz ziemas dusu\\
Ierīkojies. Bet rīt dzīve skaļā\\
No šīm dažām alām nenobīsies,\\
Atraus durvis, atraus logus vaļā,\\
Miega pārņemtajās sienās ielauzīsies.

Tā ir mūsu zemes vēju brīve,\\
Un tās skarbumā nav nekā neparasta,\\
Jo no daudzu, daudzu ļaužu dzīvēm\\
Mūsu lielās zemes dzīve sastāv.\\
Un lai nemiers kvēlotu ikvienā\\
Tajā mūsu lielās zemes šūnā,\\
Lai mums neļauj puteņainā dienā\\
Apgulties uz miegainības dūnām,\\
Lai tev istabā jūt vēju brīvo,\\
Lai jūt dzīves nemieru kā senāk,\\
Draugs lai nevar aizmirst\\
Ielu, kur tu dzīvo,\\
Atnācis lai otrreiz\\
Nevar nenākt!

\newpage

{\large \sc Ledus lūst}

... Atspīd Gaujā padebeši zili,\\
Dzelmē palojošā redzu es:\\
Upe šturmē ledus "Ziemas pili",\\
Sagrauj to un drupas lejup nes.

Ledus lūst ar dobju kanonādi,\\
Palu straumes spēku atzīstot,\\
Vienreiz gadā atnāk brīži tādi,\\
Atnāk domām palu spēku dot.

Lai cik liels ir pavasaris dabā,\\
Tas zūd, rudens salnām atnākot, ---\\
Dzīves pavasari sirdī glabā,\\
Un nekādas salnas neskars to!

Un ikreiz, kad krasta koku zari\\
Liecas palu dzelmes mutuļos,\\
Domāju par citu pavasari,\\
Citā ledus dunā ieklausos.

Dzirdu straumes izlaužamies brīvē,\\
Petrogradas ielās palojot.\\
Ir! Ir pavasaris arī dzīvē ---\\
"Aurora" ar zalvi sāka to!

VIenreiz gadā atnāk brīži tādi,\\
Bet lai cauri dzīvei dzirdam to:\\
Ledus lūst ar dobju kanonādi,\\
Palu straumes spēku atzīstot!


\newpage

{\large \sc Pavasarī}

Šķīst sniegs uz palodzes, un ārā līst,\\
Skrien strumes duļķainas,\\
Un dobji ledus plīst.\\
Šķīst sniegs uz palodzes, un ārā līst.\\
Kā šodien visa daba sniegu nīst!\\
Kaut īsto pavasara sauli\\
Tā vēl nepazīst ---\\
Šķīst sniegs uz palodzes un ārā līst.


\newpage

{\large \sc Maijā}

No savas sirds tu šorīt dziesmu ņemi,\\
Un tā lai ceļas kvēlojošai blāzmai līdz ---\\
Mēs šorīt ieraudzīsim tādu zemi,\\
Kam tūkstoš ziedu acīs asaras\\
No rīta laimes trīc.

Mēs šorīt ieraudzīsim dzimto zemi,\\
Kas maija dzīvības un plauksmes ilgās tvīkst, ---\\
Tiem tādas zemes nav, kas viņu nemīl\\
No karoga līdz smilgai pēdējai,\\
Kas ceļa malā līkst.


\newpage
{\large \sc Rudens rītā}

Tikko kļūst zilgana tumsa aklā,\\
Tiklīdz var izšķirt jau dažu krāsu,\\
Sāk zem kājām kā brīnums lāsot\\
Rudens brīnišķo lapu paklājs.

ARvienu skaistāks paklājs kļūst\\
Tās dienas staros, kas būs.

Paglaudi to, un, lapām švīkstot,\\
Rudens zelts tev pie rokām skaras,\\
Nu, bet pasakiet --- vai mēs varam\\
Līdzi paņemt šo paklāju mīksto?

Lai tad, ja kādreiz smagi kļūs,\\
Šis skaistums dara stiprus mūs!

Nevar to paņemt līdzi ceļā\\
Un pat saglabāt nevar ilgāk.\\
SKaties --- kļūst pelēka tāle zilgā ---\\
Rudens milzīgie vēji ceļas.

Jau paklājs pirmā brāzmā trīc.\\
Nē, viņu nevar paņemt līdz!

Brāzmainais vējš pa mežu ārdās,\\
Dzeltenās lapas noplēš saujām.\\
Koki sten. Tā ir varena kauja ---\\
Un ne nāves, bet dzīvības vārdā.

SImt pumpuru slēpj katrs zars,\\
Un viņiem pieder pavasars.

Katrs nākošais vēju cirtiens\\
Kokus ņem ciešāk savā varā ---\\
Jo no spēka pārpilnā zara\\
Vecām lapām ir grūti šķirties.

Krāc vējš, lūst zari, trako viss,\\
Šķīst rudens lapu ugunis.


\newpage

{\large \sc Tā puksti sirds}

Uz zemes\\
Vēl mūsu paaudzes nebij tad,\\
Kad sērās atsedza galvu\\
Jaunie un vecie.\\
Mūsu paaudze Ļeņinu\\
Nesatika nekad,\\
Bet šodien tā pārņem\\
Dzīvi uz saviem pleciem.

Un kā straume\\
Šī dzīve\\
Pret nākotni skries,\\
Glabājot aprīļa vēju\\
Sava karoga krokās.\\
Ieročus,\\
Ar ko par nākotni cīnīties,\\
Dzimtene, liec\\
Mūsu jaunajās rokās!

Liec rokās traktora stūri,\\
Un uzplauks viskrāšņākais zieds\\
Zemē, kas pirmoreiz tiekas\\
Ar cilvēka skatu,\\
Liec rokās veseri,\\
Otu vai spalvu liec\\
Un, ja vajadzēs, ---\\
Liec automātu!

Tāpat kā maijam\\
Šodienas tērces mirdz,\\
Lai skaistai, briedīgai vasarai\\
Kūp mūsu pirmā vaga.\\
Tā puksti, sirds,\\
Kā pukstētu Ļeņina sirds,\\
Ja tavā vietā\\
Pa dzīvi viņš soļotu tagad!




\newpage

\section{Dziesma}

\begin{quote}
{\em
Nāc, kad riets ar sārtu liesmu\\
\mbox{}\hspace{10pt} Tālē zied,\\
Klausīties, cik skaistu dziesmu\\
\mbox{}\hspace{10pt} Ezers dzied.

Dziesmu tad no sirdīm abām\\
\mbox{}\hspace{10pt} sāksim mēs,\\
Ezers gribēs dziedāt labāk ---\\
\mbox{}\hspace{10pt} Nevarēs!
}
\end{quote}


{\large \sc Kolhozā}

Kā mājās izkārtoju visas mūsu lietas,\\
Pat radio, ko veda līdzi draugs,\\
Bet jutu --- sirds nav tomēr savā vietā\\
Un arī paša drauga nav un nav.\\
Tad durvis pēkšņi atveras ar dārdu!\\
Viņš nosvīdis un satraukts iejoņo,\\
Tā aizelsies, ka tikai divus vārdus\\
Pār lūpām izdabūt var: --- Zini ko... ---

... Neviens tai naktī ciemā negulēja,\\
Bij palu briesmās milzīgs rudzu lauks,\\
Tai naktī atkusušās zemes vējus\\
Pa īstam abi ieelpojām, draugs,\\
Un sajutām, kā ir, kad nomet svārkus,\\
Bet karstums tomēr smagi dvašot liek.\\
No slapjām drēbēm pārvērtās par mārku\\
Mums istaba, No kājām gāza miegs.

... Mūs no šī pirmā lauku miega cietā\\
Jau vesels mēnesis, pilns steigas, šķir;\\
Mums istabā ir daudzas jaukas lietas\\
Un pavasara ziedi vāzē ir.\\
Ir kaut kā aizmirsušās Rīgas ielas\\
Un labāk patīk lauku lielceļš mīksts,\\
Bet... draugam dzīvē sācies kaut kas lielāks,\\
Un nezinu, vai par to runāt drīkst.\\
Ir tieši tāpēc ziedi mūsu mājās ---\\
Es tikai vakar uzzināju to,\\
Kad, vēlu pārnācis, viņš mani modināja\\
Un, acīm mirdzot, teica: --- Zini ko!...



\newpage

{\large \sc Pavasaris oktobrī}

Pavisam drūms un domīgs nesen biji\\
Kā spurains ezis --- gatavs katram durt;\\
Kas būtu pēkšņi visu pārmainījis\\
Un spējis tevī tādu liesmu kurt?

Jo tagad tu ar labu smaidu vari\\
Pat līdz šim nelasītās dzejas šķirt\\
Un dungo klusi: --- Dārzā pavasaris... ---\\
Bet dārzā taču zeltīts rudens ir!

Pavisam savāds tu pie virpas stājies,\\
Un zilgais tērauds, dzirkstis metot, skrien,\\
Mēs tikāmies, no darba ejot mājās,\\
Pie bulvāra. Tu nebiji tur viens.

Es tagad zinu, kāpēc virpa dziesmo\\
Un ko tu pats tik dziļi krūtīs nes,\\
Es zinu, kāpēc tavas acis liesmo\\
Kā virpas šķiltās, spožās dzirksteles.

Bet bulvārī vien lapas krīt no zara\\
Un rotaļājas rudens saules stars ---\\
Es zinu, draugs, pēc kāda kalendāra\\
Tev oktobrī ir sācies pavasars.

\newpage

{\large \sc Pie Juglas}

Rimst vēji tumši zaļās egļu burās,\\
Nāk Juglas viļņi krastā atpūsties,\\
Kvēl mākoņogles rieta ugunskurā\\
Un pamazām virs tāliem mežiem dziest.

Un, tikko nodzisusi rieta liesma,\\
Plaukst upes krastā daudzas ugunis,\\
Pavisam nedroši sāk skanēt dziesma,\\
Un drīz tai piebalso viss apvārsnis.

Ar jautru dziesmu vienmēr jābūt radiem,\\
Bet es, pie Juglas nācis, kļūstu kluss\\
Un atceros, kā vērojām pirms gadiem\\
Šos satumsušos, klusos ūdeņus.

Tāpat bij studenti, bij Līgo vakars,\\
Bij arī nezināmu biedru daudz.\\
...Bet tā, ar kuru tikāmies uz takas\\
Un kurai nepajautāju, kā viņu sauc...

Tik vienu reizi dejā tikās pirksti,\\
Un, gaišāk uzblāzmojot ugunij,\\
Vai tas bij ugunskurs, kas meta dzirkstis?\\
Bet var būt, ka tās viņas acis bij.

Kad rītā sāka kūpēt upe lielā,\\
Es jutu --- sirds ko neizteiktu tur,\\
Un pasmaidīja viņa: --- Rīgas ielās\\
Gan kaut kur tiksimies. Kaut kad... kaut kur...

Ir atkal Līgo vakars. Sirds man glabā\\
Vēl vārdus tos. Un reizēm sāpes most,\\
Jo desmit šķiršanās var panest labāk\\
Kā vienu tādu "kaut kad..." tikšanos.

Pār zaļām egļu burām krēslo vakars,\\
Un rieta ugunskurā mākoņogles dziest ---\\
Gar Juglas malu vijas daudzas takas,\\
Mēs tiešām varam kaut kur satikties!


\newpage

{\large \sc Tavā ielā}

Cik nav bradātas ceļu smiltis,\\
Cik nav bulvāra asfalts rīvēts,\\
Bet šai ielā ir sirdij silti,\\
Jo šī iela ieveda dzīvē. 

Pāri marta mākoņi vēlās,\\
Strauti caur sniegiem sev ceļu ara,\\
Bet es raudzījos naktī vēlā,\\
Kā tevi trolejbuss aiznes garām. 

Likās, vien roku pastiept lēnām ---\\
Un tavu delnu ar savu skaršu.\\
...Nezin no kurienes vēji ņēma\\
Toreiz kūpošas maizes smaržu.

Viss, ko laiks bija dziļi racis,\\
Pēkšņi pavērās maniem skatiem,\\
Un ar bērnību aci pret aci\\
Satikāmies pēc divdesmit gadiem ---

Atceros --- mēs ar brāli tvīkstam,\\
Uz donu skatāmies, acīm zibot, ---\\
No tās pat viens nevar paēst pa īstam,\\
Bet... pašiem gribas un ... runcim gribas...

Atmiņu stāsts varbūt daudz ko vijis,\\
Bet es sajūtu tikai vienu:\\
Ka man dzīvē ir daudz kas bijis,\\
Ka ir pagajis tik daudz dienu,

Ka ir izbristas ceļu smiltis,\\
Takas mazas un takas lielas ---\\
Lai tieši te sirdij būtu silti,\\
Lai es iznāktu uz šīs ielas!

Un, te gaidot, aizmirstu sevi,\\
Atrodos vienas domas varā ---\\
Nedrīkst trolejbuss aiznest tevi\\
Vienmēr garām un tikai garām!


\newpage

{\large \sc Dziesma}

Dzenā vēji viļņu vālus\\
\mbox{}\hspace{10pt} Ezerā.\\
Un tu atkal esi tālu,\\
\mbox{}\hspace{10pt} Kāpēc tā?

Rudzi san un baltā tvaikā\\
\mbox{}\hspace{10pt} Ziedēt sāk ---\\
Viens var nomirt tādā laikā,\\
\mbox{}\hspace{10pt} Mīļotā!

Nāc, kad riets ar sārtu liesmu\\
\mbox{}\hspace{10pt} Tālē zied,\\
Klausīties, cik skaistu dziesmu\\
\mbox{}\hspace{10pt} Ezers dzied.

Dziesmu tad no sirdīm abām\\
\mbox{}\hspace{10pt} Sāksim mēs,\\
Ezers gribēs dziedāt labāk ---\\
\mbox{}\hspace{10pt} Nevarēs!

Apklusīs un viņā pusē\\
\mbox{}\hspace{10pt} Dzirdēs mūs.\\
...Ezers visu nakti klusēs,\\
\mbox{}\hspace{10pt} Dusmīgs būs.

Bet, kad atkal spožā liesmā\\
\mbox{}\hspace{10pt} Atnāks rīts,\\
Nospriedīs, ka laba dziesma,\\
\mbox{}\hspace{10pt} Dziedās līdz.

Mēs ar viņu būsim rados ---\\
\mbox{}\hspace{10pt} Viņš nav ļauns,\\
Un, kaut tam jau tūkstoš gadu, ---\\
\mbox{}\hspace{10pt} Viņš ir jauns. 

Arī mēs pret gadu tāli\\
\mbox{}\hspace{10pt} Iesim tā,\\
Kā iet baltie viļņu vāli\\
\mbox{}\hspace{10pt} Ezerā.



\newpage

{\large \sc Pie tevis}

Trauc vilciens, kas mani pie tevis un Rīgas\\
Nes tuvāk un tuvāk arvien,\\
Aiz loga, kur sals audis sudraba stīgas,\\
Vien sarkanas dzirksteles skrien.

Šņāc putenis, zeme ar debesīm tiekas,\\
Un vakars ir satumsis sen.\\
Ne kurtuve karstā, bet sirds mana, liekas,\\
Uz priekšu šo vilcienu dzen. 

Caur puteni neticot šūpojas priedes,\\
Kad saku: --- Ja vajadzēs, draugs,\\
Tad vilciens spēs pamest pat ceļu un sliedes ---\\
Pie tevis pār klajumiem trauks! ---

Šņāc vēji un ledainas vērpetes svaida,\\
Caur negaisu ugunis blāv ---\\
Es zinu, tur mani uz perona gaida\\
Šai putenī mīļotais stāvs. 

Trauc vilciens, kas mani pie tevis un Rīgas\\
Nes tuvāk un tuvāk arvien,\\
Aiz loga, kur sals audis sudraba stīgas,\\
Jau pilsētas ugunis skrien. 



\newpage

{\large \sc Lietū}

Ir tāda steiga --- visi kaut kur skrien,\\
Un pēkšņi --- liepas baltās šaltīs tītas,\\
Kļūst visas ielas it kā izslaucītas,\\
Mēs ejam divi vien. 

Tu jautā man, vai kaut kas skaistāks šķiet\\
Par rudens lietus trako vēju brīvi\\
Un vai caur tādu negaisu pa dzīvi\\
Tev spētu blakus iet. 

Tu jautā man, bet pašai acīs spīd\\
Tāds nebēdīgs un trauksmains pavasaris,\\
Ar tādu skatienu tu taču vari\\
Man sirdi saskatīt.

Kāds brāzmains šodien rudens lietus ir!\\
Var tāda brāzma dzīvē atnākt rītu,\\
Lai mūsu abu sirdis pārbaudītu ---\\
Vai var vai nevar šķirt.

Lai tāpēc katru brīdi sirdis jūt:\\
Es esmu teksts, tu esi melodija,\\
Cik ilgi būs šī mūsu abu mija,\\
Tik ilgi dziesmai būt!


\newpage

{\large \sc Ilgas}

Tu man tādus labus vārdus\\
Atvadoties devi,\\
Bet, kad vilciens dunot\\
Aizdrāzās pār tiltu, ---\\
Vārdiem, kā jau vārdiem,\\
Pazuda viss viņu siltums,\\
Un es pēkšņi sapratu,\\
Ka ilgi neredzēšu tevi. 

Nevarēju atsaukt vairs ---\\
Par tālu tu tai mirklī biji.\\
Bet, ja varētu,\\
Tu kaut ko lielu manās acīs ieraudzītu:\\
Kādus vārdus --- Mūžā neizteicamus\\
Es tev tai brīdī pasacītu\\
Un pavisam citādi\\
No tevis būtu atvadījies.

Tā, ka nedziestošas zvaigznes\\
Uzdzirkstītu tavu acu dzīlēs,\\
Tā, ka dzīve tev pēc tam\\
Tik neciešami īsa liktos,\\
Tā, ka varētu pēc tam mēs\\
Arī nesatikties,\\
Bet uz zemes neviens nevarētu\\
Par mums skaistāk mīlēt.

Tu par tālu biji...\\
Sliedēs nevarēja saklausīt vairsDobjo takti.\\
Dūmu mākoņus jau izsvaidījusi bij\\
Vēju brāzma.\\
Tikai pāri mežiem\\
Visās krāsās uguņoja blāzma.\\
Nāca nakts,\\
Un debess uguņojot sagaidīja nakti.

Gribējās man tevi skūpstīt,\\
Bet bij tikai vēsais sliežu čuguns,\\
Auksts un nejūtīgs,\\
Bet no tā nevarēja atraut rokas.\\
Kas no slāpēm pakrīt tuksnesī,\\
Tas laikam tāpat mokās,\\
Aukstas veldzes vietā pieplokot\\
Pie smilts, kas deg kā uguns.

Satiekoties\\
No šiem vārdiem es tev neteikšu neviena,\\
Jo par slāpēm īsti runā tad,\\
Kad tās svilst krūtīs.\\
Satikšanās plūdos slāpes saprast grūti.\\
Bet, kad vilciens\\
Manam tālam ceļam pūtīs,\\
Tad tu visu sapratīsi ---\\
Tajā dienā.

\newpage

{\large \sc Sardzē}

{\em Mandžūrijas smilšu pauguros}

Kalnāji dus,\\
Miglāji tālumus sedz,\\
Pāri nemierā dzirkstošas zvaigznes\\
Zilas ugunis dedz.\\
Kalnāji dus,\\
Nemieru jūt tikai sirds ---\\
Tavu mīļoto acu spožums\\
Šurp no tālienes mirdz.

Trauksmaina nakts,\\
Redzu caur tālumu,\\
Kā pie dēla ar vēstuli manu\\
Mīļa noliecies tu.\\
Trauksmaina nakts ---\\
Zvaigznēs dzirkst debesu jums.\\
Viegli par mūžību nomodā stāvēt,\\
Mieru sargājot jums. 

Lepnumā šai\\
Sirds lielā dziesmā skan ---\\
Tūkstošu dzīve un tūkstošu laime\\
Šonakt jāsargā man.\\
Lepnumā šai\\
Acis valgas mazliet ---\\
Zinu, kaut kur pa dunošām sliedēm\\
Vilciens uz Maskavu iet...

Kalnāji dus,\\
Miglāji tālumus sedz.\\
Acis redz krūmu un akmeni katru,\\
Zāles stiebriņu redz.\\
Kalnāji dus,\\
Trako vienīgi sirds.\\
Tavu mīļotu acu spožums\\
Šurp no tālienes mirdz.

\newpage

{\large \sc Meža ceļš}

Līkst lielo egļu zari,\\
Tveicē sveķi lās,\\
Un viļņo gaiss pār meža ceļa dangām,\\
Kur satikušies tā,\\
Kā vējš ar jūru satiekas,\\
Mēs atvadījāmies,\\
Kā vētra atvadās no bangām.

Tu toreiz teici, ka mums ceļi šķirti ies,\\
Ka tajos laime mums būs katram sava,\\
Un es pēc tam\\
Nekad vairs negribēju satikties\\
Ne ar šo meža ceļu,\\
Ne ar acīm tavām. 

Bet aizmirst atnācu\\
To dziesmu, kas šeit skan,\\
Jo šajā ceļā kādreiz gājām abi,\\
Pavisam aizmirst,\\
Un lai turpmāk man\\
Ir visi ceļi vienādi un labi. 

... Aiz meža lauki šalc ---\\
Tur mierīgi klīst skats,\\
Tur mūsu soļus izdzēsis ir lemess,\\
Bet meža ceļš ir palicis tāds pats,\\
Un tāds viņš nedrīkst palikt uz šīs zemes. 

Te nedrīkst palikt pat ne krūmiņš tāds,\\
Kur zars ar atmiņām un smeldzi\\
Man pie pieres skaras,\\
Tā nedrīkst palikt, kad tu pretī nāc\\
Un tev bez sāpēm nevar paiet garām. 

Kaut kas ir palicis --- \\
Kaut kāda balss vēl teic:\\
--- Nu, atceries --- bij vakars miglas vālā... ---\\
Nē, bijušais ir vienreiz jāizbeidz ---\\
Ar tādu balsi nevar doties tālāk. 

Kaut kas ir palicis ---\\
Vēl kaut kāds ziediņš mirdz\\
Caur daudzu rudeņu un pavasaru salnām.\\
... Bet tev ir šodien skaidrai jākļūst, sirds,\\
Tik tīrai un tik skaidrai,\\
Kāds ir avots kalnā. 

Tad iesim tālāk, sirds,\\
Kad teiksi tā ---\\
Aiz manis tikai labas egles paliek,\\
Vien parasts meža ceļs ---\\
Un cita it nekā.





\newpage

\section{Atrastās vēstules}

\begin{quote}
{\em
Tās nedrīkstēju plēst nekādā ziņā\\
Un paturēt tās nedrīkstēju es,\\
Jo tās nav manas un var bū, ka viņām\\
Ir divas, trīs un vairāk adreses.
}
\end{quote}

{\large \sc Atrastās vēstules}

Kā tā var atrast? Jāpazaudē kādam,\\
Jo nevar taču atrast tāpat vien,\\
Bet, lai nu cik jums dīvaini tas rādās, ---\\
Šīs vēstules nav zaudējis neviens. 

Un tomēr tās ir atrastas --- gan miestā,\\
Kur ļaudis pulkā dodas laukus sēt,\\
Gan krustceļos, kur autobusi piestāj\\
Un vīri iziet ārā uzpīpēt. 

Gan šeit, uz mūsu Rīgas parku soliem,\\
Kad klusāks vakars --- čuksti tālu skan...\\
Un viena otra --- tur jau nav ko noliegt ---\\
Sirds kaktiņā bij paslēpusies man. 

Tās nedrīkstēju plēst nekādā ziņā\\
Un paturēt tās nedrīkstēju es,\\
Jo tās nav manas un var būt, ka viņām\\
Ir divas, trīs un vairāk adreses.


\newpage

{\large \sc Dzimšanas dienā}

{\em Vēstule mātei}

--- Tu šodien laimīgs, --- varbūt kāds man sacīs,\\
Es tiešām nezinu, bet klausies, māt,\\
Reiz lācīti ar brūnām stikla acīm\\
Jūs man ar tēvu nesāt dāvināt. 

Un tad es biju laimīgs ļoti, ļoti,\\
Jo nejutu, cik drūms ir lācēns tas,\\
Ka santīmi vispēdējie tam doti,\\
Bet bērna laimei vajadzēja maz ---

Vien pļavas, kurās skraidīt mazām pēdām,\\
Vien krūma, kurā putnu dziesmas skan.\\
... Bet pagastā ar saviem priekiem, bēdām\\
Šalc sabangotais dzīves okeāns.

Kaut kad, kaut kā, kaut kāds tur karš bij sācies,\\
Un ceļi nebij kūpējuši tā,\\
Bet bērns bij laimīgs --- viņam bij savs lācis,\\
Un vairāk viņam nevajag nekā...

Bet drīz vien lācim trūka vienas kājas...\\
Un, šalcot vējā maija lapotnei,\\
Es teicu, ka par šauru kļuvis mājās,\\
Ka vajag ceļa, kas pret tāli skrej. 

Tu visu saprati, mums nebij vārdu asu,\\
Un atceros, tas rīts bij ļoti kluss,\\
Kad bridām mēs pa balto auksto rasu\\
Un mazliet nokavējās Rīgas autobuss. 

Es tagad mācos te ar biedru saimi,\\
Kā stingri savas dzīves soļus lemt.\\
Jūs man ar tēvu šodien vēlat laimi,\\
Cik labi gan, ka viņu nevar ņemt.

Un ielikt rokās kā to lāci seno,\\
Māt, labi, māt, bet var jau būt, ka rīt,\\
Tāpat kā lāci, nezinādams cenu,\\
Es varu arī laimi sapostīt. 

Un, ja ne sapostīt, tad nebūs prieka\\
Par katru soli, ko uz priekšu sper,\\
Rīt gaida darbu daudz, un, māt, man liekas,\\
Ka laime tur un tā ir jānotur. 

Bet parīt atkal darbu būs bez skaita,\\
Tie jādara ar sirdi kaistošo,\\
Māt, laime ir šai nepārtrauktā gaitā,\\
Pašreiz es nezinu, bet ejot jūtu to!



\newpage

{\large \sc Tālā maršruta autobusā}

{\em Andrim Vējānam}

Ceļš savus līkumus steigā vija,\\
Kalnā devās un lejā drīz,\\
Birzis rudens ugunī bija,\\
Debesis --- mākoņu ugunīs. 

Un, kad tumsa aizsedza skatam\\
Bērzu, kam dzeltenas lapas lās,\\
Ugunīm mirdzot, uz ceļa platā\\
Skrejošās mašīnas sasaucās. 

Cilvēki bija ar savu steigu,\\
Katram prieka un rūpju daudz\\
Tajā ceļā, kuram nav beigu,\\
Ko par ceļu uz nākotni sauc. 

Bija tik ļoti dažādas sejas,\\
Tajās dažāda blāzma krīt.\\
... Tas nebij brauciens, bet tā bij dzeja,\\
Kuru vajag uzrakstīt. 

Ja par kādu var vispār sacīt:\\
Jaunību viņa skatienā jūt,\\
Tad es toreiz tev redzēju acīs\\
Īsti liesmojošo varbūt. 

... Tādu, kam patīk jūra dziļa,\\
Vajag viļņu, kas šalcot plīst,\\
Tādu, kam vajag, lai zibens šķiļas\\
Dzīves un dzejas debesīs!

Tādu portretu pats tu devi,\\
Un, lai darbu ir vesels mežs, ---\\
Tikai tādu es pazīšu tevi,\\
Citāds būsi man tomēr svešs. 


\newpage

{\large \sc Tu --- Ļeņina kalnos}

Skries pelēki viļņi, un decembra salā\\
Man priecīga ziedoņa sajūta ir:\\
Tu --- Ļeņina kalnos, es --- Daugavas malā,\\
Mūs pilsētas, meži un klajumi šķir.\\
Bet izšķirt mūs nespēj un tālums ir nieki ---\\
Mums kopīga skola, ko atmiņas vij:\\
Tur kopējas bēdas un kopēji prieki,\\
Un kopmītnē kopēja istaba bij.\\
Tu zini --- pie Gaujienas Gauja vēl nedus\\
Un pelēka migla pār ūdeņiem brien,\\
Bet, šķiet, pāri dīķim jau dzirksteļo ledus\\
Un slidas pār zilgano spoguli skrien.\\
...Tu atceries taču vēl Gaujienas krastu\\
Un runas, kad kopmītnē novakars tumst,\\
Kā reiz mēs ap pusnakti gājām uz pastu ---\\
Lai grāmatas sūta no Maskavas mums.\\
Tev patika kalnāju varenums staltais,\\
Kas drosmīgos rūda un drosmīgos sauc, ---\\
Tev grāmatu plauktā bij Tjanšans un Altajs\\
Un dažādu Dzimtenes attēlu daudz.\\
Mēs ceļojām domās --- gan dienvidu tveicē,\\
Gan rāpāmies kalnos, kur mūžīgais sniegs.\\
Es atceros --- reiz mūsu maršrutu beidza\\
Un Elbrusā kāpšanu pārtrauca miegs...\\
\mbox{}\hspace{10pt} Tad pavērās dzīve, tās plašumā lielā\\
Tas viss, par ko sapņojām, pretī mums zied.\\
Tu --- Ļeņina kalnos, es --- Ļeņina ielā,\\
Un ejam mēs abi, kurp gribējām iet.


\newpage

{\large \sc Meitenes pirmā vēstule}

Tu piedod, ja pārāk strauja biju\\
Un varbūt kaut ko neredzēja skats,\\
Bet sirds ir sirds, es viņai paklausīju,\\
Un, ko man atbildēt, --- tu zini pats.

Kad ledus lūst, tad upē miera nava,\\
Tai vajag tālāk putodamai steigt,\\
Es arī tāpēc nācu dzīvē tavā\\
Un teicu to, ko vajadzēja teikt. 

Jo vēlāk grūti būs par visu taujāt,\\
Cik ļoti sāp, tu sapratīsi pats,\\
Kad iziesi par savu laimi kaujā\\
Un līdzi tev ies manu acu skats.

Tu redzēsi --- būs grūti tālāk doties\\
Un reizēm prieka vietā skumji kļūs,\\
Bet no šī skata grūti atbrīvoties,\\
Un var jau būt, ka tad par vēlu būs. 

Ja vajag, vari atrakstit ar smaidu,\\
Ja vajag, esi atturīgs un vēss,\\
Bet zini, ka šīs vēstules, ko gaidu,\\
Man varbūt visu mūžu vajadzēs. 

Jo spēks, ko tavu acu skatiens deva,\\
Ar ugunīgu elpu manu dzīvi skar.\\
--- No pirmā skata iemīlēties nevar, ---\\
Tā kādreiz teica. Tici man, ka var.


\newpage

{\large \sc Draudzība izjūk}

Kad tu būsi visu izlasījis,\\
Mazliet padomā, kaut tāpat vien, ---\\
Var būt, ka tev daudz ir draugu bijis,\\
Man tu, zini, biji tikai viens. 

Negribējās strīdēties ar tevi ---\\
Daudzreiz atliku --- nu rīt, nu rīt.\\
Nevar taču ilgāk apkrāpt sevi,\\
To, ko vajag, --- jāvar uzrakstīt!

Zini ko, var tikai bērni mazi\\
Nedomājot drauga vārdu šķiest ---\\
Mēs, kam dzīvē nācies daudz ko pazīt,\\
Nedrīkstam ar vāŗdiem mētāties!

Draugs --- tad draugs! Tu jūti viņa plecu,\\
Viņa solim blakus --- solis tavs.\\
Draudzību sev nevar pakļaut vecums,\\
Ja tā ir... Bet ja nu viņas nav?

Nogurstu no dienas gaitas naskās,\\
Kāpēc iesit vārdi, kuri glauž?\\
Un man liekas, ka zem drauga maskas\\
Tu slēp sirdi, kurai kaut kas skauž. 

Zini ko, kad sanāk lielas raizes,\\
Drauga vārds tad ir gandrīz tas pats,\\
Kas ir izsalkušam gabals maizes,\\
Kas ir nogurušam uzmudinošs skats.

Redzot drauga jautājošo seju,\\
Tu ar vārdiem nekautrējies sist,\\
Parādīji ceļu, kurā ejot\\
Šodien, rīt --- man vajadzēja krist. 

Nepakritu, jo tu nesaprati,\\
Steigdams melus labos vārdos vīt,\\
Nesaprati to, ka dzīve pati\\
Katru vārdu atnāk pārbaudīt. 

Kad tu būsi visu izlasījis,\\
Mazliet padomā, kaut tāpat vien, ---\\
Var būt, ka tev daudz ir draugu bijis,\\
Man, tu zini... bija tikai viens.


\newpage

{\large \sc Tu aizmirsi mūs}

Vējš atkal lapas trenc gar Gaujas krastu,\\
Kopš izšķīrāmies --- otrais rudens jau.\\
Es šodien atkal aizgāju uz pastu,\\
Bet vēstules no tevis nav un nav. 

Nav četrus mēnešus no tevis ziņas ---\\
Tu esot precējies, tas --- viss, ko zināt drīkst.\\
Bet kā tad tā? Vai tiešām tev bez viņas\\
Neviena drauga nebūs vajadzīgs?

Ja cilvēks mīl --- plaukst februārī koki,\\
Nakts ugunis mirdz pēkšņi citādāk...\\
Bet tu mums pēkšņi atrauj savu roku,\\
Tu taču mīli... Nu, bet kā tad tā?

Nekas, ja aizmirsti vien mūsu saimi,\\
Ja citi draugi tev, jo gadi skrien, ---\\
Ir labi zināt savas dzīves laimi,\\
Bet slikti --- zināt tikai tādu vien. 

Jo divu lūpas nespēj visu sacīt\\
Un acis, lai cik arī redzīgas, --- \\
Tās tomēr ir un paliek četras acis,\\
Lai visu redzētu --- to ir par maz. 

Nekā mēs negribam no tavas laimes,\\
Kas četrām acīm --- lai zem četrām ir,\\
Bet roku daudz ir tavai biedru saimei ---\\
Un daudzas rokas vieglāk ceļu šķir. 

Ir vieglāk iet, ja biedrus jūti blakām,\\
Un arī nespēks ceļu neaizsedz ---\\
Ir lūpu desmitiem tik daudz kas sakāms\\
Un desmits acu --- tālu, tālu redz.

\newpage

{\large \sc Meitenei no manas klases}

Ir aizaugušas takas, redzi pati ---\\
Tur katru pavasari jauna zāle dīgs,\\
Bet uz to puisi, kas reiz tevi satiks,\\
Es laikam būšu mazliet greizsirdīgs...

Ne tāpēc, ka mēs skatus mijām biežāk\\
Un tagad gandrīz nepārmainām tos, ---\\
Nē, divi ceļi bieži projām griežas,\\
Lai tālumā ar citiem sastaptos. 

Ne tāpēc, ka pār Gaujas krasta pļavām,\\
Kad palu migla atnāk upi tīt,\\
Es gāju, jūtot tavu roku savā\\
Un baidīdamies... tevi noskūpstīt. 


Nē, toreiz krūtīs neliesmoja slāpes,\\
Kaut kāda trauksme, kaut kas dīvains bij,\\
Jo šķīrāmies un it nekādas sāpes\\
Tās tālās dienas sauli neapvij. 

Bet tagad šeit --- uz lielās dzīves trases ---\\
Man, tevi atceroties, silti mirkļi būs,\\
Tu esi meitene no manas klases,\\
Kaut zvans jau sen uz klasi nesauc mūs. 

Un dzīve visus --- viņu, tevi, mani ---,\\
Kur vajadzēja, nostādīja, draugs.\\
Mūs skolas klusie koridoru zvani\\
No tādām tālēm kopā nesasauks.

Nu, atbrauc! Nē... tu sapratīsi pati ---\\
Drīz mūsu gravās atkal zāle dīgs,\\
Un uz to puisi, kas reiz tevi satiks,\\
Es tomēr laikam esmu greizsirdīgs. 


\newpage

{\large \sc Aizbrauci}

Kur acu mirdzums tavs?\\
\mbox{}\hspace{20pt}Tālu.\\
\mbox{}\hspace{10pt}To es zinu ---\\
Ielā, istabā, tramvajā kaut kā nav,\\
Kaut kā milzīgi liela nav.\\
Kur acu mirdzums tavs?

Vajag abu mūs
\mbox{}\hspace{20pt}Kopā.\\
\mbox{}\hspace{10pt}To es zinu ---\\
Man bez tevis rindas mierīgāk plūst,\\
Kaut kā pavisam bez uguns plūst.\\
Vajag abu mūs. 

Kur acu mirdzums tavs?\\
\mbox{}\hspace{20pt}Tālu.\\
\mbox{}\hspace{10pt}To es zinu ---\\
Man bez tevis krūtīs nemiera nav.\\
Viss --- mums diviem. VIenam nekā man nav.\\
\mbox{}\hspace{10pt}Vētras bez tevis nav!\\
Kur acu mirdzums tavs?

\newpage

{\large \sc Vīramātei}

Jūs nesmejieties --- savu svētku kleitu\\
Es atkal uzvelku, jo divi gadi būs,\\
Kopš svešu meiteni Jūs sākāt saukt par meitu\\
Un es kā māti iemīlēju Jūs.

Daudz laba man šie divi gadi deva,\\
Bet tagad bieži ir ap sirdi tā,\\
Ka nerakstīt šo vēstuli vairs nevar,\\
lai cik Jums sāpīgi būs lasīt, māt.

Pret dēlu laba bijāt, tā Jums likās,\\
Jo vienmēr teicāt tikai: ``Mīļo Rem!''\\
Ja debesīs tam zvaigzne iepatikās,\\
Pat to Jūs nonesāt un: ``Dēliņ, ņem!''

Un, lai cik labi, māt, tas domāts būtu,\\
Lai kāda mīla tajos vārdos skan,\\
Bet tik daudz stundu neciešami grūtu\\
Bij Jūsu vietā jāpiedzīvo man. 

Sirds siltumu --- to prot viņš krūtīs aiznest,\\
To siltumu, ko dodu viņam es;\\
Kad viņam reizēm iepatīkas zvaigznes,\\
Tās tagad man ir lejā jānones.

Ir dzīvē brīži blakus saules jomām,\\
Kad grūti ir un slāpst pēc drauga balss.\\
Bet Remam labi, viņš par sevi domā\\
Un neredz, ka man auksti, ka man salst...

Jums otrs dēls ir, tāpēc rakstu asi.\\
Vēl kamēr viņš uz pirmo klasi skrej,\\
Lai tādi vārdi, kādus pašreiz lasāt,\\
Nav jāsaka vēl kādai meitenei. 


\newpage 

{\large \sc Lai stipra sirds}

Es labi atceros, kāds vakars bija:\\
Aiz loga šalko pielijušais mežs,\\
Stāv pudeles kā barga baterija,\\
Un sēdi tu tik nožēlojams, svešs. 

Tu stāsti man, kā sapņojāt jūs abi,\\
Kā likās jums, ka gadi pretī skrien,\\
Ka naktīs zvaigžņotās bij klejot labi,\\
Un to, ka tagad pēkšņi esi viens...

Ir grūti skumjās domas projām gainīt,\\
Un, rakstot tev, es zinu --- grūti tas,\\
Bet alus glāzei sirdi nepārmainīt\\
Un simtu gramu --- mierinājums mazs. 

Mums pārcilvēku nav, bet esam stipri ļaudis,\\
Ne visu smago vienmēr malā stumt ---\\
Ja nākas kaut ko ļoti tuvu zaudēt,\\
Tad nākas gluži cilvēcīgi skumt. 

Vai tāpēc vajag pamest dzimto ielu,\\
No fakultātes draugiem projām skriet,\\
Jo tieši tas ir mūsu jūtu lielums,\\
Ka arī sāpēs spējam tālāk iet!

Ir Rīgā tādas pašas skaistas naktis,\\
Pie fakultātes durvīm spuldze mirdz ---\\
Es ticu sliežu ceļa dobjai taktij ---\\
Tu atbrauksi. Un tev būs stipra sirds.


\newpage

{\large \sc Tramvajā}

Šodien tramvajā mēs esam blakām\\
Un viens otru neredzēsim rīt ---\\
Kaut Jūs spētu visu, kas man sakāms,\\
Svešā līdzbraucēja acīs saskatīt!

Jums vēl acīs bērnišķīgas dzirkstis,\\
Lūpas krāsotas un sejā bālums tāds,\\
Ka man cimdu ciešāk sažņaudz pirksti,\\
Jūsu skatiens jautā: kas es tāds?

Varbūt tas, kam patīk glāsta maigums,\\
Varbūt tas, kam Jūsu nav pat žēl,\\
Varbūt tas, kam patīk klusais zaigums ---\\
Tieši tas, kas Jūsu acīs kvēl!...

... Nu, bet Jūs vēl esat tango šūpās,\\
Un tā ritums Jūs no manis šķir.\\
Jūsu mazās, daudzu skartās lūpās\\
Kaut kas vēl no pievilcības ir. 

Dziīvojiet! Jūs taču ļoti jauna,\\
Nezināt pat, ko par laimi sauc,\\
Nezināt, ka dzīve ļoti ļauna\\
Tiem, kas reibumā pa viņu trauc. 

Kam tās rotas, kas Jums viz ap kaklu?\\
Acu spožums katru dienu kūst.\\
... It kā kucēnu --- vēl mazu, aklu ---\\
Savu laimi vīnā metāt Jūs. 

Izjuks Jūsu slikto biedru saime ---\\
Viņam,\\
Īstam draugam atnākot,\\
Jūsu acīs saskatot to laimi,\\
Ko Jūs tam vairs nevarēsiet dot.

Nē, es netraucēšu Jūsu namu,\\
Bet viņš --- atnāks. Pavērs durvis tās,\\
Atradīs Jūs, dzerot simtu gramu,\\
Sabrukušo jūtu krāsmatās...

Neskatieties manī acīm kairām ---\\
Esmu cits un nepavisam tas ---\\
Bet var būt, ka man par visiem vairāk\\
Jūsu žēl un Jūsu jaunības.

Šodien tramvajā mēs esam blakām\\
Un viens otru neredzēsim rīt ---\\
Kaut Jūs spētu visu, kas man sakāms,\\
Svešā līdzbraucēja acīs saskatīt!


\newpage

{\large \sc Vēstules no slimnīcas}

{\bf 1}

Man ilgi nerādīja tavas vēstules,\\
Tās jau pa murgiem biju sācis prasīt,\\
Bet tagad murgu nav. Es esmu atkal es,\\
Un man ir šodien atļauts mazliet lasīt. 

Un arī rakstīt. Tas jau ir visviens,\\
Tu laikam bargi skatīsies man acīs ---\\
Nu, tici --- atļāva, bet droši vien\\
Bij darba daudz un aizmirsa man sacīt. 

Tu mātei tādu nesaki neko\\
Un reizēm paskaties, kā viņai klājas,\\
Lai viņa nesadomā nezin ko ---\\
Rīt viss būs labi. Rīt es būšu kājās.

Es būšu kājās! Dzirdi! Būšu rīt!\\
Ne šajā klusumā, bet ielu dunā spalgā\\
Tu nāksi, nē --- tu skriesi sagaidīt,\\
Un tad mēs iesim... Kur? Tas būs vienalga. 


\newpage

{\bf 2}

Tu saproti --- te klusums spiež vai nost\\
Un tramvajiem te ir tik klusi zvani,\\
Ka tikai vakarā var dzirdēt tos.\\
Tie kaut ko sauc --- un, liekas, tieši mani. 

Vēl kājas nejūt, nu, bet tas nekas,\\
Viss taču labi būs, ja tā ir cerēts.\\
Tu, lūdzu, atnes manas lekcijas\\
Un labāk ienes tā, lai ārsti neredz. 



\newpage 

{\bf 3}

Nu, nedusmojies. Nedusmojies atkal jau ---\\
Es taču varu rakstīt un man vajag,\\
Jo --- bija lekcijas un viņu nav...\\
Neko man neļauj. Smoku telpā šajā. 

Nu, nedusmojies. Nedusmojies atkal jau\\
Un atnes, lūdzu, nav jau smaga krava, ---\\
Ja manu lekciju vairs mājās nav,\\
Uz kādu brīdi, lūdzu atnes savas.

\newpage

{\bf 4}

Man tagad kaimiņš pavisam cits.\\
Viņam ir smagi, bet mums vēl smagāk ---\\
Liekas, viņš savai slimībai tic.\\
Ļoti slikti pie mums ir tagad. 

Spīta un spēka tam nav nemaz ---\\
Slimība, ko tikai grib, to dragā,\\
Pretoties nespēj un negrib tas...\\
Ļoti slikti pie mums ir tagad. 

Tu jau saproti --- arī mums\\
Sirdī iearta šaubu vaga,\\
Tie, kas smaidīja, tagad skumst ---\\
Ļoti slikti pie mums ir tagad. 


\newpage 

{\bf 5}

Agrāk likās --- tad, kad grūti šķiet,\\
Sāpes pāries --- vajag vienu zināt:\\
Dzīvi mīlot, sāpes saņemt ciet,\\
Ņemt un dzīves mīlā noslīcināt. 

Jā, bet kaimiņš? Kā tad īsti tas?\\
Šaubu uguns daudzu krūtis skāris:\\
Laikam jau uz zemes ir kaut kas,\\
Kam ar mūsu spēkiem netiek pāri...

Atnāks rīts un pēc tam atnāks cits,\\
Atblāzmosies logā vakarblāzmas kāvi...\\
Saki --- kas viņš ir un kam viņš tic,\\
Ja bez cīņas laiž tik tuvu nāvi?

\newpage

{\bf 6}

Kaimiņš guļ, un šodien viņu jau\\
Taisās kaut kur nolikt vienu pašu,\\
Laikam ārstiem cerību vairs nav...

Vienu pašu... Norims pulsa takts,\\
Gaisma atvadoties paskatīsies acīs ---\\
Un pēc tam uz visiem laikiem nakts...

Spilvenu ar zobiem gribas kost ---\\
Nē, nav tiesa! Un kā smagu sapni\\
Gribas nokratīt no sevis nost!

Vienu pašu... Nedod mieru tas,\\
It neko man negribas vairs darīt,\\
Es vairs nelasīšu tavas lekcijas. 

Vienu pašu.... Nē, tas ir par daudz ---\\
Dzīve, mūsu nemierīgā dzīve.\\
Kāpēc tu tik klusu šodien sauc?!


\newpage 

{\bf 7} 

Visu, kas jādara, nomet nost,\\
Un, ja nav naudas, tad draugi iedos ---\\
Paņem sev līdzi visskaistākos,\\
Pašus labākos zemes ziedus. 

Nedomā vairāk, kā rītu ies,\\
Ja varbūt plānāks būs maizes rieciens, ---\\
Varbūt šais ziedos aizķersies\\
Nāves pēdējais stiprākais trieciens. 

Kaimiņu drīz varbūt aizved jau,\\
Viņam pašreiz ir ļoti slikti,\\
Viņam liekas, ka dzīves nav, ---\\
Nāc, mums abiem vajaga tikties!

Lai viņš sadzird, kā īsti skan\\
Dzīves dziesma no lūpām tavām,\\
Nāc, un lai viņš sajūt, kā tvan\\
Dzīvi ziedi no dzimtenes pļavām!

Nāc, lai viņš paskatās, ka arvien\\
Lūpas dedzina tikšanās svelme,\\
Nāc --- lai cauri palātai skrien\\
Mūsu dzīve kā vētraina dzelme!

Nāc ar lūpām, kas mīļi čukst,\\
Nāc ar laimi, ko dzīvē jūti,\\
Nāc ar sirdi, kas rītam pukst,\\
Nāc --- ar visu, kas ir tavās krūtīs!

Un, kad pie kaimiņa gultas šīs\\
Mūsu dzīve un jaunība stāvēs,\\
Redzēs visi --- ko mīlēt, ko nīst,\\
Kas ir stiprāks --- vai mīla vai nāve. 

Bet, ja arī par vēlu būs\\
Un ja kaimiņu nāksies aiznest,\\
Nāc, lai kā agrāk dedzina mūs\\
Mūsu mērķu un sapņu zvaigznes.

Nepamet mājās it nekā,\\
Steidzies un steigā netaupi sevi ---\\
Daudzi tevi šeit gaida tā,\\
Kā līdz šim tikai es\\
Biju gaidījis tevi. 

\newpage

{\large \sc Vēstules latviešiem ārzemēs}

{\bf Tam, kas aizmaldījies}

Vienalga, vai tev apkārt sniegi palsi\\
Vai stepes ceļš ar karstu smilti klāts,\\
Šai brīdī tavas dzimtās zemes balsī\\
Man vajag, draugs, ar tevi parunāt. 

Vai tas var būt, ka tajā tālā pusē,\\
Kur svešas zvaigznes uzliesmo un dziest,\\
Nevienu brīdi tev nav nācies klusēt\\
Un tālu vēju dziesmā klausīties?

Tas nevar būt! Lai kādas gaitas mēro,\\
Kaut vienu reizi --- ilgas tomēr most,\\
Tu vari simtiem svešu straumju vērot,\\
Bet, ja reiz spoguļojies Gaujas ūdeņos,

Simts pirmo straumi skatot, krūtīs iedzels\\
Un jūtas satvers tevi, stāstīs daudz,\\
Uz svešā krasta tevi kājās piecels\\
Un noņems cepuri. Tur --- dzimtā zeme sauc!

Bet šaubas tevi grauzīs ļoti, ļoti\\
Un satvers bailes, dzirdot balsi to, ---\\
Nav visi maldu soļi noskaloti ---\\
Tos arī gadi nevar noskalot. 

Var tikai viens. To man ir lieki sacīt,\\
Un ja tev ceļš uz mūsu pusi ies,\\
Es ticu, ka tev dzimtai zemei acīs\\
Būs pirmā brīdī grūti skatīties.

Būs Rīga savā parastajā dunā,\\
Par to, kas bijis, nerunāsim mēs,\\
Bet krūtīs tava sirdsapziņa runās ---\\
Tai atbildēt tev pašam vajadzēs. 

Un, ja tu šurpu atnāc, liepām ziedot,\\
Un tev pēc darba slāpst, un meklē skats,\\
Tad mums vairs nav ko nepiedot un piedot,\\
Tu piedosi vai nepiedosi pats. 

Šī vēstule lai liek tev biežāk klusēt\\
Un tālu vēju dziesmā klausīties,\\
Lai tā tev nedod mieru tālā pusē,\\
Kur svešas zvaigznes uzmirgo un dziest. 

\newpage

{\bf Ienaidniekam}

Mums nav ko ``jūsot''. Mēs jau sen uz ``tu'',\\
Es toreiz biju mazāks katrā ziņā,\\
Kad ieskaidroji man ar pātagu,\\
Ka teļi nedrīkst bizot āboliņā. 

Kad uzbrēci man: --- Puika, dziedi tā! ---\\
Es drebēju un pīkstēju ar mokām:\\
--- Pie tēvu zemes dārgās ķeries klāt... ---\\
Jā, Kļaviņa kungs, bet ar kalpu rokām!

Mēs tātad pazīstami --- redzi nu,\\
Es vakar gribu paklausīties valsis\\
Un griežu radio, bet sadzirdu\\
Ne mūziku, bet tieši tavu balsi. 

Kā būtu pazīstama, tā kā ne...\\
Es šaubījos, bet tā nav mana vaina,\\
Man likās, ka tu vienmēr tikai brēc,\\
Bet... gadi iet un balss jau ļaudīm mainās. 

Un manējā vairs arī nebūs tā,\\
Ar kuru nesaprazdams pīkstēju, tu dzirdi:\\
--- Pie tēvu zemes dārgās ķeries klāt,\\
To turi ciet ar visu savu sirdi! ---

Kāpēc?\\
Zin tagad kalpu puika tavs\\
(Jo toreiz vienkārši vēl bij par mazu) ---\\
Ka saimniekam sirds vai nu numaz nav,\\
Vai arī kalpiem ādu plēšot, pazūd.

Un kalpi zemei tai, kas sulās briest,\\
Ar sirdi pieķērās, un sirds tev trūka.\\
Tev, tiesa, gribas slepus nolaisties\\
No kādas svešas lidmašīnas lūkas.

Jo gribas zināt to, kā Venta blāv,\\
Jo gribas redzēt jūras smilti liego,\\
Kur mūsu krasta baterijas stāv,\\
Vai, vaļsirdīgāk izsakoties, spiegot. 

Tev ļoti gribas... Nu, bet tas nav viss,\\
Nav visi mirkļi gadu miglā rakti ---\\
Bij nakts, kad varēji reiz izliet asinis,\\
Tas bija sen. Ir tagad citas naktis ---

Un kritīs svešā lidmašīna tā,\\
Un tevi pašu atradīs un pazīs ---\\
Pie tēvu zemes vajag ķerties klāt,\\
Bet ne pie kakla tai un ne ar nazi. 

Daudz gadu pagājis, bet, redzi nu, ---\\
Es tavu seju ļoti labi zinu,\\
Ja rīt pie mums tu slepus ielīstu,\\
Mums būtu īsa saruna --- ar svinu!

Mums nav ko ``jūsot''. Mēs jau sen uz ``tu''.



\newpage

\section{Pāri pasaulei --- saule}

\begin{quote}
{\em
Visu, ko jaunība mīl,\\
Pie sirds tā sasildīt var,\\
Visu, ko jaunība nīst,\\
Ar sirdi var ---\\
Sadedzināt!
}
\end{quote}


{\large \sc Ceļš uz komjaunatni}

{\bf Skolnieki}

Bij četrdesmit pirmais, un jau tikos es\\
Ar komjauniešu bezbailīgām sirdīm,\\
Kas vienās krūīs divu spēku nes.\\
Daudz gadu pagājis, bet vēl es dzirdu,\\
Kā stundu pārtrauc nesaprotams zvans.\\
Pēc kārtas  aktu zālē stājas klases,\\
Vismazākie --- tie stājas blakus man...\\
Un direktors par kaut ko runā asi,\\
Bet klases sačukstas, ir savāds satraukums,\\
Tad vācu virsnieks iestreipuļo iekšā\\
Un aizsmakušā balsī uzbrēc mums:\\
--- Komjaunieši, soli uz priekšu! ---

Ir kaut kas stindzinošs, ir briesmīgs drauds\\
Šī piedzērušā feldfēbeļa acīs.\\
... Sper soli komjaunieši! Un cik daudz\\
Viens solis vīrišķīgs var visiem sacīt!\\
Ir milzīgs spēks tiem rokās sažņaugtās\\
Un acis, acis!... Tās ir ogles karstas.\\
Kļūst pēkšņi direktors kā bērza tāss,\\
Ar roku krampjaini gar krēsle malu tvarsta. 

Raud skolotāja, klasē vedot mūs,\\
Caur asarām mūs brīdina no briesmām,\\
Bet šurp no zāles valdonīgi plūst\\
``Uz cīņu mostiet...'' --- bargā, skaistā dziesma.\\
Tad daudzus aizveda... Bij zālē kluss,\\
Un dzirdēt varēja, kā muša sīc pie sienas.\\
Un kaut kur tālus, tālus šāvienus...\\
Mēs baušļus nemācējām tajā dienā,\\
Un, izdomājot smagās lietas tās,\\
Mēs, otrā klase, secinājām tieši:\\
--- Ir no šīs dienas daudz kas jāmācās,\\
Jo arī mēs reiz būsim komjaunieši...

\newpage

{\bf Padomju armijas karavīri}

Bij četrdesmit ceturtais, un stāvais krasts\\
Viss liesmoja un dunēja kā smēdē,\\
Kā verdot Gauja meta šļakatas,\\
Un gāja tanki, žvadzot kāpurķēdēm.\\
Bij upe jāpārpeld, kaut nodreb sirds,\\
Kaut dzelme krācot gaisu rauj uz iekšu,\\
Vads nostājās, un stingru balsi dzird:\\
--- Komunisti, soli uz priekšu! ---\\
... Bij krēsla tad. Un pulki gāja turp,\\
Kur pāri Gaujai dega daudzas mājas,\\
Bet, kas bij lodes dzelts, to nesa šurp,\\
Tam šineli zem lazdas krūma klāja\\
Un kasku noņēma. Sirds skaudru smeldzi kur ---\\
Mirst gvards, bet sejā nav ne vēsts no mokām,\\
Vien roka glauž un cieši, cieši tur\\
To, ko pat mirstot neizlaiž no rokām, ---\\
Glauž biedra karti, lūpas skar\\
Vēl kādu vēstuli, no kā --- to viegli zināt.\\
Tie --- gvarda dārgumi. Ja mīli tos, tad var\\
Ar sirdi reizē tanku uzspridzināt.\\
... Pirms stundas viņš vēl komjaunietis bij,\\
Bet, ļimstot ``tīģerim'' ar kāpurķēdi sistu,\\
Tā bruņas skaujot zilai ugunij,\\
Gvards sevi pieteica par komunistu. 


\newpage

{\bf Uz komjaunatnes komiteju}

Bij četrdesmit asotais, un putināja tā,\\
Ka sniegs ar asu nātri pēra seju.\\
Mēs --- četri. Braucām smagā mašīnā\\
Uz tālo komjaunatnes komiteju.\\
Bij janvāris, bij ziemas skarbums īsts\\
Un ielejās --- tur kupenu bez gala.\\
Bij reizēm jārok stundas divas trīs,\\
Un kā pa tuneli bij jābrauc tālāk.\\
Bij grūti? Nezinu. Lai vēji elš ---\\
Mēs savai laimei attīrījām ielu!\\
Tas nebij aizputināts ziemas ceļs,\\
Bet ceļš uz kaut ko varenu un lielu.\\
Ja dzīvi spētu otrreiz atkāŗtot,\\
Es nemainītu atstātu šo posmu vienu,\\
Ne mirkli nešauboties, izvēlētos to,\\
To pašu janvāri, to pašu dienu.\\
... Kad sniegputenis atkal svilptu tā\\
Un sniegs ar asu nātri pērtu seju,\\
Mēs četri brauktu smagā mašīnā\\
Uz tālo komjaunatnes komiteju...\\
\mbox{}\hspace{10pt}Tas brīdis beidzot bija sagaidīts,\\
Kad vaigos šķita ugunskuri kurti.\\
Mēs redzējām, kā pašiem rokas trīc,\\
Kā komjaunatnes nozīmē deg pieci burti.\\
Bet, skatoties, kā mazā zvaigzne mirdz,\\
Drīz jutām visu uztraukumu plokam,\\
Sirds krūtīs kļuva --- komjaunieša sirds\\
Mums, pirmo reizi turot\\
Biedra karti rokās. 


\newpage

{\bf Lielajā gadadienā}

Dzirkst šodien acīs nesavaldāms prieks ---\\
Mēs lielās dzīves, lielās rosmes skarti.\\
Lai komjaunatne mūsu priekšā liek\\
Un atritina dzimtās zemes karti,\\
Lai rāda maršrutu, kur grūtāk ir,\\
Vai tveicē dienvidos, vai polārnaktī baigā,\\
Mēs iesim jaunu lapas pusi šķirt\\
Gan auditorijā, gan tuksnesī, gan taigā.\\
Šai dienā komjaunatnes vārdu teic\\
Un lepnums starojot skauj mūsu seju,\\
Šai dienā lielo gadadienu sveic\\
Un reizē arī --- mūsu jubileju.\\
...Tev varbūt jau ir divdesmitais gads,\\
Bet svētku karogam pār ielu plīvot,\\
Tu sajūti, tu labi zini pats ---\\
Pirms dažiem gadiem sāki īsti dzīvot.\\
... Bij četrdesmit astotais, un putināja tā,\\
Ka sniegs ar asu nātri pēra seju.\\
Jūs --- četri. Baucāt smagā mašīnā\\
Uz tālo komjaunatnes komiteju.\\
\mbox{}\hspace{10pt}Tad sākās dzīves palu straujums īsts,\\
Tam visu mūžu tādam straujam virmot,\\
Lai acīs deg tās dienas ugunis,\\
Kaut arī matiem iepatīkas sirmot.\\
Tu esi dzimis tad! Kaut ceļi tālu skrej,\\
To janvāri un sniega auku dzirdi\\
Un drošu soli visu mūžu ej\\
Ar vienmēr jauno\\
Komjaunieša sirdi!


\newpage 

{\large \sc Vētrainā sirds}

{\bf Dzimtajā pusē}

Stāj autobuss.\\
Un jau pēc brīža steidzot\\
Aiz kalna sārtā uguntiņa zūd,\\
Tu paliec viens uz ceļa,\\
Un šis ceļš nu beidzot\\
Pēc gadiem pazīstamos soļus jūt. 

Stāv egles sniegotas,\\
Un zilas zvaigznes klusē,\\
Tev liekas --- šeit nav simtiem gadu būts,\\
Nav simtiem gadu justs,\\
Kā vēji šalko dzimtā pusē\\
Un cik pēc tiem ir izslāpusi krūts.

Varbūt ir arī citur tādi vēji,\\
Bet tu tos īsti jūti tikai te,\\
Kur pirmo reizi juti, kalnā skrējis,\\
Cik grūti sasniedzama virsotne.

Kur, lūk, šai mežā, kas steidz sarmā tīties\\
Un nolīkušiem zariem sniegu skar,\\
Tev pirmo reizi nācās apmaldīties\\
Un nācās salt un raudāt tovakar.

Jo toreiz likās, ka no meža drūmā\\
Tu laikam ārā netiksi tik drīz,\\
Ka melna nakts drīz gulsies katrā krūmā\\
Un atnāks vilks un tevi apēdīs...

Tad likās, ka nav glābiņa nekādas\\
Un tu vairs mājās neatnāksi rīt.\\
... Bet sāki skatīties, kur zari rāda\\
Un kurā pusē Polārzvaigzne spīd.

Tāds rūgtums tajā naktī kāpa kaklā\\
Un bailes bija tā kā vēl nekad,\\
Bet ceļu atrast pusnakts tumsā aklā\\
Par visu vairāk vajadzēja tad!

Un tad tu nejuti vairs, ka bez mitas\\
Pār vaigiem asaras tev skrej un skrej\\
Un ka aiz vienas kupenas nāk cita,\\
Jo priekšā kaut kur tālu uguns spīd\\
Un suņi rej. 


\newpage


{\bf Nolādētā sirds}

Tu paņem čemodānu. Gribi mājup doties\\
Un tikai saprast nevari kaut ko ---\\
Ar dzimto pusi šonakt satiekoties,\\
Sirds pirmo atcerējās tieši --- to...

Sirds atcerējās un vairs nedod miera:\\
Bet ja nu tagad ceļus sajauktu\\
Un mežā, kas no bērnu dienām pierasts,\\
Pavisam negaidīti apmaldītos tu?

Ja kādu satiktu, kas zin šo vietu,\\
Tu droši steigtos ceļu paprasīt,\\
Bet, ja neviena nebūtu, ---\\
Tad mierīgs ietu?\\
Ja šodien neaizies, tad aizies rīt...

Tu paņem čemodānu, ej un klusē,\\
Un atmiņas pie visiem velniem dzen ---\\
Tik reti iznāk pabūt dzimtā pusē,\\
Vai tad ir vērts par to, kas bijis sen...

Tu ej, un pāri egļu zari plešas,\\
Un īsti nevar saprast, vai ir mežs\\
Par dažiem gadiem kļuvis krietni svešāks\\
Vai pats tu esi kļuvis krietni svešs. 

Nu skaidrs --- taču bērnība ir garām\\
Bet atkal atmiņas ir jāsatiek...\\
Ko lai ar tādu sliktu sirdi dara,\\
Un kur lai viņu, nolādēto liek?

Tu ej, un tev ir saprast ļoti grūti,\\
Tikgrūti, ka pat ilgāk nevar ciest:\\
Kas pēkšņi noticis ir sirdi krūtīs,\\
Kas viņai liek pret tevi sacelties?

Var būt, ka tava sirds nīst sniega vālus,\\
Caur kuriem šodien tu to projām nes\\
Uz dzimto pusi gan, bet toties tālu, tālu\\
No kādas ļoti labas meitenes. 

Jā, arī tas. Sirds vēlas tā kā vakar\\
Smelt gaismu mīļo acu ugunīs,\\
Bet tas nav viss. Vēl kaut ko viņa saka,\\
Vēl kaut ko citu ļoti, ļoti nīst. 

Var būt, ka sirdij netīk, ka tev acīs\\
Šai satikšanās brīdī\\
Vien sniega pārslas trīc,\\
Bet var jau būt, ka viņa grib tev sacīt,\\
Ka pats tu viņai ---\\
Ļoti nepatīc. 

Nu, velns ar viņas nemierīgo takti ---\\
Tu labāk skatīsies, kā zvaigznes zib,\\
Un nebojāsi Jaunā gada nakti,\\
Bet sirds lai niķojas, cik viņa grib.

... Caur mākoņiem vairs zvaigznes pamirdz reti.\\
Mežs beidzas. Biežāk, biežāk pārslas krīt,\\
Un kamanās, kas tagad slīd tev pretī,\\
Brauc kāda māte dēlu sagaidīt. 

\newpage

{\bf Vai tiešām gadi mainās?}

Guļ lauki vecā gada sniegu klāti,\\
Bet gaisā jaunā gada sniegi snieg.\\
Slīd kamanas. Un abiem jums ar māti\\
Šai brīdī Jaunais gads ir jāsatiek.

Un liekas, krūtīs vētra norimusi,\\
Sirds mierīga un atkal labi viss,\\
Bet varbūt tāpēc, ka tev priekšā klusi\\
Mirdz tava ciema svētku ugunis?

Tev liekas --- viss uz zemes pašreiz smaida,\\
Un tādā labā naktī jau var smiet ---\\
Uz ceļa zaķis Jauno gadu gaida\\
Un netaisās no ceļa projām iet. 

Zirgs, kamanas --- tas viss ir viņam nieki,\\
Bet, kad tu beidzot iesāc klaigāt pats,\\
Tad zaķa pacietībai arī pietiek ---\\
Te bij, te nav. Tāpat kā vecais gads. 

Klāt dzimtie dārzi, dziļu sniegu tīti,\\
Un, kad jau zirgs gar mājas stūri skrien,\\
Tu izlec. Visām kupenām par spīti,\\
Ir vienai tomēr cauri jāizbrien. 

Tu tagad beidzot būtu savās mājās,\\
Gandrīz žņaudz elpu atgriešanās prieks,\\
Skrien pretī Duksis, slienas pakaļkājās,\\
Un, kuru ķepu gribi, to viņs sniegs. 

Jo, kaut pēc gadiem atbrauci tu ciemā,\\
Viņš tevi pazīst, priecīgs asti kuļ,\\
Un apkārt ---\\
Kā jau katrā dziļā ziemā ---\\
Zem dziļiem sniegiem kaut kur zeme guļ. 

Sirds meklē katru bērnu takas oli\\
Un katru tavu seno pēdu tver,\\
Un salīdzina vakardienas soli\\
Ar to, ko šidien tu uz zemes sper.

Tā upe, kas ir dzimusi no strauta,\\
Līdz pašai jūrai strautu sevī nes,\\
Un tāpēc sirds tev, lielu jūtu skauta,\\
Ir pie šīs apsnigušās mājeles.

Kā viesis un kā dēls tu durvis atver,\\
Un sarmas mākonis tev ieiet līdz,\\
Un pēkšņi skuju smarža tevi satver ---\\
Tu esi atnācis un esi sagaidīts. 


\newpage

{\bf Ierunājas tālumi}

Tev visi mājinieki apkārt stājas\\
Un spriež, ka vecāks izskaties mazliet,\\
Bet māte teic, ka esi kļuvis vājāks.\\
Ak, mātes, mātes, tā jums visām šķiet. 

Un tad tu saproti, cik labi būtu,\\
Ja mājinieku vidū būtu tā,\\
Pret kuru tev ir tik daudz lielu jūtu,\\
Ka...\\
Vienkārši to nevar izrunāt. 

Kad satikšanās kņada mazliet pieklus,\\
Tu raksti vēstuli un stāsti daudz;\\
Tu dzirdi viņas tālos, mīļos smieklus,\\
Bet nedzirdi, ka blakus māte sauc. 

Šai vēstulē sirds ieliek savus pukstus,\\
Bet visi tādi uzmanīgi kļūst,\\
Un tēvs ar māti kaktā nospriež čukstus:\\
--- Jā, laikam gan, ka pārmācījies būs... --- 

Tev satikšanās priekš ar ilgām jaucas,\\
Un, kā ir vairāk, --- sirds vairs neizšķir.\\
Tā nevar istabā. Tā ārā traucas,\\
Tai šodien četrās sienās grūti ir. 

... Ir satumsusi ziemas tāle zilgā,\\
Ar tādām lielām pāŗslām sācis snigt,\\
Tā sācis snigt, ka grūti būs\\
Pat ilgām\\
Caur tādu puteni uz Rīgu tikt. 

Bet --- var. Sirds var caur tādu tāli skatīt,\\
Kā Rīgas ielās šonakt spuldzes zied\\
Un meitene, kas tev vislabāk patīk,\\
Caur Jaunā gada sniegputeni iet. 

Un tādām domām jau nekad nav gala...\\
Tad kāda balss pie pašām ausīm skan:\\
--- Dēls, visu nakti nevar stāvēt salā,\\
Nu, sirds jau nenosals, bet kājas gan. 


\newpage

{\bf Kas noticis?}

Tā sen jau bija: steiga mājās valda,\\
Un, negaidot, kad tēvam pagurs plecs,\\
Dēls saprata, ka malka jāsaskalda,\\
Un nebij jāsaka, jo pats jau redz. 

Bet šodien --- ne. Ir tāda laba diena,\\
Tu saposies un slēpes paņem jau,\\
Un māte it kā taisnodamās pienāk:\\
--- Dēls, varbūt vēlāk... Pašreiz malkas nav. ---

Tu vēlreiz atceries,\\
Ka pāri zari plešas,\\
Un īsti nevar saprast, vai ir mežs\\
Par dažiem gadiem kļuvis krietni svešāks\\
Vai... pats tu esi kļuvis krietni svešs...

Plīst bērza šķila --- ne jau nu bez mokām\\
Var sīksto zaru pieveikt cirvis tavs,\\
Ar spalvu, mīļais, saradusi roka ---\\
Un malku skaldīt diez cik viegli nav. 

Svīst piere drusku, vēlāk sākas pali.\\
Tēvs smaida: --- Draugs, ja tevi laidīs klāt,\\
No baļķa pāri paliks tikai skali\\
Un nolādēti ciets kļūs cirvja kāts. ---

Un tēva joks kā karstās knaiblēs saņem ---\\
Tu tikko gribēji mest šķilu nost.\\
--- Nu labi, tēv, tu logaritmus paņem,\\
Un tad es sēdēšu un skatīšos. --- 

Tēvs smaidot iet ``pie gudrām lietām ciemā'',\\
Grauž zīmuli un teic: --- Jā... likās --- ies.\\
Vai tiešām tici tu,\\
Ka četras cara skolas ziemas\\
Var iet ar taviem gadiem cīkstēties? --- 

Tēvs smaida, burza papīru pa rokām:\\
--- Tas rēķins ir priekš manis pārāk sīks. ---\\
Un visa cīņa kaut kā kļūst pa jokam,\\
Bet šajā cīņā ir kaut kas,\\
Kas tēvam nepatīk. 

Un viņš to saka?\\
Nē, nekādā ziņā,\\
Ja pats tu nesaproti, tad nekā.\\
Un māte arī steidz ar vakariņām,\\
Jo tēvu taču kaut kā vajag glābt.


\newpage

{\bf Vakariņas}

Uz galda viss, kas ir uz galda liekams,\\
Pat grūti kaut ko izvēlēties sev ---\\
Daudz visa kā. Un viss ir tavam priekam\\
Un varbūt ilgi glabāts tieši tev. 

Te neļauj tukšākam kļūt šķīvim tavam ---\\
Neko tu neiestāstīsi, te visu zin ---\\
Tēvs uzgrūdies bij vecam draugam savam,\\
Un, protams, tikšanās bij jānosvin. 

Pie satikšanās simtu gramiem sēdot,\\
Starp citu bija pastāstījis tas,\\
Ka studenti pa gadu kaut ko ēdot,\\
Bet eksāmenu laikā neēdot nemaz. 

Neko tu neiestāstīsi --- tā ir --- un beigas:\\
Un --- neēst mēnesi! Var taču nomirt tā!\\
Un velti taisnosies, ka darba steigā\\
Jau vienu otru reizi neiznāk. 

Tu smaidi: jums jau var to pašu sacīt,\\
Kad darba sanāk daudz un roku maz.\\
Bet māte nezin kāpēc nolaiž acis ---\\
Vai tad no darba steigas jākaunas?

Un klusums.\\
Ilgi vārda nav neviena.\\
Tēvs lēnām it kā slimu vietu skar:\\
--- Tā... mazāk iznāca uz darba dienas,\\
Un nav jau vienmēr tā kā šovakar. 


{\bf Tēvs ļoti ticēja}

Tēvs gribēja, lai pats tu meklē pēdas,\\
Un, ticēdas, ka atradīsi drīz,\\
Viņš noslēpa no tevis visas bēdas\\
Un nolēma, ka tev tās nerādīs.

Viņš ļoti ticēja, ka ceļu platu\\
Tev dzīve dos --- vien skolu vajag beigt.\\
Ka atbrauks dēls ar drošu, asu skatu\\
Un it neko tam nevajadzēs teikt. 

Viņš ļoti ticēja --- vienalga kādā druvā\\
Dēls darbu darīs tā, lai visi jūt:\\
Par mūsu dzimtās zemes mērķiem tuvāks\\
Nevienam šodien mērķis nedrīkst būt.

Bet dēls ne tā ar dzimto pusi tiekas, \\
Dēls negrib vētrām pleu pretī griezt,\\
Bet nē --- varbūt tas viss\\
Vien pirmā brīdī liekas.\\
Nē droši vien tā nevajaga spriest. 


{\bf Tikai par sevi}

Ir svēdiena un saules staru zibā\\
Līkst logā koku sarmas mirdzums kluss.\\
Nemaz nav gulēts. Gulēt ļoti gribas,\\
Jo rīt jau ies uz Rīgu autobuss. 

Un kā tu gulēsi, ja naktī pavadītā\\
Nekādā jēgā nevarēji tikt\\
Un jūs ar tēvu runājāt līdz rītam\\
Par to, kas kolhozā ir labs un slikts. 

Un arī šorīt desmitreiz un simtreiz\\
Tēvs savas brigadiera lapas šķir,\\
Un lapas rāda tev, ka kolhozs dzimtais\\
No komunisma diezgan tālu ir. 

Tēvs stāsta, ka šeit reizem iet kā cirkā,\\
Ka tādi jauni agronomi brauc,\\
Kas ļoti labprāt pajautātu zirgam,\\
Uz kura gala lai tam sakas mauc.

Tēvs stāsta visus priekus, visas raizes\\
Un domā, ka pats pamanīsi tu,\\
Cik neērti ir teikt, ka pietrūkst maizes...\\
Cik ļoti labprāt viņš to neteiktu!

Tu skaties, klausies. Tēvu saprast gribas,\\
Bet miegs kā nāve --- nevar atvairīt.\\
Aiz loga ceļas vējš un saules zibā\\
No kokiem sarma spīguļojot krīt. 

Tu vakar domāji, ka šorīt skriesi\\
Un koki sarmas pilnus zarus sniegs,\\
Bet tēvs te runā... Kā tu tagad iesi?\\
Birst sarma.\\
Nobirst ---\\
Un kļūst vienkāršs sniegs. 

Rīt autobusā arī nebūs prieka,\\
Ja miegs jau tagad plakstus kopā glauž.\\
Var kādā pieturā kāds students iekāpt ---\\
Un tu tur snaudīsi, kā veči snauž. 

--- Dēls, neklausies, ja tev tik slikta oma, ---\\
Tev pēkšņi sirds par karstu ogli kļūst,\\
Jo tēvs ir tevi noķēris pie domas:\\
``Jā, ir jau tā... bet gan jau labi būs.''

Tad tēvam acīs kaut kas iemirdz asāk,\\
Tās sāpīgi zem sirmām skropstām kvēl ---\\
Dēls negrib klausīties, ko dzīve prasa,\\
Ne kaut kas svešs,\\
Ne kaut kas cits,\\
Bet dēls...

Un, it kā tev kāds būtu pļauku devis,\\
Kauns sārtojot pār tavu seju skrien ---\\
Tā,\\
Visu aizmirstot\\
Un redzot tikai sevi,\\
Nav zem šī jumta domājis neviens. 

\newpage

{\bf Divkauja ar sirdi}

Tēvs klusām sabāž savas lapas somā,\\
Ne vārda nesacījis, iet kaut kur.\\
Tu paliec viens.\\
Viens pats ar savām domām,\\
Un no šīm domām neizbēgt nekur. 

Viss ataust citā gaismā,\\
Un vairs nav kur dēties.\\
Tēvs teica?\\
Ko tu atbildēji tam?

Un tu sāc visu, visu atcerēties\\
Līdz pašam beidzamajam sīkumam ---\\
Tu atceries un šoreiz jau bez smaida,\\
Kā iesāk autobuss caur sniegiem skriet.\\
Kā zaķis savu Jauno gadu gaida\\
Un netaisās no ceļa projām iet...

Tu atceries, kā desmitreiz un simtreiz\\
Tēvs savas brigadiera lapas šķir\\
Un lapas rāda tev, ka kolhozs dzimtais\\
No komunisma diezgan tālu ir. 

Tēvs stāstīja, ka reizēm pietrūkst maizes...\\
Bet tu?\\
Tu nevarēji miegu atvairīt\\
Un domāji --- līdz komunismam aizies,\\
Ja šodien neaizies, tad aizies rīt...\\
Ja dzīve tev vien tādus vāŗdus atnes,\\
Ja, kaudamies ar miegu, dzirdi tos ---\\
Nav tevī it nekā no komjaunatnes ---\\
Un noliec savu biedra karti nost!

Tas tev ir papīrs!\\
Met no sevis projām,\\
No komjaunieša vārdiem savu dzēs ---\\
Ar tādu biedra karti mira Zoja,\\
Ar tādu biedra karti ejam mēs. 

Uz sīkas sirds to gļēvie nedrīkst vazāt.\\
Met projām,\\
Un vienalga, kaut tā dubļos krīt, ---\\
Tā dubļos tomēr notraipīsies mazāk,\\
Var tavas rokas vairāk aptraipīt!

Nu ņem un met!\\
Bet kāpēc sirds tev mokās?\\
Varbūt tu acu priekšā redzi jau,\\
Ka ceļas visu tavu draugu rokas\\
Un komjauniešu rindās tevis ---\\
Nav!

Tev gribas kaut vai dienu attaisnot\\
No gadiem taviem,\\
Un saki, ka ir skaisti sapņi tev,\\
Bet vai tā ir,\\
Ja tu no sapņiem saviem\\
Smel siltumu un gaismu tikai sev?

Nē, velti, velti bezspēcīgā spītā\\
Pret sirdi cīnies ---\\
Sirds tev skaidri jūt:\\
No dzimtās zemes sapņiem atdalīti\\
Nevienam šodien sapņi nedrīkst būt!


\newpage

{\bf Dīvainā pastaiga}

Klāj māte steigā galdu. Vakariņām?\\
Pa kuru laiku vakars atkal jau?\\
Tēvs nerunā. Ir pašreiz grūti viņam.\\
Tu nerunā. Tev arī viegli nav. 

Tēvs tomēr reizēm slepus vēro tevi\\
Un ļoti cenšas neizrādīt to,\\
Viņš taču zin, ka, cīnoties ar sevi,\\
Ir uzvaru visgrūtāk uzkarot. 

Nav par šo cīņu grūtākas nevienas,\\
Un māte pat vēl apgriezusies nav,\\
Kad tava mēteļa vairs nav pie sienas\\
Un tevis paša istabā vairs nav. 

Bet drīz vien māte jau ar lielām bažām\\
Sāk just, ka tās nav labas pastaigas ---\\
Nupat uz naglas vēl bij tēva kažoks,\\
Bet pazudis ir pēkšņi arī tas. 

... Ar niknumu, kas laikam ilgi krājies,\\
Pār zemi satrakoti vēji skrien.\\
Bet tev?\\
Uz kuru pusi tev ir jāiet?\\
Visviens. 

\newpage 

{\bf Tagad gadi mainās}

Krāc dzimtā puse sniega vētras skavās.\\
Tu nedzirdi, kā koki vētrā dzied,\\
Jo senas dienas garām acīm tavām\\
Kā vēja trenkti pārslu bari iet. 

Un bijušais\\
Vēl tādu bargu skatu\\
Un tādu tiesu līdz šim nepazīst ---\\
Kas bijis labs, to sirds šai naktī patur,\\
Kas bijis slikts ---\\
To aizmet prom un nīst. 

Daudz sīku dienu ir gar acīm traucies,\\
Un daudzu nakšu miegs\\
Ir bijis velti zagts...\\
Un lūk, jau nāk šis Jaunā gada brauciens.\\
Nē...\\
Nē! Tā nebij Jaunā gada nakts!

Tur bija daudz visdažādāko nieku ---\\
Gan zaķis bija kaut kur ceļmalā\\
Un muka projām, tev par lielu prieku...\\
No Jaunā gada nakts tur nebija nekā!

Bij tosti, jā.\\
Par daudzām lielām jomām,\\
Par vilcieniem, kas tālā ceļā skrien,\\
Bet galveno ---\\
Par to, ko tagad domā, ---\\
Tai naktī tostu necēla neviens. 

Vēl svētku nav, kad darba drēbes pakar\\
Vai kalendāra beigu lapa krīt, ---\\
Tie sākas tad,\\
Kad pārskatījis savu ``vakar'',\\
Pilns spēka pievērs seju\\
Savam ``rīt''!

... Traks vējš ir visus sniegus uzvandījis,\\
Un acīs triecas pārslu trakā takts ---\\
Tev Jaunā gada nakts nav tiešām bijis,\\
Ja ir ---\\
Tad šī ir Jaunā gada nakts!

\newpage

{\bf Nav divu siržu}

Tu vari bū par pašu dzelzi cietāks,\\
Bet tomēr būsi nemanāms un sīks,\\
Ja nenostāsies tajā dzīves vietā,\\
Kur tu visvairāk esi vajadzīgs. 

Bet, ja tu tikai meklēsi arvienu,\\
Kur pašam siltāk un tev vieglāk ies,\\
Tev droši nāksies atkal kādā dienā\\
Ar savu sirdsapziņu cīnīties. 

Būs atkal jāpārskata dzīve tava,\\
Būs daudz kas jānomet,\\
Pirms tālāk steigt,\\
Jo ``divas sirdis cilvēkam jau nava'',\\
Kā tas ir kādā labā dzejā teikts. 

Tev apkārt dzimtā puse sniega vētrā krāc.\\
Ir tikai viena sirds.\\
Un tā pukst krūtīs tavās,\\
Neko tev nepārmet --- tu viņai patīc tāds.

No dzīves gadus nosvītrot ir grūti ---\\
Daudz viņu nav\\
Un ātri viņi skrej ---,\\
Bet reizēm jānosvītro...\\
Un ar stipru sirdi krūts,\\
Ar dzimtās zemes ilgām krūtīs\\
Tālāk ej. 

Sirds lika gadus mainīt te,\\
Kur pajumts\\
No vētrām sargāja, kad biji mazs un vārgs,\\
Un tev uz mūžu kļūst šis sniega klajums\\
Līdz pašiem dziļumiem,\\
Līdz sāpēm dārgs. 

Un it kā smaga slimība ir cauri ---\\
Tev gribas kājās celties,\\
Gribas lielam kļūt ---\\
Par dzimtās zemes horizontiem šaurāks\\
Nevienam šodien skatiens nedrīkst būt!


\newpage 

{\large \sc Pāri pasaulei --- saule}

{\bf Mums būs svētki}

Pāri pasaulei --- saule,\\
Tās staros ik asnam ir silti,\\
Ziedoņa nemiers visapkārt,\\
Ziedoņa nemierā sirds.\\
Tā šeit izauga liela\\
Uz bērnībā mītajām smiltīm,\\
Un bērna apvāršņu vietā\\
Milzīgi apvāršņi mirdz. 

Pāri pasaulei --- vēji\\
Ar ziedoņa pērkona dārdiem,\\
Baltās zibšņu ugunīs\\
Debesis kaist.\\
Un ar vārdiem ``es mīlu'',\\
Ar jaunības dārgākiem vārdiem,\\
Pāri pasaulei nolaižas\\
Ziedošais maijs. 

Mums tas iesākās martā,\\
Kad zeme jau atkusnī valga,\\
Bet reizēm veltīgā spītā\\
Cīruļu puteņi elš,\\
Mums tas bij martā,\\
Jums tas bij maijā --- vienalga,\\
Satikās divi, un viņiem\\
Iesākās kopīgais ceļš. 

Pims tam bija puteņi asi,\\
Pērkona lieti lija,\\
Bet tie aizgāja projām,\\
Sastopot abus mūs,\\
Aizgāja visi tie tālē,\\
Kur mēs saucām par ---\\
``Bija'',\\
Un priekšā pavērās tāle,\\
Kuru mēs saucām par ---\\
``Būs''.

Satumst.\\
Uz tikšanos sirds\\
Savos spārnos mūs aiznes,\\
Satiekas rokas,\\
Un acīs tāds mīļums spīd ---\\
Redzi, nokaunas\\
Debesīs mirdzošās zvaigznes\\
Un pat dažas --- tu redzi ---\\
Skaudībā krīt. 

Lielāka prieka par mūsējo\\
Uz zemes, šķiet, nav nevienam ---\\
Tad, kad vēji pār rudziem\\
Ziedu mākoņus dzīs, ---\\
Tie nebūs vienkārši svētki,\\
Tā nebūs vienkārša diena,\\
Mīļotā, drīz...

... Un mēs klusējot raugāmies\\
Ielas asfaltā cietā,\\
Zinot, ka mūsu kāzās\\
Istaba jautrībā dūks\\
Un ka pie  galda nebūs\\
Nevienas tukšas vietas.\\
... Bet mūsu jaunības svētkos\\
Viena cilvēka trūks.

Kāpēc viņa te nav,\\
Kur viņš varēja pazust?\\
Varētu viņš pie šī galda\\
Vislielāko jautrību sēt.\\
...Reiz viņam izstiepi pretī\\
Savus pirkstiņus mazos,\\
Pirmo vārdu šai pasaulē\\
Viņam tu sacīji:\\
--- Tēt! ---

Pēc tam līksmi tu skrēji\\
Pa savu rotaļu taku,\\
Tēvam, kas pārnāca vēlu,\\
Seja bij cieta kā varš.\\
Kāpēc viņš šineli toreiz\\
Pie gultas sev nolika blakus,\\
Kāpēc aiz loga bij dziesma:\\
--- Ja mums rītu draud karš... ---



\newpage

{\bf Bij tādas dienas}

Apklust radio,\\
Kaut būtu sakāms daudz,\\
Simtiem staciju peronu drebina:\\
--- Uz cīņu, zeme varenā,\\
Uz nāves kauju trauc! --- 

Noliec malā, draugs,\\
Kāzām pirkto kausu,\\
Paņem kareivja krūzīti,\\
Pielej un izdzer sausu.\\
Dzirdi ---\\
Dzimtene sauc.

Māte, noskūpsti dēlu\\
Un neraudi daudz:\\
Uz cīņu, zeme varenā,\\
Uz nāves kauju trauc! ----

... Palika pusvārdā dziesma,\\
Palika nepļauta pļava,\\
Latvijā toreiz ne stūrītis\\
Nebij no dūmiem tīrs.

Maskavā\\
No tavas mātes,\\
No mīļotās sievas skavām\\
Pirmoreiz dzīvē\\
Ar spēku izrāvās vīrs...

Kaut gan pēc mīļuma todien\\
Sirds krūtīs kliedza visskaļāk,\\
Šineļos tērptie atgrūda rokas,\\
Kas tik cieši vij...\\
Toreiz tās, likās,\\
Ja satvers,\\
Tad mūžību nelaidīs vaļā,\\
Bet līdz atvadu svilpei\\
Dažas sekundes bij. 

Pavisam mazai tev nācās\\
Pie frontes vilciena stāvēt,\\
Kam pēc minūtēm dažām\\
Kaut kāds N virziens būs dots.\\
... Todien Vidzemes debesīs\\
Riņķoja krustaina nāve\\
Pār mūsu puišeļa galvu\\
Bet vai tēvs zināja to?

Var būt, ka iešāvās doma?\\
Nezinu. Mirušie klusē.\\
... Gara kā liesmaina čūska\\
Izstiepās fronte starp mums,\\
Savilkās bargi mākoņi\\
Pār mūsu galvu šai pusē,\\
Pār tavām bizītēm apmācās\\
Taigas debesu jums.

Divas mājas bez tēviem:\\
Tavs --- pie Maskavas sniegos,\\
Kur mīnu rūcošās šķembas\\
Decembra sērsnu ar;\\
Manējais --- Vidzemes mežos,\\
Līdz padusēm kupenās stiegot.\\
... Cīnās tēvi,\\
Lai bērni dzīvot var. 

Asiņo abi un nezi,\\
Pārciešot milzīgas mokas,\\
Ka uz to izcirstā ceļa,\\
Kas sprādzienu ugunīs mirdz, ---\\
Uz mūžu kopīgam ceļam\\
Satiksies bērnu rokas,\\
Uz mūžu kopīgiem pukstiem\\
Satiksies bērnu sirds. 


\newpage

{\bf Raiņa bulvārī}

Šeit mēs iesākām dziesmu,\\
Kuru negribam mainīt,\\
Šeit mums iesākās ceļš,\\
Kurš pretī nākotnei trauc. 

Liekas, ka pretī nekas\\
Nebūtu arī Rainim,\\
Ja par tikšanās bulvāri\\
Viņa bulvāri sauc.\\
Šodien debesīs sajucis ---\\
Sniegs nāk kopā ar lietu,\\
Jaungads pie durvīm jau,\\
Ziemas vēl nav un nav.\\
... Bet, lai kurp šodien abi\\
Mīļotā pilsētā ietu, ---\\
Vairās ar manējo sastapties\\
Skatiens tavs. 

Bija pirms gadiem\\
Šī pati decembra diena,\\
Nekādās gadu tālēs\\
Šī diena nevar dzist.\\
Ilgs bij klusums,\\
Tad frontes vēstule viena:\\
--- Brauciet steidzīgi. ---\\
Viss. 

Jā, tas bij šodien...\\
Kad caur puteni sīvo\\
Paspējāt atlidot laikā.\\
Tikšanās brīdis bij īss ---\\
Pārāk maz stundu šai pasaulē\\
Tēvam bij atlicis dzīvot,\\
Ārsts teica: --- Divas, trīs... ---

Sēdēja karavīrs kāds,\\
Droši no tēva vada,\\
Decembra puteņa plūkāts,\\
Krāca aiz loga mežs.\\
Lielā vētrā pat koki\\
Viens otram ir rada,\\
Cilvēks, kas sāpēs sēž blakus,\\
No tā brīža\\
Nav svešs. 

--- Esiet laimīgi! ---\\
Tēvs kļuva pēkšņi bālāks...\\
Ilgi tu glaudīji roku,\\
Kas kļuva vēsa jau.\\
Jā, no tālumiem visiem\\
Tikai šis tālums\\
Vistālāks.\\
Tikai pēc šķiršanās šīs\\
Nekad vairs tikšanās nav. 

Redzot, kā cilvēku laimei\\
Karš cērt nāvīgus robus,\\
Cik ļoti satriekta māte,\\
Kā meitas bizītes trīc,\\
Karavīrs izgāja laukā\\
Nobālis, zobus griežot,\\
Naidu,\\
Briesmīgu naidu\\
Uz fronti paņēmot līdz. 


... Pagāja gadi, un šodien\\
Raiņa bulvārī ejam ---\\
Mūsu tikšanās bulvārī,\\
Un tālā diena tā\\
Stingrākus vaibstus ievelk\\
Tavā jaunajā sejā,\\
Domājam abi par vienu,\\
Nerunājot nekā. 

Zinu, kāpēc šai dienā\\
Dziļās pārdomās slīgi,\\
Nākošos gadu desmitos\\
Centies saskatīt mūs, ---\\
Jo ir atkal\\
Pasaulē nemierīgi\\
Un jau nopūšas dažs:\\
--- Nezin, kas rītu būs... --- 

Taisnība --- ir kaut kur pasaulē\\
Tādi, kas grib mums\\
Rakt bedri,\\
Kuri mūs slaucīt no zemes\\
Par visu vairāk tvīkst.

Nepietiek, ja mēs tiem šodien\\
Liekam sadzirdēt:\\
--- Nedrīkst! ---\\
Viņiem ir sajucis sen,\\
Ko tie nedrīkst un drīkst. 

--- Esiet laimīgi... ---\\
Todien,\\
Kad vēl pie Maskavas šāva,\
Teica mirstošais tēvs,\\
Domādams visus mūs, --- \\
Tāpēc mums skaidrībā jābūt\\
Ar visu, kur šodien stāvam,\\
Un ar visu, kas rīt būs,\\
Ar visu, kas parīt būs. 


\newpage 

{\bf Himna jaunībai}

Pāri pasaulei --- saule,\\
Peld mākoņi debesu dzīlē,\\
Varenus lokus pār pļavām\\
Palu ūdeņi liec,\\
Reibina zemes smarža,\\
Kad aprīļa vakars sauc mīlēt,\\
Un tu pirmoreiz kļūsti\\
Kā izplaucis zieds.

Pāri pasaulei --- vēji,\\
Kas kokiem rauj pumpurus vaļā.\\
Kuriem sirds valgumu uztic\\
Mostošās zemes krūts.\\
... Kaut kur granātas gaudo,\\
Bet par visu\\
Pasaulē skaļāk\\
Raiņa bulvārī sula\\
Liepu stumbros dūc. 

Paklausies savu sirdi,\\
Tā dzied dzīvībai slavu,\\
Draugs,\\
Kā zeme pēc saules ---\\
Pēc laimes ilgojas tā,\\
Laimei mēs tiecamies pretī\\
Ar visu jaunību savu,\\
Un par laimi uz zemes\\
Dārgāka nav it nekā.\\
Un, ja tā,\\
Tad mēs nevaram\\
Palikt klusi un mazi,\\
Kad pie pamales kaut kur\\
Melni mākoņi tūkst,\\
Kad priekš drauga vai tevis\\
Kaut kur aiz jūras trin nazi,\\
Lai atkal mūsu dzīvē\\
Dārgu cilvēku trūkst. 

Nevar klusēt,\\
Vārdi dedzina krūtis,\\
Nedod mieru\\
Krūtīs vētrainā takts.\\
...Vai tēviem Ziemas pilī\\
Ielauzties nebij grūti?\\
Kā bij Sivašā toreiz ---\\
Jūra, lodes un nakts!

Atceries, vecie kā gāja\\
Pa Eiropas ceļu oļiem,\\
Kad ``Katjušu'' pa radio\\
Kādreiz dzied,\\
Un viņu jaunība\\
Ar saviem liesmainiem soļiem\\
Vēlreiz caur veco sirdi\\
Atmiņu parādē iet. 

Klus tikai tie,\\
Kuri bez skaita\\
Svešos un dzimtajos pakalnos\\
No Naras līdz Berlīnei dus.\\
Viņi uz šodienu nāca,\\
Bet pusceļā aprāvās gaita ---\\
Nē, nav taisnība,\\
Ka viņi klus!

Savu ir teikuši sirmie\\
Un gaida, ko jaunība sacīs,\\
Tā, kuras laimei aizdegtas\\
Nākotnes ugunis,\\
Tā, kura visam, kas notiek,\\
Grib skatīties tieši acīs ---\\
Pasaulē nemierīgi?\\
Jaunībai jāredz viss!

Jaunība veido šīs dienas,\\
Zin cīņas, kādās tās tapa,\\
Zin, cik dārgs ir\\
Mierīgais debesu jums,\\
Un ar darbu un dzīvi\\
Tā šodien ceļ Zoju no kapa,\\
Ceļ no pakalniem visus,\\
Kas gāja un krita par mums. 

Kur ir tie, kas grib\\
Mūsu dzīvi grūst\\
Kara liesmainā smēdē?\\
Lai skatās šurp,\\
Kādu draudzību iekur sirds:\\
Tāda --- sprādzienu mutuļos,\\
Tāda --- zem tanku ķēdēm,\\
Arī stingušā sejā\\
Tīra un patiesa mirdz! 

Kur ir tie, kas ar briesmām\\
Cenšas iebaidīt mūs,\\
Stāstot, ka mēs šajos draudos\\
Palēnām drūpam, ---\\
Kas viņus paglābs tai dienā,\\
Pasakiet, kurš tas būs,\\
Kad mēs iesim\\
Ar šķiršanās skūpstu uz lūpām!

Ja pēc mīļuma kliegs\\
Sirds tajā dienā visskaļāk,\\
Stipri kā mūsu tēvi\\
Arī mēs būsim tad,\\
Kad no mīļotām rokām\\
AR spēku jāraisās vaļā,\\
Nezinot, kuru no mums\\
Tās vairs neskaus nekad.

Un ikviens, kas grib nelūgts\\
Līst mūsu nākotnes ēkā,\\
Lai zin, ka jaunība mūsu\\
Tās pamatos likta jau,\\
Ka par mūsējo nav\\
Pasaulē lielāka spēka\\
Un par mūsējo taisnību\\
Lielākas taisnības nav!

Pāri pasaulei --- saule,\\
Un nemiera vēji mūs skar,\\
Tiem, kas šo laimi grib atņemt,\\
Nāksies saprast un zinā ---\\
Visu, ko jaunība mīl,\\
Pie sirds tā sasildīt var,\\
Visu, ko jaunība nīst, ---\\
Ar sirdi var sadedzināt!



\newpage

{\large \sc Prologs}

Strautiņš\\
Kalnu spraugā\\
Pilienu pie piliena krāja,\\
Līz bedrē tam kļuva par šāuru:\\
Izkāpa viņš no bedres\\
Un gāja. 

Lēca pār akmeņiem,\\
Caur spraugām lauzās,\\
Sākumā liecās pauguriem apkārt,\\
Bet vēlāk ---\\
Cauri grauzās. 

Un visu ceļu\\
Pilienu pie piliena krāja.\\
Nenogurstot\\
Līdz pirmajam ūdenskritumam\\
Gāja. 

Lēca lejā,\\
Sakūlās putās balts,\\
Lieliem akmeņiem traucās pret pakausi,\\
Nenodarot nekādu grēku, ---\\
Lai šalc!

Lielie akmeņi sakustējās,\\
Un strautelis\\
Pirmoreiz sajuta savu spēku...

Atpakaļ paskatījās ---\\
Smieklīgi:\\
Apiets no sākuma\\
Katrs smilšu graudiņš. 

Nē, tā nevar!\\
Par sīku iets ---\\
Jāsāk no gala!

Skrēja tālāk,\\
Akmeņus gāza no ceļa,\\
Lielus akmeņus gāza,\\
Neskrēja vairs,\\
Bet brāza.

Atpakaļ paskatījās ---\\
Smieklīgi:\\
Apieta katra akmens grēda,\\
Katra klints mala. 

Nē, tā nevar!\\
Par sīku iets ---\\
Jāsāk no gala!

Brāzās tālāk,\\
Un klints gabali drupa\\
Cits aiz cita\\
Čupā!

Vairs ne strautiņš,\\
Bet varena upe\\
No kalnāja jūrā krita,\\
Un jūrā tai dienā\\
Nebija mierīgi ---\\
Viļņi mazgājās putu pienā.\\
Skatījās atpakaļ upes ūdeņi ---\\
Šaura, klintīs izlauzta iela. 

Bet --- jūra!\\
Tāda spēcīga,\\
Tāda liela,\\
Ka tās bangu briesmīgos sitienos\\
Sadrūp neuzmanīga sala.

Nē, tā nevar!\\
Par sīku iets ---\\
Jāsāk no gala!

Un, ja mēs klausāmies\\
Savas trauksmainās asins ritumā\\
Vai mums nav kā strautam\\
Pirmajā ūdenskritumā?






}











\end{document}
