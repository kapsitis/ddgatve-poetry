\documentclass[14pt]{extarticle}
\usepackage{ucs}
\usepackage[utf8x]{inputenc}
\usepackage{changepage}
\usepackage{graphicx}
\usepackage{amsmath}
\usepackage{gensymb}
\usepackage{amssymb}
\usepackage{enumerate}
\usepackage{tabularx}
\usepackage{lipsum}

\setlength{\parskip}{\baselineskip}%
\setlength{\parindent}{0pt}%

\oddsidemargin 0.0in
\evensidemargin 0.0in
\textwidth 6.27in
\headheight 1.0in
\topmargin 0.0in
\headheight 0.0in
\headsep 0.0in
%\textheight 9.69in
\textheight 9.00in

\setlength\parindent{0pt}

\newenvironment{myenv}{\begin{adjustwidth}{0.4in}{0.4in}}{\end{adjustwidth}}
\renewcommand{\abstractname}{Anotācija}
\renewcommand\refname{Atsauces}

\renewcommand{\baselinestretch}{1.2}

\newenvironment{uzdevums}[1][\unskip]{%
\vspace{3mm}
\noindent
\textbf{#1:}
\noindent}
{}

\newcommand{\subf}[2]{%
  {\small\begin{tabular}[t]{@{}c@{}}
  #1\\#2
  \end{tabular}}%
}



\newcounter{alphnum}
\newenvironment{alphlist}{\begin{list}{(\Alph{alphnum})}{\usecounter{alphnum}\setlength{\leftmargin}{2.5em}} \rm}{\end{list}}


\makeatletter
\let\saved@bibitem\@bibitem
\makeatother

\usepackage{bibentry}
%\usepackage{hyperref}


\begin{document}

\begin{center}
{\LARGE \bf Dzegužlaiks}
\end{center}


{\large \sc * * *}

Kāds mani atkal sauc uz satikšanos. 

It kā pēc izcirtuma izklausās,\\
Pār kuru\\
Ir atkal saulē notrīcējis gaiss. 

It kā pēc mežu pļavās aizmirstajiem šķūnīšiem,\\
Kas stāv un stāv kā vecas baravikas\\
Un reiz no vientulības\\
Saļogās un krīt. 

Tā balss ir veca.\\
Es reiz dzegužlaikā dzirdēju,\\
Ka manu vectēvu\\
No pamales tā sauca. 

Un vectēvs aizgāja.\\
Un līdz pat mūža mājām\\
Nekādi labumi, ne sliktumi\\
To nevarēja atsaukt\\
No turienes, kurp aicināja balss.

Es izliekos, ka nedzirdu,\\
Bet dzirdu.\\
Es izliekos, ka neredzu,\\
Bet klūpu\\
Pār manā ceļā izliktajiem staipekņiem. 

Un krītu\\
Ar seju atmiņpilnos vaivarājos.

Un tad?\\
Un tad --- rīt sāpēs galva\\
Un tāla balss uz satikšanos sauks. 


\newpage

{\large \sc Ievads}

Piedod, dzimtene,\\
Ka tu man biji\\
Kartona vākos,\\
Kādās divsimt lappusēs,\\
Un pusmiljonā\\
Miega un nomodas sapņu.

Kā tu varēji paciest,\\
Kad es tev skatījos acīs,\\
Bet skatījos garām kā blēdis ---\\
Izdomātajās?

Kā tu varēji saprast,\\
Kad čukstēju ausī tev,\\
Bet zemes vēji noplēsa vārdus\\
No stērbeles,\\
Pie kuras es biju pieklupis?

Un, dieva dēļ,\\
Kur es turēju roku,\\
Kad zvērēju tev, ---

Varbūt asnam uz galvas?\\
Varbūt skudrai uz ceļa?\\
Varbūt sev\\
Un saviem draugiem uz acīm?

Tā ir briesmīgi runāt,\\
Bet vēl briesmīgāk būtu ---\\
Dzīvot,\\
To nesaprotot.

Tagad es zinu,\\
Ka man vairs nav jākāpj Everestā,\\
Lai redzētu.

Es biju kalnā.

Tagad es zinu,\\
Ka man vairs nav jāatkāpjas,\\
Lai saprastu.

Es biju tālu.

Es varēju atnākt noguris,\\
Bet esmu atnācis izsalcis.


\newpage

{\large \sc Dziesmu grāmata}

Vecāmāte\\
Man varēja atstāt\\
Šo vienīgo mantu,\\
``Ko rūsa un kodes maitā''.

Un vēl pāris bezkrāsas brunču,\\
Kuru krāsa izbalējusi\\
Spiedīgu vasaru saulē.

Ko es varēju iesākt ar tādiem\\
Un vispār --- brunčiem,\\
Bet dziesmu grāmatu\\
Paņēmu.

Nelasiet tur tās blēņas.\\
Es vienīgais zinu,\\
Kas tur ir iekšā.

\newpage

{\bf Dievs kungs ir mūsu stipra pils ---}

Un vienīgi dievs tas kungs.\\
Jo nelīdzēja nekādi klīsteri\\
Un janvāra vēji pa istabu staigāja\\
Tāpat kā pa āru.

Un arī no šīs\\
Mūžīgi izvēdinātās pils\\
Saimnieks, kad tikai gribēja,\\
Varēja izsviest.

Par to,\\
Ka viņš atradis citu kalpu,\\
Par to,\\
Ka nav laikā izslauktas govis,\\
Par to,\\
Ka ar saimnieka dēlu neiet uz šķūni,\\
Par to,\\
Ka vienkārši gribas izsviest.

Un tad paliek dievs tas kungs\\
Un pazīstams meldiņš,\\
Kur savu nedienu kucēnus slīcināt.

Un tad nu ir, pie kā tverties,\\
Jo vīrs būs no meža mājās\\
Un, nosviezdams sveķainos cimdus,\\
Prasīs:\\
--- Nu ko, lika mierā? ---

Un tad būs klusēšana.\\
Un tad ---\\
Būs viss miers pie velna\\
Un meža cirtēja cirvis dauzīsies\\
Saimnieka durvīs.

Pēc tam visas pinckas būs kupenā\\
Un pāri zvaigžņota janvāra debess\\
Un stipra pils,\\
Un vienīgi dievs tas kungs.


\newpage 

{\bf Ak, galva asiņaina...}

Viena pēc otras.\\
Pie vienas būs vainīga priede,\\
Pie otras ---\\
Dūmi aiz meža.

Cik priecīgi barona mēbeles,\\
Tik baismīgi šautenes sprakstēja.

Ak, galva,\\
Ak, galvas,\\
Ko jūs augāt uz trakiem pleciem!...

Ak, galva ---\\
Un šoreiz ne Jēzus Kristus,\\
Bet pašas galvā\\
Ar katru asiņaino\\
Ir deniņos sadzīti ērkšķi,\\
Tik asi,\\
Ka tajā vietā\\
Aug vienīgi sirmi mati.

Ak, galva,\\
Tu palice viena\\
Un atbildi\\
Par trim mazām galviņām.

Tu tausti\\
Un nevari saprast,\\
Vai arī zem tām\\
Pleci nav traki.

Un laikam jau ir\\
Un laikam jau būs\\
No tā paša kaula,\\
No tās pašas miesas.

Un priežu bez jēgas\\
Un muižu arī vēl pietiekoši.

Ak, galva asiņaina!\\
Bet arī asiņainai tev\\
Jādomā.


\newpage

{\bf Pie rokas ņem un vadi ---}

Ņem, ja reiz es tā saku,\\
Un vadi.\\
Bet, kad es skatos, kur mani ved,\\
Tad jākliedz ---\\
Lai velns tevi parautu,\\
Kungs!

Pie rokas ņem,\\
TIkai nesien.\\
Uz kā es nesīšu savus bērnus!

Tu tos pat neesi paturējis,\\
Kad gāžos no kājām.

Un es nesūdzos, kungs,\\
Viņiem ne reizi vien būs jākrīt.

Uz savas zemes\\
Tu savām avīm\\
Esi saracis tik daudz\\
Vilka bedru,\\
Ka neviens nevar iztikt\\
Bez krišanas.

Lai zin, kas ir krist

Pret zemi nosisties\\
Vienmēr ir svētīgāk\\
Nekā pret nelieša rokām.

Krītiet, mani bērni,\\
Es papūtīšu uz ziluma ---\\
Tā nav nemaz tik briesmīga krāsa.\\
Briesmīga ir tā rozā\\
Uz taukiem purniem.

Ja ņem pie rokas,\\
Tad izraujiet roku,\\
Ja izraut nevar ---\\
Noraujiet roku,\\
Bērni.

Tu nekur nevari aizvest,\\
Kungs, mani pats,\\
Bet es gribu, lai vienu mirkli\\
Kāds mani ved\\
Un lai es\\
Varu ticēt,\\
Ka mani neved uz bedri.

Kungs,\\
Tā ir liela laime ---\\
Ticēt kaut kam.


\newpage

{\bf Vāks ---}

Pret visādiem pasaules galdiem\\
Un pret ceļgalu deldēts,\\
Un deldēta sāpe\\
Un pārestība,\\
Un vientulība,\\
Un izmisums,\\
Un bezcerība,\\
Un apspiests kliedziens:\\
--- Dzīve,\\
Kāpēc tev ir tik cieta sirds,\\
Kāpēc tu man nedod\\
Nevienu gabaliņu,\\
Nevienu drusciņu,\\
Nevienu puteklīti,\\
Un kāpēc man visu\\
Pa melnu nakti,\\
Pa šautriem sapņiem\\
Vajag izdomāt, izdomāt, izdomāt ---\\
Un no rīta\\
Tu atnāc ar koku\\
Un sasit...




\newpage

{\large \sc Vectēvam}

Vai tu atceries vēl, kalēj,\\
Večuku ar baltu bārdu?\\
Iekal, dzeni, labi dziļi\\
Resnā priedē viņa vārdu.

Stāmerienas muižas liesmās\\
Baltā bārda rūsga svila,\\
Vēl no Piektā gada dūmiem\\
Gadu pamale ir zila.

Viņa alus mucu piemin\\
Tajā pusē, kur es augu,\\
Un vēl katru pavasari\\
Zeme rūgst ar viņa raugu.

Vēl to dēlu asins piles,\\
Kuri dzima muižas vergam,\\
Atmin saplosītā zeme\\
Breslavā un Kēnigsbergā.

Vēl nav kodes saēdušas\\
Viņa gadu meitu sagšas,\\
Vēl jau viņa taisīts kāts ir\\
Dēladēlu siena dakšās.

Viņa vārdam akmens neder.\\
Iekal dziļi priedē, dzeni,\\
Tāpēc ka viņš pats jau bija\\
Ļoti līdzīgs laukakmenim.


\newpage

{\large \sc Mantojums}

Pletne, ar ko manu vectēvu pēra,\\
Guļ muzeja zālē zem stikla.\\
Lode, kas bija domāta tēvam,\\
Ieurbās priedē,\\
Un vēl tagad tek sveķi.

Pletne vēl atmiņas uzpātago,\\
Un lode vēl kādreiz zāģim\\
Izlauzīs zobus.

Bet tas arī ir viss.

Zilu velvi pār galvu es esmu saņēmis,\\
Steigā dunošu zemi\\
Tēvs un vectēvs zem kājām man nolicis.

Un man pa naktīm sapņos nāk ---\\
Apvārsnis,\\
Apvārsnis,\\
Apvārsnis....

Es dzirdu vectēva domīgo balsi,\\
Es redzu tēva pētošo skatu,\\
Es jūtu mātes saudzīgo roku.\\
Un tie it kā saka:

--- Tev sākas liela cilvēka sapņi.\\
Tu esi pārkāpis slieksni,\\
Un tagad tev jāpārkāpj apvārsnis.


\newpage

{\large \sc Dzērves}

Ap manu bērnību apkārt ir apkritis\\
Pelēku lapu pelēks laprkitis.

Kokos lapas vēl bija.\\
Krita dzērves.\\
Un fronti jau varēja dzirdēt.

Droši vien no šīs skaņas\\
Katrs automāts kļuva ārprātīgs\\
Un meklēja visu dzīvo.

Kā vāgūzī\\
Uz zirgu sviedros sabrauktām sakām\\
Palika Taņa ---\\
Es neredzēju.

Es redzēju, kā krita dzērves,\\
Kā sniga līdz pašam tālākam mežam\\
Pelēkas pārslas.

Kā piesēja kaķi pie ozola\\
Un kā viņš kūleņoja ---\\
Es redzēju.

Pār krūmu, kurš mani slēpa,\\
Krita pelēki sniegi,\\
Pār vāgūzi krita,\\
Pār kaķa raibajiem sāniem,\\
Pār visiem, kas krita,\\
Dzērvju pelēkās spalvas krita.

Un neļāva kliegt\\
Un neļāva nostāties tiem,\\
Kam automāts nebija ieskatījies\\
Vēl acīs.

Es nezinu, vai tā ir diena vai gads,\\
Bet manī ir palicis nomiris laiks.\\
Tas ir šis,\\
Kad ap bērnību bija apkritis\\
Pelēku lapu pelēks lapkritis.


\newpage

{\large \sc Minhenes tosti}

Minhenes pagrabos\\
Atkal alu dzer\\
Pirms un pēc vārda,\\
Pirms un pēc cīsiņa.

Atmiņa, dzer!\\
Lai šīs zāles tev der.\\
Lai tu kļūsti, kā vajag, ---\\
Īsiņa.

Pieskandini, mans čehu draugs,\\
Ko tur šņākt,\\
Daudz jau par visu to\\
Īdēts.\\
Viss ir tik tālu,\\
Ka var sākt.\\
Viņiem --- kārtībā durkļi,\\
Un tev ir kārtībā lādiņi.

Pieskandini, mans angļu draugs,\\
Atmiņas rūs.\\
Zārkiem pietiekoši\\
Ir savākts cinka,\\
Un tas jau nav svarīgi,\\
Kur tas būs,\\
Nav taču obligāta\\
Dinkerka.

Pieskandini, mans vācu draugs,\\
Jau ir pārkausēts tavs tēva tanks.\\
Saskandināsim,

Lai krūzes nosten\\
Par {\em nach Westen drang},\\
Par {\em nach Süden drang}\\
Un katrā ziņā\\
Par {\em drang nach Osten}.

Atmiņa, tu esi īsa jau?\\
Arī mums jau trauki ir sausi.

Minhenes alus pagrabos\\
Kausu nav.\\
Minhenes pagrabos\\
Stāv galvaskausi.


\newpage

{\large \sc Kadri no trofeju kara kinohronikas}

{\bf Ļoti jūsmīgs kadrs ar diemžēl laikā 
nenošautā sarkanā briesmoņa piebildēm}

Smaidiet... tautietes...

Lido puķes\\
Kā viņu rokas\\
Pret debesīm:\\
--- Heil! ---

Skatieties augšup uz viņiem\\
Kā uz debesīm,\\
Sarkstiet,\\
Lai zin, kādā krāsā\\
Pēc tam\\
Degt jūsu mājām.

Dieva dēļ, negrieziet muguras,\\
Grieziet priekšas...\\
Vēl jau tāds laiciņš ir.

Un viņu lodes ir gudras ---\\
Tās pašas prot atrast\\
Muguras.

Smaidiet... tautietes...

Nu, kas tad tas ir par smaidu?\\
Ar tādu smaidu\\
Pārdodas katra netikle...

Vismaz smaidiet, kā nākas\\
Velns rāvis ---\\
Ar jūsu smaidu\\
Vesela tauta iet pārdoties.


\newpage

{\bf Ļoti rotaļīgs kadrs, kas noteikti ir uzņemts
pirms Paulusa armijas sagrāves}

--- Alu, šņabi, friktvaseri ---\\
Reiz lai skan viss krogs!\\
Alu, šņabi, fruktvaseri ---\\
Karavīrs nav koks! ---

Blaukt!\\
Un šķūņa galā, zieniet,\\
Džimlairūdi, dzirksts.\\
Spārdās, savās zarnās pinies,\\
Džimlairūdi, zirgs.

Džimlairūdi, puto brūni\\
Sveša alus stops,\\
Spārdās svešas zemes šķūnī\\
Svešas zemes lops.

--- Alu, šņabi, fruktvaseri ---\\
Reiz lai skan viss krogs! ---

Blaukt!\\
Tas karš ir ļoti derīgs,\\
Ļoti derīgs joks.

Džimlairūdi, lielgabali,\\
Džimlairūdi, šauj.\\
Viens pussamalts grāvī paliek,\\
Džimlairūdi, bļauj.

--- Alu, šņabi, fruktvaseri ---\\
Reiz lai skan viss krogs! ---

Blaukt!\\
Bet ko ar mātēm darīt,\\
Ko tām rakstīt? Ko?

Džimlairūdi, kaujās visās,\\
Ejot slavu vīt,\\
Vienam otram arī misas,\\
Džimlairūdi, --- krīt.

Džimlairūdi, neciet kaunu ---\\
Lai var rindā stāt,\\
Labi ātri dzemdē jaunu,\\
Džimlairūdi, māt!

--- Alu, šņabi, fruktvaseri ---\\
Reiz lai skan viss krogs!\\
Alu, šņabi, fruktvaseri ---\\
Karavīrs nav koks!


\newpage

{\bf Kadrs, pilns svēta sašutuma, 
ar dažiem šo svētumu apgānošiem atgādinājumiem}

Skatieties!\\
Tur viņi lido ---\\
Četrmotorīgie cilvēkēdāji!

Klausieties!\\
Tā ir viņu --- četrmotorīgo cilvēkēdāju\\
Ķērkstošā balss.

Skatieties! Klausieties!\\
Nīstiet!

Redziet, kā viņiem vajag!\\
Redziet šo mīlīgo {\em frau}?\\
Viņa ratiņos ved savu bērnu.

... Vai tas patiešām ir bērns?\\
Bērns?\\
Tad kāpēc viņš nav uz durkļa?

Redziet, kā viņiem vajag!\\
Redziet šo mājīgo klasi\\
Un šo mierīgo ābeci!

... Vai tā tiešām ir ābece?\\
Ābece?\\
Tad kāpēc tai nav\\
Pārogļojušās lapas?

Redziet, kā viņiem vajag!\\
Redziet šīs vācu meitenes,\\
Redziet, cik stalti tās iet,\\
Cik vingri kustina rokas!

... Vai tās patiešām ir meitenes?\\
... Vai tās patiešām kustina rokas?\\
Ja tās ir meitenes\\
Un tām ir rokas ---\\
Tad kāpēc tām rokas\\
Nav sasietas ar telefona stiepli\\
Un kāpēc\\
Tās iet tik stalti\\
Pēc izvarošanas?

Skatieties!\\
Viņi ir tuvu ---\\
Četrmotorīgie cilvēkēdāji!

... Vai tiešām viņi ir tuvu?\\
Vai viņi nav klāt?\\
... Vai tiešām tie ir\\
Četrmotorīgie bumbvedēji?\\
Vai tiešām jūs pareizi redzat?

Rūc!\\
Visa pamale rūc ---\\
Veļas verdošu asiņu vilnis.



\newpage

{\large \sc Kapitulācija}

Vakar\\
Zem vārdiem ``Mēs beidzam karu''\\
Keitela pildspalvas tinte\\
Nožuva.

Šorīt\\
Sieviņa gāja pie krāsns cept maizi\\
Un palika...

Šodien\\
Pie kapa uz soliņa apsēdās vecenīte\\
Un uzsprāga...

Šovakar\\
Puika dārzā pie ābeles gāja\\
Un pazuda...

Rūras tērauds\\
Ar sievām un zīdaiņiem,\\
Un mūsu bērnību\\
Karoja.

Un pat tiem,\\
Kuri četrdesmit piektajā\\
Piedzima,\\
Karš pie šūpuļa\\
Un virs šūpuļa\\
Stāvēja.

Viņu kritušos uz lielgaballafetēm\\
Neveda.

Viņu zārku tēvs\\
Cēla plecā un\\
Aiznesa...

Vai ir karā bijuši ---\\
To viņiem šodien\\
Neprasa.

Jā, ir bijuši.\\
Kad pērkonīgās frontes norima\\
Un klusēja.\\
Kad karā kritušie zem zemes\\
Dusēja.\\
Kad zeme bumbu bedrēm zāli uzsēja,\\
Kad viņus vēl ar krūti māte\\
Baroja,\\
Viņi karoja!



\newpage

{\large \sc Nodedzinātā Rūjiena}

Katrs akmens šeit ---\\
Kapakmens:\\
Kāda cilvēka dzīves gabalam\\
Un paša cilvēka gabalam,\\
Un dažam ---\\
Pašam.\\
Un Prometejam\\
Un Olimpiskajai ugunij,\\
Un lāpu gājieniem.

\newpage

{\bf Pirmais kapakmens}

Te guļ mans darbs\\
Un laime,\\
Kādu to es biju iedomājies.

Pie santīma santīms ---\\
Tā tas izskatījās.

Paša kroplums.\\
(Limbažu mežos,\\
Pa tumsu vēl gāzdams kokus,\\
Es lauzu kāju\\
Un kļuvu klibs.)

Sievas nāve.\\
(Viņa dzemdēja mežā\\
Un turpat uz zariem nomira.)\\
Dēla bārenība.\\
(Viņš aizgāja\\
Uz Rīgas pusi,\\
Un es nezinu,\\
Cik tālu viņš ir ticis.)

Bet es uzcēlu māju.\\
Tas laikam ir viss,\\
Ko es varēju.\\
Un līdz tai naktij\\
Es biju lepns,\\
Ka devis laimi\\
Sievai un dēlam.

Mani nošāva šaipus Saulkrastiem.

\newpage

{\bf Otrais kapakmens}

Es esmu sadegusi vista.\\
Olas es dēju\\
Un pavasarī izperēju cāļus.\\
Būtu izvārījuši zupā ---\\
Tāds ir visu vistu liktenis.\\
Būtu izcepuši...\\
Bet man it nekad nebūs skaidrs,\\
Par ko mani sadedzināja.

Var būt, ka pārāk daudz\\
Ēdu graudus?\\
Bet graudus man tikai vienreiz iedeva.\\
Vai tad tas būtu par daudz?

\newpage

{\bf Trešais kapakmens}

Te guļ mana manta.\\
Žēl gan.

Vislabāk, ja viss būtu sadedzis, ---\\
Karā ne tas vien deg.

Bet nav manā dabā,\\
Ja kāds ņem manu krūzi,\\
Nes nēšos manu spaini\\
Un lepojas ar manu krēslu.

Lai pelna pats!\\
Katrs ir savas laimes kalējs.

Žēl jau ir,\\
Bet labi, ka krieviem netika.


\newpage

{\bf Ceturtais kapakmens}

Es esmu puikas skumjas\\
Par koka zirgu.

Saimnieki tā neskumst\\
Par īstiem ---\\
Kā puikas par koka zirgiem.

Koka zirgs sadegdams\\
Skatījās mūžībai acīs\\
Un bija briesmās mierīgs:\\
\mbox{}\hspace{10pt}--- I-ā, es kļūstu ogle.\\
\mbox{}\hspace{10pt}--- I-ā, es kļūstu pelni.\\
\mbox{}\hspace{10pt}Mani nes vēji.\\
\mbox{}\hspace{10pt}Es baroju koku.\\
\mbox{}\hspace{10pt}Mani sazāģē dēļos.\\
\mbox{}\hspace{10pt}No manis iztaisa zirgu.\\
\mbox{}\hspace{10pt}Un nāk atkal\\
\mbox{}\hspace{10pt}Puika... ---\\
Bet puikas skumjas es esmu.\\
Tajā naktī\\
Viņš pēdējoreiz skuma\\
Bērnišķīgi.


\newpage

{\bf Piektais kapakmens}

Te guļ mana dzimtene.\\
Šeit dzīvoja mana māte\\
Tajā naktī,\\
Kad rados.

Un, ja arī sabruks\\
Šis pusapdegušais mūrītis,\\
Kad nezāles nomāks\\
Vai citi izraks\\
Un aiznesīs savā dārzā\\
Tos ķeizarkroņus,\\
Kurus es ostīju kādreiz\\
Un nosmērēju sev degunu brūnu\\
Ar putekšņiem,

Pat tad,\\
Kad te nobrauks ar buldozeru\\
Un uztupsies jauna māja\\
Uz mauriņa,\\
Kurā pliks dibens man mirka rasā, ---\\
Te būs mana dzimtene.

Jauna mirušās vietā,\\
Bet --- dzimtene,\\
Jo nevar izkustināt no vietas\\
Pirmo bļāvienu\\
Un nav apmaināms\\
Nosmērēts deguns\\
Un rasā samērcēts dibens.


\newpage

{\bf Sestais kapakmens}

Es esmu suns\\
Un neesmu sadedzis ugunī.

Es sadegu savādāk\\
Un nosprāgu pēc kāda mēneša.

Es ilgi meklēju\\
Savus saimniekus tonakt.

Bet bija palikušas\\
Vienīgi smakas.

Malkas bluķis\\
Oda pēc saimnieka.

Viena lupata\\
Un koks ar slotu ---\\
Pēc saimnieces.

Un lejkanna dārza stūrī\\
Oda pēc viņas ---\\
Pēc saimnieka meitas,\\
Ar kuru mēs runājāmies\\
No rīta līdz vakaram.

Tā bija mana vismīļākā smaka.\\
Bet, kad viens gribēja kannu aiznest,\\
Es nedevu,\\
Un man pārsita muguru.

Es nevarēju vairs paiet,\\
Un kannu aiznesa.

Tad es gulēju\\
Un domāju,\\
Ko teikšu saimniecei,\\
Kad viņa atnāks,\\
Bet es nebūšu kannu nosargājis.


\newpage

{\bf *}

Te beidzās viena no ieliņām,\\
Un te es atradu šķiltavas\\
No zaļas čaulītes.

Un redzēju, ja tās palabo,\\
Tad vēl var šķilt.

Salabojis vēl neesmu,\\
Nēsāju kabatā\\
Un reizēm domāju ---\\
Ko man tās izšķils? ---

Vai dieviem nozagto,\\
Vai olimpisko,\\
Vai aizdegs lāpu gājiena lāpas,\\
Vai pielaidīs uguni\\
Kādai Rūjienai klāt?



\newpage

{\large \sc Bedre pie hospitāļa}

Un šoreiz es to pierakstu sev pašam,\\
Jo nav man it neviena cita veida,\\
Kā sev no atmiņas un acīm izsviest\\
To balto, sarmas mežģīņoto rītu\\
Un melno kara hospitāļa bedri,\\
No kuras sanitāri projām gāja\\
Kā divi miera laika sniegavīri...

Un nolādu to brīdi, kad es paskatījos,\\
Jo bedrē bija sasviests juku jukām\\
Gan asiņainas marles melnās skrandas,\\
Gan kaut kas sarkans, notraipīts ar brūnu,\\
Gan brūns ar sarkanu. Un viena kāja ---\\
Tik neciešami vesela, ka likās,\\
Ņem, liec sev klāt --- un vari staigāt tālāk...

Pie visa tā un visa kā bij pierasts,\\
Jo reizēm šāva bērnu acu priekšā,\\
Lai viņu acu vietā --- citas acis\\
Un viņu siržu vietā citas tiktu.\\
Bet šermuļi jau bija. Un tad pēkšņi,\\
Es nezinu, no kurienes tās radās,

Bet bariņš zīlīšu jau metās bedrē\\
Un sāka knābāt tieši balto kāju.\\
Ko mēs par cilvēkēdējiem tur muldam!\\
Ko mēs par cilvēkēdējiem tur rakstām!\\
Būs brīdis --- un mēs paši atkal rēksim\\
Un dursim, kamēr pienāks atkal diena ---\\
Par cilvēkēdējiem būs padarīti stārķi.\\
Par cilvēkēdējiem būš padarīti...\\
Ko zīlītēm lai saka, tās ir labi putni,\\
Tās ielidoja laikā, kuru devām,\\
Un ēda savu kara laika speķi.


\newpage

{\large \sc * * *}

Cik tev pāri vajag bumbu līt,\\
Mana Galaktikas kniepadatas galviņa,\\
Manu zemeslodīt,\\
Manu bumbulīt?

Es jau zinu, ne jau tu tur vainīga,\\
Vainīgs ir tas divkājainais gudrais ---\\
Apšauj četrkājainos,\\
Kārta divkājainiem,\\
Apnicis ir mīdīt tikai skudras,\\
Dzīvei taču jābūt mainīgai.

Nu, tad iesim šauties, sisties, durties,\\
Kožas viena otrai rīklē\\
Brīvība un brīvība,\\
Kauli žļakst pret kauliem.\\
Apnīk klausīties šo mūziku!\\
Un aiziet\\
Dzīvība.

Nu, un tiem, kas paliek?\\
Ja kāds paliek...\\
Kas tiem paliek?

Savu lielumu reiz apzināties\\
Kad tie sāks?\\
Nē, es neticu, ka cilvēks dzīvībai ir kronis,\\
Pats visindīgākais augs\\
Un draņķīgākais zvērs\\
Ir patiesāks.

Kādu izgudrot tev bumbu rīt ---\\
Super?\\
Super-super? ---\\
Mana Galaktikas kniepadatas galviņa,\\
Manu zemeslodīt,\\
Manu bumbulīt?


\newpage

{\large \sc  * * *}

Tu esi labs,\\
Mans dzimtais gabaliņ zemes.\\
Uz globusa pakaļas\\
Niecīgs ielāps ---\\
Tu aizlāpīji man dvēselē izšautu caurumu,\\
Lai likteņa caurvēji\\
Neizrauj dvēseli laukā.

Tu man teici ---\\
Ja ēdot kāds šūpo kājas,\\
Tad tas šūpo velnu.

Tagad viena\\
Man stāsta par dzīvi,\\
Šūpo kājas\\
Un gvelž.

Tu man teici ---\\
Ja kāds sit bites,\\
Viņš daudz ko vairāk var nosist.

Tagad es skatos uz vienu\\
Un redzu, ka viņš ir sitis.

Tu man teici ---\\
Ja zārka nesējs jūt svaru,\\
Viņš nav mīlējis nomirušo.

Nezin kāpēc\\
Man visi zārki\\
Bez svara.

Tikai to pārāk daudz.

Es odu tavējo zemi,\\
Kad tā smaržoja pēc siltas zāles\\
Un kūpošā granātas bedrē\\
Pēc sēra,\\
Un sapratu,\\
Cik līdz ārprātam tuvu es esmu bijis\\
No zvēra.


\newpage

{\large \sc Oda priedei}

Dzegužu silos --- stīga\\
Un jūrmalas vējos šķība,\\
Un Rainim --- lauztā\\
Un nesalauztā,\\
Un pirmā latviešu elektrība,\\
Un zaļi mati\\
Uz kāpu skausta,\\
Un galdi, un zārki, un grozi...

Visu, kas jāpanes lepniem,\\
Tu panes.\\
Mūžam ar vējiem tu karoji\\
Un mūžos karosi,\\
Mūžam tu vētrā\\
Kā ērģeles skanēsi.

Un vienā vētrā\\
Vai --- no viena zāģa\\
Tu dabūsi galu.\\
Bet tavas beigas\\
Nav nāves stingums un aklums.

Ja manam vectēvam būtu pietrūcis\\
Tavu\\
Ar sarkanu uguni degošu skalu ---\\
Tad lords Bairons\\
Pie manis nebūtu varējis\\
Atklibot.

Kādos zaros tad varētu\\
Manas ziemeļu dzeguzes mesties?\\
Un bez dzeguzēm ---\\
Kas tad tie ir par siliem?\\
Kādā zārkā man savus mīļos\\
Uz kapsētu nest,\\
Un kur dzintaram dzimt,\\
Ja nesastingst\\
Tavu saulaino asaru piliens?

Bez taviem ugunsgrēkiem\\
Es būtu daudz mazāk svilis.\\
Man būtu grūti elpot,\\
Ja plaušas man neiztīrītu priedājs\\
Un ja nesvilptu vēji.

Ja dzeguzes nebūtu kūkojušas,\\
Ja nebūtu šalkuši sili ---\\
Nezin vai cilvēkam ienāktu prātā,\\
Ka viņš var dziedāt.

Nezin vai ienāktu.


\newpage

{\large \sc  Tetera kāposti}

Ja tos dabū pa īstam\\
Un vēl no šmaugāka bērza ---\\
Nenosēdēt\\
Pat uz paša mīkstākā\\
Ar sūnu apzaļojuša akmens.

Tagad par to, protams, var tērzēt.\\
Kā nekā ---\\
Bērnības atmiņa...

Bet bija tā ---\\
Mēs zāģējām aviobumbu\\
Ar dzelzs zāģīti,\\
Un arī dzirkstele nezin vai\\
Būtu mums gadījusies.

Nedzirdējām,\\
Kā blakus mums piečīkstēja vāģi,\\
Un nezinājām, ka teterkāposti\\
Nedēļu sapņos rādīsies...

Bet kas tad tur bija vainīgs!\\
Ar kara padarīšanām\\
Piebāztas visas vietas\\
Ar mīnām un bumbām\\
Piebērtas visas priedes.\\
Un mēs bijām tādā vecumā,\\
Kad dod rotaļlietas, ---\\
Mēs arī spēlējāmies ar to,\\
Ko laikmets mums bija iedevis.

Un aviobumbu mēs pārzāģējām\\
Pēc nedēļas\\
Vecajos kapos.\\
Un es neteikšu, kas mūs vadīja ---\\
Slikta vai laba tieksme.

Bet katrā ziņā tur piedalījās\\
Tetera kāposts\\
Un mūsu --- pret netaisnu pēršanu ---\\
Attieksme.


\newpage

{\large \sc  Dziesma}

Ko tu domā šai pelēkā, pelēkā lietū,\\
Mana dzimtā un bezgala tālā puse?

Lapas\\
Nobirst kā notis ---\\
Un katra sev meklē vietu.\\
Kā operas partitūra\\
Ziemeļvidzemes augstiene klusē.

Stāv meži kā kori.\\
No Alūksnes līdz pašai Rīgai\\
Stiepjas vilcienu stigas\\
Ar gulšņu melnajiem taustiņiem,\\
Stiepjas tālruņa vadu\\
Tūkstoš vijoļu stīgas.

Manai dzimtajai pusei\\
Pelēkā lietū ir auksti.

Kad uzspīdēs saule,\\
Asaras nožūs\\
No bērzu zeltaini smalkām\\
Un apšu sarkanām notīm.

Tad upēs un ezeros iedzirkstīs\\
ģēnija spožums

Man ļoti gribas dzirdēt to operu.\\
Ļoti.

Kad pacelsies gaisā\\
Kapsētas nošu sēras\\
Un jaunā osēna\\
Dauzoņas svilpiens smalkais,\\
Un noslēpumainais bass ---\\
Tumšais Pārgaujas vēris\\
Savu himnu\\
Kā okeānu\\
Uz zariem šalkos.

Bet tagad līst lietus.\\
Līst dzeltenīgs lieplapu lietus.\\
Par ko tu domā\\
Šai pelēkā, pelēkā lietū?


\newpage

{\large \sc Balsij bez pavadījuma}

{\bf Pusdivos}

Dzīvi dzerot,\\
Es esmu palicis skaidrā...\\
Dodiet man vēl simt gramu sapņu.

--- Sapņu...\\
... apņu... ---

Bufete ir slēgta,\\
Un ar mani tirgojas vienīgi atbalss.\\
Tā iet, kad nezin bufetes slēgšanas laiku.

--- Laiku...\\
... aiku... ---

Atbalss, tu nemēdies,\\
Cilvēki to lietu pieprot par tevi labāk.

Šinī vietā tu mani nesakaitināsi,\\
Es zinu --- kad stulbeņi dzīves krogā vairs nedod dzert,\\
Tad var aiziet uz Nebūtības restorānu.\\
Riebīgi tikai, ka tur iekšā laiž vienīgi miroņus.

Bet, ja šim, kas stāv pie durvīm\\
Un saucas par šveicaru,\\
Kādreiz ir bijusi sirds,\\
Gan viņš iznesīs laukā.

Sapņu jau tajā restorānā nav.\\
Lai iznes pusstopu nopūtu suslas,\\
Vai nav vienalga, no kā paliek nelabi...

--- Nelabi...\\
... elabi...\\
... labi... ---

Nemēdies, atbalss, bet velc mēteli,\\
Un iesim.

\newpage

{\bf Divos}

Paklausies, kaulu kambar,\\
Es varu apgalvot, ka tu reiz esi par cilvēku saukts.

Neatceries?\\
Nu, tad tici man,\\
Pīlītes pūst es te nebūtu nācis.\\
Aizgrabi, lūdzu, līdz bufetei\\
Un atnes vienu pudeli pašas draņķīgākās kaislības...

Tu tādu dzērienu nezini?\\
Ak, pareizi, tas ir tajā otrajā krogā,\\
Bet to mūlāpi ir aiztaisījuši ciet.\\
Nu, visviens,\\
Atnes vienalga ko,\\
Ko tavi miroņi vēl nav izlakuši.

Kur tu esi aizmaldījusies, mana atbalss,\\
Tūlīt viņš nesīs,\\
Protams, tas būs kaut kas riebīgs,\\
Un pie tam dzerot būs jātaisa priecīgs ģīmis,\\
Bet vai tad mazumu mēs abi tā esam darījuši?

--- Darījuši...\\
... arījuši... ---

Skaidrs, ka darījuši.\\
Redzi, viņš jau nāk.\\
Cienījamais, nenāciet laukā,\\
Vējš sāks svilpot caur jūsu ribām,\\
Un gan jau es šinī mūzikā\\
Dabūšu klausīties līdz apnikumam.

Cik man jāmaksā?\\
Divas ciešanu kapeikas?\\
Ņemiet, cienījamais, veselu sauju.\\
Sevišķi daudz man viņu nav atlicis,\\
Bet, kamēr vēl ir,\\
Tikmēr jūsu krogā mani nelaidīs iekšā.

Saspiediet pirkstus ciešāk,\\
Jūs tomēr neesat pieradis,\\
Ka uz jūsu pirkstiem vairs sen nav miesas.

Nu, ko es teicu?\\
Manas kapeikas nobira zemē.\\
Cienījamais, lasiet tās kopā pats,\\
Atšķirībā no dzīves kroga\\
Jums ir uz pusi mazāk darba ---\\
Jums tacu vienīgi jālaiž iekšā.

No nebūtības restorāna es neesmu redzējis nevienu\\
Ne iznākam, ne izrāpojam.

Uz redzēšanos, cienījamaos,\\
Pēc cik ilga laika ---\\
To jūs paprasiet kādam citam.



\newpage

{\bf Pustrijos}

Es domāju, atbalss,\\
Ka mēs apsēdīsimies tepat uz soliņa\\
Un neiesim ne pie viena.

Ir --- pie kā gribas,\\
Bet es domāju, ka viņus nevajag traucēt\\
Tik vēlā stundā.

Es viņiem varu būt vai nebūt ---\\
Atradīs citu,\\
Un piecas minūtes tas liksies tikpat labs,\\
Tikpat vienīgs,\\
Kā es kādreiz likos.\\
Bet pēc tam...\\
Arī tas kļūs ik stundu sliktāks.

Labāk mēs dzersim divatā.\\
Tu esi mana atbalss,\\
Un, gribam vai negribam,\\
Mēs vienīgie esam pa īstam salaulāti.

Bet ko viņš mums īsti ir iedevis?

Tas ir visbrīnišķīgākais dzēriens ---\\
``Bija'',\\
Tu nemaz nevari iedomāties,\\
Cik pasakaini no viņa reibst...

Dzīves krogā arī ir ``Bija'',\\
Tikai daudz sliktāks.

Atbalss, tu, neiedomājami tukšā vecene,\\
Ja tu no šitā nekļūsi pilna...

--- Pilna...\\
... ilna... ---

Neklaigā pirms laika,\\
Šitas vēl ir cilvēku parks,\\
Un milicim tu neiestāstīsi, cik esi man nepieciešama.

--- Dzeram...\\
... eram...\\
... ram...\\
... ram... ---

Es tev teicu, nedziedi,\\
Bez manis tik un tā tev nekas nesanāks.



\newpage

{\bf Trijos}

Un tagad dziedāsim par to, kas bija.

--- Bija...\\
... ja... ---

Jā, tas reiz bija.

--- Bija...\\
... ja... ---

Zini, tu mazāk jaucies man pa vidu ---\\
Un redzēsi, ka būs labāk.

Es dziedāšu par zvaigznēm.\\
Vai tu zini, kas ir zvaigznes?\\
Faktiski tev nav ko prasīt, jo tu esi dumja.

--- Dumja...\\
... ja... ---

Pat nelabojami.\\
Mācies no cilvēkiem.\\
Viņi nekad neatzīstas šinī grēkā.

Bet es dziedāšu par zvaigznēm.

Nolādēts!\\
Mēle neklausa.\\
Tas dzēriens ir tik stiprs,\\
Ka es varu dziedāt vienīgi par nokritušajām.\\
Tātad, tas ir, par tukšu vietu ---\\
Vai arī par to spožumu, kas krītot bija?

--- Bija...\\
... ja... ---

Zini, man ar tevi ir neinteresanti,\\
Tu vienmēr man piekrīti.

--- Piekrīti...\\
... iekrīti...\\
... krīti... ---

Jā, man līdz riebumam uzticīgā,\\
Dari, ko zini,\\
Bet par tukšu vietu es neprotu dziedāt.

Ja debesīs būtu tas mirdzums,\\
Kas sensenā gadsimtā bija.

--- Bija...\\
... ja ... ---

Ja saliktu atpakaļ zvaigznes,\\
Kas rudeņu pusnaktīs lija.

--- Lija...\\
... ja... ---

Ja atdotu atpakaļ starus,\\
Ko zeme ar zvaigznājiem mija.

--- Mija...\\
... ja... ---

Nu un kas tad būtu?\\
Mēs te gaudojam kā divas vientuļas kuces,\\
Un es tomēr piederu pie vīriešu kārtas.

Pie kā mēs palikām?

--- Palikām...\\
... alikām...\\
... likām... ---

Atbalss, nu paskaties man acīs.\\
Tā jau es domāju...\\
Ja tev dod, cik tu gribi,\\
Tad tev noteikti jāpiedzeras līdz nesamaņai.



\newpage

{\bf Pusčetros}

Biedri milici,\\
Mēs neesam zagļi\\
Un, ja apzogam kādu, tad vienīgi paši sevi.

... Un te nu mēs esam, mana atbalss,\\
Kamera tāda pati kā dzīve,\\
TIkai ar to starpību,\\
Ka šeit visu var aptaustīt.

--- Taustīt...\\
Taustīt... ---

Tu runā tikpat dumji kā visas sievietes.

Neko es netaustīšu.\\
Te ir tik auksts,\\
It kā čoks domāts dvēselei,\\
Nevis miesai.

--- Miesai...\\
Miesai... ---

Miesai, lai cik viņas mums katram būtu,\\
Te ir par aukstu,\\
Un dvēselei cietumi ir citādi.

--- CItādi?... ---

Un ko tu brīnies?\\
Varbūt tu nopietni domā,\\
Ka šis ir kaut kāds cietums?

--- Cietums?...\\
Cietums... ---

Biedri milici, laidiet mani laukā,\\
Mana atbalss netic,\\
Ka mēs abi jokojam.

Laidiet mani laukā ---\\
Un es viņai parādīšu to dzīves cehu,\\
Kur taisa kropļus.

Tu redzi to gaišo logu,\\
Tātad tur vēl neguļ.\\
Tu dzirdi varenus akordus.\\
Tātad tā ir ``Ungāru rapsodija''.

Muļķe, atjēdzies taču,\\
Kopš kura laika mēs sāksim klauvēt\\
Pie aizslēgtām durvīam,\\
Mans amats šai ziņā ir visuvarens ---\\
Ver vaļā.\\
Un, kamēr mēs ciemosimies,\\
Lai atslēgas stāv uz ceļiem...

Dzirdi --- kā viļņu valzivis\\
Veļas lielā ungāra jūtas.\\
Ieplet spārnus ---\\
Un akordi pacels tevi.\\
Pārgriez vēnas ---\\
Un akordi pielies tevi...

Un tagad paskaties\\
Uz šiem gremojošiem žokļiem,\\
Uz šo pārīti,\\
Kura acīs aust mīlestība\\
Runča un kaķenes apjomā.\\
Paskaties uz šo ``Leonardo da Vinči'',\\
Kurš aizspiež klavierēm rīkli,\\
Paskaties uz šiem podiem,\\
Kāda slepkavnieciska ekstāze ir viņu\\
Porcelāna fizionomijās,\\
Un, cienībā noliecot galvu,\\
Paskaties šinī tumšajā kaktā ---\\
Tur sēž ar galvā uzmauktu kastroli\\
Lists.

Ejam.\\
Atbalss, tu esi meitene,\\
Pamodini pie durvīm snaudošo vecenīti,\\
Lai viņa nelaiž šai mājā nevienu kurlo,\\
Jo tas var būt arī Bēthovens.

Un tagad kāpsim\\
Pa šīm lielīgajām kāpnēm,\\
Kuras izliekas vedam debesīs.

Atbalss, tu uzmanies,\\
Te dzīvo meitene,\\
Kurai viss aizliegts,\\
Atskaitot vienu --- kļūt vecai,\\
Un divi cilvēki, kuri šo kļūšanu pūlas nosargāt.

Labvakar, cienījamais,\\
Jums esot skaista meita,\\
Paturiet, lūdzu, manu cepuri.\\
Un nelieciet kāju priekšā,\\
Te nav dzelzceļa pārbrauktuve.

Labrīt, meitene, kurai viss aizliegts!\\
Es laikam iztraucēju vienu no jūsu dzidrajiem sapņiem.

Meitene, jums ir tik lielas un izbaiļu pilnas acis\\
Kā stirnai.

Es zinu, jūs baida no tālumā skanošām taurēm,\\
Pulveru smakas un dīvainu zvēru rejām, ---\\
Daži no tiem, kuriem jāmirst,\\
Ir netaisni greizsirdīgi uz palicējiem,\\
Bet tur nekā nevar darīt.

Rīt izejiet, stirniņ, savā šalcošajā mežā,\\
ejiet turp, kur skan draudošās taures,\\
Un mācieties\\
Skriet.

Un nekas, ja jūs saskrambās zari,\\
Nolīs pa sarkanai lāsei...\\
Toties jūs zināsit,\\
Kā garšo avota ūdens,\\
Kāda ir sirds cīņas kaisles un azarta laimīgā dziesma,\\
Citādi rīt jums būs jāaiziet bojā\\
Un lielo atmodas dziesmu\\
Pavasarī\\
Stirnas dziedās bez jums...

Meitene, kurai viss aizliegts,\\
Neskaties manī tik cieši ---\\
Es neesmu atnācis ne pēc kā,\\
Un pēc kaut kā es pie tevis nemaz nevaru atnākt,\\
Man tagad vajag tik daudz,\\
Lai aizrautos elpa.

Tu esi stirna,\\
Es iešu meklēt niknāku zvēru.\\
Skaties tālāk savu visdzidrāko sapni\\
Un neklausi vecumam,\\
Vecums ir labs vienīgi tādēļ,\\
Ka drīz vien jāmirst.

Atbalss, tālāk pa ielu visi logi ir tumši.

--- Tumši...\\
Tumši... ---

Bet mēs esam gandrīz kaķi,\\
Vismaz mums ir tāds amats,\\
Kas atļauj redzēt arī pa nakti

Atbalss, neklaudzini papēžus.\\
Gultā var gulēt daudz kas,\\
Šinī gultā guļ mīlestība...

Nevar saprast, vai pukst\\
Viena vai divas sirdis,\\
Lai viņas izrunā pašas.\\
Atbalss, bet paskaties apkārt,\\
Vai tu redzi pelēkās ikdienas pīķus,\\
Ikdienas rūpju trulos asmeņus:\\
Plītis,\\
Veļas baļļas\\
Un tālumā --- televizora haubices draudošo purnu...

Šausmas!

Paskaties mirkli ---\\
Viņi taču gandrīz vēl bērni,\\
Neskarta, pirmatnēja uguns\\
Deg viņu sakļautajās, aizmigušajās lūpās.

Paskaties palmā pār viņiem ---\\
Palmas lapas kā zaļas rokas ar izplestiem pirkstiem.\\
Bet es redzu aukstus ikdienas ģindeņa pirkstus ---\\
Slepkavas pirkstus.

Guli, mīlestība,\\
Rīt tevi vārīs uz plīts,\\
Tveicēs netīrās veļas garaiņos\\
Un indēs ar siltās ligzdiņas indi,\\
No katras šīs lietas\\
Tevī kāds mazumiņš paliks.

Guli, mīlestība,\\
Guli saldāk un ātrāk,\\
Rīt tevi vārīs uz plīts.



\newpage

{\bf Četros}

Atbalss, pasaki man pilnīgi godīgi,\\
Ko tu domā par manu sirdi.

--- Sirdi? ...\\
Sirdi? ...

Es tev to neprasu prieka pēc ---\\
Man ir grūti ar viņu.

Bērnībā man bija cita --- ļoti skanīga --- sirds,\\
Ja es nebūtu nabagu ģimenē piedzimis,\\
Varētu teikt, ka no sudraba.

--- Sudraba? ...\\
.... ba-a-a... ---

Es jau zinu, ka nebija.

Tikai atceros --- lietus lija un skanēja,\\
It kā kristāla traukos kāds bērtu\\
Vissmalkākās skrotis.

Pagalmā stāvēja veca ieva\\
Ar nokārušamies zariem,\\
Un es šūpojos šinīs bizēs,\\
Gan tumšās un zaļās, gan ziedošās un sirmās,\\
Un ieva neteica man, ka sāp,\\
Jo es šo vārdu vēl nebiju dzirdējis,\\
Un ieva negribēja būt pirmā,\\
Kas man to pasaka.

Pēc tam es saslimu\\
Un daudz ko neatceros,\\
Tikai kad kļuvu vesels,\\
Sirds bija it kā cita.

Lietus vairs nebēra skrotis\\
Kristāla traukos,\\
Bet aumaļām plūda pa loga rūtīm,\\
Un man šķita, ka aiz loga krāc okeāns\\
Un ar brāzmainām šaltīm\\
Noskalo stiklu,\\
Lai es varētu labāk redzēt.

Un ievai es jau lauzu zarus...

Atbalss, pasaki man ---\\
Varbūt sirdi man apmainīja tas ārsts,\\
Kurš man dūra plecā\\
Tad, kad māte atstāja mūs abus vienatnē?

--- Vienatnē...\\
... nē... ---

Nu, bet kas tad tas varētu būt?

Es biju ļoti kauslīgs,\\
Un man nebija lielāka prieka\\
Kā --- sist tos, kuri ikdien varēja paēst, cik grib.\\
Es viņus situ\\
Un tajā brīdī pats sev likos paēdis...

Un tad viņi mani aicināja cept ugunskurā kartupeļus\\
Un nogrūda mani no kraujas.

Atbalss, bet viņi taču pa to laiku,\\
Kamēr es neatceros, kur biju,\\
Varēja nozagt manu skanīgo sirdi, vai ne?

--- Vai ne? ...\\
... nē... ---

Tad es nezinu.\\
Tikai pēc tam lietus varēja līt vai nelīt ---\\
Mani interesēja tikai zibeņu baltie raksti,\\
Un ar savādu bijību\\
Es liku roku pie zibens saspertas apses,\\
Liku tā kā pie nepazīstama suņa purna,\\
Kā pie sarkani sakurinātas plīts.

Un ievas?\\
Es neatceros vairs --- ko,\\
Bet kaut ko es drāzu no ievas koka.

Un pēc tam...

Lielceļu smiltis ir aprijušas daudz manu pēdu,\\
Man jau, atbalss, nav tā kā tev ---\\
Ej, un neviens nevar pateikt, ka iets.

Es atstāju pēdas ---\\
Un manī atstāj.

Jā.\\
Bija reiz tāda stunda:\\
Trotuāru asfalts bija mīksts no karstuma,\\
Un es atvadījos no meitenes,\\
Kura no manis gribēja par maz.

Šķirties...\\
Ja no divu cilvēku lūpām\\
Vienas ir teikušas --- šķirties,\\
Tad tas jau ir noticis.

Un es uzliku sirdi uz delnas.\\
Ir grūti bez sirds,\\
Un uz mirkli man satumsa acīs,\\
Sagrīļojās zem kājām...

Kad es atjēdzos,\\
Mīkstajā asfaltā bija vienīgi pēdas\\
Un uz delnas nekā.

Atbalss,\\
Vai var būt, ka manu skanīgo sirdi\\
Paņēma līdzi šī meitene?

--- Meitene? ...\\
... nē... ---

Droši vien sirds ir ielēkusi atpakaļ krūtīs,\\
Lai neredzētu, kā viņa aiziet.

Atbalss, un tagad man ir pavisam citāda sirds ---\\
Tāda pati kā Rīga, pa kuru mēs ejam.\\
Tāpat saucas ---\\
Manu asiņu galvaspilsēta.

Viņai tāpat --- savi parki un izsisti logi,\\
Savi tilti un savas pārdēvētās ielas,\\
Savi taksometri un savi spekulanti.\\
Un gluži tāpat ---\\
Tajā dzīvo bez gala, kas varētu nebūt,\\
Un nedzīvo tie, kuriem vajag būt.

Smaga ir tāda galvaspilsēta,\\
Kura jānēsā sevī,\\
Kuras stacijām jāpieņem\\
Straujie un sarkanie asiņu vilcieni,\\
Jābruģē jaunas ielas\\
Un jāizķer ielasmeitas,\\
Kuras ne vienmēr ir sliktākas par slavenībām.

Ak, atbalss, atbalss,\\
Kuro kilometru mēs jau ejam,\\
Manī sāk pāriet reibums,\\
Un sāk gribēties gulēt.

--- Gulēt...\\
Gulēt... ---

Kāpēc tu neteici agrāk?\\
Pilsēta.\\
Galvaspilsēta.\\
Vai tev ir viena dzeltena lapu kaudze\\
Priekš diviem blandoņām?

Cieta?\\
Tad pasauc vēju,\\
Lai sapūš\\
Labi smaržīgu un labi lielu ---\\
Mēs ar atbalsi gribam skatīties smaržīgus sapņus.



\newpage

{\bf Puspiecos}

Atbalss, par ko tu domā?\\
Par to pašu?

--- To pašu...\\
To pašu... ---

Es domāju par to, kāpēc atnācu uz pilsētu.

Vai tu atceries, kā pavasarī\\
Strazdi sviež zvirbuļus ārā no būriem?\\
Atceries?

--- Atceries... ---

Vai tu atceries, kā mēs ar bērzu sulām\\
Iedzērām dzimtenes garšu?\\
Atceries?

--- Atceries... ---

Un, ka vasarā birzī no četrām pusēm\\
Mūs sargāja zaļgans un caurspīdīgs aizkars,\\
Atceries?

--- Atceries... ---

Un skaisto, zilacaino netikli rudzupuķi,\\
Uz kuru rudeņos visas vārpas sāk skatīties,\\
Atceries?

--- Atceries... ---

Un vienu rudzu salmu, spīguļojošu mēness naktī uz ceļa,\\
Un kuļmašīnu dūkoņu\\
Atceries?

--- Atceries... ---

Un rudeņos, kad peļķes aizstiklojas\\
Un cilvēki, kuriem viens logs jau ir,\\
Ķēmodamies peļķēm pakaļ, liek otru,\\
Atceries?

--- Atceries... ---

Un ziemā, kad sniegs vienādi snieg\\
Uz veciem un jauniem,\\
Uz izpildkomitejām un kapsētām,\\
Atceries?

Vai tu tiešām neatceries?\\
Atbalss, tu esi nejauka.\\
Kā tev nav kauna sākt skatīties sapni bez manis...

\mbox{}\hspace{10pt}... Rudzi zied.\\
\mbox{}\hspace{10pt}Ar tūkstoš, tūkstoš ziediem rudzi zied.\\
\mbox{}\hspace{10pt}Pret mākoņiem, kas savu ceļu iet,\\
\mbox{}\hspace{10pt}Ar tūkstoš, tūkstoš ziediem rudzi zied.

Zeme, tev šodien viens mākonis vairāk.

\mbox{}\hspace{10pt}... Ceļi kūp.\\
\mbox{}\hspace{10pt}Ar baltiem, baltiem dūmiem ceļi kūp.\\
\mbox{}\hspace{10pt}Uz galda ceļamaizes rieciens drūp ---\\
\mbox{}\hspace{10pt}Ar baltiem, baltiem dūmiem ceļi kūp.

Zeme, tev šodien viens ceļinieks vairāk.

\mbox{}\hspace{10pt}... Zvaigznes krīt.\\
\mbox{}\hspace{10pt}Ar spožām, spožām šautrām zvaigznes krīt.\\
\mbox{}\hspace{10pt}Līst zvaigžņu lietus, nevar saskaitīt ---\\
\mbox{}\hspace{10pt}Ar spožām, spožām šautrām zvaigznes krīt.

Zeme, tev šodien simt akmeņu vairāk.

Un pēkšņi --- nav.\\
Un pēkšņi --- līdzens asfalts.\\
Uz visām četrām debespusēm --- asfalts.\\
Līdz visiem četriem horizontiem --- asfalts.\\
Asfalt, ja arī tev nav sirds,\\
Atdod vismaz manu atbalsi.\\
Atdod.

Par mani nabagāka vairs uz zemes nav ---\\
Pat klaiņojošam sunim\\
Sava ēna ir\\
Un arī atbalss.

Sāk trīcēt asfalts.\\
Milzu mājas aug...

Es nolecu no tikko izdīguša jumta,\\
Un atkal man zem kājām asfalts trīc ---\\
Dīgst atkal jumts,\\
Es lecu atkal,\\
Atkal trīc,\\
Es lecu,\\
Trīc,\\
Es lecu...

Un vairs vietas nav, kur lēkt, ---\\
Aiz jumta malas sestā stāva augstums...

Bet sīkas tirpas skrien pār ielu bruģi,\\
Man piere aukstus baiļu sviedrus svīst ---\\
No ielu bruģa rudzu druva dīgst.

Es kliedzu:\\
--- Neaudz šinī vietā, mīļais lauks,\\
Tev nejūtīgas riepas pāri brauks,\\
Te --- pilsēta\\
Un viņas apkvēpušās debesis\\
Pret lūgšanām ir kurlas... ---

Aug rudzu lauks,\\
Jau stiepjas rudzupuķes...

--- Lien ātrāk zemē, palaidnīgā skuķe,\\
Te --- pilsēta\\
Un tavas netiklības pilsētai par maz... ---

Dārd vias ielas,\\
Vārpas krīt uz ceļiem ---\\
Mirst stāvošie,\\
Mirst lūdzošie ---\\
Iet pāri dārdošs metāliskais dēmons,\\
Un dobji runā asfaltētā balss:

--- Neesi sentimentāls, manu zēn,\\
Katram ir dzīvē jāzin sava vieta.\\
Kurš nezin, tam jāiemāca.\\
Ja nevar dzīvam --- jāmāca mirušam ---

Dārd dobji, dobji asfaltētā balss:

--- Es atmācīšu tevi, manu zēn,\\
No desmit gudrībām, ko lauki iemācīja,\\
Un tūkstoš nekrietnības iemācīšu tajā vietā. ---

Es saucu --- man nav balss,\\
Es lecu lejā --- nevar nosisties,\\
Un šinī brīdī atbalss mani pamodina.



\newpage

{\bf Sešos}

Iet pirmie tramvaji,\\
Tā ir bijusi viņu rūkoņa...

Atbalss, lien ārā!\\
Draņķīgā stundā mēs esam pamodušies,\\
Bet vēlāk te būs daudz cilvēku.

Paskaties, atbalss, šis blēdis\\
Kaut kā tomēr iešugulējies dzīves restorānā\\
Un ir laimīgi pillā vēl šorīt.

Miers viņa pīšļiem!\\
Paskaties, atbalss, šinī bālajā sejā.\\
Baidos minēt, bet izskatās,\\
Ka viņš bijis pie savas nelaimes ciemā.

Vispār, pēc acīm spriežot,\\
Viņš pats nesaprot,\\
Kur laime, kur nelaime,\\
Un kamēr tas nav saprasts,\\
Mums it nekā nav, ko saprast.

Un lai!\\
Teiksim skaļi,\\
Ka ikviens dzejnieks\\
Ir vairāk vai mazāk Kvazimodo.

Lai dzīve paciestu viņa kroplumu,\\
Viņš maksā ar visu sevi.

Šie jaunekļi arī ir kropļi,\\
Tikai tie negrib maksāt,\\
Un dzīve pret viņiem arī nebūs\\
Diez cik pašaizliedzīga.

No nekurienes viņi nāk,\\
Uz nekurieni viņi aiziet?

Pulkstens seši,\\
Un pāri pilsētai skan\\
Atgriešanās maršs.

Atgriežas katrs pie tā, kas viņam ir.



\newpage

{\large \sc * * *}

Drīz būs zili, zili ūdeņi,\\
Drīz būs balti, balti pūpoli,\\
Drīz es beigšu sevi mierināt.

Zirgs es esmu.\\
Ziemās man no izkaltēta āboliņa\\
Siltā stallī bezdievīgi slāpst.

Ķīvītes ir prom.\\
Un nedūc kamene ap ausīm.\\
Tikai mašīnas.

Bezspalvaini\\
Bezdvēseles\\
Zirgi\\
Dūc ap manu stalli,\\
Dūc to pašu, ko es sen jau zinu:\\
``Var-r-ram, var-r-ram, var-r-ram...''

Jūs par manām slāpēm\\
Smieties varat,\\
Bet, kad ārā nobirs visas sarmas,\\
Ko jūs, pļavā iziedami, darīsit:\\
``Var-r-ram, var-r-ram...''?

Man būs vēji krēpēs,\\
Man būs zāle apkārt,\\
Man būs kumeļš blakām.

Rūciet, kamēr vēl nav zili ūdeņi,\\
Rūciet, kamēr vēl nav balti pūpoli,\\
Rūciet, kamēr man vēl jāiztiek\\
Ar sausu,\\
Izkaltušu\\
Pagājušo\\
Vasaru.



\newpage

{\large \sc Svilpojamas dziesmiņas}

{\bf Dziesmiņa par trako Līzi}

Pērkonlietus šonakt līs,\\
Trako Līz, trako Līz,\\
Nāc pie manis labi drīz,\\
Trako Līz.

Tavos matos, trako Līz,\\
Trako Līz, trako Līz,\\
Velni ūdensrozes pīs,\\
Trako Līz.

Manā sirdī, trako Līz,\\
Trako Līz, trako Līz,\\
Velni tevi apbedīs,\\
Trako Līz.

Kamēr dievs to sapratīs,\\
Trako Līz, trako Līz,\\
Mūs vairs nebūs, trako Līz,\\
Trako Līz.

Saskries bērni, trako Līz,\\
Trako Līz, trako Līz,\\
Un mums pāri govis dzīs,\\
Trako Līz.



\newpage

{\bf Svilpiens par lielo olu}

Vienu lielu olu\\
Zvirbulis dēja,\\
Septiņas dienas\\
Apkārt tai skrēja,\\
Uzsēdās perēt ---\\
Bet izperēt\\
Nevarēja.\\
Fuit!



\newpage

{\bf Dziesmiņa par labo roku}

Ar labo roku\\
Apņemt var sievu,\\
Ar labo roku\\
Nolauzt var ievu,\\
Ar labo roku\\
Var glaudīt pa spalvu,\\
Ar labo roku\\
Nocirst var galvu,\\
Un, darot labu,\\
Pa galvu var dabūt.



\newpage

{\bf Dziesmiņa par strazdu}

Es neprotot laist ziepju burbuļu krāsainās domas pie
kafijas tases un zilganu actiņu bezdibens tukšumā
verdziski izdomāt pērles, un vispār uz spoguļu parketa
es esot kājslauķa mērogā prasts. 

Bet es svilpoju tālāk, jo es esmu būris un manī ir
apmeties strazds. 

Es neesot prātīgs un neprotot izskaitļot satiktos 
ziedus un smaidus, un, oticot parastai nopūtai, es
esot aiznests pār draudosu jūru tik tālu, ka vairs arī 
rūdītu vanagu acīm nekur neesot samanāms krasts. 

Bet es svilpoju tālāk, jo, kamēr strazds manī nes mazuļiem
paēst, man kaut vai ar šūpuļa dziesmu ir jāaizstāj
strazds. 

Es nemīlot strādāt, jo pamanīts esot, ka rokas pa reizei
gar sāniem man krītot kā virves un pats reizēm
gāžoties cilvēku priekšā kā pensijā izdzīts un vētrai pa grābienam gadījies masts. 

Bet es svilpoju klusi, jo manī ir bēres, jo priecīga puišeļa 
šaujamā palaista priecīga akmeņa ceļā bij aizmirsies 
parasts un savējai mūzikai līdz pašai pēdējai spalviņai
atdevies strazds. 



\newpage

{\bf Dziesmiņa par minisvārkiem}

Pland matu baltie lini,\\
Iet meitene iekš mini\\
Un modri skatās apkārt,\\
Kurš skatiens viņai tiks.

Ak, likteņkleitas īsās,\\
Es jūtu sevī trīsas\\
Un, pasaulei par spīti,\\
Pats arī ģērbjos pliks.

Jo, ja reiz viss ir mini,\\
Pie velna, lai ir mini!\\
Mūs, tāpat kā reiz senčus,\\
No paradīzes trieks,\\
Bet mums būs minibučas\\
Un pēc tam miničuča ---\\
Un pēc tam tikai bērni\\
Un morālisti kliegs.

Bet tad nāks miniraizes,\\
Jo pietrūks minimaizes,\\
Un glābjoties mums abiem\\
Kļūs apkārblandošs skats.\\
Bet tas jau ir tāds likums,\\
Kā izpārdodams plikums.\\
Zem visiem dzīves mini\\
Ir apmēram tas pats.

Es ķeros koku zaros\\
Un raibā šlipsē karos,\\
Un izveļos no gultas,\\
Un tūlīt mostos ar...\\
Tad pērku skaistu kleitu\\
Un eju meklēt meitu,\\
Ko satikt, precēt apģērbt\\
Un arī izģērbt var.


\newpage

{\bf Dziesmiņa par svilpošanu}

Kad puikām no biksēm\\
Ārā aug lieli,\\
Puišeļi svilpo,\\
Jo vēlas būt lieli.

Kad vīriešiem dzīve\\
Sirdī lej mieles,\\
Vīrieši svilpo,\\
Jo vēlas kļūt lieli.

Nosvilpjas akmens,\\
No kalna grūstot,\\
Nosvilpjas lode,\\
Par nāvi tev kļūstot.

Bet, ja nu tev kādreiz\\
Kaklā mauc cilpu,\\
Izkļūsti ārā.\\
Ja nevari ---\\
Svilpo.


\newpage

{\large \sc Dziesma}

Pie nātres skāries,\\
Neraudi, ka skāries, ---\\
Kas tevi dzina\\
To ziedu plūkt!\\
Reiz sāpes pāries,\\
Visas sāpes pāries,\\
Tik nevajaga\\
Nevienam lūgt.

Lai tevi sargā\\
Tā, kā sevi sargā,\\
Kā sargā ausis\\
No pērkonbalss,\\
Kā sargā acis\\
Tie, kas zin, kas --- acis,\\
Kā sargā dzīvi,\\
Kad tai ir gals.

Lai tevi aizskar\\
Tas, kas visus aizskar,\\
Lai lūpās smaidu\\
Vai naidu liek,\\
Bet tēraudaizkars ---\\
Tas ir visam aizkars,\\
Ne cauri lode,\\
Ne smarža tiek.

Kur paliks smaržas\\
Pasaulē bez smaržas?\\
Ko ļaudīm elpot,\\
Ko vējam nest?\\
Tad dienu baržas ---\\
Smagi krautās baržas ---\\
Melns, miris ūdens\\
Var tālāk vest...

Pie nātres skāries,\\
Neraudi, ka skāries, ---\\
Ir visām nātrēm\\
Savs savāds maijs.\\
Ja sāpes pāries\\
Un pavisam pāries,\\
Tad aizies arī\\
Viss pārējais.


\newpage

{\large \sc * * *}

Drīz negants vējš no ziemeļpuses skries\\
Un kritīs manam bulvārim ap kaklu,\\
Pa naktīm raudās un pa dienām smies\\
Un ezeriem pēc bezmiega nāks aklums.

Un cauri sniegputeņiem iesim mēs\\
Pa aizmigušās zemes dzīvām acīm.\\
Caur sniegiem mūsu pavasari neredzēs,\\
Un, kur tas dzīv, mums to nepasacīs.

Ciet mūsu pēdas sniegputeņi vilks,\\
Lai citiem mūsu ceļu atrast grūtāk,\\
Un ziema bezdievīgi ilgi ilgs\\
Gan zemes klajumos, gan mūsu jūtās.

Pa vēnām lēnām asinis mums skries\\
Un brīži būs, kad liksies --- nu vairs nevar,\\
Bet paliks viens, viens vienīgs glābjošs miers ---\\
Mēs vēl pirms ziemas zemei kaut ko devām.

Un tāpēc nāvei --- nē! Un salam --- nē!\\
Lai baltā nāves dvaša kūp no mutes!\\
Mēs zīli ielikām reiz zemei padusē,\\
Un zeme modīsies, jo viņai kutēs.


\newpage

{\large \sc * * *} 

Saki ---\\
Cikreiz starp mums nodegs pīlādži,\\
Cikreiz pērkona negaisi noplandīs,\\
CIkreiz vasaras dzīs\\
Kāpu vieglo un sanošo smilti?

Es šai neziņā brienu\\
Kā vīgriežos.

Pāri man\\
Ar visbaltāko baltumu\\
Debess jūra\\
Savas bangotnes putotos mākoņus nes.

Kaut kur\\
Pēc lietus vālodze kliedz.

Vēl viens lietus\\
No tūkstoš lietiem,\\
Kas tecēs no matiem ---\\
Bet tikai no maniem.

Saki ---\\
Vai tu dzirdi tos šalcošos soļus?\\
Tur nāk pļavas bez tevis,\\
Tur nāk rudens un ziema bez tevis,\\
Tur nāk gadi bez tevis.

Un uz tā soliņa parkā,\\
Uz kura mums jāsēd,\\
Ir divi.

Bez manis un tevis.




\newpage

{\large \sc * * *}

Šajos ceriņos rīt ziedi būs ---\\
Mēļas liesmas no tumšzaļas pekles.\\
Un pa debesīm eņģeļus --- mūs\\
Un pa zemi mūs --- grēcīgos meklēs.

Viņu smaržu šķels tavējais ils,\\
Viņu ēnu mans autobuss sabrauks.\\
Krūtīs sāpēs kāds aromāts zils,\\
Un viens atlidos, un otrs atbrauks.

Vārdu lavīna nogrūs un mirs,\\
Ļaujot rokām, ko vēls, to darīt.\\
Vien no ceriņiem smaržojot birs\\
Mēļas zvaigznes ar nepāra stariem.

Kvēlos liesmā un plaiksnīsies sārts\\
Zilā debesu sejā mums pāri,\\
Un uz ceriņu krūma būs kārts\\
Viena eņģeļa lieks spārnu pāris.



\newpage

{\large \sc Palāses}

Krīt rudens ar pavasari\\
Gar logu,\\
Jo lāse ir viršu zila.\\
Aiz prieka par sauli\\
Tos vēji ir sajaukuši.\\
Krīt lāse un nosten ---\\
\mbox{}\hspace{10pt}tu skaties uz mani\\
\mbox{}\hspace{10pt}caur aizejoša vagona stiklu.\\
Krīt lāse un smaida ---\\
\mbox{}\hspace{10pt}tu nāc,\\
\mbox{}\hspace{10pt}un es tevi redzu\\
\mbox{}\hspace{10pt}caur lidostas stikla sienu,\\
\mbox{}\hspace{10pt}bet mēs vēl netiekam kopā.

Un tāpēc\\
Krīt dzintarā mirdzoša lāse ---\\
Mūsu pārakmeņojušies sveķi ---\\
No brūcēm,\\
Ko rāvuši visādi braucamie,\\
Visādi ļāudis\\
Un visādi strīdi,\\
Nelaizdami mūs kopā.

Bet tagad\\
Līdz nākošai reizei neko\\
Viņi mums nevar izdarīt.

Mums ir mūžība.\\
 
Un tikai nākošā šķiršanās brūce\\
Mums sacīs,\\
Cik bija šī mūžība īsa.


\newpage

{\large \sc * * *}

Es gaidu tevi\\
Uz šī pavasarīgā stūra\\
Jau visu dzīvi.

Var būt, ka tu\\
Mani gaidi citur,\\
Bet, tikai šeit satikušies,\\
Mēs satiksimies.

Mirguļo kupenas,\\
Un upēs ir daudz ūdens.

Lieli kuģi var peldēt\\
Pa tādu dziļu un brūnu ūdeni.

Bet es gaidu tevi\\
Un skatos\\
Uz papīra kuģīti\\
Marta ledos.



\newpage

{\large \sc * * *}

Apturēsim autobusu. Klausa.\\
Palūgsim, lai tālāk dodas. Brauc.\\
Šausmas, cik man mute tagad sausa,\\
Šausmas, cik tai tagad vajag daudz! ---

Ezera, kur ir viens miljons salu,\\
Lūpu, kuras miljons skūpstu ļauj,\\
Līču, kur visvairāk lakstīgalu,\\
Pļavas, kur visskaļāk sienu pļauj,

Lietu, kas vistīrāk mazgā pieri,\\
Zibens, kas vistaisnāk priedē krīt,\\
Nakti, kurā it kā miljons mieru,\\
Miljons zilu zvaigžņu pārī spīd.

Man nav it nekā. Un tā ir labāk.\\
Tikai tad jau īsto skaņu jūt.\\
Pareizi jau saka --- kur der nabags,\\
Kurš par miljonāru negrib kļūt!


\newpage

{\large \sc * * *}

Ja tu zini, kāds ir Gaujas alts,\\
Tad tu zini, kā mākoņi šalc,\\
Tad tu zini, cik silts katrs zieds,\\
Tad tu zini, kā vējš pļavas liec, ---\\
Tad tu zini. Un nevarēs posts\\
Tavā atvērtā dvēselē kost.\\
Grūtā brīdī tad varēšu es\\
Slapju biti tev nožāvēt nest.\\
Lieti līst, salnas ir, sniegi snieg,\\
Skropstās līst, vaigos kož, sirdī tiek.\\
Bet tev neziņā netumsīs skats,\\
Ja kaut reizi tu būsi viens pats,\\
Gaujas balss ilgi meklēts un saukts,\\
Baltu mākoņu kopenās jaukts,\\
Ziedu sildīts un viesuļu nests, ---\\
Ja tu būsi. Bet saredzu es ---\\
Katrā solī, ko tālāk tu liec,\\
Pelēks, silts, samīts zaķītis kliedz.\\
Tā jau nav kāda varena balss ---\\
Tā kliedz viss,\\
Kur iet sals,\\
Kur iet sals.


\newpage

{\large \sc Skabargas}

Kad pirkstā tu ierausi skabargu ---\\
Aizej pie mammas.

Jods nav visbriesmīgākais,\\
Ko cilvēki liek uz pušuma.

Kad sirdī tu ierausi lāsteku ---\\
Aizej pie mīlestības,

Ja tu vēl vari paiet,\\
Ja lāsteka nav pārāk dziļi.

Bet, ja būs pārāk dziļi,\\
Tad tevis dēļ\\
Kāds ieraus pirkstā skabargu,\\
Kāds ieraus sirdī lāsteku\\
Un vēl kāds\\
Var sirdī ieraut durkli.

No viena baļķa neiznāk māja,\\
Bet iznāk trīs miljoni skabargu.


\newpage

{\large \sc Šķiršanās}

Pagājušas divdesmit septiņas stundas\\
Varbūt...

Bet ir pagājis pavasars ---\\
Es redzēju čurkstes,\\
Kas ņudzēja Gaujas kraujā,\\
Un ķīvītes kliedza.

Varbūt blakus istabā raudāja bērns.

Pēc tam sāka smaržot siens\\
Un gaiss dūca.\\
Ar basu kāju es laikam uzkāpis biju\\
Uz kameņu perēkļa.

Varbūt garām brauca mašīna

Pēc tam\\
Kļavas lapa\\
Dzinās pakaļ autobusam.\\
Aizkusa, apstājās\\
Un atkal dzinās ---\\
It kā autobusā būtu aizbraucis\\
Tās pavasaris.

Varbūt es vēl ticu, ka tu atnāksi.

Pēc tam bija krusa.\\
Tā krita patvertnes uzbēruma zaļajā zālē,\\
Ripoja lejup pa Doma zaļgano jumtu,\\
Sitās cauri caur divu koķetējošo\\
Mākslīgo vārdu saharīnu,\\
Bira skursteņos kuģiem\\
Un kā trokšņu stacija dūdinājās uz palodzes,\\
Slāpējot sirdsapziņu diviem,\\
Kas zaga mīlestību ``kaut kāpēc'',\\
Kam pietrūka ņemt to tāpat.

Var būt, ka es neatšķiru vairs galveni,\\
Pēc tam nāca sniegs.\\
No sākuma kusa uz rokas,\\
Bet vēlāk vairs nekusa.\\
Tāpat kā toreiz es laikam jūtas Rīga,\\
Kad pa viņu brauc saldētavas.\\
Es pat nezinu,\\
Uz kuras ielas es biju apstājies,\\
Bet garāmgājējs bij asprātīgs:\\
--- Viņas nav mājās. ---

Varbūt šodien viņu visu nav mājās.

Un tad atkal lidoja čurkstes,\\
Spietoja bites ---\\
Un divi iemīlējušies\\
Dzimtsarakstu nodaļā\\
Teica viens otram ``jā''\\
Un apmainījās ar televizoriem.


\newpage

{\large \sc * * *}

Ja man rādītu ceļu\\
Kāda no Rīgas ugunīm,\\
Ja tā nebūtu tikai elektrība...

Ja nebūtu.

Ja mani varētu aizvest pie tevis\\
Kāds no daudzajiem tramvajiem,\\
Ja tie nebūtu tikai satiksmes līdzekļi...

Ja nebūtu.

Es tev pieklauvētu pie loga\\
Tik klusu,\\
Ka tu jautātu man:\\
``Ziema, vai tā esi tu?''

Es nebūtu ziema\\
Es būtu vēl rudens.\\
Viena vējos kvēloša lapa\\
No tā, kas šalca\\
Un ziedēja reiz.

Ja ugunis rādītu ceļu.\\
Ja tramvaji vestu.

Bet tagad es palieku\\
Salna\\
Uz zāles, pa kuru ej,\\
Uz durvīm, kuras tu ver.\\
Uz deniņiem taviem,\\
Kur tikko redzami pulsē\\
Zila bērnība tava,\\
Sārta jaunība tava\\
Un zils vakars tavs.


\newpage

{\large \sc * * *}

{\em Visvaldim Lāmam}

Kad graudu samaļ miltos\\
Un miltus mājās ved,\\
Lai briedīgs zemes siltums\\
Tev kaulos spēku met.

Lai sirds kā kumeļs kārpās\\
Un vēji krēpes šķir ---\\
Kamēr uz zemes vārpas\\
Un rudzu maize ir.

Tev nedrīkst prasīt vājums,\\
Cik netic tev, cik tic,\\
Šis gājums ir tavs gājums,\\
To nevar noiet cits.

Būs nevienādi ceļi,\\
Cits --- dūkšņains un cits --- sauss,\\
Bet neizmelos meļi\\
Par svešu, kas ir tavs.

Būs soģi krēslos platos\\
Un platu tiesu lems,\\
Bet vēju tavos matos\\
Neviens tev neatņems.

Būs asiņaini kāvi,\\
Tu paņem acīs tos,\\
Bez tavas --- citu nāvi\\
Neviens tev neiedos.

Nemaz nav briesmīgs šķīrums\\
Starp ``dzīvot'' un starp ``mirt'',\\
Ja tevi apsēts tīrums\\
Un mīlestība dzird.


\newpage

{\large \sc Tikai}

Tikai āboli krīt\\
Rudens rasainā zālē.\\
Rasa zila kā tāle,\\
Kurp tu aizbrauksi rīt.

Tikai mākonis snauž\\
Tālam mežam uz kakla,\\
Diena izliekas akla,\\
Dienai drusciņ vēl skauž.

Tikai veltīgi viss.\\
Stājas asiņu dima,\\
Un tās zvaigznes, kas dzima,\\
Visas nolemtas dzist.

Tikai pagaidi vēl,\\
Ne jau tāpēc, ka baidos,\\
ne jau tāpēc, ka svaidos,\\
Ne jau tāpēc, ka žēl.

Tikai tāpēc, ka ir\\
Man no tevis kas iedzimts\\
Un vēl reizi ir piedzimts,\\
Lai vel reizi var mirt.

Tikai neteiksim --- ``rīt'',\\
Kaut gan lūpas jau plešas.\\
Krīt kā planētas svešas,\\
Tikai āboli krīt.


\newpage

{\large \sc Salna}

Atkal ir zāle no skārda.\\
Atkal naktī ir vārītas puķes.

Kļavas lapa\\
Liek logiem plaukstu uz acīm:\\
--- Nav ko skatīties bērni,\\
Tā gadās... ---

Bet bērni ir slimi\\
Ar ``kāpēc?''.

Kāpēc skārda zālē\\
Nav kameņu?

Kaut vai skārda kameņu\\
Kāpēc nav?

Kāpēc jāvāra puķes,\\
Ja abrā rūgst maize?

Vectēv, tev ir gara dzīve\\
Un mums ir tik īsi ``kāpēc?'',\\
Saki.

Bet vectēvam pašam\\
Gar deniņiem baltums\\
Un rokas uz spieķa\\
Kā puķes pēc salnas.

Un viņa ``kāpēc?'' vēl lielāks.\\
Un nav neviena tik liela,\\
Kas atbildētu.

Un kad viņš uz mūrīša lien,\\
Sauc vedekla bērnus:\\
--- Nav ko skatīties,\\
Gadās... ---

Un grūti gadās,\\
Kad cilvēkiem jāiet\\
Pa vārītām puķēm\\
Un skārda zāli.


\newpage

{\large \sc * * *} 

--- Tu par mani pārāk daudz zini. ---\\
Un Bordžija izdomā indi.\\
Un ķīmija, vēl nemaz neapjēgdama,\\
Kas viņa pati ir, ---\\
Dod.

Bet skalpeļi paver spraugu\\
Un ierauga indi.

Šķir vēsture vecas dzeltenas lapas\\
Un ierauga\\
Kardinālu.

Nojauš.

Es arī nojaušu,\\
Ka man pilina indi\\
No rīta debesīs,\\
Dienā acīs un ausīs\\
Un naktī ber pilnu galvu.

Ne lapas, ne skalpeļi nelīdz.\\
Un neatsedzas\\
Ne trauks, no kā lej,\\
Ne melnā kapuce,\\
Zem kuras slēpjas\\
Tā draudīgā seja,\\
Kas par mani var sacīt:\\
--- Tu esi vientiesis,\\
Tu esi muļķis,\\
Tāpēc\\
Tu par mani pārāk daudz zini.


\newpage

{\large \sc * * *}

Viena minūte, protams, būs pēdējā,\\
Es jau dzirdu, ka viņa nāk.\\
Garām sētai, kur mēness sēdēja,\\
Garām liepai, kas ziedēt sāk,\\
Nāk uz sirdi, kuras vairs nevajag.\\
Klāt ir brīdis, ko gudrie sauc\\
Vienkārši --- ``viss dotais atdodas devējam''.\\
Tagad var redzēt, ka tā nav daudz.\\
Drusku vienaldzības un drusku vājības,\\
Drusku izlikšanās, uz sirds ko segt,\\
Drusku par daudz tās remdenās mājības,\\
Kas gandrīz vienmēr neļauj degt.\\
Drusku par maz domu jaunās vingrības,\\
Rokas spiediens sasteigts un ass,\\
Drusku par daudz pret citiem stingrības ---\\
Tik daudz, ka pašam tās paliek maz.\\
Nu, un viss pārējais --- tas ir kā maize dots\\
Un līdz tam brīdim, kad skropstas vēl trīc.\\
Un tas, kas teikts nav, --- to mēs visi aizejot\\
Arī uz turieni paņemam līdz.


\newpage

{\large \sc * * *}

Kamēr vēl visi lēkti nav padzīti,\\
Kamēr es dzirdu vēl, kā šalc bērzs,\\
Kādos grēkos man vēl vajag atzīties\\
Tā, lai sanāktu pilns soda mērs ---\\
Nepārsūdzamais un visaugstākais,\\
Nepielūdzamais augstais sārts?\\
Prasiet, ja tas būtu pats visaukstākais\\
Un visgrūtāk sakāmais vārds, ---\\
Es to teikšu. Jau lūpas paplešas,\\
Un nemaz vairs tik grūti nav,\\
Man paša lūpas jau ir diezgan pasvešas,\\
Un kaut kur tās ir mirušas jau.\\
Nolādēts siltums, kas gaisīgi aulēkšo\\
Un pēc kādas tur atbalss kliedz,\\
Mostas uz gaišu un laimīgu saullēktu,\\
Bet dienu nobeidz tumšs, nelaimīgs riets.\\
Sirdi spiež mākoņi paši zemākie,\\
Tukšas stundas pēc atbildes nāk.\\
Nē, ja reiz pats tu ziedēt nemāki,\\
Kļūsti par zemi tiem, kuri māk.\\
Tāpēc es atzīstos ikvienā mūcībā,\\
Ikvienā grēkā. Lai visi dzird.\\
Sakņu skopumā, zaru trūcībā ---\\
Kroplam kokam ir vieglāk mirt.



\newpage

{\large \sc * * *}

Tad, kad viss būs izkliegts un kad atbalss norims,\\
Vasaras nakts migla vēsi acīs kāps\\
Un kā brīnumsveču dzirkstis zvaigznes nobirs, ---\\
Sirdij tajā brīdī vairs ne salks, ne slāps.

Noreibs sirds no sava izdomātā miera,\\
Izdomātā sapnī pievērs acis ciet.\\
Var jau būt ne tā, jo cilvēkam nav pierasts\\
Smagiem soļiem pāri savai sirdij iet.

Tacu gribas man, lai tobrīd nav ne vēja,\\
Nav ne rasas piles, kas no zariem lās,\\
Un tā Gauja, kas man sudrabsmieklus smēja,\\
Lai uz citu pusi tobrīd paskatās.

Bērzs uz īsu mirkli aiziet prom no kalna,\\
Apsei lai uz mirkli arī klusums dots, ---\\
Nebija jau manī pārāk liela salna,\\
Nav jau pārāk maigs man paša piespriests sods.

Negaiss jau ar stūri ir pār mani pleties,\\
Vai tad vajaga vēl, ja tā labi šķir,\\
Man ar dzīvi tagad vēlreiz sastrīdēties? ---\\
Iztiksim bez strīda. Pietiek to, kas ir.


\newpage

{\large \sc Atvadvārdi}

Baltā māt, veļu māt,\\
Mazgā baltu paladziņu,\\
Jau ganās mani zirgi\\
Viņas saules pļaviņā.

Auksta, auksta tā pļavaiņa,\\
Kur es jāšu pieguļā,\\
Es tur iešu, veļu māt,\\
Ar oglīti azotē.

Velti, velti, bērza tāss,\\
Tu raujies čokurā,\\
Nesildīji šai zemē,\\
Tad sasildi aizsaulē.

Baltā māt, veļu māt,\\
Ņem mani savā daļā,\\
Tur jau īd manas govis\\
Neganītas, neizslauktas.

Nelieciet rožu kroni\\
Manā galvas galiņā,\\
Es saviem teliņiem\\
Aiznesīšu pienenīti.

Viens ziediņis, simts ēdēju,\\
Kam nu dot, kam nedot?\\
Visiem došu pienenīti,\\
Un būs visi paēduši.

Baltā māt, veļu māt,\\
Aizver man acu vārtus,\\
Kad atnācu, tad atvēru,\\
Nu ir reize aizdarīt.

Jūs, dēliņi, neraudiet\\
Par tētiņa aiziešanu,\\
Katru rīt' rasa raud\\
Par visiem gājējiem.

Nesakiet savu bēdu ---\\
Es jau vairs nedzirdēšu,\\
Ciet lodziņi, ciet durtiņas,\\
Nav neviena istabā.


\newpage

{\large \sc * * *} 

Var pierast pie paša velna,\\
Kā fabrika pierod pie plāna,\\
Var pierast pie garām sapulcēm\\
Un lūpām no celofāna.

Bet pierašana ir dzērums ---\\
Velk pirmā glāzīte otru,\\
Un otra pēc tam velk trešo,\\
Un ceturto nav vairs kam noturēt.

Var pierast pie trušu kaušanas,\\
Un tad visi truši ir gaļa,\\
Var pierast pie sevis puses\\
Un pēc tam pie desmitdaļas.

Var pierast pie matu garuma,\\
Var pierast pie prāta īsuma.\\
Mums ārsti neizsniedz zīmes\\
Par dvēseles daltonismu.

Un slimojošo bez jēgas.\\
Pats sev pat reizem neprasi,\\
Lai neredzētu, ka pieradis,\\
Ka esi aizmirsis --- nepierast.



\newpage

{\large \sc * * *}

Bail.\\
Ļoti bail dzīves dziļumos jaukties,\\
Bail grambu ceļā svaidīties.\\
Ja reiz tev bail nav par cilvēku saukties,\\
Ko tad no pārējā baidīties?

Saukties par cilvēku nav tev bail. Pierasts.\\
Un ja sirds liekuļot aizkusīs,\\
Nolīdīs malā un smiltīs ieraks,\\
Ņems un kā kaķis --- aizkasīs.

Kad būs no mūža kā no rudens apses\\
Nolemtais dienu skaits nobiris,\\
Tad jau var vējs tavu darīto atsegt,\\
Var. Jo tu būsi jau nomiris.

Žagaru rīkstēm un likteņa rīkstēm\\
Tad jau var atsegt plikumus,\\
Mums vajag mācīties šodien drīkstēt\\
Un uztrīt šodienas likumus.

Citādi, elpojot apvāršņu dūmus,\\
Mēs nedzirdam tāles pašas.\\
Nedzirdam. Tāpēc ka visos krūmos\\
Čab. Pārāk daudzi tur kašājas.


\newpage

{\large \sc * * *}

Putniņu vajag noķert,\\
Citādi lidos augstu,\\
Citādi nevarēs aizlikt\\
Priekšā tā mutei plaukstu.

Citādi pavasaros\\
Atkal kaut ko tur cerēs,\\
Dēs kaut kādas tur olas\\
Un sev līdzīgus perēs.

Putniņu vajag noķert ---\\
Un noķert tieši aiz astes,\\
Citādi --- kas tad dziedās\\
Stiepļotu režģu kastēs?

Viņam ir jādzied, ko pavēl,\\
Vai arī jāaiziet bojā,\\
Viņam ir jāzin --- no dziesmām\\
Ēstgriba uzlabojas.

Putniņu vajag noķert,\\
Lai vieglāk ir pieēst punci.\\
Putniņš lai dzied, kamēr piekūst,\\
Pēc tam viņu atdos runcim.

Man pēc maltītes dzīvesprieks\\
Pāri sirds malām kūsās ---\\
Un uz tepiķa gulēs\\
Runcis ar sarkanām ūsām.


\newpage

{\large \sc Balāde par cirvja kātiem}

Jauno dienu zilais asmens nodila.\\
Večuks tirgojās ar cirvja kātiem,\\
Puskažociņš smaržoja pēc skaidām.

Milicim, kas viņu noķēra,\\
Zābakstulmi spīdēja pēc saules,\\
Pats --- pēc frizētavas smaržoja.

Pārējo\\
Neviens tur neošņāja.

Večuks atpirkās ar trijiem rubļiem,\\
Četriem nepārdotiem cirvja kātiem,\\
Pāris desmit garāmslīdošiem\\
Un nosodošiem skatieniem.

Večukam bij cirvja kātu žēl.\\
Ko ar četriem cirvjiem\\
Jaunais cilvēks darīs?

Un trīs rubļu arī bija žel.\\
Žēl to divu dēļ,\\
Kas bija palikuši.

Milicim par visu prieka nebija.\\
Vienīgi kad četrus cirvja kātus\\
Iebāza viņš pirmā atkritumu kastē,\\
Atgriezās tam labs un pierasts miers.

Dzīve atkal tālāk ritēja.\\
Saimniecības preču plaukta gulēja\\
Četru cirfju nocirstās dzelzs galvas.


\newpage

{\large \sc Sienas}

Ko jūs man stāstāt, sienas?

Dzeriet, ko gribat,\\
Bet nekliedziet, lūdzu, man ausīs\\
To, ko es zinu,\\
Un to, ko parasti čukst.

Es ar jums runāju katru dienu.\\
Nevajag bojāt mūsu attiecības.

Es ilgi stāvu\\
Pie caurumainām\\
Un nelādu jauno paaudzi,\\
Kas iet garām.

Viņi nezin, ko atstāj lodes.\\
Un lai viņi nezin ---\\
Tas nav nekas interesants.

Man ir labi,\\
Kad mājas krāso\\
Un tās sastatnēs stāv\\
Kā trākeldiegos pie drēbnieka.

Tad man šķiet,\\
Ka tās pošas uz satikšanos\\
Ar maniem ciemiņiem,\\
Kuri rīt atbrauks.

Logu gaismas un tumsas Morzi\\
Es mācījos, tikko atbraucis,\\
Reizē ar tramvaju maršrutiem.

Es esmu dusmīgs, siena,\\
Kad tu uz mani kliedz.

Es nāku no darba\\
Un eju uz mājām ēst,\\
Bet tu man pa visu sienu:\\
--- Mīliet mūsu Padomju Dzimteni! ---


\newpage

{\large \sc * * *}

Dzeja mīksta kā dīvāns ---\\
Liecies uz auss un pampsti.\\
Visādu lupatu deķi\\
Kaudzē līdz griestiem samesti.

No svešām masku ballēm ---\\
No katras atstiepts pa skrandai.\\
Vienīgi plikais karalis,\\
Kā bija pliks tā blandās.

Kad uz tā dīvāna mosties,\\
Tad visi šie krāšņie mēsli\\
Izskatās tā kā maisi,\\
No kuriem izbērta dvēsele.

Piesvīduši un pieputējuši,\\
Un arī saplēsti daži.\\
Trīs musketieru trīs zobeni ---\\
Taukaini ķēķa naži.

Viss Donkihota bruņojums ---\\
Spocīga dzelžu čupa.\\
Karaļa meitas šķidrauts ---\\
Vai kājauts, vai grīdas lupata.

Nē, es te nevaru gulēt,\\
Man tavas naktsmājas smird.\\
Es šajā dailes kapsētā\\
Vēl dažu nopūtu dzirdu.

Taču ne smaids te uzžilbst,\\
Ne asins no brūces izšļāc,\\
Kāpēc tu vāci šo orķestri,\\
Cilvēciņ, ja tev nav zižļa?


\newpage

{\large \sc Jūrasslimība}

Viņu cēla uz augšu\\
Un laida uz leju,\\
Līdz viņam palika slikti.

Tā iet visiem,\\
Kas itin nekur nav gājuši paši,\\
Bet --- likti.

Kā ziedi,\\
Kā podi,\\
Kā trauki.

Liek tevi tikmēr,\\
Kamēr uz cita plaukta\\
Paliek tavs paša prieks,\\
Tavs paša gods\\
Un tavi draugi.

Un uz pasauli dusmas\\
No savas vemšanas.

Bet pasaule jau nav vainīga,\\
Ka dzīve tev vienīgi sastāv\\
No ņemšanas.


\newpage

{\large \sc * * *}

Es esmu staigājis visādos rūpnīcu cehos\\
Būvbedrēs\\
Rijis miljardus\\
Pa gaisu skrejošu smilšu,\\
Pa rudens ceļiem\\
Mašīnas stūmis ---\\
Un iznācis tīrs.

Kāpēc smērējas rokas un sirdsapziņa\\
Šinī spoguļotajā zālē?

Te roka jāspiež visiem,\\
Bet pēc daža saldena spiediena,\\
Pats nemanot, nobraucu plaukstu\\
Gar rūpīgi izgludinātām biksēm.

Un dažs labs acu pāris\\
Ar riebumu skatās\\
Uz šo šeit tik ļoti nepiedienīgo žestu.

Bet tas ir tāds instinkts.\\
Kad sunim blusas --- kasās,\\
Es daru, kā es māku.

Un sirdsapziņa te smērējas.

Vienādi mīlīgs smaidiņš uz lūpām\\
Gan tiem, kas mīl jubilāru aiz nevarības,\\
Gan tiem, kas to skaita par lopu.

Lūpas kā vienlīdzības zīmes\\
Starp naidu un mīlestību,\\
Starp kāzām un bērēm.

Lai velns rauj šo karātavu diplomātiju!

Es eju uz garderobi\\
Un sargāju savu mēteli,\\
Lai blakus karājošies\\
Ar mīlīgu smaidu\\
Neietriec dunci tā raibajā mugurā.


\newpage

{\large \sc Briesmas}

Nu ir šausmas!\\
Manā kupejā cilvēks bez iekšām!\\
Izbāzts cilvēks ir manā kupejā!

Visu laiku viņš runā\\
Kā plate:\\
--- He, viņš skatās pa logu!\\
Visus kokus bez lapām\\
Noskatīsi.\\
Visas meitenes līdz sirmiem matiem\\
Izskatīsi.

He, viņš meitenei pasaku stāsta!\\
Tumsā nedzīvo nekādi tur\\
Tīģeri.\\
Tumsā noķers to kaut kādi\\
Brūtgāni.

He, viņš lasīs un uguni nedzēsīs!\\
Visas gudrības tikpat\\
Neizlsasīsi.\\
Visam jēgu tikpat jau\\
Neizpētīsi. ---

Tā viņš tur atraugājas,\\
Kamēr es ceļos un saku:\\
--- Mīļais,\\
Kopš tās dienas, kad tev nav iekšu,\\
Tu vairs nezini, kas ir izsalkums.

Mani neviens līdz šim vēl nav izbāzis.\\
Es esmu kokuēdājs,\\
Es esmu pilsētēdājs,\\
Es esmu cilvēkēdājs ---\\
Es esmu\\
Badmira.\\
He!


\newpage

{\large \sc Darbs}

Kā viņi bij nomocījušies,\\
Malēnieši,\\
Veldami baļķus atpakaļ kalnā\\
Un stiepdami gaismu ar maisu!

Kā zivis uz sauszemes mutes vārstīja,\\
Skaties ---\\
Un gribas raudāt aiz žēluma.

Mani nabaga tantali!\\
Mani nabaga novadnieki!

Un kas tie bija,\\
Kas ēdās caur biezputras kalniem?\\
Vai tad tiem bija vieglāk?

Nabaga Gargantija!\\
Nabaga leiputrijieši!

Un atkal vai jāraud ---\\
Cilvēki plīsdami\\
Kaujas ar savu pārpilnību.

Bet katru vasaru slejas pār mežiem\\
Pilnīgi nenopietnā\\
Varavīksne.

Par viņu vīpsnā ---\\
Nu, kas tad tur ir? ---\\
Varavīksne ir celta bez sastatnēm:\\
Ne baļķi pret kalnu velti,\\
Ne kalni ēsti.\\
Ne galvenie inženieri\\
Desmitreiz mainīti.

Un pats galvenais ---\\
Netek sviedri,\\
Trūkst pašas galvenās\\
Cilvēka nemācēšanas pazīmes.

Es spļauju uz varavīksni\\
Un žēlumā raudu:\\
--- Malēnieši!\\
Mani nabaga tantali!\\
Mani pašaizliedzīgie novadnieki!


\newpage

{\large \sc Nu labi}

Nu labi.\\
Mēs abi esam tik briesmīgi labi,\\
Ka normāliem droši vien liekamies spoki ---\\
Mēs nezinām, kas tas ir īsti --- dzert šņabi, ---\\
Tik taisni kā biezoknī auguši koki.

No mums iznāks zirgu loki.\\
Ja meistars mūs salieks...

Bet kas var mūs saliekt,\\
Ja mēs pašā vidū biezoknim augam,\\
Ja skaidri zinām, cik jābūt šmaugam\\
Un nedrīksti pārstiepties pāri draugam!\\
Kas stiepjas,\\
Tam vētra pa purnu saliek.

Nu labi.\\
Mums nezināt, kas ir salīkt,\\
Jo mums līdz kaklam ir māksla --- būt labiem.\\
Mēs zinām katrs un divkārt pa abiem\\
Mēs zinām šo mākslu,\\
Kā labiem palikt.

Nu labi.\\
Bet citi dzer šķiroties šņabi\\
Un brauc ar pirkstu pa meridiāniem,\\
Un pēc tam paši ---\\
Pa kreisi, pa labi,\\
Un mūsu meitenes iet pie to sāniem.

Nu labi.\\
Jau tad, kad mums nācās tikties,\\
Mēs zinājām --- labums jau viņām nav cieņā,\\
Tām gribas likt salā mēli pie stieņa.\\
Nu lai viņas iet...\\
Un lai aiziet pie sliktiem.

Un labi.\\
Bet mēs taču esam labi.\\
Lai sliktie pa purviem, pa ciņiem bristu.\\
Lai izlec,\\
Lai par tiem avīzes raksta!\\
Mēs neesam no tādiem.\\
Mēs abi.

Nu labi.\\
Un sliktiem par mums nespriest tiesu.\\
Pat sīkākais muskulis mums nav ļengans,\\
Mēs atbildam paši par savu miesu,\\
Par mums var liecināt katrs rentgens.

Un tomēr mūs iesūdz?\\
Nu labi.


\newpage

{\large \sc Jūs --- pie telegrāfa!}

--- Rīga jūs izsauc.\\
Ne pie Centrālā ---\\
Pie sirds telegrāfa! ---

Un es zinu un saprotu --- kāpēc.\\
Pārāk daudz\\
Ir jūsmīgi pļāpāts.\\
Jāmācās\\
Pukstošas sirds Morzes ābece.

Lūpas pabālās, lūpas vairīgās\\
Daudzkrāsu ņirboņas bīstas,\\
Izdomā Akrīgas,\\
Izdomā Airīgas,\\
Lai tikai aizšmauktu garām\\
Īstajai.

Tam, kur aptiekās galvas pulveri,\\
Tam, kur cilvēki nīst, mīl un izkaro,\\
Tam, kur pār vecu un sirmu bulvāri\\
Jauniņas lampas\\
Gaismu izstaro.

Un ko līdz šī liekuļu izdoma,\\
Ja viņš pats strādā,\\
Guļ, sēd, iemet sīvo\\
Ne jau tai pilsētā, kuru izdomā, ---\\
Nē, viņš īstajā Rīgā dzīvo.

Nevis Airīgā,\\
Nevis Akrīgā\\
Tajā, kur it neko nevajag zubrīt,\\
Tajā, kur vienā teikumā sakarīgā\\
Ir miliču svilpes\\
Un pilsētu žubītes.

Baznīcu torņu zaļganais pelējums\\
Un jauno kvartālu\\
Augšanas untums,\\
Sirdis, kur dzīvo apsūbējušas peles,\\
Un sirdis,\\
Kas pret visu mēreno buntojas.

Ej pie sirds telegrāfa!\\
Rīga izsauc.\\
Tu esi viņai kā sētnieks\\
Padots.\\
Pēc visa tā, ko vēl nezini, izsalc,\\
Pēc visa tā, ko vēl nejūti, badojies.

Telegrāfs saplosīs Airīgas rīmi,\\
Diktēs tev jauno kvartālu formu.\\
Izpēti visu pilsētas ķīmiju,\\
Lai vari pateikt\\
Tās vienkāršo formulu.

--- Rīga jūs izsauc! ---\\
Lai Morzes pīkstieni\\
Tavās asinīs mūžīgi skanētu,\\
Izpēti miglājus, mudžekļus, haosus,\\
Lai reiz tu drīkstētu\\
Uztaisīt vienu vienkāršu planētu.


\newpage

{\large \sc Arkādija}

Piemiedz acis un paver ---\\
Parīze, vai?\\
Rīga, vai?\\
Dzeltenu kļavziedu pilni mati,\\
Un bites pinas kā trakas,\\
Un pirmie negaisi danco apkārt\\
Kā līgavai.

Un upītes dzelmē guļ\\
Noslēpta sievišķība ---\\
Tāds no atvaru atvariem lielākais\\
Dziļums.

Bet saule līdz potītēm izbrien.\\
Visi strazdi un zvirbuļi\\
Un visas žubītes\\
Apbērnojas.

Un --- te tev nu bija!\\
Bet bija gan...

Kad vasaras mēness\\
Sudrabo koku kupolus,\\
Kad no brieduma, šķiet,\\
Plešas kalns,\\
Pa kuru reiz gāja\\
Rainis ar Aspaziju.

Puikām līdz sešpadsmit gadiem\\
Te nav ko meklēt.

Bet meklē gan.

Ar ko tu, neliete Arkādij, spekulē\\
Un pa kuru laiku tu paspēj,\\
Un kurā speltē\\
Tu slēp savu zeķi,\\
Ka rudeņos tev ir tāds milzums\\
Vara un zelta naudas?

Es brienu kā Rokfellers,\\
Kamēr virs manis\\
Tu lauzi kļavu un kastaņu rokas\\
Pēc tiem vējiem,\\
Ko paņēma citas.

Pati vainīga, tā nemaz nevar sanākt.\\
Bet sanāk gan...

Un tad\\
Tu palien zem vatētas segas,\\
Un tev ir silts prieciņš,\\
Ka mazbērni staigā pa tevi\\
Un ---\\
Ka tev vēl mazdrusciņ kut.


\newpage

{\large \sc Iļģuciema alus}

Vai tad Iļģuciemā mieži aug?\\
Kurā vietā?\\
Mājas saaugušas ir no vienas vietas,\\
Visa zeme\\
Mašīnu un ļaužu nopēdota,\\
Labi vēl,\\
Ja kādai puķei paliek stūris.

Kas par miežiem!\\
Mieži --- tie ir tur ---\\
Aiz tiem zilgiem torņiem\\
Zili meži,\\
Aiz tiem ziliem mežiem\\
Tāli lauki,\\
Tur, tur --- akotu pie akota\\
Iļģuciema alus mieži aug.

Apīņi, kā virves savijušies,\\
Kārtis žņaudz aiz mīlestības nost.

Kāpēc tie uz Iļģuciemu brauc?\\
To jūs paprasiet tam galdniekam.\\
Uz zārka\\
Divas biezas sviestmaizes un alus,\\
Un pats meistars ---\\
Dzīvība uz nāves.

--- Sviestmaize par sausu, ---\\
Meistars smejas,\\
--- Tumšais maizi pataisa par maizi. ---\\
Paprasiet tiem, kas uz baļķa sēž, ---\\
Kāpēc alus?\\
Bet vislabāk ---\\
Paši zāģētavā izkarstiet kā tuksneši.

Visi ``kāpēc'' dabū savus ``tāpēc''.

Un patiešām\\
Puto visi ``tāpēc'':\\
Tāpēc, ka es izaugu uz lauka,\\
Tāpēc, ka es cieši, cieši vijos,\\
Tāpēc, ka ap mani daudz ir gudrots.\\
Tāpēc, ka tev patīk ņemt to visu,\\
Kas vistaisnāk atnācis no zemes.

Ozolkoka mucā ilgi rūgstu,\\
Dzer\\
Un nepaliec par piparkūku!\\
Dzer un rūgsti,\\
Rūgsti pāri malām.

Visiem horizontiem pāri rūgsti.



\newpage

{\large \sc Paegļi}

Viņiem ir ciprešu šmaugums.\\
Tā ir\\
Un reizē nav maska.\\
Viņi reti aug garāk\\
Par cilvēka augumu ---\\
Viņiem gribas,\\
Lai cilvēki saskata\\
Paegļu sīkstumu,\\
Ko vētra nekad nevar nogāzt,\\
Paegļa ilgas pēc debesīm,\\
Kaut arī tas vairāk tic zemei,\\
Paegļu zilās,\\
Apsūbējušās ogas,\\
Kaut arī tās\\
Ne reizi nav runājušās ar zemenēm.

Ka mēs nesaskatītu,\\
Nav jau tā tīri:\\
Kad jāstāsta pasaka,\\
Tad iešalcas meža skaņa,\\
Pa gravām kāpelē\\
Zaļgani paegļu vīriņi.

Kad vajag paegļu sīkstuma ---\\
Ejam un paņemam.

Viņš visās gaitās\\
Mums līdzi nākt alcis.\\
Kāpēc mēs negribam\\
Viņa paegļa dvēselei ticēt?\\
Mežos un pasakās\\
Viņš ir palicis\\
Vienīgais koks,\\
Ko Rīgas parks nolēmis\\
Nepublicēt.



\newpage

{\large \sc * * *}

Tādas nu ir tās pasaules lietas ---\\
Dažs savu mīļo\\
Dēvē par cielaviņu,\\
Dažs par kaiju un dažs par irbi.\\
Es esmu plebejs\\
Un netaisos noliegt tās cilvēka vietas,\\
Kurām ir attāla radniecība\\
Ar ķirbi.

``Fui!'' var izpalikt.\\
``Pē!'' var izpalikt.\\
Normas?\\
Ko viņas ēd pa ziemu --- šīs normas?\\
Ne sakņu, ne augļu dārzs,\\
Ne Galaktika nemaz nevar iztikt\\
Bez šīs plebejiskās\\
Un pilnīgās formas.

Es jūs pat palūgšu\\
Ķirbjus neapcelt ---\\
Viņi ir šādi, un viņi ir tādi.

Kādā formā\\
Visas planētas velt,\\
Tas dievam ienāca prātā,\\
Kad viņš gāja ķirbi stādīt.

Bet visvairāk\\
Un nopietnāk es jūs brīdinu,\\
Lai jums nesanāk vāji\\
Un nesagribas pat kārties, ---

Ar savām acīm es redzēju:\\
Kad slēdza tirgu\\
Un aizgāja pārdevēji,\\
Ķirbji salīda kaktā\\
Un spēlēja kārtis.

Ne uz naudu ---\\
Uz knipjiem un cilvēkiem.



\newpage

{\large \sc Stūrgalvība}

Stūrgalvības arī ir visādas,\\
Dažu kārojas,\\
No dažas jāmūk projām.

Dažreiz skaties ---\\
Nu, velnts ar ārā! ---\\
Cilvēks tev acu priekšā\\
Stūrgalvīgi iet bojā.

Tu ap viņu\\
Kā mazuli līkšņā,\\
Sasauc apt to\\
Simt gudro, simt meļu,

Bet viņš speras iekšā\\
Savējā slīkšņā.

Un, godīgi sakot,\\
Man ir aizdomas,\\
Ka viņš zin tur ceļu.

Lai katrs meklē pats\\
Savas dzērvenes, savus laurus.\\
Nešpetnas maskas\\
Dažreiz nes skaistais.

Pie Torņakalna stacijas\\
Sētā ir caurums,\\
Ko katru mēnesi\\
Vienreiz aiztaisa.

Šķiet, ko tad vairāk ---\\
Sēta kā bilde,\\
Bez viena cauruma,\\
Ne sēta, bet konfekte!

Bet rīt būs caurums.\\
Un, velna milti,\\
Ja cits to neizplēsīs,\\
Tad, teiksim, es konkrēti.

Un nemaz nedibināšu\\
Plēsēju sektu.\\
Ja gribat, saskatiet caurumā šajā\\
Manas domstarpības\\
Ar arhitektiem,\\
Manu ``man patīk''\\
Un manu ``man vajag''.

Šinī ziņā\\
Man visi nav mājās,\\
Kaut kas palicis mantojumā\\
No mežoņiem ---\\
Sētas caurumu\\
Es dievinu jau no bērna kājas,\\
Sevišķi to,\\
Kas ir starp paša negaršīgajām antonovkām\\
Un kaimiņu garšīgajiem mežeņiem.


\newpage

{\large \sc Brāļu kapos}

Mana baltā un mierīgā māt,\\
Es tev zilganu asteri atnesu.\\
Ir pie Kalnciema lielceļa dārzs,\\
Veca māmuļa tur man to iedeva ---\\
Nāves salā jau asteres nav,\\
Nekopj puķes Ložmetējkalns\\
Kopš tā laika... Bet to laiku zin\\
Tavā karogā sastindzis vējš.\\
Bet pa lielceļiem gājēji iet\\
Un no rīta, kad pārgulēts šķūnītī,\\
Maizi ņem, pienu ņem, puķes ņem ---\\
Ņem tās ceļā sev līdzi līdz vītumam.\\
Bet šī astere, ko nesu tev,\\
Negribēja vai neprata līst,\\
Negribēja vai neprata vīst ---\\
Kur tad citur lai tos ziedus liek,\\
Ja ne tiem, ja ne tev, ja ne šeit?\\
Mana baltā un mierīgā māt,\\
Drīz jau leduspuķes vien ziedēs te,\\
Es tās ņemšu un nesīšu tev\\
Arī tad, ja pa ceļam tās izkusīs.


\newpage

{\large \sc Rīti}

Rīti.\\
Melnai pusnaktij dzimušie bērni\\
Iet no loga pie loga,\\
Un Rīgai logu ir daudz.\\
Velti\\
Viņus māte no krēslainiem kaktiem,\\
Klusu draudot un čukstot,\\
Pie sevis atpakaļ sauc.

Nevar.\\
Pland pār pasauli liesmojošs ``nevar''\\
Un pa cilvēku vēnām\\
Ar jaunu spirgtumu tek.\\
Pāri upēm ir pārmesti\\
Rudens appūsti tilti.\\
Pārslas iznāk uz mākoņu malām\\
Un ir gatavas lēkt.

Zeme,\\
Katru rītu tu mums esi karte ---\\
Rīts mūs paceļ tik augstu,\\
Ka horizonts vairs nav taisns.

Taisnīgs.\\
Rīts kā bērns --- maziņš, izsalcis, taisnīgs,\\
Kuram vēl nedzidrās acīs\\
Dzidruma atspīdums.

Pusnakts,\\
Tu mums uztici savējo bērnu.\\
Malku un šauteni,\\
Un izsmieklu nēsājis, es\\
Saprotu katru rītu,\\
Cik maz ir būt vienkārši stipram ---\\
Ar spēku var panest kalnu,\\
Bet bērnu ir jāmāk nest.


\newpage

{\large \sc * * *}

Kas tikai debesīs skrejošie mākoņi\\
Nevar būt!\\
Viss kaut kas,\\
Tikai ne atkārtojums.

Debesīs pasakas skrēja\\
Pa vējam.\\
Debesīs katrudien\\
To laimi, ko tikai Vecgada vakarā,\\
Lēja.

Bet, kas arī neskrietu ---\\
Pūķis vai puķe ---,\\
Viss smēja.

Un Rodēna domātājs varēja staigāt,\\
Cik grib,\\
Viņš bija krokodils,\\
Viņš bija pils,\\
Jo ne Rodēns,\\
Ne filozofs\\
Viņš vēl būt nevarēja.

Vēl nebija sācies Hitlers,\\
Vēl debesīs meitenes mātei\\
Nebija planduši vējā,\\
Vēl debesis neredzošs skats\\
Nebija skatījies tajā,\\
Kas gribēja dzīvot,\\
Bet izdzīvot nevarēja.

Vēl nebija maizei cenas\\
Un devalvācijas naudām.\\
Vēl nebija ziemas, ne rudens, ne vasaras.\\
Bija vien kūpoša zeme\\
Un trīsojošs cīruļu sudrabs debesīs.\\
Sēja.


\newpage

{\large \sc * * *}

Es pat nezinu, kā visas puķes sauc\\
Šajā pļavā.\\
Tas ir briesmīgi,\\
Bet es nezinu.

Elpoju smaržu,\\
Plūcu ---\\
Un nezinu.

Un man to atļauj,\\
Un visiem atļauj ---\\
Nezināt.

Ja tikai puķes vien...\\
Koku, ielu un cilvēku likteņiem\\
Mēs bieži sakām:\\
``Jā!''\\
Un: ``Nē!''

Tāpēc, ka, redziet, mēs nedrīkstam sacīt:\\
``Nezinu.''\\
Tāpēc, ka mēs esam tur, kur jālemj,\\
Un tur nav ---\\
``Nezinu''.

Un dzīve katru brīdi\\
Mums priekšā noliek mapi\\
Ar bezgala tuhačevskiem.


\newpage

{\large \sc Dzērvju deja}

To ļoti reti var redzēt,\\
Jo dzērvju arvien kļūst mazāk.\\
Man pietika ar vienu reizi,\\
Un man vēl šodien ir skumji.\\
Tie droši vien viņām bij prieki\\
Par dienu, kad satikušās,\\
Un tāpēc tās dejoja purvā.\\
Nu, mūsu valdoā --- purvā,\\
Bet viņas --- pasaulē savā.\\
Un dejoja tādu deju,\\
Kur krustības, kāzas un bēres,\\
Un daudz kas man ienāca prātā ---\\
Un ienāca viss vienā reizē ---\\
Tik jocīgi reizē un dziļi.\\
Un es līdz reibumam jutu,\\
Ka man it nekas tāds nav bijis.

Ne naktī skatoties zvaigznēs,\\
Ne ejot uz pirmo deju,\\
Ne aprokot pirmo draugu,\\
Ne pirmoreiz jūtoties liekam.


\newpage

{\large \sc * * *}

Piedod, saimniec,\\
Mēs lamājām tevi,\\
Ka maizei tik bieza garoza.

Mēs neatnesām tev malku.\\
Jo spēlējām domino.

Mēs neredzējām,\\
Kā skrēji\\
Apkopt kaimiņu bērnus.\\
Jo mēs bijām neprecējušies.

Mēs nepieskatījām krāsni,\\
Kamēr tu sēdēji kūtī\\
Un gaidīji brūnaļas teliņu.\\
Jo mēs bijām tikai talcinieki.

Mēs pārlaidām mīklu pār abru\\
Un priecājāmies,\\
Kāda tev seja būs,\\
Kad tu atnāksi.\\
Jo mēs bijām svešinieki.

Par kūpošo maizes mīkstumu\\
Mēs nedomājām.\\
Mēs ēdām\\
Tik sveši,\\
It kā tu nelūgta būtu\\
Atnākusi pie mums.

It kā viss, kas mums apkārt, ---\\
Nelūgts.\\
It kā pie mums ---\\
Četriem jaunekļiem ---\\
Nelūgta, nelūgta\\
Būtu atnākusi\\
Šī zeme.



\newpage

{\large \sc * * *}

Vispirms\\
Ļauj sasildīties un tad dod ēst.

Tu nemāki\\
Rakstīt uz papīra,\\
Pat ja tu dvēselē nemāki\\
Lasīt un rakstīt ---\\
Vispirms ļauj sasildīties\\
Un tad --- dod ēst.

Pēc tam\\
Var aprunāties par gaidāmo laiku,\\
Un vēl pēc tam\\
Var arī atminēties.

Un nedrīkst\\
Šo kārtību sajaukt ---\\
Kas dodams vispirms\\
Un pēc tam.

Reiz iedeva bisi\\
Eskimosiem,\\
Un nopļāva briežus\\
Bises, no vienas,\\
Un mēris, no otras puses.

Un tikai tad sūtīja\\
Maisus ar turku pupām.

Un eskimosi tās lādēja\\
Bisē.


\newpage

{\large \sc * * *}

Es tevi iemīlēju\\
Vasarā, redzēdams baltu.

Es domāju ---\\
Tā gan ir dūšā:\\
Nenomest ziemas glābjošo\\
Balto spalvu\\
Un nekļūt rudai.

Un zināt,\\
Ka tagad vai katrā mājā\\
Pa bisei.

Un turpat vai katrs no medniekiem\\
Šauj nevis tad, kad drīkst,\\
Bet --- kad var.

Es tev piegāju klāt,\\
Baltais brīnum,\\
Un redzēju,\\
Ka bises tu esi saskaitījusi.

Tu teici,\\
Ka paliksi balta, ---\\
Un dzīvosi\\
Mirkli.\\
Bet --- ne pēc bisēm,\\
Bet --- tā, kā tu gribi.

Un es tevi iemīlēju vēl vairāk.


\newpage

{\large \sc * * *}

Izstāsti man savu pasauli, vecenīt!\\
Lai tirgus pastāv!

Es par tavām zālēm un sēklām brīnos.\\
Vai tu patiešām zini,\\
Kā visas tās sauc?\\
Vai tikai ej gar pļavmalu\\
Kā lēna pasaka ---\\
Un pļava pati tev visu stāsta?

Ko tu dari,\\
Kad kupenas gāžas no jumtiem?

Un, kad tu izej pa vārtiem,\\
Uz kuru pusi tu skaties?

Visur jau kaut ko redz pirmo.

Mašīnu jūklis.\\
Reklāmas uguns.\\
Taksometru pietura.\\
Tas ar pelēko, tas ar zaļgano mēteli.

Bet ko tu redzi?

Jā, ja tu gribi ---\\
Es esmu piedzēries.

No tām pļavām, kas tev uz galda,\\
No tā sniega, ko ved no pilsētas ārā,\\
No tiem nosmulētajiem autobusiem,\\
Kuri nāk nepārtraukti,\\
No tā visa, ko avīzēs raksta.

No tām sēklām, kas tev uz galda,\\
Un no tā, kas no viņām izaugs.\\
Un no visa,\\
No kā kaut kas izaugs.

Vecenīt,\\
Velti tu sēdi,\\
Savā mētelī ierāvusies,\\
It kā tas viss vairs uz tevi neattiecas.

Es pats redzēju,\\
Kā tu baroji\\
Tur to pelēko kaķi,\\
Kas pieder visiem.


\newpage

{\large \sc Naktsmājas}

Mums ir šķūnis ar salmiem,\\
Mēs guļam un rukšķam\\
Par to visu,\\
Kas mūs iztrencis\\
Šajās oktobra lietavās sildīties.

Izrādās ---\\
Nav nemaz jābūt\\
Pārlieku tukšam,\\
Lai pasauli celtu pie acīm\\
Un liktu sev pildīties.

Izrādās,\\
Ka daudz ķīmijas saradies\\
Mūsējās jūtās,\\
Mūsējā skatīšanā un mūsu ēdienā.

Lai dzīvo ķīmija!\\
Lai ķīmiķiem labi būtu!\\
Bet lai dzīvo ar mugursomu,\\
Ar kilometriem\\
Un zemi saaudzināta svētdiena!

Gadsimtam ir metālbalss\\
Un dzelžaina roka,\\
Pārāk daudz jēdzienu: pakļaut, pieveikt, piespiest ---\\
Un pārāk maz --- saudzēt.

Bet ne velti pats gadsimts\\
Kliedz pēc māla un koka\\
Un pēc vienas dzeltenas\\
Bērzu ataudzes.

Mēs paši --- daļa no laika\\
Un to sevī nesam,\\
Bet, kādi mēs nebūtu melšas un melši,\\
Pienāk reiz tādas naktsmājas,\\
Kurās mēs domājam,\\
Nevis --- kas kļuvuši esam,\\
Bet --- no kā esam cēlušies.


\newpage

{\large \sc * * *}

Es redzēju nošaušanu.\\
Sapnī.\\
Un pamodies\\
Lēnām kā koks\\
Augu cauri šīm šausmām.

Par kādiem tik vējiem,\\
Zvaigznēm\\
Un mākoņiem\\
Es sev stāstīju augdams!

Stāstīju\\
Ar tādu spītību,\\
Ar kādu bērni skaita līdz simtam\\
Migdami.

Bet man vajadzēja to skaitīt\\
Mostoties.

Bet tikai, uz darbu ejot,\\
Vēss, pelēks vējš,\\
Spuldžu zvaigznes\\
Un kvartālu dzirkstošie mākoņi\\
Man palīdzēja.

Tas nebija sapnis\\
Kaut kur uz pasaules šonakt\\
Nošāva vienu ---\\
Tik bezgala, bezgala līdzīgu man,\\
Ka es dzirdēju.



\newpage

{\large \sc * * *}

Kuru miljono reizi\\
Es tevī esmu iemīlējies, pasaule?\\
Dumjā,\\
Mīļā\\
Un nolādētā.

Atgriežas saule.\\
Un asni kausē\\
Ap sevi visspilgtāko sniegu.

Bērnus ved ārā\\
Rādīt saulei.

Purvos uz ciņiem\\
Lien sildīties čūskas.

Tanki izlien uz tankodromiem.

Jo atgriežas saule.

Un miljono reizi, pasaule,\\
Tu esi mana mīlestība.

Es atdodu sevi\\
Un ņemu tavu\\
Dzīvības lūdzošo apspīdēto\\
Draudīgumu.

\newpage

{\large \sc Monologs}

Ne vien negribas --- man riebjas mirt\\
Tikpat stipri,\\
Cik riebjas uz cilvēkiem tēmēt.\\
No dzīves un dzīvības puses\\
Man ticēts\\
Un uzticēts ir līdz ārprātam ---\\
Manas vēnas\\
Saknes turpina zemē\\
Neiznīdējamas\\
Kā grāvmalas kārklam un vārpatai.

Cērtiet.\\
Es esmu dadzis birztalām\\
Un sprādzīte meiteņu matos,\\
Un, tulkojot latviski veco\\
{\em Ad astra per aspera},\\
Manī ir sastrīdējies\\
Vīna un gāzēta ūdens patoss.

Un es skaidri zinu,\\
Kurš no tiem ne vien spundi,\\
Bet visu mucu man saspēra.\\
Saspēra tas, kas bij īsts,\\
Ar dzīvām kājām mīdīts\\
Un ar dzīvām rokām brūvēts,\\
Reibstošs pēc velna,\\
Kā es to apreibdams atzinu.

Galaktiku\\
Saturēt kopā\\
Nevar ne ar kādām skrūvēm,\\
Tapēc to remontmehāniķi,\\
Kas lien pie manis kā uts,\\
Es iekaustīju un padzinu.

Es esmu laimīgs kā ķērpis uz akmens,\\
Kā vēsai zivij gar sāniem\\
Ūdeņi vēsi.\\
Un, mani pie sirds\\
Vai uz šautenes grauda ņemot,\\
Mehāniķi, es mīlēju jūs,\\
Modri esiet,\\
Es zinu to praksi ---\\
No sākuma iztaisīt kropli\\
Un pēc tam remontēt.


\newpage

{\large \sc * * *}

Pie dabas vajaga griezties\\
Jo laiks ļerkšķ par daudz par dzelžiem\\
Un mēs pārāk lielīgi augstu\\
Esam pār sevi cēlušies.

Tik augstu, ka paši vairs nevaram\\
Saredzēt savas kājas,\\
Tik tālu, ka, uz zemes esot,\\
Mēs īsti vairs neesam mājās.

Tik daudz, ka bail vienam no otra\\
Un visiem kopā --- no rītdienas,\\
Un katra puķe, mums smaržojot,\\
Ne tik daudz smaržo, cik brīdina.

Un glābjošo muļķību, cik tik lien,\\
Mēs sevī un citos uzejam\\
Un paši no savas bezbiogrāfijas\\
Šķīstāmies Brīvdabas muzejā.

Bet stirnēns zin, ko tam saka zars,\\
Kad simts soļu attālāk notrīc,\\
Jo viņa sakars ar visu, kas dzīvs,\\
Kā dzīvība vārīgs un noturīgs.

Mēs ceram uz melu hipnozi.\\
Mēs ceram uz sienu biezumu.\\
Bet nav jau zu zemes akmeņu\\
Tik daudz, cik uz zemes ir briesmu.


\newpage

{\large \sc * * *}

Uz bīstama ceļa stājies,\\
No bailīgas kraujas vēlies,\\
Tu nedomā tik daudz par briesmām,\\
Bet vairāk, ko vēlies --- vēlies.

Ir cilvēkā viena groža,\\
Par ko vajag stingri rūpēt,\\
Ja palaid, tad katrā krūmā\\
Kāds šaušalīgs briesmonis tupēs.

Un uz tavu vārgo miesu\\
Kā kaķis uz baldrijāņiem\\
Trīs nagus. Un locīsies sviedri\\
Kā auksti zalkši gar sāniem.

Tev stomīsies kāja. Tu lūgsies,\\
Lai dievs tevi drīzāk pieņem,\\
Bet jābūt no sava mērķa\\
Kā bērnam no mates ieņemtam.

Neviens tik briesmīgi nekāro\\
Pēc tavas brīnišķās gaļas.\\
Bailes pie cilvēka ciemojas,\\
Kad cilvēkam pārāk daudz vaļas.

Ja tu kaut ko gribi, tad droši,\\
Lai ceļš tev taisns, lai slīps ir,\\
Tu būsi uz lauvas jāšus\\
Un kobra tev būs par šlipsi.


\newpage

{\large \sc * * *}

Es nevaru apklusināt prātu\\
Un nevaru pievaldīt mēli,\\
Kad laikmets ar virsskaņas ātrumu\\
Laiž savu biogrāfiju dēlī.

Ir atkal uzraksti ``mīnēts''\\
Un bumbas ar haizivs smeceri,\\
Ir atkal pie svētās Inerces\\
Kāds aizdedzis lūgšanu sveci.

Ir raķešu tik daudz sakrāts,\\
Ka četras Zemes var sagāzt,\\
Un tāpēc, biedri, šai sakarā\\
Mums vajaga tālāk dragāt.

Laiks grumbās ar zaldātu pieri,\\
Par ģenerāļiem tos darot,\\
Un tik daudz ir pļāpāts par mieru,\\
Ka nupat jau sākt var karot.

Būs slava un vīrišķība,\\
Un visādas citādas lustes,\\
Uz apgruzdējušām čībām\\
Kāds mazulis varbūt kustēs.

Viņš nomirs, jo liesmā rudā\\
Jauns sprādziens ar zemi to sakausēs.\\
Un zemeslode kļūs gluda\\
Kā sklerotiķa pliks pakausis.

Mēs izvēlamies, kā degt, kā dzist.\\
Mēs pakaram sevi vai nepakaram.\\
Es labāk būšu beigts utopists\\
Nekā dzīvs palicis slepkava.


\newpage

{\large \sc Divas odas skorpionam}

{\bf I}

No sirds novēlās akmens,\\
No lūpām rūgts pieneņu piens ---\\
Pie visa vainīgs ir skorpions.\\
Un tāpēc --- lai dzīvo skorpions!

Kas par brīnišķu dzīvnieku!\\
Man šķita, ka skaists ir cilvēks,\\
Bet, kad mūs radiācija saplosīs,\\
Viņš paliks vienīgais Apollons.

Kad mūs kā ūdens pilienu\\
Uz kara pirts ceriem iztvaicēs\\
Un kūpēsim savā nespēkā ---\\
Viņš paliks vienīgais Herkuless.

Un vispār --- viņš spriedīs par visu:\\
Cik atomu šķelt ir prātīgi,\\
Jo cilvēcei vēl nav prātiņa.\\
Jo viņš paliks vienīgais spriedējs.

Jo, izrādās, viņam nekaitīgas\\
Simts mūsu nāves porciju.\\
Var nebūt ne krievu, ne latviešu,\\
Bet dzīvos skorpionu nācija.

Es skatos uz viņa fotogrāfiju ---\\
No sirds novēlies akmens:\\
Es domāju, zeme kļūs nedzīva,\\
Bet, redziet, mūs izglābs skorpions.


\newpage

{\bf II}

Es meklēju mirušu skorpionu.\\
Varbūt man atļaus to preparēt.\\
Man jāredz, ka viņa smadzenēs\\
Nav iebūvēts mūsu neprāts.

Un viņa mazajā galviņā\\
Ar kādām ir iepūsta plēšām\\
Tā viņa drausmīgā intuīcija\\
Un ģēnija paredzēšana,

Ka cilvēki nokāps no koka\\
Un iztēsīs laivu no baļķa,\\
Un vēlāk sāks būvēt mājas\\
No ādām, no zemes, no kaļķa.

No ķieģeļiem un no betona.\\
Un tad būs laiva bez airiem,\\
Un tad sāksies kari un kariņi.\\
Un katrā no tiem šaus vairāk.

Un tad pienāks lielā diena\\
Ar indīgu gāzu smaržām,\\
Kad smadzeņu rievas komandēs\\
Ar dibena rievu seržants.

Kā skorpionam tai galviņā\\
Var cilvēces vēsturi salikt?\\
Viņš salika. Un šausmās redzēja,\\
Ka vajaga dzīvam palikt.

Un tagad viņam pie durvīm,\\
Pie pasaules vienīgām durvīm\\
Nav jāstāda bruņota sardze.

Lai dzīvo tās mazās galviņas,\\
Kas tomēr tik daudz var paredzēt!


\newpage

{\large \sc Mēmais muzikants}

Mūsu dzīvē viens muzikants mīt,\\
Kurš uz izšautas čaulītes spēlē.\\
Viņš ir izgājis cauri\\
Simt ugunīm\\
Un tai simtajā zaudējis mēli.

Tur, kur kādreiz bij pulveris sauss,\\
Indīgs dzīvei kā pulveris kodei, ---\\
Tagad dziesma.\\
Tā dūc mums pie auss,\\
Un šai dziesmai ir skaņa kā lodei.

Tad, kad uzvaru salūti dimd,\\
Dziesma pazūd starp orķestra paliem,\\
Bet, kad svinības paiet un rimst,\\
Tukšās čaulītēs mūzika paliek.

Lai ir kritušais mierīgs un zin,\\
Ka jauns asins pār zemi nav izšļākts.\\
Ka vel čaulītes galā nav svins,\\
Ka vēl ir viņa čaulīte tukša.

Tā pa zemi šis muzikants iet,\\
Un lai katrs nams viņam ir atvērts;\\
Tiklīdz viņš savu muti vērs ciet,\\
Simti šaujamo savējās atvērs.

Mīļie cilvēki, noticiet man,\\
Asins kvēlē un uguņu kvēlē\\
Tas nav vējš, kas aiz loga jums skan, ---\\
Tas ir viņš, kas uz čaulītes spēlē.


\newpage

{\large \sc * * *}

Cūka apēda zīli,\\
Un cilvēks apēda cūku.\\
Bet ozols bij spītīgs,\\
Viņš zināja ---\\
Cūku ir daudz\\
Un neēdis cilvēks ir briesmīgs ---\\
Viņš ne tikai cūkas ēd\\
Un ozolus cērt.

Vēdera dēļ\\
Viņš uz pasaules\\
Vispār visvairāk ir paveicis.

Un ozols taisīja zīles\\
Cūkām,\\
Cilvēkiem\\
Un klusēja par visu pārējo.

Taču vienunakt,\\
Kad noguris nolikos gulēt\\
Uz ozola sakņu virvēm,\\
Es dzirdēju čukstu:

--- Tagad ej.\\
Viņš ir aizmidzis.\\
Bet nav ēdis.

Esi uzmanīgs, dēls!

Tiklīdz tu nonāksi zemē,\\
Apsedzies ar manu lapu\\
Un stipri domā par debesīm.

Ātri domā.\\
Rudens naktis ir garas,\\
Bet, kad viņš tur modīsies,\\
Viņš pasauks cūku.

Tev vajag briesmīgi steigties. ---

Un blakus man nopakšķēja.\\
Un tajā brīdī\\
Nezin kāpēc\\
Es sāku stipri domāt par debesīm.


\newpage

{\large \sc Nakts reiss}

Viņš nedzird,\\
Kā ap kravas kasti plīkšķ brezents\\
Un atbalsis pliķē gravas.

Viņš ir ieķēries stūrē kā ērce\\
Un šoseja priekšā\\
Kā dzejniekam lapa,\\
Kā iemīlējušamies dzīve,\\
Kā pūcei tīrums.

Kā visiem nakstīgiem\\
Kaut kas ir priekšā:\\
Kam --- rinda,\\
Kam --- skūpsti,\\
Kam --- pele.

Un sekunžu pāris līdz snaudam.\\
Un tikpat tālu līdz nāvei.

Šofera snauds.\\
Un --- kaps.\\
Dzejnieka snauds.\\
Un --- kaps.\\
Iemīlējušos snauds.\\
Un --- kaps.\\
Pūces snauds ---\\
Kaps.

Katastrofa.\\
Dvēseles trulums\\
Sašķaidīts sapnis.\\
Bads.

Cik visiem naksnīgiem tas ir tuvu,\\
Un cik dažādas asinis\\
Lāso no viņiem,\\
Ja viņi --- nakts putni ---\\
Aizmirst, ka viņi ir nakts putni!

Un dara,\\
Ko visi pārējie dara pa nakti, ---\\
Aizmieg.


\newpage

{\large \sc * * *}

Es negribu\\
Būt tēsts no bērza pagales.

No dzīvas miesas\\
Dzīvs es esmu plēsts.

Kā muca pārvelties pār tiltu\\
Negribu.\\
Kā paša tukšums apdullina dzirdi,\\
Nezinu.\\
Ka dzīvot var ar aizslēģotu sirdi,\\
Neticu.

Kā uguns --- malku,\\
Tā ēst maizi\\
Negribu.

Jo tilts ir celts,\\
Lai es, vairs nebūdams,\\
Uz zemes būtu atstājis\\
Kaut vienu tiltu.

Ir maize iesēta,\\
Lai arī reiz aiz manis\\
Paliek lauks.\\
Ar manu elpu\\
Plaukstošajās vārpās.

Es dzīvoju\\
Ne mācīdams, bet ---\\
Mācīdamies.

No visām ābecēm,\\
Kur ir kaut vienīgs burts.

Es negribu\\
Būt tēsts no bērza pagales,\\
Jo --- ej nu zini,\\
Ko tas tēsējs iztēš.


\newpage

{\large \sc * * *}

Zvans sev sita pie krūtīm,\\
Ka ir redzējis dievu.

Zvans bija saelpojies\\
Pārāk daudz Gaujas ievu.\\
Elpot, tad dziļi elpot ---\\
Tāda mums visiem paraša,\\
Taču zilskābes inde\\
Ir katras ievas smaržā.\\
Tāpēc no ievām sāp galva\\
Un knišļi no ievām mirst,\\
Un tomēr no jauna un jauna\\
Mēs ievās ar seju nirstam.\\
Un ceļas pierautās plaušas.\\
Kā šļirces adata tieva\\
Dur zilskābe mūs caurcauri,\\
Un tikai aiz indes nāk ieva.\\
Cērt lūpu pomādei cauri\\
Ar saldumu salaulāts rūgtums,\\
Es nezinu, kas būtu ieva,\\
Ja tai šī rūgtuma trūktu.

Es nezinu.\\
Tikai tornī\\
Vēl ilgi, ilgi un grūti\\
Par to, ka redzējis dievu\\
Zvans sita un sita pie krūtīm.


\newpage

{\large \sc Čaks. I}

Nekāpiet virsū ---\\
Tā sirds uz trotuāra ir Čaka.

Uz visiem trotuāriem\\
No Ziepniekkalna līdz Sarkandaugavai\\
Un, kad apbūvēs Ķengaragu,\\
Neslampājiet ---\\
Uz trotuāra būs sirds.

Daudzi pa to jau ir slampājuši.\\
Tie, kam vienalga, kur slampāt ---\\
Pa grīdas tepiķi,\\
Pa tribīnes pakāpieniem,\\
Pa cilvēka jūtām\\
Un atvērtām brūcēm.

Tagad tie puķītes laista\\
Un klusām slampā pie sevis.

Tālāk par Čaku tie netiks,\\
Un mazāk par Čaku tie atstās.

Jo savu lielo un vārīgo sirdi\\
Uz pasaules trotuāriem\\
Visu\\
Ir atstājis dzejnieks,\\
Kam piere kā planetārijs.

Kad tu ar savējo ej\\
Un kad tavējā klusē,\\
Kad Rīga uz pusēm\\
Savu akmeņu smaidus un drūmumu dala,\\
Tu apstājies pēkšņi\\
Pirmā puteņa švīkotā ielā:\\
--- Čaks. ---

Virmo deklāmu raganisks ņirbums:\\
--- Čaks. ---

Verdošs asfalts kutina nāsi:\\
--- Čaks. ---

Meitene skatās uz tevi,\\
Un acis tai mēmi kliedz,\\
Ka beidzot tas esi tu:\\
--- Čaks. ---

Tevī kāpj mīļums\\
Un plēš tevi puši\\
Kā dzīvsudrabs termometru:\\
--- Čaks. ---

Un miglā asaro logs.\\
Un nav ko patiešām liegties,\\
Ka mīlējis esi.

Ne jau velti šai pilsētai\\
Trotuāri ir karsti,\\
Uz katra no tiem ---\\
Sirds.



\newpage

{\large \sc Čaks. II}

Ko viņš gribēja ar to sacīt?

Viņš arī pateica.\\
Viņš mira un neatstāja pēc sevis\\
Vēsā dvēseles betonā\\
Ar bezjūtu roku\\
Iekaltus baušļus,\\
Viņš atstāja\\
Izbrīna pilnas acis,\\
Romantiķus\\
Un kaušļus,\\
Pats būdams reizēm tik trausls\\
Kā stikla mats,\\
Pats sevī briezdams un šaubīdamies\\
Kā milzīga koka pumpurs.

Bet, kad Vecpilsētā\\
Ir čakisks vakars,\\
Kuru ne izteikt var,\\
Ne izdziedāt un ne izskūpstīt,

Cilvēku domas\\
Kā vindas smeļ dzejas akā,\\
Smeļ kādu gaišu un smeldzošu pasauli\\
Visu.

Ar veciem gadiem\\
Tur samaisīts kopā\\
Forštates fabrikas gaudiens\\
Un muskuļu trulā smeldze,\\
Un puisis kā nodzīts kumeļš,\\
Un meitene --- orhideja,\\
Un mēness tiem pāri ---\\
Kristāla lustra,\\
Un līdz pašam rītam\\
Tā mīlējis nav neviens karalis.

Un plīstoša stikla šķindoņa...\\
Visādi gorodovoji\\
Joza pa dzejnieka dvēseles lecektīm,\\
Paši ķerdami\\
Savus murgus.\\
Un plīsa lecekšu stikli\\
Kā ledus pār agrām peļķēm.

Bet tajās lecektīs Čaks\\
Audzēja cilvēkiem\\
Retas puķes\\
Ar neatkārtojamu smaržu.

Jūtiet,\\
Cauri gadiem un kariem, un nāvēm\\
Šī smalkā un dīvainā smarža\\
Dzīvo.

Ko viņš gribēja ar to sacīt?

Viņš gribēja sacīt,\\
Ka vajag būt blēžiem,\\
Vajag izlikties tā kā viņš,\\
Ka esi nomiris.

Bet --- palikt.\\
Kā jasmīnu smaržai\\
Iekļūt cilvēku dvēselē\\
Pa kādu spraugu\\
Un mūžīgi palikt.


\newpage

{\large \sc Talismans}

Pie arheologiem es redzēju\\
Tavu brāli\\
Ar tādām pat acīm\\
Pa visu pieri,\\
Ar tādu pat varavīkšņainu glazūru\\
Pa visu ģīmi.

No tāda pat Ukrainas māla.

Viņš sargāja zemi no krusām\\
Bērnus no čūskām,\\
Sievas no neauglības\\
Un pilsētas vārtus no tatāriem.

Es tevi nelūgšu\\
Ne par zemi, ne bērniem, ne sievu.\\
Un, lai cik dīvaini neizklausītos,\\
Arī par to,\\
Lai tatāri nenāk.

Ja gribi, tad sargā.

Es tevi lūgšu ---\\
Visās briesmās\\
Sargi mani no klaja lauka.

Pie arheologiem es redzēju\\
Blakus ar tavu brāli\\
Mugurkaulus\\
Ar tatāru bultām tajos.\\
Gadsimti glazūras maina,\\
Bet likumi paliek.

Mīļais, smaidīgais sargātāj,\\
Dod pašā baigākā stundā\\
Vienu klinti,\\
Vienu koku,\\
Vai vienu stenderi,\\
Kur atspiest muguru.

Es esmu tipisks cilvēks ---\\
Es esmu bezspēcīgs\\
Gan pret vēju, kas palīdz,\\
Gan pret lodi, kas nobeidz\\
No mugurpuses.




%%\( c \) 1968 Ojārs Vācietis; Ояр Вацетис; 沃齐的斯·奧雅

\end{document}